\section{August 22, 2014}

\renewcommand{\S}{\mathcal{S}}
\renewcommand{\E}{\mathcal{E}}
\begin{nothing}
  Let's think about how equivariant sheaves should be defined. Fix
  some category of spaces $\S$ to work in. Let $G$ be a group object
  in $\S$ and $X$ an object in $\S$ with a $G$-action. A
  \emph{$G$-equivariant vector bundle} over $X$ should be a vector
  bundle $\xi \c E \to X$ where $E$ is also equipped with a $G$-action
  and the map $\xi$ is $G$-equivariant, i.e. the diagram
  \[
  \begin{tikzcd}
    G \times E \ar[r] \ar[d] & E \ar[d] \\ G \times X \ar[r] & X
  \end{tikzcd}
  \]
  commutes.
  
  What does this mean in terms of sheaves? Let $\E$ be the sheaf of
  sections of $\xi$.
\end{nothing}