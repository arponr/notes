
\documentclass[10pt,leqno]{amsart}

\linespread{1.08}
\frenchspacing

\usepackage[%
  tmargin=1.25in,bmargin=1.25in,%
  lmargin=1.75in,rmargin=1.75in,%
  %marginparsep=0.2in, marginparwidth=1.8in%
]{geometry}
\setlength{\skip\footins}{3.0ex}

\usepackage{datetime2}
\usepackage{fancyhdr}
\usepackage{titlesec}
\usepackage{appendix}
\usepackage{microtype}
\usepackage{stmaryrd}
\usepackage{mathabx}
\usepackage[bottom]{footmisc}
\usepackage[hyphens]{url}
\usepackage{enumitem}
\usepackage{xspace}
\usepackage{calc}
\usepackage{etoolbox}
\usepackage{ifthen}
\usepackage{tikz}
\usepackage{tikz-cd}


\definecolor{darkred}{rgb}{0.5,0.0,0.0}
\usepackage[%
  hyperfootnotes=false,
  colorlinks,%
  linkcolor=darkred,%
  citecolor=darkred,%
  urlcolor=darkred,%
]{hyperref}
\urlstyle{rm}

\usepackage{amsmath,amsthm,amssymb}
\usepackage[%
  scr=euler, scrscaled=0.98,%
]{mathalfa}
\usepackage[textsize=footnotesize,backgroundcolor=orange!50]{todonotes}
\usepackage{cleveref}
\usepackage{xr}




\AtBeginDocument{%
  \setlength{\abovedisplayskip}{1.5ex plus 0.3ex minus 0.3ex}%
  \setlength{\abovedisplayshortskip}{1.0ex plus 0.3ex minus 0.3ex}%
  \setlength{\belowdisplayskip}{1.5ex plus 0.3ex minus 0.3ex}%
  \setlength{\belowdisplayshortskip}{1.0ex plus 0.3ex minus 0.3ex}%
}

\let\theoldbibliography\thebibliography
\renewcommand{\thebibliography}[1]{%
  \theoldbibliography{#1}%
  \setlength{\parskip}{0ex}
  \setlength{\itemsep}{0.5ex plus 0.2ex minus 0.2ex}
  \small
}
\apptocmd{\thebibliography}{\raggedright}{}{}

\pagestyle{fancy}
\renewcommand{\headrulewidth}{0pt}
\renewcommand{\footrulewidth}{0pt}
\fancyhf{}
\fancyfoot[C]{\vspace{1ex}\small\thepage}

\renewcommand{\title}[1]{\newcommand{\thetitle}{#1}}
\renewcommand{\author}[1]{\newcommand{\theauthor}{#1}}
\renewcommand{\date}[1]{\newcommand{\thedate}{#1}}

\renewcommand{\maketitle}{%
  \begin{center}
    {\Large\bfseries{%
      \thetitle}}\\[3ex]
    {\theauthor}\\[2ex]
    \ifthenelse{\equal{\thedate}{}}{}{%
      \small%
      % \setlength{\tabcolsep}{0.2em}%
      % \begin{tabular}{rl}
        original: \thedate \\
        updated: \today
      % \end{tabular}
    }
  \end{center}
  \vspace{2.5ex}
  \thispagestyle{fancy}
}

\renewenvironment{abstract}{\section*{Abstract}}{}


% ---------------------------------------------------------------------

\newlength{\tagsep}
\setlength{\tagsep}{1em}

% center display math with respect to full page
% -- amsart.cls
\makeatletter
\def\fullwidthdisplay{\displayindent\z@ \displaywidth\columnwidth}
\edef\@tempa{\noexpand\fullwidthdisplay\the\everydisplay}
\everydisplay\expandafter{\@tempa}
\makeatother

% equation numbering in left margin
% -- http://tex.stackexchange.com/questions/59244
\makeatletter
\let\mytagform@=\tagform@
\def\tagform@#1{\maketag@@@{\hbox{\llap{\small\ignorespaces#1\unskip\@@italiccorr\hspace{\tagsep}}}}\kern1sp}
\renewcommand{\eqref}[1]{{\mytagform@{\ref{#1}}}}
\makeatother

\titleformat{\section}{\bfseries\large}{\llap{\S\thesection\hspace{\tagsep}}}{0em}{}
\titlespacing*{\section}{0pt}{*6}{*2}
\titleformat{\subsection}{\scshape}{\llap{\S\thesubsection\hspace{\tagsep}}}{0em}{}
\titlespacing*{\subsection}{0pt}{*4}{*2}

% Display format for equations
\newcommand{\crefeqfmt}[1]{
  \crefformat{#1}{(##2##1##3)}
  \Crefformat{#1}{(##2##1##3)}
  \crefrangeformat{#1}{(##3##1##4--##5##2##6)}
  \Crefrangeformat{#1}{(##3##1##4--##5##2##6)}
  \crefmultiformat{#1}{(##2##1##3}{, ##2##1##3)}{, ##2##1##3}{, ##2##1##3)}
  \Crefmultiformat{#1}{(##2##1##3}{, ##2##1##3)}{, ##2##1##3}{, ##2##1##3)}
  \crefrangemultiformat{#1}{(##3##1##4--##5##2##6}{, ##3##1##4--##5##2##6)}{, ##3##1##4--##5##2##6}{, ##3##1##4--##5##2##6)}
  \Crefrangemultiformat{#1}{(##3##1##4--##5##2##6}{, ##3##1##4--##5##2##6)}{, ##3##1##4--##5##2##6}{, ##3##1##4--##5##2##6)}
}
% Display format for sections
\newcommand{\crefsecfmt}[1]{%
  \crefformat{#1}{\S##2##1##3}
  \Crefformat{#1}{\S##2##1##3}
  \crefrangeformat{#1}{\S\S##3##1##4--##5##2##6}
  \Crefrangeformat{#1}{\S\S##3##1##4--##5##2##6}
  \crefmultiformat{#1}{\S\S##2##1##3}{ and~##2##1##3}{, ##2##1##3}{ and~##2##1##3}
  \Crefmultiformat{#1}{\S\S##2##1##3}{ and~##2##1##3}{, ##2##1##3}{ and~##2##1##3}
  \crefrangemultiformat{#1}{\S\S##3##1##4--##5##2##6}{ and~##3##1##4--##5##2##6}{, ##3##1##4--##5##2##6}{ and~##3##1##4--##5##2##6}
  \Crefrangemultiformat{#1}{\S\S##3##1##4--##5##2##6}{ and~##3##1##4--##5##2##6}{, ##3##1##4--##5##2##6}{ and~##3##1##4--##5##2##6}
}
\crefeqfmt{equation}
\crefeqfmt{enumi}
\crefeqfmt{enumii}
\crefsecfmt{section}
\crefsecfmt{subsection}
\crefsecfmt{appendix}
\crefname{part}{Part}{Parts}
\crefname{chapter}{Chapter}{Chapters}
\crefname{figure}{Figure}{Figures}

\makeatletter


\newcommand{\thmnumfont}{\small\normalfont}
\newcommand{\thmheadfont}{\bfseries}
\newcommand{\thmnotefont}{\scshape}
\newcommand{\thmhorizspace}{0.4em}
\newcommand{\thmnotespace}{0.25em}
\newcommand{\thmsep}{.}

\newtheoremstyle{block}%
  {1.5ex plus 0.1ex minus 0.1ex}% Space above
  {1.5ex plus 0.1ex minus 0.1ex}% Space below
  {}% Body font
  {}% Indent amount
  {\thmheadfont} % Theorem head font
  {}% Punctuation after theorem head
  {0em}% Space after theorem head
  {\llap{\thmnumber{{\thmnumfont #2}\hspace{\tagsep}}}%
    \thmname{#1}%
    \thmnote{\hspace{\thmnotespace}\thmnotefont(#3)}%
    \@ifnotempty{#1}{\thmsep\hspace{\thmhorizspace}}%
  }% Custom head spec

\newcommand{\defemph}[1]{\textit{#1}}

\renewenvironment{proof}[1][Proof]{\par
  \pushQED{\qed}%
  \normalfont%
  \topsep1ex plus 0.2ex minus 0.1ex\relax%
  \labelsep \thmhorizspace\relax%
  \trivlist
  \item[\hskip\labelsep\thmheadfont#1\@addpunct{\thmsep}]\ignorespaces
}{%
  \popQED\endtrivlist\@endpefalse%
}

\makeatother

\theoremstyle{block}

\newcounter{block}
\newcounter{subblock}
\numberwithin{subblock}{block}
\numberwithin{equation}{subblock}

\newcommand{\defthm}[2]{%
  \newtheorem{#1}[block]{#2}%
  \crefeqfmt{#1}%
  \newtheorem*{#1*}{#2}%

  \newtheorem{sub#1}[subblock]{#2}%
  \crefeqfmt{sub#1}%
  \newtheorem*{sub#1*}{#2}%
}

\defthm{algorithm}{Algorithm}
\defthm{assumption}{Assumption}
\defthm{case}{Case}
\defthm{claim}{Claim}
\defthm{conjecture}{Conjecture}
\defthm{construction}{Construction}
\defthm{convention}{Convention}
\defthm{corollary}{Corollary}
\defthm{definition}{Definition}
\defthm{definitions}{Definitions}
\defthm{example}{Example}
\defthm{examples}{Examples}
\defthm{exercise}{Exercise}
\defthm{fact}{Fact}
\defthm{intuition}{Intuition}
\defthm{lemma}{Lemma}
\defthm{notation}{Notation}
\defthm{nothing}{}
\defthm{observation}{Observation}
\defthm{philosophy}{Philosophy}
\defthm{proposition}{Proposition}
\defthm{question}{Question}
\defthm{recall}{Recall}
\defthm{remark}{Remark}
\defthm{remarks}{Remarks}
\defthm{situation}{Situation}
\defthm{terminology}{Terminology}
\defthm{theorem}{Theorem}

\makeatletter
\long\def\XR@test#1#2#3#4\XR@{%
  \ifx#1\newlabel
    \xr@cref#2@cref\relax#3\@nil
  \else\ifx#1\@input
     \edef\XR@list{\XR@list#2\relax}%
  \fi\fi
  \ifeof\@inputcheck\expandafter\XR@aux
  \else\expandafter\XR@read\fi}

\def\xr@@cref{@cref}
\def\xr@cr@add#1{{[\XR@prefix\@gobble#1}}
\def\xr@cref#1@cref#2\relax#3\@nil{%
\def\tmp{#2}%
\ifx\tmp\xr@@cref
  \edef\tmp{\noexpand\newlabel{\XR@prefix#1@cref}{\xr@cr@add#3}}%
  \tmp
\else
 \newlabel{\XR@prefix#1}{#3}%
\fi
}
\makeatother

% Display format for equations
\newcommand{\externalcrefeqfmt}[2]{
  \crefformat{#1:#2}{[\texttt{#1}, ##2##1##3]}
  \Crefformat{#1:#2}{[\texttt{#1}, ##2##1##3]}
  \crefrangeformat{#1:#2}{[\texttt{#1}, ##3##1##4--##5##2##6]}
  \Crefrangeformat{#1:#2}{[\texttt{#1}, ##3##1##4--##5##2##6]}
  \crefmultiformat{#1:#2}{[\texttt{#1}, ##2##1##3}{, ##2##1##3]}{, ##2##1##3}{, ##2##1##3]}
  \Crefmultiformat{#1:#2}{[\texttt{#1}, ##2##1##3}{, ##2##1##3]}{, ##2##1##3}{, ##2##1##3]}
  \crefrangemultiformat{#1:#2}{[\texttt{#1}, ##3##1##4--##5##2##6}{, ##3##1##4--##5##2##6]}{, ##3##1##4--##5##2##6}{, ##3##1##4--##5##2##6]}
  \Crefrangemultiformat{#1:#2}{[\texttt{#1}, ##3##1##4--##5##2##6}{, ##3##1##4--##5##2##6]}{, ##3##1##4--##5##2##6}{, ##3##1##4--##5##2##6]}
}
% Display format for sections
\newcommand{\externalcrefsecfmt}[2]{%
  \crefformat{#1:#2}{[\texttt{#1}, \S##2##1##3]}
  \Crefformat{#1:#2}{[\texttt{#1}, \S##2##1##3]}
  \crefrangeformat{#1:#2}{[\texttt{#1}, \S\S##3##1##4--##5##2##6]}
  \Crefrangeformat{#1:#2}{[\texttt{#1}, \S\S##3##1##4--##5##2##6]}
  \crefmultiformat{#1:#2}{[\texttt{#1}, \S\S##2##1##3}{ and~##2##1##3]}{, ##2##1##3}{ and~##2##1##3]}
  \Crefmultiformat{#1:#2}{[\texttt{#1}, \S\S##2##1##3}{ and~##2##1##3]}{, ##2##1##3}{ and~##2##1##3]}
  \crefrangemultiformat{#1:2}{[\texttt{#1}, \S\S##3##1##4--##5##2##6}{ and~##3##1##4--##5##2##6]}{, ##3##1##4--##5##2##6}{ and~##3##1##4--##5##2##6]}
  \Crefrangemultiformat{#1:#2}{[\texttt{#1}, \S\S##3##1##4--##5##2##6}{ and~##3##1##4--##5##2##6]}{, ##3##1##4--##5##2##6}{ and~##3##1##4--##5##2##6]}
}

\newcommand{\GRAINROOT}{/Users/arpon/Documents/grain}
\newcommand{\externalgrain}[1]{
  \externaldocument[#1:]{\GRAINROOT /nodes/#1/#1}
  
  \externalcrefeqfmt{#1}{algorithm}
  \externalcrefeqfmt{#1}{case}
  \externalcrefeqfmt{#1}{conjecture}
  \externalcrefeqfmt{#1}{construction}
  \externalcrefeqfmt{#1}{convention}
  \externalcrefeqfmt{#1}{corollary}
  \externalcrefeqfmt{#1}{definition}
  \externalcrefeqfmt{#1}{definitions}
  \externalcrefeqfmt{#1}{example}
  \externalcrefeqfmt{#1}{examples}
  \externalcrefeqfmt{#1}{exercise}
  \externalcrefeqfmt{#1}{fact}
  \externalcrefeqfmt{#1}{intuition}
  \externalcrefeqfmt{#1}{lemma}
  \externalcrefeqfmt{#1}{notation}
  \externalcrefeqfmt{#1}{nothing}
  \externalcrefeqfmt{#1}{proposition}
  \externalcrefeqfmt{#1}{question}
  \externalcrefeqfmt{#1}{recall}
  \externalcrefeqfmt{#1}{remark}
  \externalcrefeqfmt{#1}{remarks}
  \externalcrefeqfmt{#1}{situation}
  \externalcrefeqfmt{#1}{theorem}

  \externalcrefeqfmt{#1}{equation}
  \externalcrefeqfmt{#1}{enumi}
  \externalcrefeqfmt{#1}{enumii}
  \externalcrefsecfmt{#1}{section}
  \externalcrefsecfmt{#1}{subsection}
  \externalcrefsecfmt{#1}{appendix}
}

\setlist{%
  parsep=0ex, listparindent=\parindent,%
  itemsep=0.75ex, topsep=0.75ex,%
  leftmargin=2.5em,%
}

\setlist[enumerate, 1]{%
  label=(\emph{\alph*}),%
  ref={\emph{\alph*}},%
  widest=d,
}
\setlist[enumerate, 2]{%
  leftmargin=*,%
  label=(\theenumi.\arabic*),%
  ref=\theenumi.\arabic*,%
}
\setlist[itemize, 1]{%
  label=$\vcenter{\hbox{\footnotesize$\blacktriangleright$}}$,%
}
\setlist[itemize, 2]{%
  label=--,%
}

% ---------------------------------------------------------------------

\makeatletter

\let\ea\expandafter

\newcount\foreachcount

\def\foreachletter#1#2#3{\foreachcount=#1
  \ea\loop\ea\ea\ea#3\@alph\foreachcount
  \advance\foreachcount by 1
  \ifnum\foreachcount<#2\repeat}

\def\foreachLetter#1#2#3{\foreachcount=#1
  \ea\loop\ea\ea\ea#3\@Alph\foreachcount
  \advance\foreachcount by 1
  \ifnum\foreachcount<#2\repeat}

% Roman: \rA is \mathrm{A}
\def\definerm#1{%
  \ea\gdef\csname r#1\endcsname{\ensuremath{\mathrm{#1}}\xspace}}
\foreachLetter{1}{27}{\definerm}
\foreachletter{1}{27}{\definerm}
% Script: \sA is \mathscr{A}
\def\definescr#1{%
  \ea\gdef\csname s#1\endcsname{\ensuremath{\mathscr{#1}}\xspace}}
\foreachLetter{1}{27}{\definescr}
% Calligraphic: \cA is \mathcal{A}
\def\definecal#1{%
  \ea\gdef\csname c#1\endcsname{\ensuremath{\mathcal{#1}}\xspace}}
\foreachLetter{1}{27}{\definecal}
% Bold: \bA is \mathbf{A}
\def\definebold#1{%
  \ea\gdef\csname b#1\endcsname{\ensuremath{\mathbf{#1}}\xspace}}
\foreachLetter{1}{27}{\definebold}
% Blackboard Bold: \lA is \mathbb{A}
\def\definebb#1{%
  \ea\gdef\csname l#1\endcsname{\ensuremath{\mathbb{#1}}\xspace}}
\foreachLetter{1}{27}{\definebb}
% Fraktur: \ka is \mathfrak{a}, \kA is \mathfrak{A}
\def\definefrak#1{%
  \ea\gdef\csname k#1\endcsname{\ensuremath{\mathfrak{#1}}\xspace}}
\foreachletter{1}{27}{\definefrak}
\foreachLetter{1}{27}{\definefrak}
% Sans serif: \iA \is \mathsf{A}
\def\definesf#1{%
  \ea\gdef\csname i#1\endcsname{\ensuremath{\mathsf{#1}}\xspace}}
\foreachletter{1}{6}{\definesf}
\foreachletter{7}{14}{\definesf}
\foreachletter{15}{27}{\definesf}
\foreachLetter{1}{27}{\definesf}
% Bar: \Abar is \overline{A}, \abar is \overline{a}
\def\definebar#1{%
  \ea\gdef\csname #1bar\endcsname{\ensuremath{\overline{#1}}\xspace}}
\foreachLetter{1}{27}{\definebar}
\foreachletter{1}{8}{\definebar} % \hbar is something else!
\foreachletter{9}{15}{\definebar} % \obar is something else!
\foreachletter{16}{27}{\definebar}
% Tilde: \Atil is \widetilde{A}, \atil is \widetilde{a}
\def\definetil#1{%
  \ea\gdef\csname #1til\endcsname{\ensuremath{\widetilde{#1}}\xspace}}
\foreachLetter{1}{27}{\definetil}
\foreachletter{1}{27}{\definetil}
% Hats: \Ahat is \widehat{A}, \ahat is \widehat{a}
\def\definehat#1{%
  \ea\gdef\csname #1hat\endcsname{\ensuremath{\widehat{#1}}\xspace}}
\foreachLetter{1}{27}{\definehat}
\foreachletter{1}{27}{\definehat}
% Checks: \Achk is \widecheck{A}, \achk is \widecheck{a}
\def\definechk#1{%
  \ea\gdef\csname #1chk\endcsname{\ensuremath{\widecheck{#1}}\xspace}}
\foreachLetter{1}{27}{\definechk}
\foreachletter{1}{27}{\definechk}
% Underline: \Aund is \underline{A}, \aund is \underline{a}
\def\defineul#1{%
  \ea\gdef\csname #1und\endcsname{\ensuremath{\underline{#1}}\xspace}}
\foreachLetter{1}{27}{\defineul}
\foreachletter{1}{27}{\defineul}

\makeatother

% ---------------------------------------------------------------------

\usetikzlibrary{calc,decorations.pathmorphing,shapes,arrows}
\tikzcdset{
  arrow style=tikz,
  diagrams={>={stealth}},
}

\newcommand{\arrlen}{1.25em}
\newcommand{\shortarrlen}{0.85em}
\renewcommand{\to}{\mathrel{\tikz[baseline]%
    \draw[>=stealth,->](0,0.5ex)--(\arrlen,0.5ex);}}
\newcommand{\limto}{\mathrel{\tikz[baseline]%
    \draw[>=stealth,->](0,0.4ex)--(\shortarrlen,0.4ex);}}
\newcommand{\from}{\mathrel{\tikz[baseline]%
    \draw[>=stealth,<-](0,0.5ex)--(\arrlen,0.5ex);}}
\renewcommand{\mapsto}{\mathrel{\tikz[baseline]%
    \draw[>=stealth,|->](0,0.5ex)--(\arrlen,0.5ex);}}
\newcommand{\inj}{\mathrel{\tikz[baseline]%
    \draw[>=stealth,right hook->](0,0.5ex)--(\arrlen,0.5ex);}}
\newcommand{\surj}{\mathrel{\tikz[baseline]%
    \draw[>=stealth,->>](0,0.5ex)--(\arrlen,0.5ex);}}
\newcommand{\fromto}{\mathrel{%
  \begin{tikzpicture}[baseline]%
    \draw[>=stealth,<-](0,0.15ex)--(\arrlen,0.15ex);%
    \draw[>=stealth,->](0,0.85ex)--(\arrlen,0.85ex);%
  \end{tikzpicture}}}
\newcommand{\doubto}{\mathrel{%
  \begin{tikzpicture}[baseline]%
    \draw[>=stealth,->](0,0.15ex)--(\arrlen,0.15ex);%
    \draw[>=stealth,->](0,0.85ex)--(\arrlen,0.85ex);%
  \end{tikzpicture}}}
\newcommand{\goesto}{\mathrel{%
  \begin{tikzpicture}[baseline= {( $ (current bounding box.south) + (0,-0.3ex) $ )}]%
    \draw[>=stealth,->,decorate,%
          decoration={zigzag,amplitude=0.15ex,segment length=0.35em,pre=lineto,pre length=.15em,post=lineto,post length=.3em}](0,0.15ex)--(\arrlen,0.15ex);%
  \end{tikzpicture}}}
\newcommand{\lblto}[1]{\mathrel{%
    \begin{tikzpicture}[baseline= {( $ (current bounding box.south) + (0,-0.5ex) $ )}]
      \node[inner sep=.4ex] (a) {\,$\scriptstyle #1$\,};
      \draw[>=stealth,->] (a.south west) -- (a.south east);
    \end{tikzpicture}}}
\newcommand{\isoto}{\lblto{\sim}}

\newcommand{\simpl}[3]{
  \begin{tikzcd}[ampersand replacement=\&, column sep=small]
    #1 \&
    #2 \ar[l, shift right=0.35ex]
       \ar[l, shift left=0.35ex] \&
    #3 \ar[l, shift right=0.70ex]
       \ar[l, shift left=0.70ex]
       \ar[l] \&
    \cdots \ar[l, shift right=0.35ex]
           \ar[l, shift left=0.35ex]
           \ar[l, shift right=1.05ex]
           \ar[l, shift left=1.05ex]
  \end{tikzcd}
}
\newcommand{\cosimpl}[3]{
  \begin{tikzcd}[ampersand replacement=\&, column sep=small]
    #1 \ar[r, shift right=0.35ex]
       \ar[r, shift left=0.35ex] \&
    #2 \ar[r, shift right=0.70ex]
       \ar[r, shift left=0.70ex]
       \ar[r] \&
    #3 \ar[r, shift right=0.35ex]
       \ar[r, shift left=0.35ex]
       \ar[r, shift right=1.05ex]
       \ar[r, shift left=1.05ex] \&
    \cdots
  \end{tikzcd}
}

\newcommand{\tto}{\mathrel{\tikz[baseline]%
    \draw[>=stealth,->,double, double distance = 0.3ex](0,0.5ex)--(\arrlen,0.5ex);}}
\newcommand{\doubfrom}{\mathrel{%
  \begin{tikzpicture}[baseline]%
    \draw[>=stealth,<-](0,0.15ex)--(\arrlen,0.15ex);%
    \draw[>=stealth,<-](0,0.85ex)--(\arrlen,0.85ex);%
  \end{tikzpicture}}}
\newcommand{\tripfrom}{\mathrel{%
  \begin{tikzpicture}[baseline]%
    \draw[>=stealth,<-](0,0.00ex)--(\arrlen,0.00ex);%
    \draw[>=stealth,<-](0,0.50ex)--(\arrlen,0.50ex);%
    \draw[>=stealth,<-](0,1.00ex)--(\arrlen,1.00ex);%
  \end{tikzpicture}}}


\renewcommand{\l}{\left}
\renewcommand{\r}{\right}
\renewcommand{\o}{\overline}
\renewcommand{\u}{\underline}
\newcommand{\til}{\widetilde}
\renewcommand{\hat}{\widehat}
\newcommand{\del}{\partial}
\newcommand{\bul}{\bullet}
\newcommand{\dash}{\text{-}}
\renewcommand{\c}{\colon}
\newcommand{\lc}{\,:\!}
\newcommand{\ce}{\mathrel{:=}}%{\coloneq}
\newcommand{\ec}{\mathrel{=:}}%{\eqcolon}
\newcommand{\iso}{\simeq}
\newcommand{\lbliso}[1]{\overset{#1}{\simeq}}
\newcommand{\dual}[1]{#1^\vee}
\newcommand{\ldb}{\llbracket}
\newcommand{\rdb}{\rrbracket}
\newcommand{\shimplies}{\Rightarrow}
\newcommand{\shimplied}{\Leftarrow}

\newcommand{\Obj}{\operatorname{Obj}}
\newcommand{\Hom}{\operatorname{Hom}}
\newcommand{\Map}{\operatorname{Map}}
\newcommand{\Fun}{\operatorname{Fun}}
\newcommand{\Aut}{\operatorname{Aut}}
\newcommand{\End}{\operatorname{End}}
\newcommand{\Iso}{\operatorname{Iso}}
\newcommand{\Ext}{\operatorname{Ext}}
\renewcommand{\id}{\mathrm{id}}
\renewcommand{\im}{\operatorname{im}}
\newcommand{\pt}{\star}
\newcommand{\op}{\mathrm{op}}
\newcommand{\univ}{\mathrm{univ}}
\newcommand{\colim}{\operatorname*{colim}}
\newcommand{\coeq}{\operatorname*{coeq}}
\newcommand{\cofib}{\operatorname*{cofib}}
\newcommand{\coker}{\operatorname*{coker}}
\newcommand{\fib}{\operatorname*{fib}}
\newcommand{\eq}{\operatorname*{eq}}
\newcommand{\dlim}{\displaystyle\lim}
\newcommand{\dcolim}{\displaystyle\colim}
\newcommand{\Spec}{\operatorname{Spec}}
\newcommand{\Spf}{\operatorname{Spf}}

% ---------------------------------------------------------------------


