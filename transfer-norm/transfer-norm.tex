
%%%%%%%%%%%%%%%%%%%%%%%%%%%%%%%%%%%%%%%%%%%%%%%%%%%%%%%%%%%%%%%%%%%%%%

\renewcommand{\A}{\mathbb{A}}
\renewcommand{\O}{\mathcal{O}}

\renewcommand{\a}{\mathfrak{a}}
\newcommand{\p}{\mathfrak{p}}
\newcommand{\q}{\mathfrak{q}}

\newcommand{\height}{\operatorname{ht}}

%%%%%%%%%%%%%%%%%%%%%%%%%%%%%%%%%%%%%%%%%%%%%%%%%%%%%%%%%%%%%%%%%%%%%%


\title{Duality, transfer, and norm}
\author{Arpon Raksit}
\date{2017-04-17}

\numberwithin{block}{section}

\begin{document}
\maketitle

\newcommand{\BG}{{\rB G}}
\newcommand{\hG}{{\rh G}}
\newcommand{\Loc}{\operatorname{Loc}}
\newcommand{\Nm}{\operatorname{Nm}}
\newcommand{\Spaces}{\mathrm{Spaces}}
\newcommand{\Spectra}{\mathrm{Spectra}}
\newcommand{\tG}{{\rt G}}

% ---------------------------------------------------------------------

In these notes we'll see how certain transfer and norm maps in stable homotopy theory arise from Spanier-Whitehead duality applied in a parametrized setting.\footnote{Thanks to Allen Yuan for explaining some of this to (and discussing all of it with) me.}

% ---------------------------------------------------------------------

\section{Transfers}

% ---------------------------------------------------------------------

\section{The norm}
\label{norm}

\begin{notation}
  \label{norm-setup}
  Let $\bG$ be a compact Lie group. Let $G$ denote the underlying homotopy type (\textsc{aka} fundamental $\infty$-groupoid) of $\bG$; this has the structure of a group space.
\end{notation}

\begin{nothing}
  \label{norm-goal}
  For $X$ a spectrum with (``doubly na\"ive'') $G$-action, there is a \emph{norm map}
  \[
    \Nm_X \c (\Sigma^\infty\rS^\kg \otimes X)_{\hG} \to X^\hG,
  \]
  where $\kg$ denotes the Lie algebra of $\bG$, and $\rS^\kg$ denotes the one-point compactification of $\kg$, viewed as a pointed space with $G$-action via the adjoint representation of $\bG$ on $\kg$. This arises, for instance, in the definition of the \emph{Tate spectrum}, which is the cofiber of this map: $X^\tG \ce \cofib(\Nm_X)$. Our goal in this section is to give a construction of this norm map.
\end{nothing}

\begin{nothing}
  \label{norm-naive}
  As indicated above, we will only need to work with $G$-equivariant objects of the most na\"ive nature.

  \begin{subnotation}
    \label{norm-naive-local-systems}
    As usual, $\BG$ denotes the classifying space (\textsc{aka} delooping) of $G$. If $C$ is a category, an \emph{object in $C$ with $G$-action} is a functor $\BG \to C$, i.e. a $C$-valued local system on $\BG$; the category of these is denoted $\Loc(\BG;C)$.

    The example of primary interest here is $C = \Spectra$. The smash product $\otimes$ on $\Spectra$ defines a symmetric monoidal structure also on $\Loc(\BG;\Spectra)$, formed pointwise, which we also denote $\otimes$.

    We will also (perhaps implicitly) be considering the examples $C = \Spaces$ and $C = \Spaces_\pt$. The functors $\Sigma^\infty \c \Spaces_\pt \to \Spectra$ and $\Sigma^\infty_+ \c \Spaces \to \Spectra$ induce functors $\Sigma^\infty \c \Loc(\BG;\Spaces_\pt) \to \Loc(\BG;\Spectra)$ and $\Sigma^\infty_+ \c \Loc(\BG;\Spaces) \to \Loc(\BG;\Spectra)$. \footnote{Note that these notational conventions were employed already in \cref{norm-goal}, in particular in the object $\Sigma^\infty \rS^\kg \otimes X \in \Loc(\BG;\Spectra)$ appearing in the source of the norm map.}
  \end{subnotation}

  \begin{subconvention}
    \label{norm-left}
    In line with our definition in \cref{norm-naive-local-systems} of objects with $G$-action as functors out of $\BG$ (rather than $\BG^\op$), we will only ever consider \emph{left} group actions; this goes not only for $G$-actions but for $\bG$-actions (on manifolds) too.

    We let $\bG^\rr$ denote $\bG$ as a manifold equipped with the $\bG$-action given by inverted right-multiplication; that is, the action $\bG \times \bG^\rr \to \bG^\rr$ is given by $(g,h) \mapsto hg^{-1}$. We analogously define $G^\rr$ as a space with $G$-action.
  \end{subconvention}
\end{nothing}

\begin{nothing}
  \label{norm-dual}
  We begin the actual work by computing the Spanier-Whitehead dual of $\Sigma^\infty_+ G^\rr$, via Atiyah duality. This calculation is responsible for the appearance of $\rS^\kg$ in the norm map.

  \begin{sublemma}
    \label{norm-dual-tangent-bundle}
    View $\kg$ as a representation of $\bG$ via the adjoint action. Then there is an isomorphism of $\bG$-equivariant vector bundles $\rT\bG^\rr \iso \bG^\rr \times \kg$ over $\bG^\rr$.

    \begin{proof}
      As on any Lie group, we have a trivialization of the tangent bundle
      \[
        \bG \times \kg \isoto \rT\bG, \quad (h,\xi) \mapsto \rd l_h(\xi),
      \]
      where $l_h \c \bG \to \bG$ denotes left-multiplication by $h$. We just need to check that this isomorphism is $\bG$-equivariant when we equip $\bG$ with the $\bG$-action of $\bG^\rr$ and $\kg$ with the adjoint action. That is, we need to check that the diagram
      \[
        \begin{tikzcd}
          \bG \times \bG^\rr \times \kg \ar[d, "\alpha"] \ar[r, "\gamma"] &
          \bG^\rr \times \kg \ar[d, "\delta"] \\
          \bG \times \rT\bG^\rr \ar[r, "\beta"] &
          \rT\bG^\rr
        \end{tikzcd}
      \]
      commutes, where
      \begin{align*}
        \alpha(g,h,\xi) \ce (g,\rd l_h(\xi)), \qquad
        \beta(g,v) \ce \rd r_{g^{-1}}(v), \\
        \gamma(g,h,\xi) \ce (hg^{-1},\rd c_g(\xi)), \qquad
        \delta(h,\xi) \ce \rd l_h(\xi),
      \end{align*}
      with $r_{g^{-1}} \c \bG \to \bG$ denoting right-multiplication by $g^{-1}$ (i.e. the action of $g$ on $\bG^\rr$) and $c_g \c \bG \to \bG$ denoting conjugation $k \mapsto gkg^{-1}$ (whose derivative at the identity in $\bG$ is by definition the adjoint action of $g$ on $\kg$). This follows from the equality of maps $r_{g^{-1}} \circ l_h = l_{hg^{-1}} \circ c_g$.
    \end{proof}
  \end{sublemma}
\end{nothing}

% ---------------------------------------------------------------------

% \bibliographystyle{amsalpha}
% \bibliography{refs}

\end{document}
