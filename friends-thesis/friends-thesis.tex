%%%%%%%%%%%%%%%%%%%%%%%%%%%%%%%%%%%%%%%%%%%%%%%%%%%%%%%%%%%%%%%%%%%%%%

\newcommand{\ob}{\oper{ob}}
\renewcommand{\hom}{\oper{hom}}
\newcommand{\id}{\oper{id}}
\newcommand{\im}{\oper{im}}
\newcommand{\op}{\oper{op}}

\newcommand{\Top}{\oper{Top}}
\newcommand{\Set}{\oper{Set}}
\newcommand{\Ab}{\oper{Ab}}
\newcommand{\Grp}{\oper{Grp}}
\newcommand{\Mod}{\oper{Mod}}
\newcommand{\Simplex}{\Delta}
\newcommand{\s}{\oper{s}}
\newcommand{\Ch}{\oper{Ch}}

\newcommand{\Sing}{\oper{Sing}}
\renewcommand{\H}{\mathrm{H}}

%%%%%%%%%%%%%%%%%%%%%%%%%%%%%%%%%%%%%%%%%%%%%%%%%%%%%%%%%%%%%%%%%%%%%%


\title{Comparing shapes}
\author{Arpon Raksit}
\date{April 24, 2015}

\begin{document}
\maketitle

%%%%%%%%%%%%%%%%%%%%%%%%%%%%%%%%%%%%%%%%%%%%%%%%%%%%%%%%%%%%%%%%%%%%%%

\section{Introduction}

I found (and continue to find) the topic of my thesis so interesting
because it is a story which brings together characters from several
different areas of mathematics, e.g. algebraic topology, algebraic
geometry, and representation theory. But if I tried to start telling
this story, you'd immediately (and reasonably) interrupt me to ask
something like: Who are these characters? What are their motivations?
Why should we care about them? Answering these questions and really
appreciating the story necessitates a substantial prologue. My goal in
this talk is just to relate part of this prologue, by giving an
introduction (hopefully a somewhat intuitive one) to a fundamental
idea in algebraic topology which plays a main role in my thesis, that
of a \emph{cohomology theory}.


%%%%%%%%%%%%%%%%%%%%%%%%%%%%%%%%%%%%%%%%%%%%%%%%%%%%%%%%%%%%%%%%%%%%%%

\section{Shapes/spaces}

\begin{nothing}[Shapes]
  Maybe you've heard this a million times but let me say it anyway:
  topology is about studying shapes (mathematically and
  precisely). The basic problem is that of telling when two shapes are
  the same or distinct. Of course I haven't even said what I mean by
  \emph{shape}, or what it means for two shapes to be \emph{the same};
  i.e. I haven't said anything.
  
  Probably the most intuitive/familiar way to think about a shape is
  as a subset of euclidean space $\lR^n$, e.g. a circle in $\lR^2$
  (the blackboard) or some solid object in $\lR^3$, or maybe even
  $\lR^4$ if we think about time.

  Then one guess for a notion of sameness for shapes would be
  set-theoretic equality, i.e. two shapes are the same if they
  literally consist of the same points in $\lR^n$. But this is clearly
  far too stringent: if I translate or rotate a subset I get a
  different subset, but certainly have not changed its shape. And
  actually in topology we're even more lenient than that: roughly
  speaking, we say two shapes are the same if each can be continuously
  deformed (molded without tearing, puncturing, etc.)  into the
  other. E.g. a circle is the same (topologically) as a square and a
  triangle; a solid disk is the same as a solid square and solid
  triangle, which are all the same as a single point.
\end{nothing}

\begin{nothing}[Spaces]
  One interested thing to note from the last example (a disk contracts
  to a point) is that two shapes can be the same without having the
  same dimension (for our intuitive notion of dimension). In
  particular, the dimension of the euclidean space in which we embed
  our shape is \emph{not} a fundamental property of the
  shape. E.g. all the solid disks/balls $D^n \subseteq \lR^n$ are the
  same topologically (as a point). This indicates a conceptual shift
  that one should make: the fundamental properties of a shape can be
  thought of \emph{intrinsically}, independent of the way in which it
  is embedded inside euclidean space. I.e. we should think of shapes
  themselves as \emph{spaces} in/on which something can live, and the
  fundamental properties of these spaces are things which can be
  tested by the things which live inside of it. But of course for
  intuition it often helps to actually see things, and for that we
  automatically think with embeddings because that's how our brains
  work.

  So for the remainder of the talk we'll switch terminology from
  \emph{shapes} to \emph{spaces}, which we'll think of more abstractly
  as sets of points with some extra structure which tells us when we
  should think of two points as \emph{close}. (Note we shouldn't speak
  of actual distances, since these can change when we deform the
  space.) If we have two spaces $X$ and $Y$, there is a notion of a
  \emph{map(ping)} from $X$ to $Y$: it is a way of assigning to each
  point of $X$ a point of $Y$, such that points that are close in $X$
  are taken to points which are close in $Y$.
\end{nothing}

\begin{nothing}[Homotopy]
  I've been vague about what exactly a space is, but if you allow me
  leave that there I can tell you more precisely what I mean by
  continuous deformation and sameness of spaces.
\end{nothing}

\begin{nothing}[First invariants: $\pi_0,\pi_1$]
  Hi
\end{nothing}

\begin{nothing}[Higher homotopy groups]
  
\end{nothing}

%%%%%%%%%%%%%%%%%%%%%%%%%%%%%%%%%%%%%%%%%%%%%%%%%%%%%%%%%%%%%%%%%%%%%%

\section{Cohomology theories as invariants}

\begin{nothing}[Criteria for computability]
  
\end{nothing}

%%%%%%%%%%%%%%%%%%%%%%%%%%%%%%%%%%%%%%%%%%%%%%%%%%%%%%%%%%%%%%%%%%%%%%

\section{Cohomology theories as topological objects}

%%%%%%%%%%%%%%%%%%%%%%%%%%%%%%%%%%%%%%%%%%%%%%%%%%%%%%%%%%%%%%%%%%%%%%

\bibliographystyle{amsalpha}
\bibliography{refs}

\end{document}
