%%%%%%%%%%%%%%%%%%%%%%%%%%%%%%%%%%%%%%%%%%%%%%%%%%%%%%%%%%%%%%%%%%%%%%

\newcommand{\ob}{\oper{ob}}
\renewcommand{\hom}{\oper{hom}}
\newcommand{\id}{\oper{id}}
\newcommand{\im}{\oper{im}}
\newcommand{\op}{\oper{op}}

\newcommand{\Top}{\oper{Top}}
\newcommand{\Set}{\oper{Set}}
\newcommand{\Ab}{\oper{Ab}}
\newcommand{\Grp}{\oper{Grp}}
\newcommand{\Mod}{\oper{Mod}}
\newcommand{\Simplex}{\Delta}
\newcommand{\s}{\oper{s}}
\newcommand{\Ch}{\oper{Ch}}

\newcommand{\Sing}{\oper{Sing}}
\renewcommand{\H}{\mathrm{H}}

%%%%%%%%%%%%%%%%%%%%%%%%%%%%%%%%%%%%%%%%%%%%%%%%%%%%%%%%%%%%%%%%%%%%%%


\title{Comparing shapes}
\author{Arpon Raksit}
\date{April 23, 2015}

\begin{document}
\maketitle

%%%%%%%%%%%%%%%%%%%%%%%%%%%%%%%%%%%%%%%%%%%%%%%%%%%%%%%%%%%%%%%%%%%%%%

\section{Introduction}

\begin{nothing}[Abstract]
  The goal of my thesis is to rigorously explain a certain theorem in
  algebraic topology. But there is a great deal of language and
  background with which one must familiarize oneself in order for to
  appreciate this theorem, and there's no way I can relay this
  background in twenty minutes. Thus my goal today is just to give
  some kind of intuitive, heuristic sense of what this theorem is
  about, and in particular introduce the sort of mathematical objects
  it concerns. So please forgive me for the utter lack of precision
  that pervades following.
\end{nothing}

\begin{nothing}[Acknowledgements]
  I must thank my advisor Jacob Lurie for helping me enormously in
  writing my thesis, and in preparing this talk. And I must thank my
  parents, who happen to be here in Cambridge this weekend, for
  helping me in everything else, and for happily sitting through this
  talk.
\end{nothing}

%%%%%%%%%%%%%%%%%%%%%%%%%%%%%%%%%%%%%%%%%%%%%%%%%%%%%%%%%%%%%%%%%%%%%%

\section{Spaces}

\begin{nothing}[Shapes]
  Maybe you've heard this a million times but let me say it anyway:
  topology is the (mathematical and precise) study of shapes. The
  basic problem is that of telling when two shapes are the same or
  distinct. Of course I haven't even said what I mean by \emph{shape},
  or what it means for two shapes to be \emph{the same}; i.e. I
  haven't said anything.
  
  Probably the most intuitive/familiar way to think about a shape is
  as a subset of euclidean space $\lR^n$, e.g. a circle in $\lR^2$
  (the blackboard) or some solid object in $\lR^3$, or maybe even
  $\lR^4$ if we take time into consideration.

  Then one guess for a notion of sameness for shapes would be
  set-theoretic equality, i.e. two shapes are the same if they
  literally consist of the same points in $\lR^n$. But this is clearly
  far too stringent: if I translate or rotate a subset I get a
  different subset, but certainly have not changed its shape. And
  actually in topology we're even more lenient than that: roughly
  speaking, we say two shapes are the same if each can be continuously
  deformed (molded without tearing, puncturing, etc.)  into the
  other. E.g. a circle is the same (topologically) as a square and a
  triangle; a solid disk is the same as a solid square and solid
  triangle, which are all the same as a single point.
\end{nothing}

\begin{nothing}[Spaces]
  One interested thing to note from the last example (a disk contracts
  to a point) is that two shapes can be the same without having the
  same dimension (for our intuitive notion of dimension). In
  particular, the dimension of the euclidean space in which we embed
  our shape is \emph{not} a fundamental property of the
  shape. E.g. all the solid disks/balls $D^n \subseteq \lR^n$ are the
  same topologically (as a point). This indicates a conceptual shift
  that one should make: the fundamental properties of a shape can be
  thought of \emph{intrinsically}, independent of the way in which it
  is embedded inside euclidean space. I.e. we should think of shapes
  themselves as \emph{spaces} in/on which something can live, and the
  fundamental properties of these spaces are things which can be
  tested by the things which live inside of it. But of course for
  intuition it often helps to actually see things, and for that we
  automatically think with embeddings because that's how our brains
  work.

  So for the remainder of the talk we'll switch terminology from
  \emph{shapes} to \emph{spaces}, which we'll think of more abstractly
  as sets of \emph{points} with some extra structure which tells us
  when we should think of two points as \emph{close}. (Note we
  shouldn't speak of actual distances, since these can change when we
  deform the space.) If we have two spaces $X$ and $Y$, there is a
  notion of a \emph{map(ping)} from $X$ to $Y$: it is a way of
  assigning to each point of $X$ a point of $Y$, such that points that
  are close in $X$ are taken to points which are close in $Y$.
\end{nothing}

\begin{nothing}[Homotopy]
  I've been vague about what exactly a space is, but if you allow me
  to leave that there I can tell you more precisely what I mean by
  continuous deformation. Firstly, a deformation from a point $x$ to
  another point $x'$ inside a space $X$ is a (continuous) \emph{path}
  between them; letting $I$ denote the interval of real numbers $t$
  between $0$ and $1$ (which one can think of as parameterizing time),
  such a path is just a map $I \to X$ which to $0$ assigns $x$ and to
  $1$ assigns $x'$.

  We can bootstrap from this to define a deformation between two maps
  $f,g \c X \doubto Y$. Letting $X \times I$ denote the space of pairs
  $(x,t)$ for $x \in X$ and $t \in I$, it's a map
  $h \c X \times I \to Y$ which when restricted to time $t=0$ is the
  map $f \c X \to Y$ and when restricted to time $t=1$ is the map
  $g \c X \to Y$. In other words, it consists of deformations of (i.e.
  paths from) the point $f(x) \in Y$ to $g(x) \in Y$ varying
  continuously in $x \in X$. Such a map $h$ is called a
  \emph{homotopy} from $f$ to $g$, and when such a thing exists we say
  $f$ and $g$ are \emph{homotopic}.

  Finally, we can use this to say more precisely what it means for two
  spaces to be the same. We say spaces $X$ and $Y$ are \emph{homotopy
    equivalent} if there are maps $f \c X \to Y$ and $g \c Y \to X$
  such that the composites $gf \c X \to X$ and $fg \c Y \to Y$ are
  homotopic to the respective identity maps. If we demanded them to be
  \emph{equal} to the identity maps, we would get a stricter notion,
  requiring a bijection on the set of points. Allowing homotopy is
  what allows continuous deformation in our notion of sameness.
\end{nothing}

%%%%%%%%%%%%%%%%%%%%%%%%%%%%%%%%%%%%%%%%%%%%%%%%%%%%%%%%%%%%%%%%%%%%%%

\section{Homology theories}

\begin{nothing}[Invariants]
  So now what should our strategy be in telling whether two spaces are
  homotopy equivalent or not? Well, let's consider how we do this
  intuitively in easy examples:
  \begin{itemize}
  \item Certainly the space consisting of two discrete points should
    be different from the space consisting of just one
    point. Similarly the spaces consisting of two copies of a circle
    should be different from just one circle, or just one point. We
    say this because the former have ``two parts'' and the latter have
    ``one part''.
  \item Then why is one circle different from one disk or one point?
    Because a circle has a hole in the middle, and continuous
    deformation can't create or destroy a hole.
  \end{itemize}
  
  We make this kind of reasoning precise by defining \emph{invariants}
  of spaces. An invariant is some machine $M$ which takes as input a
  space $X$ and outputs some kind of object $M(X)$, such that if $X$
  and $Y$ are homotopy equivalent then $M(X)$ and $M(Y)$ are ``the
  same''. That way, if we ever feed two spaces $X$ and $Y$ in to $M$
  and get different outputs, we know $X$ and $Y$ cannot be homotopy
  equivalent. For instance:
  \begin{itemize}
  \item In the first example above we considered the ``number of
    parts'' in the given space. More generally our space might have an
    infinite number of parts, so we can't just assign a number, but we
    can define a machine $\pi_0$ which to a space $X$ associates the
    ``set of parts'' $\pi_0(X)$ comprising $X$. This is fairly easy to
    define rigorously: we define an equivalence relation on the points
    of $X$ by saying two points are in the same component if there is
    a path in $X$ connecting them; then the equivalence classes are
    the ``parts'' we are looking for.
  \item Making the second example rigorous is a bit more involved. But
    the basic idea is the same: we want some machine $M$ such that
    $M(X)$ is a measure of the number of holes in a space
    $X$. Actually defining this machine led to the discovery of an
    invariant known as \emph{singular homology}.
  \end{itemize}
\end{nothing}

\begin{nothing}[Homology theories]
  Singular homology is the first example of a fundamental notion in
  algebraic topology, that of a \emph{homology theory}. A homology
  theory is an invariant $E$ (more precisely a collection of
  invariants $E_n$ indexed by the integers) satisfying certain
  axioms. One can think of these axioms essentially as criteria that
  give a recipe for actually computing these invariants, which is of
  course important if they are actually going to be useful in
  distinguishing spaces of interest. Let me just give you a flavor of
  some these criteria:
  \begin{itemize}
  \item Linearity: $E$ takes values in abelian groups, i.e. each set
    $E_n(X)$ is not only a set but a set which has a commutative,
    associative, unital law of addition. The ability to add things is
    often very useful in mathematics.
  \item Functoriality: in addition to assigning abelian groups $E(X)$
    to each space $X$, $E$ also assigns to every map of spaces
    $f \c X \to Y$ a map/homomorphism of abelian groups
    $E(f) \c E(X) \to E(Y)$. This should behave well with respect to
    composition of maps and identity maps:
    \begin{itemize}
    \item $E$ assigns to the composition of two maps $f \c X \to Y$
      and $g \c Y \to Z$ the composite of the maps
      $E(f) \c E(X) \to E(Y)$ and $E(g) \c E(Y) \to E(Z)$;
    \item $E$ assigns to the identity map $X \to X$ the identity map
      $E(X) \to E(X)$.
    \end{itemize}
  \item Homotopy invariance: if $f,g \c X \doubto Y$ are homotopic,
    then $E(f)$ and $E(g)$ are the same homomorphism. It's easy to
    check that this together with functoriality imply that a homotopy
    equivalence $f \c X \isoto Y$ induces an isomorphism
    $E(f) \c E(X) \isoto E(Y)$, so that $E$ is indeed an invariant.
  \item Locality: If we can break up our space $X$ into two pieces,
    $X = A \cup B$, then $E(X)$ is (in some sense) determined by
    $E(A)$, $E(B)$, and $E(A \cap B)$.
  \end{itemize}
  

  There's no time to define any examples, but be assured that there
  are plenty! In fact, there are so many that studying them is really
  quite fun! I can attest to this: the theorem explained in my thesis
  is essentially a comparison of homology theories. Of course,
  comparing invariants is always useful, especially if one is much
  easier to compute and you want to know information about the other.
\end{nothing}

\begin{nothing}[Rationalizing]
  I'll end by trying to say something a little more precise about what
  kind of comparison the theorem in my thesis gives. There's a sense
  in which the study of homology theories is a ``linearization'' of
  the study of spaces; you might get a bit of a sense of this from the
  linearity axiom above. That is, studying homology theories is like
  studying spaces except where we've allowed ourselves to add. This
  makes things easier (i.e. homology theories are fairly computable),
  but as always in mathematics, we'd really like to make things even
  easier (i.e. obtain even more computable invariants). One natural
  way to do this is the following: adding and subtracting lets us do
  things like multiply by $2$ or $3$ or $-5$ or $n \in \lZ$, but we
  can't necessarily divide by $n$. It'd be nice if we could.

  This is already a process we use in studying abelian groups. If we
  have an abelian group $A$, we can adjoin elements $a/n$ for all
  $a \in A$ and $n \in \lZ$ in a natural way and obtain its
  \emph{rationalization} $A_\lQ$, a $\lQ$-vector space (i.e. an
  abelian group in which we can divide by integers). For example, when
  we do this to $\lZ$ we obtain the rational numbers $\lQ$.

  It turns out there is indeed an analogous rationalization process
  for homology theories: from a homology theory $E$ we get a homology
  theory $E_\lQ$ which takes values in $\lQ$-vector spaces. But the
  relationship between $E$ and $E_\lQ$ is not so simple in general:
  one might na\"ively guess that we could just define
  $(E_\lQ)_n(X) \ce E_n(X)_\lQ$, but this doesn't actually define a
  homology theory, and indeed this is not what $E_\lQ$ is. I.e., the
  processes of rationalization and evaluating the homology theory do
  not commute. So although the rationalized theory $E_\lQ$ is indeed
  often easier to compute and study, it's not clear how to use this to
  obtain information about the original theory $E$.

  The theorem in my thesis describes a situation in which we can
  actually state a precise relationship between $E$ and $E_\lQ$. It
  roughly says, that for certain naturally arising homology theories
  $E$, though it is not true that $E_n(X)_\lQ$ is given by
  $(E_\lQ)_n(X)$, it is given by $(E_\lQ)_n(Y)$ for some other space
  $Y$. In fact we can explicitly say what $Y$ is in terms of $X$; it's
  something like the ``space of maps'' from the $k$-fold torus
  $\rT^k \ce \rS^1 \times \cdots \times \rS^1$ into $X$, for some
  positive integer $k$ attached to $E$. Thus the theorem gives us a
  way of obtaining a good amount of information about the homology
  theory $E$ from its rationalization $E_\lQ$.
\end{nothing}



%%%%%%%%%%%%%%%%%%%%%%%%%%%%%%%%%%%%%%%%%%%%%%%%%%%%%%%%%%%%%%%%%%%%%%

% \bibliographystyle{amsalpha}
% \bibliography{refs}

\end{document}
