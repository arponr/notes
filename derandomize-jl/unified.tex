%%%%%%%%%%%%%%%%%%%%%%%%%%%%%%%%%%%%%%%%%%%%%%%%%%%%%%%%%%%%%%%%%%%%%%

\newcommand{\ob}{\oper{ob}}
\renewcommand{\hom}{\oper{hom}}
\newcommand{\id}{\oper{id}}
\newcommand{\im}{\oper{im}}
\newcommand{\op}{\oper{op}}

\newcommand{\Top}{\oper{Top}}
\newcommand{\Set}{\oper{Set}}
\newcommand{\Ab}{\oper{Ab}}
\newcommand{\Grp}{\oper{Grp}}
\newcommand{\Mod}{\oper{Mod}}
\newcommand{\Simplex}{\Delta}
\newcommand{\s}{\oper{s}}
\newcommand{\Ch}{\oper{Ch}}

\newcommand{\Sing}{\oper{Sing}}
\renewcommand{\H}{\mathrm{H}}

%%%%%%%%%%%%%%%%%%%%%%%%%%%%%%%%%%%%%%%%%%%%%%%%%%%%%%%%%%%%%%%%%%%%%%

%%%%%%%%%%%%%%%%%%%%%%%%%%%%%%%%%%%%%%%%%%%%%%%%%%%%%%%%%%%%%%%%%%%%%%

\newcommand{\ob}{\oper{ob}}
\renewcommand{\hom}{\oper{hom}}
\newcommand{\id}{\oper{id}}
\newcommand{\im}{\oper{im}}
\newcommand{\op}{\oper{op}}

\newcommand{\Top}{\oper{Top}}
\newcommand{\Set}{\oper{Set}}
\newcommand{\Ab}{\oper{Ab}}
\newcommand{\Grp}{\oper{Grp}}
\newcommand{\Mod}{\oper{Mod}}
\newcommand{\Simplex}{\Delta}
\newcommand{\s}{\oper{s}}
\newcommand{\Ch}{\oper{Ch}}

\newcommand{\Sing}{\oper{Sing}}
\renewcommand{\H}{\mathrm{H}}

%%%%%%%%%%%%%%%%%%%%%%%%%%%%%%%%%%%%%%%%%%%%%%%%%%%%%%%%%%%%%%%%%%%%%%


\title{CS 229, Project Notes}
\author{Arpon Raksit}
\date{\today}

\begin{document}
\maketitle
\thispagestyle{fancy}

%%%%%%%%%%%%%%%%%%%%%%%%%%%%%%%%%%%%%%%%%%%%%%%%%%%%%%%%%%%%%%%%%%%%%%

\part{Unified theory of sparse dimensionality reduction
  \cite{bourgain-2013}}

%%%%%%%%%%%%%%%%%%%%%%%%%%%%%%%%%%%%%%%%%%%%%%%%%%%%%%%%%%%%%%%%%%%%%%

\section{Background}

Recall the JL lemma: for a finite set $T \subset S^{n-1} \subset \R^n$
and $\epsilon \in (0, 1/2)$ there exists $m \lesssim
\log(|T|)/\epsilon^2$ and a linear map $\Phi : \R^n \to \R^m$, i.e., a
matrix $\Phi \in \R^{m \times n}$, such that $\sup_{x \in T} |\|\Phi
x\|^2 - 1| < \epsilon$.

\medskip
In fact Gordon has shown it is possible to take $m \lesssim
(\gamma_2(T)^2+1)/\epsilon^2$ for certain $\Phi$, where $\gamma_2$ is
to be defined later.

\medskip
The general strategy is to define a suitable \emph{distribution} of
$\Phi$ so that the desired inequality holds in expectation:
\begin{equation}
  \label{dist-JL}
  \E_\Phi \sup_{x \in T} |\|\Phi x\|^2 - 1| < \epsilon.
\end{equation}

\medskip
For purposes of efficiency, it is desirable to have $\Phi$ be sparse,
in particular to have a small number $s$ of nonzero entries per column
to support fast matrix-vector multiplication.

\medskip
The best result so far in this direction is $s \lesssim
\log(|T|)/\epsilon$ and $m \lesssim \log(|T|)/\epsilon^2$, due to
Kane-Nelson. The construction, called the SJLT (sparse JL transform),
is simple: each column of $\Phi$ has $s$ nonzero entries, their
locations chosen uniformly at random without replacement,
independently of the other columns, and their values uniformly and
independently chosen from $\pm 1$.

\begin{question}
  What conditions must $s$ and $m$ satisfy, in terms of the geometry
  of $T$, for $\Phi$ the SJLT to satisfy (\ref{dist-JL})?
\end{question}

The paper gives the following answer to the question, which
qualitatively explains known results about sparse dimensionality
reduction. Here ``qualitatively'' means that the bounds are correct up
to a $\log^c(n)$ factor in $m$ and $\log^c(n)/\epsilon$ factor in $s$.

\begin{theorem}
  The following conditions are sufficient:
  \begin{align*}
    m &\gtrsim \log^3(m)\log^6(n)/\epsilon^2 +
    \log^3(m)\log^4(n)\gamma_2(T)^2/\epsilon^2, \\ s &\gtrsim
    \log^6(m)\log^4(n)/\epsilon^2, \\ \epsilon &\ge
    \log^2(m)\log^{5/2}(n) \max_{q \le \frac{m}{s} \log(s)}
    \frac{1}{\sqrt{qs}} \left[ \E_\eta \left( \int \sup_{x \in T}
      \left| \left\langle x, \sum_{j=1}^n \eta_j g_j(\omega) e_j
      \right\rangle \right| \,d\omega \right)^q \right]^{1/q},
  \end{align*}
  where $g_j(\omega)$ are gaussian, $\eta_j$ are iid bernoulli of
  expectation $qs/(m\log(s))$.
\end{theorem}

%%%%%%%%%%%%%%%%%%%%%%%%%%%%%%%%%%%%%%%%%%%%%%%%%%%%%%%%%%%%%%%%%%%%%%

\section{Setting up}

\renewcommand{\A}{\mathcal{A}}
\renewcommand{\N}{\mathcal{N}}
\newcommand{\conv}{\operatorname{conv}}
\newcommand{\diam}{\operatorname{diam}}
\newcommand{\gmw}{\operatorname{gmw}}
\newcommand{\op}{\mathrm{op}}
\newcommand{\trip}{\interleave}

\begin{enumerate}[leftmargin=*]
\item Fix once and for all $T \subseteq S^{n-1}$ and $\epsilon \in
  (0,1/2)$ as in (\ref{dist-JL}).
\item By default, $\|{-}\|$ denotes $\ell_2$ norm for vectors and
  operator norm (with respect to $\ell_2$) for matrices.
\item The convex hull of $A \subseteq \R^n$ is denoted $\conv(A)$.
\item For $A \subseteq \R^n$ the norm $\|{-}\|_A$ on $\R^n$ is defined
  by $\|x\|_A \coloneqq \sup_{a \in A} |\langle x,a \rangle|$. Note
  that $\|{-}\|_A = \|{-}\|_{\conv(A)}$.
\item For a norm $\|{-}\|_X$ on $\R^n$, denote the induced metric by
  $d_X$ and diameter in this metric by $\diam_X$.
\item For $S \subset \R^n$ and $u > 0$, denote by $\N(S, \|{-}\|_X,
  u)$ the minimum number of $d_X$-balls of radius $u$ centred at
  points in $S$ required to cover $S$.
\item For $S \subset \R^n$ bounded:
  \begin{enumerate}
  \item Define the \emph{gaussian mean width} of $S$ by $\gmw(S)
    \coloneqq \E_g \sup_{x \in S} \langle g,x \rangle$ with $g$ a
    gaussian vector with identity covariance matrix.
  \item Define the \emph{Talagrand functional} for $p \ge 1$ by
    \[
    \gamma_p(S, \|{-}\|_X) \coloneqq \inf_{\{S_r\}} \sup_{x \in S}
    \sum_{r=0}^\infty 2^{r/p} d_X(x,S_r),
    \]
    where the infimum is over all $S_0 \subseteq S_1 \subseteq \cdots
    \subseteq S$ with $|S_0| = 1$ and $|S_r| \le 2^{2^r}$.
  \item Note that $\gmw(S) \sim \gamma_2(S,\|{-}\|_2)$, with constant
    independent of $S$.
  \end{enumerate}
\item We set $\Phi$ to be the SJLT matrix: $\Phi_{i,j} \coloneqq
  \frac{1}{\sqrt{s}} \delta_{i,j} \sigma_{i,j}$, with $\sigma_{i,j}
  \in \{-1,1\}$ and $\delta_{i,j} \in \{0,1\}$ such that $\sum_{i=1}^m
  \delta_{i,j} = s$.
\item For $x \in T$ define the $m \times mn$ matrix $A_{\delta,x}
  \coloneqq \frac{1}{\sqrt{s}} \sum_{i=1}^m \sum_{j=1}^n \delta_{i,j}
  x_j e_i \otimes e_{i,j}$. So you have $m$ adjacent $m \times n$
  matrices---in the $i$-th one, only the $i$-th row has possibly
  nonzero entries, and those are the $\delta_{i,j}x_j$.
  \begin{enumerate}
  \item Then if you write $\sigma = (\sigma_{i,j})$ as a vector, you
    have $\|\Phi x\| = \|A_{\delta,x} \sigma\|$.
  \item Writing $\A_\delta \coloneqq \{A_{\delta,x} \mid x \in T\}$,
    it follows that
    \[
    \sup_{x \in T} | \|\Phi x\|^2 - 1 | = \sup_{A \in \A_\delta} |
    \|A\sigma\|^2 - \E\|A\sigma\|^2 |,
    \]
    whose expectation is bounded up to a constant by
    \[
    \gamma_2(\A_\delta)^2 + \gamma_2(\A_\delta) +
    \diam_\op(\A_\delta).
    \]
  \item Define $\trip x \trip \coloneqq \| A_{\delta,x} \| =
    \frac{1}{\sqrt{s}} \max_{1 \le i \le m} \left( \sum_{j=1}^n
    \delta_{i,j} x_j^2 \right)^{1/2}$.
  \end{enumerate}
\end{enumerate}

%%%%%%%%%%%%%%%%%%%%%%%%%%%%%%%%%%%%%%%%%%%%%%%%%%%%%%%%%%%%%%%%%%%%%%

\bibliographystyle{amsalpha}
\bibliography{refs}

\end{document}
