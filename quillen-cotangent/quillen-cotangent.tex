%%%%%%%%%%%%%%%%%%%%%%%%%%%%%%%%%%%%%%%%%%%%%%%%%%%%%%%%%%%%%%%%%%%%%%

\newcommand{\ob}{\oper{ob}}
\renewcommand{\hom}{\oper{hom}}
\newcommand{\id}{\oper{id}}
\newcommand{\im}{\oper{im}}
\newcommand{\op}{\oper{op}}

\newcommand{\Top}{\oper{Top}}
\newcommand{\Set}{\oper{Set}}
\newcommand{\Ab}{\oper{Ab}}
\newcommand{\Grp}{\oper{Grp}}
\newcommand{\Mod}{\oper{Mod}}
\newcommand{\Simplex}{\Delta}
\newcommand{\s}{\oper{s}}
\newcommand{\Ch}{\oper{Ch}}

\newcommand{\Sing}{\oper{Sing}}
\renewcommand{\H}{\mathrm{H}}

%%%%%%%%%%%%%%%%%%%%%%%%%%%%%%%%%%%%%%%%%%%%%%%%%%%%%%%%%%%%%%%%%%%%%%


\title{Defining the cotangent complex}
\author{Arpon Raksit}
\date{December 17, 2013}

\begin{document}
\maketitle
\thispagestyle{fancy}

%%%%%%%%%%%%%%%%%%%%%%%%%%%%%%%%%%%%%%%%%%%%%%%%%%%%%%%%%%%%%%%%%%%%%%

\newcommand{\s}{\mathrm{s}}
\newcommand{\Top}{\mathrm{Top}}
\newcommand{\Set}{\mathrm{Set}}
\newcommand{\Alg}{\mathrm{Alg}}
\newcommand{\Mod}{\mathrm{Mod}}
\newcommand{\Ab}{\mathrm{Ab}}
\newcommand{\Sing}{\operatorname{Sing}}

%%%%%%%%%%%%%%%%%%%%%%%%%%%%%%%%%%%%%%%%%%%%%%%%%%%%%%%%%%%%%%%%%%%%%%

\section*{Opening nonsense}

I've recently learned a bit about the general framework of model
categories. It's sort of amazing, in the sense that seemingly a lot of
the basic homotopy theory of spaces which I've seen can be axiomatised
in a totally formal and categorical way. This includes something like
Whitehead's theorem (and its converse)---that a map of CW complexes is
a homotopy equivalence if and only if it's a weak homotopy
equivalence---which I never would have thought of as a categorical
fact.

Anyway, this axiomatisation and generalisation is beautiful and all,
but of course one hopes that it leads to new homotopy theory, in
addition to providing a nice framework for existing homotopy theory to
live in. Obviously it did. My goal in writing these notes was to learn
about one of the earliest examples of this, the development of a
(co)homology theory of commutative algebras by Andr\'{e}
\cite{andre-cotangent} and Quillen \cite{quillen-cotangent}.

\renewcommand{\C}{\mathcal{C}}

\begin{notation*}
  \begin{itemize}[leftmargin=*]
  \item Denote by $\Set$ the category of sets.
  \item All rings (and algebras) are commutative.
  \item For a ring $R$ denote by $\Mod(R)$ the category of $R$-modules
    and $\Alg(R)$ the category of $R$-algebras.
  \item For a category $\C$ denote by $\s\C$ the category of
    simplicial objects in $\C$.
  \end{itemize}
\end{notation*}

%%%%%%%%%%%%%%%%%%%%%%%%%%%%%%%%%%%%%%%%%%%%%%%%%%%%%%%%%%%%%%%%%%%%%%

\section{Homology and abelianisation}
\label{homology-abelian}

In hopes of generalising the notion of homology via the theory of
model categories, we first ask the following.

\begin{question}
  How can we view singular homology in a homotopyish way?
\end{question}

\renewcommand{\Ab}{\operatorname{Ab}}
\newcommand{\ab}{\mathrm{ab}}

\begin{nothing}
  \label{sing-doldthom}
  Perhaps the most immediate answer is the following. Suppose $X$ is a
  topological space. We can choose a CW approximation, that is, a CW
  complex $Y$ and a weak equivalence $Y \to X$. Let $\Ab(Y)$ denote
  the free topological abelian group on $Y$. Then one version of the
  Dold-Thom theorem \cite{mccord-doldthom} states that
  \[
  \pi_*(\Ab(Y)) \simeq H_*(Y) \simeq H_*(X).
  \]
  Thus we should view $\Ab(Y)$ as the object which gives, via its
  homotopy, singular homology. Note that to apply Dold-Thom we had to
  choose a CW approximation to $X$ before applying $\Ab$. From the
  point of view of model categories, this is precisely choosing a
  cofibrant replacement (in the standard model structure on
  topological spaces). So finally we should view singular homology as
  the derived functor of $\Ab$.
\end{nothing}

\begin{nothing}
  \label{sing-doldkan}
  Another answer to the question goes through simplicial sets rather
  than topological spaces. Suppose $K \in \s\Set$, perhaps thinking $K =
  \Sing(X)$ for some topological space $X$. Let $\Ab(K) \ce \Z
  K\in \s\Mod(\Z)$ the free simplicial abelian group on $K$.  Then the
  Dold-Kan correspondence\footnote{Many thanks to Albrecht Dold in this
    section!} tells us that
  \[
  \pi_*(\Ab(K)) \simeq H_*(K) \simeq H_*(X).
  \]
  So again we view $\Ab(K)$ as giving homology via its homotopy. Here
  we did not have to choose a cofibrant replacement, but this is
  simply because (in the standard model structure) all $K \in \s\Set$
  are cofibrant!
\end{nothing}

These two points of view on singular homology motivate the following
general philosophy, due to Quillen.

\begin{definitions}
  \begin{enumerate}[leftmargin=*]
  \item An \emph{abelian group object} in a category $\C$ is an object $A
    \in \C$ such that the functor $\hom_\C(-,A) \c \C^\op \to \Set$
    factors through the forgetful functor $\Mod(\Z) \to \Set$, i.e.,
    $\hom_\C(B,A)$ is naturally an abelian group.
  \item A \emph{morphism of abelian group objects} $A,A' \in \C$ is a
    morphism $A \to A'$ in $\C$ for which the induced map $\hom_\C(B,A)
    \to \hom_\C(B,A')$ is a group homomorphism for $B \in \C$.
  \item The resulting \emph{subcategory of abelian group objects} in
    $\C$ is denoted by $\C_\ab$.
  \end{enumerate}
\end{definitions}

\begin{nothing}
  \label{abgpob-mult}
  One can check that if $\C$ has a terminal object $*$ and binary
  products, then $A \in \C$ is an abelian group object if and only if
  there exist morphisms
  \begin{itemize}
  \item $\mu \c A \times A \to A$ (multiplication),
  \item $\epsilon \c * \to A$ (identity),
  \item $\iota \c A \to A$ (inverse),
  \end{itemize}
  such that the usual diagrams encoding abelian group structure
  commute, and a morphism of abelian group objects is one which
  respects multiplication.
\end{nothing}

\begin{nothing}
  \label{simplab}
  Let $\C$ be a category with a terminal object and binary
  products. It is immediate from (\ref{abgpob-mult}) that
  $(\s\C)_\ab \simeq \s(\C_\ab)$ That is, the two subcategories of
  $\s\C$ consisting of:
  \begin{itemize}
  \item abelian group objects
  \item objects which are degree-wise abelian group objects and whose
    face and degeneracy maps are morphisms of abelian group objects
  \end{itemize}
  can be identified in the natural manner. We won't distinguish
  between the two, and will denote them by $\s\C_\ab$.

  Suppose moreover that the inclusion $\C_\ab \to \C$ has a left
  adjoint $\Ab \c \C \to \C_\ab$. It is easy to see then that the
  inclusion $\s\C_\ab \to \s\C$ has left adjoint $\s\C \to
  \s\C_\ab$ given by applying $\Ab$ degree-wise.
\end{nothing}

\renewcommand{\L}{\mathbb{L}}

\begin{definition}
  \label{derivedab}
  Let $\C$ be a model category. Suppose there is a model structure on
  $\C_\ab$ such that the inclusion $\C_\ab \to \C$ is a right
  Quillen functor, with left adjoint $\Ab \c \C \to \C_\ab$, called
  \emph{abelianisation}. Then \emph{homology} is the total left
  derived functor $\L \Ab$, i.e., the homology of $X \in \C$ is the
  object $\L \Ab(X) \in \C_\ab$.
\end{definition}

\begin{examples}
  This philosophy is of course a generalisation of
  (\ref{sing-doldthom}) and (\ref{sing-doldkan}). If $\C = \Top$ then
  $\C_\ab$ is the subcategory of topological abelian groups and if $\C
  = \s\Set$ then $\C_\ab$ is the subcategory of simplicial abelian
  groups. In both cases the natural model structure on $\C_\ab$ is
  defined precisely such that $\C_\ab \to \C$ is right Quillen, with
  left adjoints $\Ab \c \Top \to \Top_\ab$ and $\Ab \c \s\Set \to
  \s\Set_\ab \simeq \s\Mod(\Z)$ the free topological and simplicial
  abelian group functors, respectively.
\end{examples}

\section{K\"ahler differentials}
\label{abelianalgebra}

Of course we must then ask: to what other categories $\C$ can we
fruitfully apply this notion of homology to? As stated earlier, the
goal here is to describe what happens when we consider the category of
commutative rings or algebras. So first we need to understand what
abelianisation is in this context.

\begin{notation}
  We fix a ring $R$.
\end{notation}

\begin{proposition}
  The only abelian group object in $\Alg(R)$ is the zero ring.
\end{proposition}

\begin{proof}
  The terminal object in $\Alg(R)$ is the zero ring $0$, and the only
  $A \in \Alg(R)$ for which there exists a morphism $0 \to A$ is $A =
  0$ itself.
\end{proof}

\begin{nothing}
  So we've already run into a subtlety in this category. We'll resolve
  this issue by working in a slightly different category. Fix an
  $R$-algebra $A$. Let
  \[
  \C \ce \Alg(R)_{/A}
  \]
  be the overcategory of $R$-algebras over $A$. I.e., an object of
  $\C$ is a sequence of ring morphisms $R \to B \to A$ (which we of
  course abusively refer to as $B$) whose composition is the structure
  morphism $R \to A$.
\end{nothing}

Before continuing the discussion on abelianisation, we review
derivations and K\"ahler differentials.

\newcommand{\Der}{\operatorname{Der}}

\begin{definitions}
  \label{derivations}
  Let $B$ be an $R$-algebra and $M$ a $B$-module.
  \begin{enumerate}[leftmargin=*]
  \item An \emph{$R$-derivation} $d \c B \to M$ is a morphism of
    $R$-modules satisfying the Leibniz rule, $d(xy) = x\,d(y) +
    y\,d(x)$ for $x,y \in B$.
  \item The set $\Der_R(B,M)$ of $R$-derivations has a natural
    $B$-module (in particular abelian group) structure.
  \item \label{kahdiff} The $B$-module $\Omega_{B/R}$ of
    \emph{K\"ahler differentials} is defined by the universal property
    \[
    \hom_{\Mod(B)}(\Omega_{B/R}, M) \simeq \Der_R(B,M).
    \]
    In fact, if we let $\mu \c B \otimes_R B \to B$ be the product map
    and $I \ce \ker(\mu)$, then there is an $R$-derivation $d \c
    B \to I/I^2$ sending $x \mapsto 1 \otimes x - x \otimes 1$ such
    that the map
    \[
    \hom_{\Mod(B)}(I/I^2, M) \to \Der_R(B,M), \quad \phi \mapsto \phi
    \circ d
    \]
    is an isomorphism, so $\Omega_{B/R} \simeq I/I^2$. We omit the
    verification of this construction.
  \end{enumerate}
\end{definitions}

\begin{nothing}
  \label{square-zero}
  We define a functor
  \[
  \Mod(A) \to \C, \quad M \mapsto A \oplus M,
  \]
  where the ring structure on $A \oplus M$ is given by $(a,x)(b,y) =
  (ab,ay+bx)$. Then the structure morphism $R \to A$ and projection $A
  \oplus M \to A$ determine $R \to A \oplus M \to A$ as an object in
  $\C$. If we have a morphism $\phi \c M \to N$ in $\Mod(A)$ then
  $\id_A \oplus \phi \c A \oplus M \to A \oplus N$ evidently gives a
  morphism in $\C$.
\end{nothing}

\begin{proposition}
  \label{modules-abelian}
  The functor $\Mod(A) \to \C$ defined in (\ref{square-zero}) factors
  through the inclusion $\C_\ab \to \C$, and furthermore induces an
  equivalence of categories
  \[
  \Mod(A) \simeq \C_\ab.
  \]
\end{proposition}

\begin{proof}
  The first claim is that $A \oplus M$ is an abelian group object for
  any $M \in \Mod(A)$. Let $B \in \C$. By definition of $\C$ and $A
  \oplus M$, giving a morphism $B \to A \oplus M$ is equivalent to
  giving an $R$-linear map $d \c B \to M$ satisfying $d(xy) = x\,d(y) +
  y\,d(x)$ for $x,y \in B$. (Here we have viewed $M$ as a module over
  $R$ and $B$ via restriction of scalars.) I.e., we have a natural
  isomorphism
  \begin{equation}
    \label{sqzero-der}
    \hom_{\C}(B, A \oplus M) \simeq \Der_R(B,M),
  \end{equation}
  which is an abelian group, giving our claim.

  We next show that if $B \in \C_\ab$ is an abelian group object
  then it is isomorphic to $A \oplus M$ for some $M \in \Mod(A)$. Let
  $\eta \c B \to A$ be the map over $A$. Note the terminal object of
  $\C$ is $A$ and the product is given by fibred product $B \times_A
  B$. To have an abelian group object we in particular need morphisms
  $\mu \c B \times_A B \to B$ and $\epsilon \c A \to B$.  Since
  $\epsilon$ is a morphism over $A$, it must be a section of
  $\eta$. Thus if we set $M \ce \ker(\eta)$ we have a splitting
  $B \simeq A \oplus M$ as $A$-modules. Next, $\epsilon$ being a
  ``two-sided identity'' means
  \[
  \mu \circ (\epsilon \eta, \id_B) = \id_B = \mu \circ (\id_B,
  \epsilon \eta),
  \]
  and hence $\mu(x,0) = x = \mu(0,x)$ for $x \in M$. This implies $xy
  = \mu(x,0)\mu(0,y) = \mu(0,0) = 0$ for $x,y \in M$, and hence for
  $a,b \in A$ we have
  \[
  (\epsilon(a)+x)(\epsilon(b)+y) = \epsilon(ab) + \epsilon(a)y +
  \epsilon(b)x
  \]
  It follows that in fact $B \simeq A \oplus M$ as objects of $\C$.

  Finally we must show the functor is fully faithful. Faithfulness is
  clear. A morphism $\psi \c A \oplus M \to A \oplus N$ over $A$ must
  be of the form $\id_A \oplus \phi$. And since
  \[
  (1,x)(1,y) = (1,x+y), \quad (a,0)(1,x) = (1,ax)
  \]
  for $a \in A$ and $x,y \in M$ or $x,y \in N$, if $\psi$ is a
  morphism of abelian group objects, then $\phi$ is $A$-linear.
\end{proof}

\begin{proposition}
  \label{adjointkahler}
  The assignment $B \mapsto \Omega_{B/R} \otimes_B A$ defines a
  functor $\C \to \Mod(A)$ which is left adjoint to inclusion $\Mod(A)
  \simeq \C_\ab \to \C$.
\end{proposition}

\begin{proof}
  Functoriality is clear from the definition
  (\ref{derivations}.\ref{kahdiff}) of $\Omega_{B/R}$ and the Yoneda
  lemma, as one can pull back derivations. Then adjointness follows
  from (\ref{sqzero-der}):
  \begin{align*}
  \hom_\C(B, A \oplus M) &\simeq \Der_R(B,M) \\ &\simeq
  \hom_{\Mod(B)}(\Omega_{B/R}, M) \\ &\simeq
  \hom_{\Mod(A)}(\Omega_{B/R} \otimes_B A, M). \qedhere
  \end{align*}
\end{proof}

\begin{nothing}
  We now have a convenient (and pretty intruiging, actually) notion of
  abelianisation for the category $\C = \Alg(R)_{/A}$. As per the
  philosophy of \textsection \ref{homology-abelian}, we should now
  take the derived functor. But wait! We don't even have a model
  structure on $\C$! And why would we expect to? Indeed, we should
  extend the above discussion
  (\ref{square-zero})--(\ref{adjointkahler}) to \emph{simplicial}
  objects via (\ref{simplab}) to get an adjunction
  \[
  \begin{tikzcd}
    \Omega_{-/R} \otimes_- A \ : \ \s\Alg(R)_{/A} \rar[yshift=2pt] &
    \s\Mod(A) \ :\ A \oplus -. \lar[yshift=-2pt]
  \end{tikzcd}
  \]
  (Recall that the functors extend to the simplicial categories by
  degree-wise application.) It seems much more natural to give model
  structures to these categories, and in fact this is what we will do
  in the following section.
\end{nothing}

%%%%%%%%%%%%%%%%%%%%%%%%%%%%%%%%%%%%%%%%%%%%%%%%%%%%%%%%%%%%%%%%%%%%%%

\section{Promoting model structures}

Why does it seem more natural to give model structures to simplicial
categories? Well, we certainly have a familiar model structure on
simplicial sets, so hopefully we can leverage this to our advantages,
via the forgetful functors from the categories at hand to
$\s\Set$. This is precisely what we do here, following
\cite{goerssjardine}, though in a bit more generality, as stated in
\cite{goerss-modelsimplicial}.

\renewcommand{\F}{\mathcal{F}}
\renewcommand{\G}{\mathcal{G}}

\begin{definition}
  \label{small}
  Let $\C$ be a category and $\F$ a class of morphisms in $\C$. We say
  $X \in \C$ is \emph{small} for $\F$ if (assuming $\C$ has enough
  colimits), for any sequence of morphisms $Y_1 \to Y_2 \to Y_3 \to
  \cdots$ in $\F$, the canonical morphism
  \[
  \colim \hom_\C(X,Y_n) \to \hom_\C(X,\colim Y_n),
  \]
  is an isomorphism, or equivalently, if any morphism $X \to \colim
  Y_n$ factors through some inclusion $Y_m \to \colim Y_n$.
\end{definition}

\begin{example}
  \label{finitesset}
  By the equivalent restatement in (\ref{small}), it is clear that any
  $X \in \s\Set$ which has finitely many nondegenerate simplices is
  small for $\F$ the class of \emph{all} morphisms in $\s\Set$.
\end{example}

\renewcommand{\I}{\mathcal{I}}
\renewcommand{\J}{\mathcal{J}}

\begin{definition}
  Let $\C$ be a model category. Let $\F$ be the class of cofibrations
  in $\C$ and $\G$ the class of acyclic cofibrations. We say $\C$ is
  \emph{cofibrantly generated} if there exist sets of morphisms $\I$
  and $\J$ in $\C$ such that:
  \begin{enumerate}
  \item for any morphism $X \to Y$ in $\I$, $X$ is small for $\F$;
  \item a morphism $X \to Y$ in $\C$ is an acyclic fibration if and
    only if it has the right lifting property against $\I$;
  \item for any morphism $X \to Y$ in $\J$, $X$ is small for $\G$;
  \item a morphism $X \to Y$ in $\C$ is a fibration if and only if it
    has the right lifting property against $\J$;
  \end{enumerate}
  In particular, $\I$ and $\J$ generate (in the relevant sense) $\F$
  and $\G$, respectively.
\end{definition}

\begin{example}
  By (\ref{finitesset}), the standard model structure on $\s\Set$ is
  cofibrantly generated with
  \[
  \I \ce \{\del\Delta^n \inj \Delta^n \mid n \ge 0\}, \quad \J
  \ce \{\Lambda^n_k \inj \Delta^n \mid n \ge 1, 0 \le k \le n\},
  \]
  the sphere and horn inclusions, respectively.
\end{example}

\renewcommand{\D}{\mathcal{D}}

\begin{nothing}
  \label{promote-setup}
  Let $\C$ be a model category and $\D$ be any category. Assume we
  have a functor $G \c \D \to \C$ with a left adjoint $F \c \C \to
  \D$. Define a morphism in $f$ in $\D$ to be a:
  \begin{enumerate}
  \item weak equivalence if $G(f)$ is a weak equivalence in $\C$;
  \item fibration if $G(f)$ is a fibration in $\C$;
  \item cofibration if it has the left lifting property against all
    acyclic fibrations (as just defined).
  \end{enumerate}
  Thinking of $\C$ as $\s\Set$ and $\D$ as our simplicial algebraic
  category, with $G$ the forgetful functor, our hope is to prove that
  this defines a model structure on $\D$. The precise result, due to
  Quillen, is the following.
\end{nothing}

\begin{theorem}
  \label{promote-thm}
  In the situation of (\ref{promote-setup}). Assume:
  \begin{enumerate}
  \item $\D$ is complete and cocomplete;
  \item $\C$ is cofibrantly generated, with generating sets $\I$ and
    $\J$ of cofibrations and acyclic cofibrations, respectively;
  \item \label{seqcolim} $G$ preserves sequential colimits;
  \item \label{accof} a cofibration in $\D$ with the left lifting
    property against all fibrations is acyclic.
  \end{enumerate}
  Then the given definitions indeed give a model structure on $\D$,
  and hence $F$ and $G$ form a Quillen adjunction.
\end{theorem}

For convenience, we isolate two steps of the proof.

\begin{lemma}
  \label{factor1}
  A morphism $f \c X \to Y$ in $\D$ can be factored as
  \[
  X \lblto{i} Z \lblto{q} Y,
  \]
  where $q$ is an acyclic fibration and $i$ has the left lifting
  property against all acyclic fibrations.
\end{lemma}

\begin{proof}
  Define $Z_0 \ce X$ and $q_0 \ce f$. Inductively define
  $i_n \c Z_{n-1} \to Z_n$ and $q_n \c Z_n \to Y$ for $n \ge 1$ as
  follows. Consider all diagrams $D$ of the form
  \[
  \begin{tikzcd}
    F(U) \rar{\alpha} \dar{F(j)} & Z_{n-1} \dar{q_{n-1}} \\ F(V)
    \rar{\beta} & Y
  \end{tikzcd}
  \]
  where $j \in \I$. Then we can form the pushout
  \[
  \begin{tikzcd}
    \coprod_D F(U) \rar{(\alpha)} \dar{(F(j))} & Z_{n-1} \dar{i_n}
    \drar{q_{n-1}} \\ \coprod_D F(V) \rar{(\beta)} & Z_n \rar{q_n} &
    Y.
  \end{tikzcd}
  \]
  Define $Z \ce \colim Z_n$, $i \c X = Z_0 \to Z$ the canonical
  morphism, and $q \c Z \to Y$ induced by the $q_n$. Then we have a
  factoring
  \[
  X \lblto{i} Z \lblto{q} Y
  \]
  of $f$. Each $j \in \I$ has the left lifting property against
  acyclic fibrations in $\C$, implying via the adjunction and the
  definition of acyclic fibrations in $\D$ that $F(j)$ has the same
  property in $\D$. Since the left lifting property is stable under
  coproducts, pushouts, and sequential colimits, $i$ also has it.

  So we just need to show $q$ is an acyclic fibration, i.e., that
  $G(q)$ is an acyclic fibration. By assumption (\ref{seqcolim}) we
  have $G(Z) \simeq \colim G(Z_n)$, so by definition of $\I$ it
  suffices to show any lifting problem
  \[
  \begin{tikzcd}
    U \rar{\alpha} \dar{j} & \colim G(Z_n) \dar{G(q)} \\ V \rar{\beta}
    & G(Y)
  \end{tikzcd}
  \]
  with $j \in \I$ can be solved. Now, $U$ must be small for
  cofibrations in $\C$, and again by the adjunction and lifting
  properties we see that $G(i_n)$ is a cofibration for all $n \ge
  1$. Thus $\alpha$ in fact factors through some $G(Z_m) \to \colim
  G(Z_n)$. But then applying the adjunction to
  \[
  \begin{tikzcd}
    U \rar{\alpha} \dar{j} & G(Z_m) \dar{G(q_m)} \\ V \rar{\beta} &
    G(Y)
  \end{tikzcd}
  \]
  and considering the definition of $Z_{m+1}$, we see that we can lift
  $\beta$ to a morphism $V \to G(Z_{m+1}) \to \colim G(Z_n)$, solving
  the original lifting problem.
\end{proof}

\begin{lemma}
  \label{factor2}
  A morphism $f \c X \to Y$ in $\D$ can be factored as
  \[
  X \lblto{i} Z \lblto{q} Y,
  \]
  where $q$ is a fibration and $i$ has the left lifting property
  against all fibrations.
\end{lemma}

\begin{proof}
  Replace $\I$ with $\J$ in the proof of (\ref{factor1}).
\end{proof}


\begin{proof}[Proof of (\ref{promote-thm})]
  We have assumed $\D$ is complete and cocomplete, giving the first
  model category axiom. The second and third (two-out-of-three and
  retract) axioms are immediate from the definitions and these holding
  in $\C$.

  The fourth axiom is the lifting axiom. One half is simply our
  definition of a cofibration, so let's prove the other half. Suppose
  $f \c X \to Y$ is an acyclic cofibration in $\D$. Let $q \circ i \c X
  \to Z \to Y$ the factorisation of $f$ given by (\ref{factor2}). Then
  assumption (\ref{accof}) gives that $i$ is a weak equivalence. By
  two-out-of-three then so is $q$. But then we know a lift exists in
  \[
  \begin{tikzcd}
    X \rar{i} \dar{f} & Z \dar{q} \\ Y \rar{\id_Y} & Y,
  \end{tikzcd}
  \]
  exhibiting $f$ as a retract of $i$. Since lifting properties are
  stable under retracts, we win.

  The fifth axiom is the factorisation axiom, which is immediate from
  (\ref{factor1}) and (\ref{factor2}), along with the definition of
  cofibrations in $\D$ and assumption (\ref{accof}).
\end{proof}

\begin{nothing}
  \label{relevant-model}
  Now consider the example $\C = \s\Set$ and $\D = \s\Mod(R)$ or $\D =
  \s\Alg(R)$.\footnote{Or $\D$ any number of other simplicial
    algebraic categories, which don't happen to be relevant to our
    current discussion.}  We have forgetful functors $\D \to \C$ with
  left adjoints given by free functors $\C \to \D$, which commute with
  filtered colimits. Then to define a model structure on $\D$ via
  (\ref{promote-thm}), the only condition which needs checking is
  (\ref{accof}). We omit this verification for the sake of concision,
  but one can see \cite{goerssjardine} for a proof.
\end{nothing}

\begin{nothing}
  \label{over-model}
  We state one more result without proof to end this discussion of
  defining model structures. Suppose $\C$ is a model category and $X
  \in \C$. If we define a morphism $f \c X \to Y$ in the overcategory
  $\C_{/X}$ to be a weak equivalence, fibration, or cofibration if it
  is such in $\C$, this defines a model structure on $\C_{/X}$
  \cite{nlab-modelover}.
\end{nothing}

%%%%%%%%%%%%%%%%%%%%%%%%%%%%%%%%%%%%%%%%%%%%%%%%%%%%%%%%%%%%%%%%%%%%%%

\section{The cotangent complex}

With (\ref{relevant-model}) and (\ref{over-model}) we've managed to
construct natural model structures on the categories $\s\Alg(R)_{/A}$
and $\s\Mod(A)$ we were discussing back in \textsection
\ref{abelianalgebra}.\footnote{I've glossed over a technical detail
  here. To get the model structure on $\s\Alg(R)_{/A}$ via
  (\ref{over-model}) we should first observe that there is a natural
  identification $\s\Alg(R)_{/A} \simeq (\s\Alg(R))_{/c(A)}$, where
  $c(A)$ is the constant simplicial object as defined below in
  (\ref{constant}).} (Recall $R$ is a ring and $A$ an $R$-algebra.)

\newcommand{\In}{\operatorname{In}}

\begin{notation}
  We recall that in the current setting we have an adjunction
  \[
  \begin{tikzcd}
    \Ab \ : \ \s\Alg(R)_{/A} \rar[yshift=2pt] & \s\Mod(A) \ :\ \In,
    \lar[yshift=-2pt]
  \end{tikzcd}
  \]
  where $\In$ is the inclusion functor $M \mapsto A \oplus M$ and
  $\Ab$ is the abelianisation (or K\"ahler differentials) functor $B
  \mapsto \Omega_{B/R} \otimes_B A$---and both of these are defined by
  applying the definitions (\ref{square-zero}) and
  (\ref{derivations}.\ref{kahdiff}) degree-wise.
\end{notation}

The next step is to derive $\Ab$ as in (\ref{derivedab}), to get our
``homology object''. But to be able to do this, we first need to check
that the adjunction is in fact a Quillen adjunction.

\begin{lemma}
  The functor $\In$ is right Quillen, and hence $\Ab$ is left
  Quillen.
\end{lemma}

\begin{proof}
  Let $\phi \c M \to N$ be any morphism in $\s\Mod(A)$ and $\id_A
  \oplus \phi\c A \oplus M \to A \oplus N$ the morphism in
  $\s\Alg(R)_{/A}$ given by applying $\In$. Consider any lifting
  problems
  \[
  \begin{tikzcd}
    K \rar{b} \dar & M \dar{\phi} \\ L \rar{d} & N
  \end{tikzcd}
  \hspace{40pt}
  \begin{tikzcd}
    K \rar{(a,b)} \dar & A \oplus M \dar{\id_A \oplus \phi} \\ L
    \rar{(c,d)} & A \oplus N
  \end{tikzcd}
  \]
  in $\s\Set$. If there is a lift $e \c L \to M$ in the first problem,
  then clearly $(c,e) \c L \to A \oplus M$ is a lift in the second
  problem. Since fibrations and acyclic fibrations in $\s\Mod(A)$ and
  $\s\Alg(R)_{/A}$ are defined by lifting properties in $\s\Set$, this
  shows that $\In$ is right Quillen.
\end{proof}

Finally it makes sense to make the following definition.

\begin{definition}
  The \emph{cotangent complex functor} is the total left derived
  functor
  \[
  \L\Ab \c \ho(\s\Alg(R)_{/A}) \to \ho(\s\Mod(A)).
  \]
\end{definition}

But hold on, why do we care about that? At the start, we wanted a
homology theory for the algebra $R \to A$. Somehow we've ended up with
a functor on the \emph{simplicial} $R$-algebras \emph{over} $A$. Well
ok, don't worry, we just have a bit more work to do.

\begin{nothing}
  \label{constant}
  Recall for any category $\C$ we have an embedding $c \c \C \to \s\C$
  by taking the constant simplicial objects. That is, for $X \in \C$
  we can define $c(X) \in \s\C$ by $c(X)_n \ce X$ for $n \ge 0$
  with all face and degeneracy maps just the identity $\id_X$.
\end{nothing}

Now we get to the definitions we've really been waiting for.

\begin{definitions}
  \begin{enumerate}[leftmargin=*]
  \item The \emph{cotangent complex} of the $R$-algebra
    $A$ is
    \[
    L_{A/R} \ce \L\Ab(c(A)),
    \]
    i.e., the cotangent complex functor evaluated on the constant
    simplicial object determined by $A$. Note $L_{A/R} \in
    \s\Mod(A)$, but we will identify it via the Dold-Kan
    correspondence with the associated chain complex of $A$-modules.
  \item As (the value of) a total left derived functor, $L_{A/R}$
    is well-defined up to weak equivalence, and hence we can define
    the \emph{Andr\'e-Quillen homology and cohomology} of $R \to A$
    with coefficients in an $A$-module $M$ by
    \begin{align*}
      D_n(A,R;M) &\ce H_n(L_{A/R} \otimes_A M),\\ D^n(A,R;M)
      &\ce H^n(\hom_{\Mod(A)}(L_{A/R}, M)).
    \end{align*}
    Note in particular that homology with coefficients in $A$ is given
    by $\pi_*(L_{A/R})$.
  \end{enumerate}
\end{definitions}

%%%%%%%%%%%%%%%%%%%%%%%%%%%%%%%%%%%%%%%%%%%%%%%%%%%%%%%%%%%%%%%%%%%%%%

\section{Simplicial resolutions}

To end our formal discussion, we remark briefly about the passage to
simplicial objects needed to define the cotangent complex.

\renewcommand{\A}{\mathcal{A}}
\renewcommand{\B}{\mathcal{B}}

Consider the theory of derived functors in abelian categories. If $F
\c \A \to \B$ is a right exact functor between abelian categories, to
compute the total left derived functor of $F$ on $X \in \A$, we apply
$F$ to a projective resolution of $X$, which we can then take homology
of. In the standard model structure on categories of chain complexes,
a projective resolution of $X$ is precisely a cofibrant replacement of
the chain complex with $X$ concentrated in degree $0$. Finally via the
Dold-Kan correspondence, this is equivalent to picking a cofibrant
replacement of $c(X) \in \s\A$.

To define the cotangent complex, we wanted to derive a functor from a
non-abelian category. The above tells us that instead of taking
projective resolutions, the correct analogue is to take a
\emph{simplicial resolution}, that is, a cofibrant replacement of the
constant simplicial object. Indeed, this is exactly how we defined
$L_{A/R}$.

%%%%%%%%%%%%%%%%%%%%%%%%%%%%%%%%%%%%%%%%%%%%%%%%%%%%%%%%%%%%%%%%%%%%%%

\section*{Concluding nonsense}

At the moment, I don't have the time or space to learn or write about
what this homology theory of algebras really means or why it's
useful. As one might expect of a left derived functor, it turns out
that Andr\'e-Quillen homology finishes the long exact sequence ending
at a certain right exact sequence which one has for K\"ahler
differentials. In addition, the cotangent complex is supposed to carry
a lot of information about the ``deformation theory'' of algebras, and
when suitably generalised, of schemes and other algebro-geometric
objects. But now we've really reached the point where I don't know
what I'm talking about.

Hopefully I can learn about these things properly in the near
future. In any case, I think just the work we did to define the
cotangent complex is a pretty awesome and convincing example of model
categories---in particular the theory of non-abelian derived functors
and simplicial resolutions---being an excellent tool for introducing
homotopical ideas into new settings.

%%%%%%%%%%%%%%%%%%%%%%%%%%%%%%%%%%%%%%%%%%%%%%%%%%%%%%%%%%%%%%%%%%%%%%

\nocite{iyengar-cotangent}
\bibliographystyle{amsalpha}
\bibliography{refs}

\end{document}
