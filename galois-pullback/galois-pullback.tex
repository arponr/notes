%%%%%%%%%%%%%%%%%%%%%%%%%%%%%%%%%%%%%%%%%%%%%%%%%%%%%%%%%%%%%%%%%%%%%%

\renewcommand{\A}{\mathbb{A}}
\renewcommand{\O}{\mathcal{O}}

\renewcommand{\a}{\mathfrak{a}}
\newcommand{\p}{\mathfrak{p}}
\newcommand{\q}{\mathfrak{q}}

\newcommand{\height}{\operatorname{ht}}

%%%%%%%%%%%%%%%%%%%%%%%%%%%%%%%%%%%%%%%%%%%%%%%%%%%%%%%%%%%%%%%%%%%%%%


\title{Pulling back a Galois correspondence}
\author{Arpon Raksit}
\date{October 25, 2014}

\begin{document}
\maketitle

%%%%%%%%%%%%%%%%%%%%%%%%%%%%%%%%%%%%%%%%%%%%%%%%%%%%%%%%%%%%%%%%%%%%%%

\newcommand{\Sub}{\operatorname{Sub}}
\newcommand{\Gal}{\operatorname{Gal}}
\newcommand{\Stab}{\operatorname{Stab}}
\newcommand{\ab}{\mathrm{ab}}
\newcommand{\Ab}{\mathrm{Ab}}

\begin{definition}
  For $G$ a topological group, we denote by $\Sub_G$ the set of open
  normal subgroups of $G$, which is a poset under inclusion.
\end{definition}

\begin{proposition}
  \label{main}
  Let $G$ be a profinite group. Suppose we have a morphism of
  topological groups $\phi \c A \to G$ such that for every
  $N \in \Sub_G$ the induced map $\phi_N \c A/\phi^{-1}(N) \to G/N$ is
  an isomorphism. Then:
  \begin{enumerate}
  \item \label{preimage-injective} the preimage map
    $\Sub_G \to \Sub_A$, defined by $N \mapsto \phi^{-1}(N)$, is
    injective;
  \item \label{join-preserved}
    $\phi^{-1}(N_1N_2) = \phi^{-1}(N_1)\phi^{-1}(N_2)$ for
    $N_1,N_2 \in \Sub_G$.
  \end{enumerate}
\end{proposition}
   
\begin{proof}
  Let $N_1,N_2 \in \Sub_G$. To prove \eqref{preimage-injective} it
  suffices to show that $N_1 \subseteq N_2$ if
  $\phi^{-1}(N_1) \subseteq \phi^{-1}(N_2)$, so assume the
  latter. Then
  \begin{equation}
    \label{intersection-preserved}
    \phi^{-1}(N_1 \cap N_2) = \phi^{-1}(N_1) \cap \phi^{-1}(N_2) =
    \phi^{-1}(N_2).
  \end{equation}
  Now consider the commutative diagram
  \[
  \begin{tikzcd}
    A \ar[d, "\phi"] \ar[r] &
    A/\phi^{-1}(N_1 \cap N_2) \ar[d, "\phi_{N_1 \cap N_2}"] \ar[r] &
    A/\phi^{-1}(N_2) \ar[d, "\phi_{N_2}"] \\
    G \ar[r] & G/(N_1 \cap N_2) \ar[r] & G/N_2
  \end{tikzcd}
  \]
  with the horizontal maps the projections. In the right-hand square,
  the vertical maps are isomorphisms by hypothesis, and the top
  horizontal map is an isomorphism by \eqref{intersection-preserved};
  thus the bottom horizontal map is an isomorphism, which implies
  $N_1 \subseteq N_2$.

  We now prove \eqref{join-preserved}. Certainly
  $\phi^{-1}(N_1N_2) \supseteq \phi^{-1}(N_1)\phi^{-1}(N_2)$. And it's
  fairly easy to see we have the sequence of identifications
  \begin{align*}
    \frac{\phi^{-1}(N_1)\phi^{-1}(N_2)}{\phi^{-1}(N_1)}
    & \iso \frac{\phi^{-1}(N_2)}{\phi^{-1}(N_1) \cap \phi^{-1}(N_2)}
      = \frac{\phi^{-1}(N_2)}{\phi^{-1}(N_1 \cap N_2)} \\
    & = \phi_{N_1 \cap N_2}^{-1}\l( \frac{N_1}{N_1 \cap N_2} \r)
      \iso \phi_{N_1}^{-1}\l( \frac{N_1N_2}{N_1} \r)
      = \frac{\phi^{-1}(N_1N_2)}{\phi^{-1}(N_1)}.
  \end{align*}
  But since $\phi^{-1}(N_1)$ has finite index in $A$, it follows
  that
  \[
  [\phi^{-1}(N_1N_2) : \phi^{-1}(N_1)\phi^{-1}(N_2)] = \l[
  \frac{\phi^{-1}(N_1N_2)}{\phi^{-1}(N_1)} :
  \frac{\phi^{-1}(N_1)\phi^{-1}(N_2)}{\phi^{-1}(N_1)} \r] = 1,
  \]
  proving the desired claim.
\end{proof}

\begin{example}
  If we take $G$ to be a Galois group in \eqref{main}, then the
  proposition says that when we have a suitable morphism $A \to G$,
  the Galois theory described by $G$ is in fact controlled by
  $A$. This is what I meant by ``pulling back a Galois correspondence''
  in the title. Let's state this in more detail in the motivating
  example.

  Let $K$ be a non-archimedean local field. Let $K^\ab$ be a maximal
  abelian extension of $K$. The Galois group $G \ce \Gal(K^\ab/K)$ is
  a profinite group, and Galois theory tells us that the poset
  $\Sub_G$ is (contravariantly) equivalent to the poset $\Ab_K$ of
  finite abelian extensions of $K$, i.e. the set of finite
  subextensions of $K^\ab$ ordered by inclusion. The ``reciprocity''
  statement in local class field theory asserts:
  \begin{itemize}
  \item existence of a morphism $\phi_K \c K^\times \to \Gal(K^\ab/K)$
    satisfying the hypothesis of \eqref{main};
  \item if $N \ce \Stab_L \in \Sub_G$ is the subgroup corresponding to
    a finite abelian extension $K \inj L$, then $\phi_K^{-1}(N)$ is
    the \emph{norm group} $N_{L/K}(L^\times) \subseteq K^\times$.
  \end{itemize}
  Thus, putting \eqref{main} and Galois theory together gives us that
  the poset $\Ab_K$ is (contravariantly) equivalent to the poset of
  norm groups in $K^\times$ by the correspondence
  $L \mapsto N_{L/K}(L^\times)$. Then there is an ``existence''
  theorem in local class field theory stating that the norm groups are
  precisely the open subgroups of finite index in $K^\times$.
\end{example}

%%%%%%%%%%%%%%%%%%%%%%%%%%%%%%%%%%%%%%%%%%%%%%%%%%%%%%%%%%%%%%%%%%%%%%

% \bibliographystyle{amsalpha}
% \bibliography{refs}

\end{document}
