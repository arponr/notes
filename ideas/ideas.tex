%%%%%%%%%%%%%%%%%%%%%%%%%%%%%%%%%%%%%%%%%%%%%%%%%%%%%%%%%%%%%%%%%%%%%%

\newcommand{\ob}{\oper{ob}}
\renewcommand{\hom}{\oper{hom}}
\newcommand{\id}{\oper{id}}
\newcommand{\im}{\oper{im}}
\newcommand{\op}{\oper{op}}

\newcommand{\Top}{\oper{Top}}
\newcommand{\Set}{\oper{Set}}
\newcommand{\Ab}{\oper{Ab}}
\newcommand{\Grp}{\oper{Grp}}
\newcommand{\Mod}{\oper{Mod}}
\newcommand{\Simplex}{\Delta}
\newcommand{\s}{\oper{s}}
\newcommand{\Ch}{\oper{Ch}}

\newcommand{\Sing}{\oper{Sing}}
\renewcommand{\H}{\mathrm{H}}

%%%%%%%%%%%%%%%%%%%%%%%%%%%%%%%%%%%%%%%%%%%%%%%%%%%%%%%%%%%%%%%%%%%%%%


\newlist{idea}{enumerate}{5}
\setlist[idea]{%
  label*=\textbf{\arabic*.},%
  leftmargin=*,%
  itemsep=1ex,%
  topsep=1ex,%
}

\begin{document}
\thispagestyle{fancy}

%%%%%%%%%%%%%%%%%%%%%%%%%%%%%%%%%%%%%%%%%%%%%%%%%%%%%%%%%%%%%%%%%%%%%%

\begin{flushleft}
  Last updated: \today.
\end{flushleft}

\medskip

\begin{idea}
\item How does one actually construct equivariant elliptic cohomology?
  \begin{idea}
    \renewcommand{\E}{\mathbb{E}}
  \item In Lurie's framework, equivariant versions of cohomology with
    coefficients in an $\E_\infty$-ring $A$ arise from a (derived)
    oriented group scheme $G$ over $A$. An elliptic cohomology theory
    $A$ comes from an elliptic curve $G_0$ over $\pi_0(A)$, so it
    seems like the question is: when I can I refine this to a derived
    elliptic curve $G$ over $A$? I.e. I need to find a pullback
    diagram: \newcommand{\Spec}{\operatorname{Spec}}
    \[
    \begin{tikzcd}
      G_0 \ar[r] \ar[d] & G \ar[d] \\
      \Spec(\pi_0(A)) \ar[r] & \Spec(A).
    \end{tikzcd}
    \]
    If I replace $A$ by it's connective cover (I can do this because
    $G$ is supposed to be flat over $A$) then I should think of the
    homotopy groups of $A$ above degree $0$ as higher nilpotents (in
    particular the sequence
    \[
    \pi_0(A) \iso \tau_{\le 0}(A) \from \tau_{\le 1}(A) \from
    \tau_{\le 2}(A) \from \cdots
    \]
    is a sequence of square-zero extensions of $\E_\infty$-rings) and
    so maybe I can approach this via some obstruction/deformation
    theory (the cotangent complex section of \emph{Higher Algebra} is
    probably relevant)?
  \end{idea}
\end{idea}

%%%%%%%%%%%%%%%%%%%%%%%%%%%%%%%%%%%%%%%%%%%%%%%%%%%%%%%%%%%%%%%%%%%%%%


\end{document}
