%%%%%%%%%%%%%%%%%%%%%%%%%%%%%%%%%%%%%%%%%%%%%%%%%%%%%%%%%%%%%%%%%%%%%%

\newcommand{\ob}{\oper{ob}}
\renewcommand{\hom}{\oper{hom}}
\newcommand{\id}{\oper{id}}
\newcommand{\im}{\oper{im}}
\newcommand{\op}{\oper{op}}

\newcommand{\Top}{\oper{Top}}
\newcommand{\Set}{\oper{Set}}
\newcommand{\Ab}{\oper{Ab}}
\newcommand{\Grp}{\oper{Grp}}
\newcommand{\Mod}{\oper{Mod}}
\newcommand{\Simplex}{\Delta}
\newcommand{\s}{\oper{s}}
\newcommand{\Ch}{\oper{Ch}}

\newcommand{\Sing}{\oper{Sing}}
\renewcommand{\H}{\mathrm{H}}

%%%%%%%%%%%%%%%%%%%%%%%%%%%%%%%%%%%%%%%%%%%%%%%%%%%%%%%%%%%%%%%%%%%%%%


\newlist{ideas}{enumerate}{5}
\setlist[ideas]{%
  label*=\textbf{\arabic*.},%
  leftmargin=*,%
  itemsep=0.5ex,%
  topsep=0.5ex,%
}

\begin{document}
\thispagestyle{fancy}

\begin{flushleft}
  Last updated: \today.
\end{flushleft}

\medskip

\begin{ideas}

%%%%%%%%%%%%%%%%%%%%%%%%%%%%%%%%%%%%%%%%%%%%%%%%%%%%%%%%%%%%%%%%%%%%%%

\item Constructing equivariant elliptic cohomology

  \begin{ideas}
  \item If we think about it as in Jacob's survey, equivariant
    versions of cohomology with coefficients in an $\rE_\infty$-ring
    $A$ arise from a (derived) oriented group scheme $G$ over $A$. An
    elliptic cohomology theory $A$ comes from an elliptic curve $G_0$
    over $\pi_0(A)$, so it seems like the question is: when I can I
    refine this to a derived elliptic curve $G$ over $A$? I.e. I need
    to find a pullback diagram:
    \[
    \begin{tikzcd}
      G_0 \ar[r] \ar[d] & G \ar[d] \\
      \Spec(\pi_0(A)) \ar[r] & \Spec(A).
    \end{tikzcd}
    \]
    If I replace $A$ by it's connective cover (I can do this because
    $G$ is supposed to be flat over $A$) then I should think of the
    homotopy groups of $A$ above degree $0$ as higher nilpotents (in
    particular the sequence
    \[
    \pi_0(A) \iso \tau_{\le 0}(A) \from \tau_{\le 1}(A) \from
    \tau_{\le 2}(A) \from \cdots
    \]
    is a sequence of square-zero extensions of $\rE_\infty$-rings) and
    so maybe I can approach this via some obstruction/deformation
    theory (the cotangent complex section of \emph{Higher Algebra} is
    probably relevant)?
  \end{ideas}

%%%%%%%%%%%%%%%%%%%%%%%%%%%%%%%%%%%%%%%%%%%%%%%%%%%%%%%%%%%%%%%%%%%%%%

\item Equivariant complex-orientability

  \begin{ideas}
  \item The condition I'm most interested in is that the cohomology of
    $\lP V$ be a finite free module for all finite-dimensional
    representations.
  \item What I would like is that the global spectra defined by
    (oriented?) derived group schemes (on abelian groups, and then
    Kan-extended to all groups) define complex-oriented $G$-spectra
    for all $G$. Then I could deduce from complex-oriented descent
    that flat base-changes of these global spectra also satisfy
    abelian descent.
  \item So I also need to sort out what the deal is with orientations
    of derived group schemes. It's just striking me at this moment
    that it's maybe not an accident that Jacob called these
    orientations.
  \end{ideas}

%%%%%%%%%%%%%%%%%%%%%%%%%%%%%%%%%%%%%%%%%%%%%%%%%%%%%%%%%%%%%%%%%%%%%%

\item Genuine global spectra

  \begin{ideas}
  \item One issue I'd like to sort out is the following. Suppose I
    have a (na\"ive) global spectrum $E$ which satisfies abelian
    descent, and that the restriction of $E$ to abelian groups can be
    extended to a genuine global spectrum. Can $E$ itself be extended
    to a genuine global spectrum?

    \begin{ideas}
      \item The first idea I had about resolving this was to think
        about some sort of ``additive Kan extension''. The issue at
        hand is that genuine global spectra are \emph{additive}
        functors on the span category, and Kan extending won't
        preserve additivity in general. Rather, I need an additive Kan
        extension which gives me the universal additive
        extension. Such a thing exists of course by an adjoint functor
        theorem, but to address the concrete question raised above, I
        need a nice colimit-type formula for computing it, like
        ordinary Kan extensions, so that I can employ cofinality
        arguments. I haven't given this idea a full go yet, but note
        that Jacob thinks it's unlikely to work out.

      \item Rather Jacob thinks that things should become clearer
        after categorifying. Something about there being universal
        properties at the level of $2$-categories which don't exist at
        the level of $1$-categories which will make things nice. Have
        to talk to him about this.
    \end{ideas}
    
  \item This is of interest to me because I'd like to know when global
    spectra arising from derived groups schemes are in fact genuine
    global spectra, and I want to think about abelian descent of
    genuine global spectra.

  \end{ideas}

%%%%%%%%%%%%%%%%%%%%%%%%%%%%%%%%%%%%%%%%%%%%%%%%%%%%%%%%%%%%%%%%%%%%%%

\item Atiyah-Segal completion theorems

  \begin{ideas}
  \item Can one prove the Atiyah-Segal completion theorem immediately
    from the description of K-theory in terms of the (derived)
    multiplicative group, i.e. without reference to vector bundles at
    all? If so, is there a general form of the completion theorem
    for spectra arising from derived group schemes?
  \end{ideas}
\end{ideas}


\end{document}
