%%%%%%%%%%%%%%%%%%%%%%%%%%%%%%%%%%%%%%%%%%%%%%%%%%%%%%%%%%%%%%%%%%%%%%

\renewcommand{\A}{\mathbb{A}}
\renewcommand{\O}{\mathcal{O}}

\renewcommand{\a}{\mathfrak{a}}
\newcommand{\p}{\mathfrak{p}}
\newcommand{\q}{\mathfrak{q}}

\newcommand{\height}{\operatorname{ht}}

%%%%%%%%%%%%%%%%%%%%%%%%%%%%%%%%%%%%%%%%%%%%%%%%%%%%%%%%%%%%%%%%%%%%%%


\newlist{idea}{enumerate}{5}
\setlist[idea]{%
  label*=\textbf{\arabic*.},%
  leftmargin=*,%
  itemsep=1ex,%
  topsep=1ex,%
}

\begin{document}
\thispagestyle{fancy}

%%%%%%%%%%%%%%%%%%%%%%%%%%%%%%%%%%%%%%%%%%%%%%%%%%%%%%%%%%%%%%%%%%%%%%

\begin{flushleft}
  Last updated: \today.
\end{flushleft}

\medskip

\begin{idea}

%%%%%%%%%%%%%%%%%%%%%%%%%%%%%%%%%%%%%%%%%%%%%%%%%%%%%%%%%%%%%%%%%%%%%%

\item How does one actually construct equivariant elliptic cohomology?

  %%%%%%%%%%%%%%%%%%%%%%%%%%%%%%%%%%%%%%%%%%%%%%%%%%%%%%%%%%%%%%%%%%%%

  \begin{idea}
  \item In Lurie's framework, equivariant versions of cohomology with
    coefficients in an $\rE_\infty$-ring $A$ arise from a (derived)
    oriented group scheme $G$ over $A$. An elliptic cohomology theory
    $A$ comes from an elliptic curve $G_0$ over $\pi_0(A)$, so it
    seems like the question is: when I can I refine this to a derived
    elliptic curve $G$ over $A$? I.e. I need to find a pullback
    diagram:
    \[
    \begin{tikzcd}
      G_0 \ar[r] \ar[d] & G \ar[d] \\
      \Spec(\pi_0(A)) \ar[r] & \Spec(A).
    \end{tikzcd}
    \]
    If I replace $A$ by it's connective cover (I can do this because
    $G$ is supposed to be flat over $A$) then I should think of the
    homotopy groups of $A$ above degree $0$ as higher nilpotents (in
    particular the sequence
    \[
    \pi_0(A) \iso \tau_{\le 0}(A) \from \tau_{\le 1}(A) \from
    \tau_{\le 2}(A) \from \cdots
    \]
    is a sequence of square-zero extensions of $\rE_\infty$-rings) and
    so maybe I can approach this via some obstruction/deformation
    theory (the cotangent complex section of \emph{Higher Algebra} is
    probably relevant)?
  \end{idea}

%%%%%%%%%%%%%%%%%%%%%%%%%%%%%%%%%%%%%%%%%%%%%%%%%%%%%%%%%%%%%%%%%%%%%%

\item How does one set up the theory of complex-orientability in the
  equivariant setting?

  %%%%%%%%%%%%%%%%%%%%%%%%%%%%%%%%%%%%%%%%%%%%%%%%%%%%%%%%%%%%%%%%%%%%

  \begin{idea}
  \item The condition I'm most interested in is that the cohomology of
    $\lP V$ be a finite free module for all finite-dimensional
    representations.
  \item What I would like is that the global spectra defined by
    (oriented?) derived group schemes (on abelian groups, and then
    Kan-extended to all groups) define complex-oriented $G$-spectra
    for all $G$. Then I could deduce from complex-oriented descent
    that flat base-changes of these global spectra also satisfy
    abelian descent.
  \item So I also need to sort out what the deal is with orientations
    of derived group schemes. It's just striking me at this moment
    that it's maybe not an accident that Jacob called these
    orientations.
  \end{idea}

%%%%%%%%%%%%%%%%%%%%%%%%%%%%%%%%%%%%%%%%%%%%%%%%%%%%%%%%%%%%%%%%%%%%%%

\end{idea}

%%%%%%%%%%%%%%%%%%%%%%%%%%%%%%%%%%%%%%%%%%%%%%%%%%%%%%%%%%%%%%%%%%%%%%


\end{document}
