%%%%%%%%%%%%%%%%%%%%%%%%%%%%%%%%%%%%%%%%%%%%%%%%%%%%%%%%%%%%%%%%%%%%%%

\renewcommand{\A}{\mathbb{A}}
\renewcommand{\O}{\mathcal{O}}

\renewcommand{\a}{\mathfrak{a}}
\newcommand{\p}{\mathfrak{p}}
\newcommand{\q}{\mathfrak{q}}

\newcommand{\height}{\operatorname{ht}}

%%%%%%%%%%%%%%%%%%%%%%%%%%%%%%%%%%%%%%%%%%%%%%%%%%%%%%%%%%%%%%%%%%%%%%

%%%%%%%%%%%%%%%%%%%%%%%%%%%%%%%%%%%%%%%%%%%%%%%%%%%%%%%%%%%%%%%%%%%%%%

\renewcommand{\A}{\mathbb{A}}
\renewcommand{\O}{\mathcal{O}}

\renewcommand{\a}{\mathfrak{a}}
\newcommand{\p}{\mathfrak{p}}
\newcommand{\q}{\mathfrak{q}}

\newcommand{\height}{\operatorname{ht}}

%%%%%%%%%%%%%%%%%%%%%%%%%%%%%%%%%%%%%%%%%%%%%%%%%%%%%%%%%%%%%%%%%%%%%%


%%%%%%%%%%%%%%%%%%%%%%%%%%%%%%%%%%%%%%%%%%%%%%%%%%%%%%%%%%%%%%%%%%%%%%

\title{Generalized character theory}
\author{Arpon Raksit}
\date{January 9, 2015}

\begin{document}
\maketitle
\thispagestyle{fancy}

%%%%%%%%%%%%%%%%%%%%%%%%%%%%%%%%%%%%%%%%%%%%%%%%%%%%%%%%%%%%%%%%%%%%%%

\section{Overview}

\renewcommand{\E}{\mathbb{E}}
\renewcommand{\G}{G}
\renewcommand{\L}{\mathcal{L}}

\begin{notation}
  If $E$ is a spectrum and $X$ is a spectrum or a space, then we'll
  denote the $E$-cohomology of $X$ (as a spectrum, i.e. the function
  spectrum) by $C^*(X;E)$.
\end{notation}

\begin{definition}
  Given a $p$-divisible group $\G$ over an $\E_\infty$-ring $E$, we
  can form a ``$\G$-twisted'' version of $E$-cohomology on spaces $X$,
  the spectrum of which we'll denote by $C_\G^*(X;E)$.
\end{definition}

\begin{definition}
  Over any $\E_\infty$-ring $E$ there is a canonical (derived)
  $p$-divisible group $\G_E$ given by
  $\G_E[p^k] \ce C^*(B\Z/p^k, E)$.\footnote{Check that $\G_E$ is
    really is a $p$-divisible group (the surjectivity/divisibility
    criterion).}
\end{definition}

Building the generalized character maps relies on the following three
observations\footnote{Do I need the orientability hypotheses in these
  parts of the argument?}:

\begin{theorem}
  \label{base-change}
  Let $\G$ be a $p$-divisible group over an $\E_\infty$-ring $E$. Let
  $E'$ be an $\E_\infty$-$E$-algebra. Let $\G' \ce E' \otimes_E \G$ be
  the base change. For sufficiently finite spaces $X$ (e.g. of the
  form $BG$), the canonical map\footnote{What is the canonical map?}
  $E' \otimes_E C_\G^*(X;E) \to C_{\G'}^*(X;E')$ is an equivalence.
\end{theorem}

\begin{theorem}
  \label{kn-local}
  Suppose $E$ is a $K(n)$-local $\E_\infty$-ring. For all spaces $X$
  the canonical map\footnote{What is the canonical map?}
  $C_{\G_E}^*(X;E) \to C^*(X;E)$ is an equivalence.
\end{theorem}

\begin{corollary}
  The conclusion of \eqref{kn-local} holds also for $E'$ a flat
  extension of a $K(n)$-local $E$.
\end{corollary}

\begin{proof}
  The ordinary side should satisfy base change too? But does the left
  side only satisfy for sufficiently finite?
\end{proof}

\begin{theorem}
  \label{splitting}
  Let $\G$ be a $p$-divisible group over an $\E_\infty$-ring
  $E$. Suppose there is a splitting
  $\G \iso \G' \times (\Q_p/\Z_p)^t$. Then for sufficiently finite
  spaces $X$ there is an equivalence
  $C_\G^*(X;E) \iso C_{\G'}^*(\L^t(X);E)$, where $\L$ is the relevant
  loop space functor.\footnote{What is the relevant loop space
    functor?}
\end{theorem}

Now fix a perfect field $k$ of characteristic $p$. Then for $n \ge 0$
we have Morava E-theories $E(n)$ and K-theories $K(n)$. We should have
that:
\begin{enumerate}
\item $E(n)$ is $K(n)$-local\footnote{I know this is supposed to be
    right for completed Johnson-Wilson theory. Is it right for more
    general Morava E- and K-theories?};
\item There is a flat extension $E(n,t)$ of $L_{K(t)}E(n)$ such that
  \[
  E(n,t) \otimes_{E(n)}\G_{E(n)} \iso \G_{E(n,t)} \times (\Q_p/\Z_p)^t.
  \]
  For this fact I certainly need an orientability result, proved by
  Stapleton.
\end{enumerate}
These observations together with the above three theorems imply that
for sufficiently finite spaces $X$,
\begin{align*}
  E(n,t) \otimes_{E(n)} C^*(X;E(n))
  & \iso E(n,t) \otimes_{E(n)} C_{\G_{E(n)}}^*(X;E(n)) \\
  & \iso C_{E(n,t) \otimes_{E(n)} \G_{E(n)}}^*(X;E(n,t)) \\
  & \iso C_{\G_{E(n,t)}}^*(\L^t(X);E(n,t)), \\
  & \iso C^*(\L^t(X);E(n,t)),
\end{align*}
which is the desired transchromatic character theory.

%%%%%%%%%%%%%%%%%%%%%%%%%%%%%%%%%%%%%%%%%%%%%%%%%%%%%%%%%%%%%%%%%%%%%%

\section{Twisted local systems}

\renewcommand{\Cats}{\mathrm{Cats}}
\renewcommand{\Spaces}{\mathrm{Spaces}}

\begin{notation}
  All category-theoretic notions in this section are by default in the
  infinity sense. E.g. by categories and functors we
  really mean $\infty$-categories and functors between them; by a
  $2$-category we mean an $(\infty,2)$-category; and so on. Also:
  \begin{enumerate}
  \item Let $\Cats$ denote the $2$-category of categories.
  \item Let $\Spaces$ denote the category of spaces.
  \end{enumerate}
\end{notation}

\begin{situation}
  Here's the setup. We have a functor $I \c \Spaces \to \Cats$.
\end{situation}

%%%%%%%%%%%%%%%%%%%%%%%%%%%%%%%%%%%%%%%%%%%%%%%%%%%%%%%%%%%%%%%%%%%%%%

\bibliographystyle{amsalpha}
\bibliography{../refs}

\end{document}
