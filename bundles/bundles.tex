%%%%%%%%%%%%%%%%%%%%%%%%%%%%%%%%%%%%%%%%%%%%%%%%%%%%%%%%%%%%%%%%%%%%%%

\newcommand{\ob}{\oper{ob}}
\renewcommand{\hom}{\oper{hom}}
\newcommand{\id}{\oper{id}}
\newcommand{\im}{\oper{im}}
\newcommand{\op}{\oper{op}}

\newcommand{\Top}{\oper{Top}}
\newcommand{\Set}{\oper{Set}}
\newcommand{\Ab}{\oper{Ab}}
\newcommand{\Grp}{\oper{Grp}}
\newcommand{\Mod}{\oper{Mod}}
\newcommand{\Simplex}{\Delta}
\newcommand{\s}{\oper{s}}
\newcommand{\Ch}{\oper{Ch}}

\newcommand{\Sing}{\oper{Sing}}
\renewcommand{\H}{\mathrm{H}}

%%%%%%%%%%%%%%%%%%%%%%%%%%%%%%%%%%%%%%%%%%%%%%%%%%%%%%%%%%%%%%%%%%%%%%

%%%%%%%%%%%%%%%%%%%%%%%%%%%%%%%%%%%%%%%%%%%%%%%%%%%%%%%%%%%%%%%%%%%%%%

\newcommand{\ob}{\oper{ob}}
\renewcommand{\hom}{\oper{hom}}
\newcommand{\id}{\oper{id}}
\newcommand{\im}{\oper{im}}
\newcommand{\op}{\oper{op}}

\newcommand{\Top}{\oper{Top}}
\newcommand{\Set}{\oper{Set}}
\newcommand{\Ab}{\oper{Ab}}
\newcommand{\Grp}{\oper{Grp}}
\newcommand{\Mod}{\oper{Mod}}
\newcommand{\Simplex}{\Delta}
\newcommand{\s}{\oper{s}}
\newcommand{\Ch}{\oper{Ch}}

\newcommand{\Sing}{\oper{Sing}}
\renewcommand{\H}{\mathrm{H}}

%%%%%%%%%%%%%%%%%%%%%%%%%%%%%%%%%%%%%%%%%%%%%%%%%%%%%%%%%%%%%%%%%%%%%%


\newcommand{\aut}{\operatorname{aut}}
\newcommand{\saut}{\operatorname{\underline{aut}}}
\newcommand{\Bun}{\operatorname{Bun}}
\renewcommand{\E}{\mathcal{E}}
\renewcommand{\F}{\mathcal{F}}
\renewcommand{\G}{\mathcal{G}}
\renewcommand{\H}{\mathcal{H}}
\renewcommand{\X}{\mathcal{X}}
\renewcommand{\Y}{\mathcal{Y}}

%--------------------------------------------------------------------%

\title{Bundles}
\author{Arpon Raksit}
\date{June 17, 2014}

\begin{document}
\maketitle

%--------------------------------------------------------------------%

\section{Bundles on a space}

We begin by defining the motivating notions from topology.

\begin{notation}
  In this section, $X$ denotes a topological space.
\end{notation}

\begin{definitions}
  \label{fibre-bundle}
  Let $F$ be a space and $G$ a topological group.
  \begin{itemize}
  \item An \emph{$F$-bundle} on $X$ is a map $\pi \c E \to X$ such
    that there is an open cover $C = \{U_i\}$ of $X$ and
    homeomorphisms $\kappa_i \c \pi^{-1}(U_i) \to U_i \times F$ over
    $X$ for each $i$. In particular the fibre $\pi^{-1}(x)$ is
    homeomorphic to $F$ for every $x \in X$.
  \item Such data $(C,\kappa)$ is called a \emph{trivialisation} of
    the fibre bundle. The \emph{transition maps} associated to a
    trivialisation are the maps
    \[
    \tau_{ij} \c U_i \cap U_j \to \aut(F),
    \]
    where $\aut(F)$ denotes the group of homeomorphisms of $F$ with
    itself, such that the automorphism $\kappa_j\kappa_i^{-1}$ of
    $(U_i \cap U_j) \times F$ is given by $(u,f) \mapsto (u,
    \tau_{ij}(u)(f))$.
  \item Suppose $F$ is equipped with a left $G$-action. Then an
    \emph{$(F,G)$-bundle}, or \emph{$F$-bundle with structure group
      $G$}, on $X$ is data $(\pi,C,\kappa)$, where $\pi \c E \to X$ is
    an $F$-bundle on $X$ and $(C,\kappa)$ is a trivialisation of $\pi$
    for which the transition maps $\tau_{ij}$ factor through the
    action $G \to \aut(F)$.
  \item A \emph{principal $G$-bundle} is a $(G,G)$-bundle, where $G$
    is a left $G$-space via left multiplication. Equivalently we may
    say a principal $G$-bundle is an $(F,G)$-bundle where $G$ acts
    simply transitively on $F$.
  \end{itemize}
\end{definitions}

\begin{remark}
  \label{principal-right-action}
  Note that a \emph{trivial} principal $G$-bundle, $X \times G \to X$,
  has both left and a right $G$-actions over\footnote{Recall that a
    left or right $G$-action \emph{over $X$} on $\pi \c E \to X$ is an
    action on $E$ which preserves the fibres of $\pi$.} $X$, via left
  and right multiplication on $G$.  Since these two actions commute
  and a principal $G$-bundle $\pi \c E \to X$ locally looks like $U
  \times G \to U$ as a left $G$-space over $X$, the right actions on
  $U \times G$ will glue together to give a global right $G$-action on
  $E$ over $X$. Thus any principal $G$-bundle is a right $G$-space
  over $X$.
\end{remark}

These are the type of objects we should have in the back of our mind
throughout. We will generalise them using the language of sheaves,
with the following definitions as our basic tools.

%%%%%%%%%%%%%%%%%%%%%%%%%%%%%%%%%%%%%%%%%%%%%%%%%%%%%%%%%%%%%%%%%%%%%%

\section{Bundles on a site}

\begin{notation}
  Until further notice, $\X$ denotes a site. If $\E$ is a sheaf on
  $\X$ and $U \in \X$, let $\E|_U$ denote the restriction of $\E$ to
  the site $\X_{/U}$.
\end{notation}

\begin{definition}
  A \emph{covering} $C$ of $\X$ is a choice of covering $C(U) = \{i \c
  U_i \to U\}$ of each $U \in \X$. Note that we denote the source of a
  morphism $i \in C(U)$ by $U_i$.
\end{definition}

\begin{definitions}
  Let $\G$ be a sheaf of groups on $\X$.
  \begin{itemize}
  \item A \emph{$\G$-sheaf} on $\X$ is a sheaf of sets $\E$ on $\X$
    together with a right action $\E \times \G \to \E$ of $\G$. A
    \emph{morphism of $\G$-sheaves} $\E \to \E'$ on $\X$ is a morphism
    of sheaves of sets which commutes with the $\G$-actions
    (i.e. which is $\G$-equivariant).
  \item A \emph{$\G$-pseudotorsor} is $\G$-sheaf $\E$ such that for
    all $U \in \X$ open, either $\E(U)$ is empty or the action of
    $\G(U)$ on $\E(U)$ is simply transitive. A $\G$-pseudotorsor $\E$
    is a \emph{$\G$-torsor} if moreover there is a covering $C$ of
    $\X$ such that for all $U \in \X$, $\E(U_i)$ is nonempty for each
    $i \in C(U)$. A \emph{morphism of $\G$-pseudotorsors or
      $\G$-torsors} is simply a morphism of the underlying
    $\G$-sheaves.
  \item There is a \emph{trivial $\G$-torsor} on $\X$ given by $\G$
    equipped with the action given by right multiplication. A
    $\G$-sheaf $\E$ is called \emph{trivial} if there is an
    isomorphism of $\G$-sheaves $\E \iso \G$.
  \item Note that if $\E$ is a $\G$-sheaf, then $\E|_U$ is a
    $\G|_U$-sheaf on $\X_{/U}$ for any $U \in \X$. If $P$ is some
    property which can be enjoyed by $\H$-sheaves on $\Y$ for any
    sheaf of groups $\H$ on any site $\Y$---e.g. the properties of
    being a pseudotorsor, torsor, or trivial---then we say a
    $\G$-sheaf $\E$ on $\X$ \emph{locally has the property} $P$ if
    there is a covering $C$ of $\X$ such that for all $U \in \X$, the
    $\G|_{U_i}$ sheaf $\E|_{U_i}$ has the property $P$ for each $i \in
    C(U)$. The same goes for morphisms of $\G$-sheaves.
  \end{itemize}
\end{definitions}

\begin{remark}
  Note we should really have called these \emph{right} $\G$-sheaves,
  $\G$-pseudotorsors, and $\G$-torsors, and we can analogously define
  \emph{left} $\G$-sheaves, $\G$-pseudotorsors, and $\G$-torsors. We
  will need to distinguish between the two later on (as we needed to
  distinguish between left and right $G$-spaces for bundles above),
  but until then we agree to leave the modifier ``right'' implicit.
\end{remark}

\begin{remarks}
  \label{torsor-equiv}
  Observe that any morphism of $\G$-torsors on $\X$ is an isomorphism,
  since clearly it must locally be an isomorphism. So suppose $\E$ is
  a $\G$-pseudotorsor on $\X$ with a section $s \in \E(U)$ for some $U
  \in \X$. Then there is a morphism of $\G|_U$-torsors $\G|_U \to
  \E|_U$ on $\X_{/U}$ given by
  \[
  \G(V) \ni 1 \mapsto s|_V \in \E(V), \quad (V \to U) \in
  \X_{/U}.
  \]
  As just observed this must be an isomorphism. I.e., any section
  locally trivialises a pseudotorsor, and in particilar a pseudotorsor
  is a torsor if and only if it is locally trivial. In fact, any
  $\G$-sheaf is a $\G$-torsor if and only if it's locally
  trivial. This is immediate from the preceding discussion and the
  following lemma, that being a pseudotorsor or torsor is a local
  condition.
\end{remarks}

\begin{lemma}
  \label{torsor-local}
  If a $\G$-sheaf $\E$ on $\X$ is locally a $\G$-pseudotorsor
  (resp. $\G$-torsor), then $\E$ is a $\G$-pseudotorsor
  (resp. $\G$-torsor).
\end{lemma}

\begin{proof}
  It's clear that the statement for $\G$-torsors follows immediately
  from the statement for pseudotorsors, so we need only prove the
  latter. Let $U \in \X$ such that $\E(V)$ is nonempty. By hypothesis
  there is a covering $\{U_i \to U\}$ of $U$ such that $\E|_{U_i}$ is
  a $\G|_{U_i}$-pseudotorsor for all $i$. Let $s, t \in \E(U)$. We
  need to show there is a unique $g \in \G(U)$ such that $t =
  sg$. For all $i,j$ Write
  \[
  U_{ij} \ce U_i \times_U U_j, \quad
  s_i \ce s|_{U_i}, \quad s_{ij} \ce s|_{U_{ij}}, \quad
  t_i \ce t|_{U_i}, \quad t_{ij} \ce t|_{U_{ij}}
  \]
  Since $\E|_{U_i}$ is a $\G|_{U_i}$-pseudotorsor, there is a unique
  $g_i \in \G(U_i)$ such that $t_i = s_ig_i$; then
  \[
  s_{ij}g_i|_{V_{ij}} = t_i|_{V_{ij}} = t_{ij} = t_j|_{V_{ij}} =
  s_{ij}g_j|_{V_{ij}},
  \]
  so uniqueness also implies $g_i|_{V_{ij}} = g_j|_{V_{ij}}$. Thus the
  $g_i$ glue to $g \in \G(U)$. The equation $t = sg$ holds when
  restricted to each $U_i$ and thus must hold on $U$, and the
  uniqueness of $g$ is immediate from the uniqueness of the $g_i$.
\end{proof}

Let's bring things back down to bundles on spaces for a second.

\begin{definition}
  \label{map-sheaf}
  To any space $Y$ we can associate a sheaf of sets $\u{Y}$ on the
  space $X$: the \emph{sheaf of continuous maps to $Y$}, given by the
  assignment
  \[
  U \mapsto \hom_\Top(U,Y), \quad U \subseteq X \text{ open}.
  \]
\end{definition}

\begin{example}
  \label{bundle-example}
  Let $G$ be a topological group. Then $\u{G}$ is in fact a sheaf of
  groups. If $\pi \c E \to X$ is a right $G$-space over $X$ then the
  sheaf $\E$ of continuous sections of $\pi$ is a $\u{G}$-sheaf in a
  canonical manner. By \pref{principal-right-action}, any principal
  $G$-bundle gives an example of such a $\pi \c E \to X$. In this case
  we will moreover have that $\E$ is locally trivial, and hence a
  $\u{G}$-torsor by \pref{torsor-equiv}.
\end{example}

We next generalise the main definitions we began with in
\pref{fibre-bundle}.\footnote{I'm not sure if there's standard
  terminology for some of these things so I'm just sticking with the
  topological terminology.}

\begin{definition}
  If $\E$ is a sheaf of sets on $\X$ then we have a \emph{sheaf of
    automorphisms} $\saut(\E)$ on $\X$ which to every $U \in \X$
  assigns the set $\aut(\E|_U)$ of isomorphisms $\E|_U \to \E|_U$ of
  sheaves on $\X_{/U}$.
\end{definition}

\begin{remark}
  If $G$ is a topological group and $F$ a space then giving a left
  $G$-action on a space $F$ is equivalent to giving a continuous map
  $G \to \aut(F)$. Observe similarly that if $\G$ is a sheaf of groups
  on $\X$ and $\F$ a sheaf of sets on $\X$ then giving a left
  $\G$-action on $\F$ is equivalent to giving a morphism of sheaves
  $\G \to \saut(\E)$.
\end{remark}

\begin{definition}
  \label{bundle-sheaf}
  Let $\F$ be a sheaf of sets on $\X$ and $\G$ a sheaf of groups on
  $\X$.
  \begin{itemize}
  \item An \emph{$\F$-bundle} on $\X$ with is a sheaf of sets $\E$ on
    $\X$ which is locally isomorphic to $\F$, that is, there is a
    covering $C$ of $\X$ such that for all $U \in \X$ we have
    isomorphisms $\kappa_i \c \E|_{U_i} \to \F|_{U_i}$ for each $i \in
    C(U)$.
  \item Such data $(C,\kappa)$ is called a \emph{trivialisation} of
    the bundle. The \emph{transition morphisms} associated to a
    trivialisation are the automorphisms $\tau_{ij} \ce
    \kappa_j\kappa_i^{-1}$ of $\F|_{U_i \times_U V_j}$, for $V \to U$
    any morphism in $\X$ and $i \in C(U), j \in C(V)$.
  \item Suppose $\F$ is equipped with a left $\G$-action. Then an
    \emph{$(\F,\G)$-bundle}, or \emph{$\F$-bundle with structure group
      $\G$}, on $\X$ is data $(\E,C,\kappa)$, where $\E$ is an
    $\F$-bundle on $\X$ and $(C,\kappa)$ is a trivialisation of $\E$
    for which the transition morphisms $\tau_{ij}$ lie in the image of
    $\G(U_i \times_U V_j) \to \saut(\F)(U_i \times_U V_j)$.
  \item A \emph{principal $\G$-bundle} on $\X$ is a $(\G,\G)$-bundle,
    where $\G$ is a left $\G$-sheaf via left
    multiplication. Equivalently we may say a principal $\G$-bundle is
    an $(\F,\G)$-bundle where $\G$ acts simply transitively on $\F$
    (on each $U \in \X$).
  \end{itemize}
\end{definition}

Actually we could have left that last lemma out since
\pref{principal-equals-torsor} below tells us we've already defined
it!

\begin{lemma}
  \label{auto-left}
  Suppose $\X$ has a final object $U$. Let $\G$ be a sheaf of groups
  on $\X$. Then any morphism of right $\G$-sheaves $\tau \c \G \to \G$
  is given by left multiplication by some $g \in \G(U)$:
  \[
  \tau_V \c\ \G(V) \ni h \mapsto g|_V \cdot h \in \G(V), \quad V \in
  \X.
  \]
\end{lemma}

\begin{proof}
  Let $1_V \in \G(V)$ denote the unit for $V \in \X$. Let $g \ce
  \tau_U(1_U) \in \G(U)$. For any $V \in \X$ and $h \in
  \G(V)$, since $1_U|_V = 1_V$ and $\tau$ is a morphism of right
  $\G$-sheaves, we have
  \[
  \tau_V(h) = \tau_V(1_V) \cdot h = \tau_U(1_U)|_V \cdot h = g|_V
  \cdot h. \qedhere
  \]
\end{proof}

\begin{proposition}
  \label{principal-equals-torsor}
  Let $\G$ be a sheaf of groups on $\X$. A sheaf of sets $\E$ on $\X$
  is a principal $\G$-bundle if and only if it is a right $\G$-torsor.
\end{proposition}

\begin{proof}
  ($\Rightarrow$) Let $\E$ be a principal $\G$-bundle. Then for any $U
  \in \X$ we are given a covering $\{U_i \to U\}$ of $U$ with
  isomorphisms $\E|_{U_i} \iso \G|_{U_i}$. Since the transition
  morphisms on $\G|_{U_i \times_U U_j}$ are given by left
  multiplication, and left and right multiplication commute, the right
  actions of $\G(U_i)$ on $\E(U_i)$ via right multiplication will glue
  together to give a right action of $\G(U)$ on $\E(U)$. Moreover
  since for any morphism $V \to U$ in $\X$ the transition morphisms on
  $\G|_{U_i \times_U V_j}$ are given by left multiplication, the same
  commuting also implies these actions will define a consistent right
  $\G$-action on $\E$.\footnote{This is precisely analogous to
    \pref{principal-right-action}.} Thus $\E$ is a right $\G$-sheaf,
  and since $\E$ is by construction locally trivial it must be a
  $\G$-torsor by \pref{torsor-equiv}.

  ($\Leftarrow$) Suppose $\E$ is a right $\G$-torsor. By
  \pref{torsor-equiv} it is locally trivial: each $U \in \X$ has a
  covering $\{U_i \to U\}$ with isomorphisms of right $\G$-sheaves
  $\E|_{U_i} \iso \G|_{U_i}$. So $\E$ is a bundle sheaf with fibre
  $\G$. Then that $\E$ has structure group $\G$ is immediate from
  \pref{auto-left}, applied to $\X_{/U_i \times_U V_j}$ with final
  object $U_i \times_U V_j$ for all morphisms $V \to U$ in $\X$.
\end{proof}

%%%%%%%%%%%%%%%%%%%%%%%%%%%%%%%%%%%%%%%%%%%%%%%%%%%%%%%%%%%%%%%%%%%%%%

\section{Comparisons}

Now, we stated that our definition \pref{bundle-sheaf} of bundles and
structure groups \emph{generalises} the topological notion
\pref{fibre-bundle} of fibre bundles and structure groups. But we've
been proceeding on analogies alone. So though the idea has been hinted
at in \pref{bundle-example}, we haven't actually proved in what sense
this is a generalisation. That is our first goal in this section. To
set up this comparison we should define some categories.

\begin{definitions}
  \label{fibre-bundle-cat}
  Let $G$ be a topological group and $F$ a left $G$-space. We define a
  \emph{morphism of $(F,G)$-bundles on $X$} $(\pi,C,\kappa) \to
  (\pi',C',\kappa')$ to be a continous map $\lambda \c E \to E'$ over
  $X$ and a common refinement $\{V_j\}$ of $C = \{U_i\}$ and $C' =
  \{U'_{i'}\}$ such that for any $i,i',j$ such that $V_j \subseteq
  U_i$ and $V_j \subseteq U'_{i'}$, the map
  \[
  \begin{tikzcd}
    V_j \times F \ar{r}{\kappa_i^{-1}} &
    \pi^{-1}(V_j) \ar{r}{\lambda} &
    \pi'^{-1}(V_j) \ar{r}{\kappa'_{i'}} &
    V_j \times F
  \end{tikzcd}
  \]
  is an automorphism $\alpha_j$ of the form $(v,f) \mapsto (v,
  g(v)f)$, where $g$ is a map $V_j \to G$.

  This defines the \emph{category $\Bun_G(X;F)$ of $G$-bundles on $X$
    with fibre $F$}.
\end{definitions}
  
\begin{definitions}
  \label{bundle-sheaf-cat}
  Let $\G$ be a sheaf of groups on $\X$ and $\F$ a left
  $\G$-sheaf. Recall \pref{bundle-sheaf} that a $\G$-bundle on $\X$
  with fibre $\F$ is a bundle $\E$ on $\X$ with fibre $\F$ equipped
  with a trivialisation $\{(U_i \to U,\kappa_i)\}$ with transition
  morphisms in $\G$. We define a \emph{morphism of $\G$-bundles on
    $\X$ with fibre $\F$} to be a morphism $\lambda \c \E \to \E'$ of
  sheaves on $\X$ and for each $U \in \X$ common refinements $\{V_j
  \to U\}$ of $\{U_i \to U\}$ and $\{U'_{i'} \to U\}$ such that for
  any $i,i',j$ with morphisms $V_j \to U_i$ and $V_j \to U'_{i'}$, the
  morphism
  \[
  \begin{tikzcd}
    \F|_{V_j} \ar{r}{\kappa_i^{-1}} &
    \E|_{V_j} \ar{r}{\lambda} &
    \E'|_{V_j} \ar{r}{\kappa'_{i'}} &
    \F|_{V_j}
  \end{tikzcd}
  \]
  is an automorphism $\alpha_j$ given by the action of an element of
  $\G(V_j)$.

  This defines the \emph{category $\Bun_\G(\X;\F)$ of $\G$-bundles on
    $\X$ with fibre $\F$}. If $\X$ is the site of open sets in $X$
  then we write $\Bun_\G(X;\F)$ for $\Bun_\G(\X;\F)$.
\end{definitions}

\begin{remark}
  Note that $\Bun_G(X;F)$ and $\Bun_\G(\X;\F)$ are groupoids, i.e. all
  morphisms of $G$-bundles with fibre $F$ and morphisms of
  $\G$-bundles with fibre $\F$ are isomorphisms. These are the notions
  of isomorphism we will adopt for these objects.
\end{remark}

\begin{remark}
  It follows from \pref{auto-left} that if $\lambda \c \E \to \E'$ is
  a morphism of sheaves on $X$ with $\E,\E'$ both principal
  $\G$-bundles (i.e. $\G$-torsors), then $\lambda$ is a morphism in
  $\Bun_\G(X;\G)$ if and only if $\lambda$ is a morphism of right
  $\G$-sheaves.

  After proving an analogous (and also easy) lemma for principal
  $G$-bundles, we can also see that a continuous map $\lambda \c E \to
  E'$ over $X$ between principal $G$-bundles $E \to X$ and $E' \to X$
  is a morphism in $\Bun_G(X;G)$ if and only if it is $G$-equivariant.
\end{remark}

We now show that the topological $\Bun_G(X;F)$ is really a special
case of the sheaf-theoretic $\Bun_\G(X;\F)$.

\begin{definition}
  Suppose $\X$ has a final object $U$.
  \begin{itemize}
  \item A collection of coverings $\{V_j \to V\}$ for each $V \in \X$
    is called \emph{consistent} if for each $V \in \X$ the covering
    $\{V_j \to V\}$ is obtained by pulling back the given covering
    $\{U_i \to U\}$ via the unique morphism $V \to U$.
  \item A trivialisation $\{(V_j \to V,\kappa_j)\}$ of bundle $\E$ on
    $\X$ with fibre $\F$ is called \emph{consistent} if the underlying
    collection of coverings $\{V_j \to V\}$ is consistent and for each
    $V \in \X$ the isomorpisms $\kappa_j \c \E|_{V_j} \to \F|_{V_j}$
    are obtained by restricting the isomorphisms $\kappa_i \c
    \E|_{U_i} \to \F|_{U_i}$.
  \end{itemize}

\end{definition}

\begin{lemma}
  \label{iso-consistent}
  Let $\G$ be a sheaf of groups on $\X$ and $\F$ a left $\G$-sheaf. If
  $\X$ has a final object $U$, then every object of $\Bun_\G(\X;\F)$
  is isomorphic to an object whose trivialisation is consistent.
\end{lemma}

\begin{proof}
  Let $\E$ be a $\G$-bundle on $\X$ with fibre $\F$ and trivialisation
  $\{(V_j \to V,\kappa_j)\}$. There is also a consistent
  trivialisation of $\E$ given by pulling back the covering $\{(U_i
  \to U, \kappa_i)\}$ of $U$ to each $V \in \X$. A common refinement
  of the two is given by $\{U_i \times_U V_j\}$, and by definition
  \pref{bundle-sheaf} of $\E$ having structure group $\G$ the
  automorphism
  \[
  \begin{tikzcd}
    \F|_{U_i \times_U V_j} \ar{r}{\kappa_i^{-1}} &
    \E|_{U_i \times_U V_j} \ar{r}{\id} &
    \E'|_{U_i \times_U V_j} \ar{r}{\kappa_{j}} &
    \F|_{U_i \times_U V_j}
  \end{tikzcd}
  \]
  is given by the action of an element of $\G(U_i \times_U V_j)$. Thus
  $\id \c \E \to \E$ gives an isomorphism in $\Bun_\G(\X;\F)$, as
  desired.
\end{proof}

Recall \pref{map-sheaf}
that to a space $Y$ we may associate the sheaf $\u{Y}$ on $X$ of
continuous maps to $Y$.

\begin{proposition}
  Let $G$ be a topological group and $F$ a left $G$-space. Taking
  sheaves of sections defines an equivalence of categories
  \[
  \Bun_G(X;F) \iso \Bun_{\u{G}}(X;\u{F}).
  \]
\end{proposition}

\begin{proof}
  Before beginning, observe that the action of $G$ on $F$ certainly
  defines an action of $\u{G}$ on $\u{F}$, so the right-hand category
  is indeed defined.

  We first show that taking sheaves of sections in fact defines a
  functor $\Bun_G(X;F) \to \Bun_{\u{G}}(X;\u{F})$. Let $\pi \c E \to
  X$ be a $G$-bundle with fibre $F$ and trivialisation
  $\{(U_i,\kappa_i)\}$. Let $\E$ denote the sheaf of sections of
  $\pi$. A section of $U \times F \to U$ is simply a continuous map $U
  \to F$, so composition with the homeomorphisms $\kappa_i \c
  \pi^{-1}(U_i) \to U_i \times F$ over $U_i$ determine isomorphisms
  $\u{\kappa}_i \c \E|_{U_i} \to \u{F}|_{U_i}$. Then the resulting
  transition morphism $\u{t}_{ij} \ce \u{\kappa}_i\u{\kappa}_j^{-1}$
  is composition with the transition map $\kappa_i\kappa_j^{-1}$,
  which is of the form
  \[
  (U_i \cap U_j) \times F \to (U_i \cap U_j) \times F, \quad
  (u,f) \mapsto (u, t_{ij}(u)f)
  \]
  for some map $t_{ij} \c U_i \cap U_j \to G$ since $\pi$ has
  structure group $G$. Thus $\u{t}_{ij}$ is given by the action of
  $t_{ij} \in \u{G}(U_i \cap U_j)$. So indeed $\E$ equipped with the
  trivialisation $\{(U_i, \u{\kappa}_i)\}$---or rather the the
  consistent trivialisation induced by it---is a $\u{G}$-bundle with
  fibre $\u{F}$.

  This defines the functor on objects. For morphisms, a map $\lambda
  \c E \to E'$ over $X$ defines (again via composition) a morphism of
  sheaves $\u{\lambda} \c \E \to \E'$. To show that the common
  refinement and condition defining a morphism in $\Bun_G(X;F)$ imply
  the condition defining a morphism in $\Bun_{\u{G}}(X;\u{F})$, we
  just apply the argument used above to show that $\E$ has structure
  group $\u{G}$ if $\pi$ has structure group $G$.

  We now sketch why this actually gives an equivalence:
  \begin{itemize}
  \item \emph{Essential surjectivity}: By \pref{iso-consistent} it
    suffices to show a $\u{G}$-bundle $\E$ with fibre $\F$ with
    \emph{consistent} trivialisation is in the essential image. Let
    the consistent trivialisation be induced by a covering $\{U_i\}$
    of $X$ with transition morphisms given by the action of $t_{ij}
    \in \u{G}(U_i \cap U_j)$. We form a $G$-bundle $\pi \c E \to X$
    with fibre $F$ by gluing the trivial bundles $U_i \times F$ over
    each $U_i$ along the transition maps $t_{ij}$. (This will be
    possible because they come from transitions on $\E$ and hence
    satisfy the cocycle condition.)  It's clear that the sheaf of
    sections of $\pi$ will be isomorphic to $\E$.
  \item \emph{Full-faithfulness}: Observe that giving a morphism in
    $\Bun_G(X;F)$ is equivalent to giving the common refinement
    $\{V_j\}$ and automorphisms $\{\alpha_j\}$ in
    \pref{bundle-sheaf-cat} such that $\alpha_j$ and $\alpha_{j'}$
    agree on $V_j \cap V_{j'}$, since the $\alpha_j$ determine
    $\lambda$. And the analogous statement holds for morphisms in
    $\Bun_\G(X;\F)$ between bundles equipped with consistent
    trivialisations. In both categories the automorphisms $\alpha_j$
    determine and are determined by an element of $\u{G}(V_j)$, which
    implies full-faithfulness. \qedhere
  \end{itemize}
\end{proof}

So any work we do in studying $\Bun_\G(\X;\F)$ for general $\X,\G,\F$
will a fortiori give results for $\Bun_G(X;F)$ for general $X,G,F$. As
a first example we discuss ``change of fibre'' for bundles, which will
allow us to relate bundles with arbitrary fibre to principal
bundles. This is also known as the ``associated bundle'' construction.

\begin{definition}
  Let $\G$ be a sheaf of groups on $\X$ and $\F$ a $\G$-sheaf. The
  action of $\G$ on $\F$ is called \emph{faithful} if the induced
  morphism of sheaves $\G \to \saut(\F)$ is injective.
\end{definition}

\begin{proposition}
  Let $\G$ be a sheaf of groups on $\X$ and $\F,\F'$ be two left
  $\G$-sheaves. If the action of $\G$ on $\F$ is faithful, then there
  is a functor
  \[
  \Theta_{\F,\F'} \c \Bun_\G(\X;\F) \to \Bun_\G(\X;\F').
  \]
  If the action of $\G$ on $\F'$ is also faithful then
  $\Theta_{\F,\F'}$ and $\Theta_{\F',\F}$ are quasi-inverse
  equivalences.
\end{proposition}

\begin{proof}
  We first define $\Theta_{\F,\F'}$ on objects. Let $(\E,C,\kappa)$ be
  an $(\F,\G$)-bundle on $\X$, with transition morphisms
  $\tau_{U_i,V_j} \in \G(U_i \cap U_j)$.
\end{proof}

%%%%%%%%%%%%%%%%%%%%%%%%%%%%%%%%%%%%%%%%%%%%%%%%%%%%%%%%%%%%%%%%%%%%%%

\begin{appendices}
  \section{Internal groups and groupoids}

  \renewcommand{\C}{\mathcal{C}}
  \begin{notation}
    In this section, $\C$ is any category admitting finite limits,
    including an initial object $* \in \C$.
  \end{notation}

  \begin{definitions}
    An \emph{internal group} in $\C$ is an object $G \in \C$ together
    with morphisms
    \[
    \mu \c G \times G \to G \qquad
    \epsilon \c * \to G \qquad
    \iota \c G \to G
    \]
    which make the diagrams
    \[
    \begin{tikzcd}
      G \times G \times G \ar[r, "\id \times \mu"] \ar[d, "\mu \times
      \id"] & G \times G \ar[d, "\mu"] \\ G \times G \ar[r, "\mu"] &
      G
    \end{tikzcd}
    \quad
    \begin{tikzcd}
      G \ar[r, "\id \times \o\epsilon"] \ar[d, "\o\epsilon \times
      \id"] \ar[rd, "\id"] & G \times G \ar[d, "\mu"] \\ G \times G
      \ar[r, "\mu"] & G
    \end{tikzcd}
    \quad
    \begin{tikzcd}
      G \ar[r, "\id \times \iota"] \ar[d, "\iota \times \id"] \ar[rd,
      "\o\epsilon"] & G \times G \ar[d, "\mu"] \\ G \times G \ar[r,
      "\mu"] & G
    \end{tikzcd}
    \]
    commute, where $\o\epsilon$ denotes the morphism $G \to *
    \lblto{\epsilon} G$. (Of course we often leave the data
    $\mu,\epsilon,\iota$ implicit in our notation.) A \emph{morphism
      of internal groups} $(G,\mu_G,\epsilon_G,\iota_G) \to
    (H,\mu_H,\epsilon_H,\iota_H)$ in $\C$ is a morphism $\phi \c G \to
    H$ in $\C$ making the diagram
    \[
    \begin{tikzcd}
      G \times G \ar[r, "\mu_G"] \ar[d, "\phi \times \phi"] & G \ar[d,
      "\phi"] \\ H \times H \ar[r, "\mu_H"] & H
    \end{tikzcd}
    \]
    commute. This defines the \emph{category $\Grp(\C)$ of internal
      groups in $\C$}.
  \end{definitions}
  
  \renewcommand{\R}{\mathrm{R}}
  \renewcommand{\L}{\mathrm{L}}
  \begin{definitions}
    Let $(G,\mu,\epsilon,\iota)$ be an internal group in $\C$. A
    \emph{right $G$-action} on an object $E \in \C$ is a morphism
    $\alpha \c E \times G \to G$ making the diagrams
    \[
    \begin{tikzcd}
      E \times G \times G \ar[r, "\id \times \mu"] \ar[d, "\alpha
      \times \id"] & E \times G \ar[d, "\alpha"] \\ E \times G \ar[r,
      "\alpha"] & G
    \end{tikzcd}
    \qquad
    \begin{tikzcd}
      E \ar[r, "\id \times \o\epsilon"] \ar[dr, "\id"] & E \times G
      \ar[d, "\alpha"] \\ & E
    \end{tikzcd}
    \]
    commute, where $\o\epsilon$ denotes the morphism $E \to *
    \lblto{\epsilon} G$. An object of $\C$ equipped with a right
    $G$-action is called a \emph{right $G$-object} in $\C$. A
    \emph{morphism of right $G$-objects} $(E_1,\alpha_1) \to
    (E_1,\alpha_2)$ in $\C$ is a morphism $\phi \c E_1 \to E_2$ in
    $\C$ making the diagram
    \[
    \begin{tikzcd}
      E_1 \times G \ar[r, "\alpha_1"] \ar[d, "\phi \times \id"] & E_1
      \ar[d, "\phi"] \\ E_2 \times G \ar[r, "\mu_H"] & E_2
    \end{tikzcd}
    \]
    commute. This defines the \emph{category $\C_G^\R$ of right
      $G$-objects in $\C$}.

    We can analogously define \emph{left $G$-actions} and the
    \emph{category $\C_G^\L$ of left $G$-objects in $\C$}.
  \end{definitions}

  \begin{examples}
    The following table enumerates the basic examples of internal
    groups and actions which appear in these notes.

    \begin{table}[h]
      \begin{tabular}{|p{90pt}|p{90pt}|p{90pt}|}
        \hline
        $\C$ & $\Grp(\C)$ & $\C_G^\R$ / $\C_G^\L$ \\
        \hline
        sets & groups & right/left $G$-sets \\[5pt]
        topological spaces & topological groups & right/left
        $G$-spaces \\[5pt]
        sheaves of sets on a site $\X$ & sheaves of groups on $\X$ &
        right/left $G$-sheaves on $\X$ \\
        \hline
      \end{tabular}
    \end{table}
  \end{examples}

  \renewcommand{\Y}{\mathcal{Y}}
  \renewcommand{\Z}{\mathcal{Z}}
  \begin{definitions}
    An \emph{internal category} $\Y$ in $\C$ consists of objects
    $\Y_0, \Y_1 \in \C$ together with morphisms
    \[
    \sigma \c \Y_1 \to \Y_0 \qquad
    \tau \c \Y_1 \to \Y_0 \qquad
    \epsilon \c \Y_0 \to \Y_1 \qquad
    \kappa \c \Y_1 \times_{\Y_0} \Y_1 \to \Y_1
    \]
    which make the diagrams
    \[
    \begin{tikzcd}
      \Y_1 \underset{\Y_0}{\times} \Y_1 \ar[r, "\kappa"] \ar[d, "p_2"]
      & \Y_1 \ar[d, "\sigma"] \\ \Y_1 \ar[r, "\sigma"] & \Y_0
    \end{tikzcd}
    \quad
    \begin{tikzcd}
      \Y_1 \underset{\Y_0}{\times} \Y_1 \ar[r, "\kappa"] \ar[d, "p_1"]
      & \Y_1 \ar[d, "\tau"] \\ \Y_1 \ar[r, "\tau"] & \Y_0
    \end{tikzcd}
    \quad
    \begin{tikzcd}
      \Y_1 \underset{\Y_0}{\times} \Y_1 \underset{\Y_0}{\times} \Y_1
      \ar[r, "\id \times \kappa"] \ar[d, "\kappa \times \id"] & \Y_1
      \underset{\Y_0}{\times} \Y_1 \ar[d, "\kappa"] \\ \Y_1
      \underset{\Y_0}{\times} \Y_1 \ar[r, "\kappa"] & \Y_1
    \end{tikzcd}
    \]
    \[
    \begin{tikzcd}
      \Y_0 \ar[r, "\epsilon"] \ar[d, "\epsilon"] \ar[rd, "\id"] & \Y_1
      \ar[d, "\sigma"] \\ \Y_1 \ar[r, "\tau"] & \Y_0
    \end{tikzcd}
    \qquad
    \begin{tikzcd}[column sep = large]
      \Y_1 \ar[r, "(\epsilon\sigma) \times \id"] \ar[d, "\id \times
      (\epsilon\tau)"] \ar[rd, "\id"] & \Y_1 \times_{\Y_0}
      \Y_1 \ar[d, "\kappa"] \\ \Y_1 \times_{\Y_0} \Y_1
      \ar[r, "\kappa"] & \Y_1
    \end{tikzcd}
    \]
    commute. Here $\Y_1 \times_{\Y_0} \Y_1$ and $\Y_1 \times_{\Y_0}
    \Y_1 \times_{\Y_0} \Y_1$ denote the pullbacks
    \[
    \begin{tikzcd}
      \Y_1 \times_{\Y_0} \Y_1 \ar[r, "p_2"] \ar[d, "p_1"] & \Y_1
      \ar[d, "\tau"] \\ \Y_1 \ar[r, "\sigma"] & \Y_0
    \end{tikzcd}
    \qquad
    \begin{tikzcd}
      \Y_1 \times_{\Y_0} \Y_1 \times_{\Y_0} \Y_1 \ar[r, "q_2"] \ar[d,
      "q_1"] & \Y_1
      \times_{\Y_0} \Y_1 \ar[d, "\tau p_1"] \\ \Y_1 \ar[r, "\sigma"] &
      \Y_0
    \end{tikzcd}
    \]
    respectively. We say $\Y$ is an \emph{internal groupoid} in $\C$
    if there is also a morphism
    \[
    \iota \c \Y_1 \to \Y_1
    \]
    making the diagrams
    \[
    \begin{tikzcd}
      \Y_1 \ar[r, "\iota"] \ar[rd, "\sigma"] & \Y_1
      \ar[d, "\tau"] \\ & \Y_1
    \end{tikzcd}
    \quad
    \begin{tikzcd}
      \Y_1 \ar[r, "\iota"] \ar[rd, "\tau"] & \Y_1
      \ar[d, "\sigma"] \\ & \Y_1
    \end{tikzcd}
    \quad
    \begin{tikzcd}
      \Y_1 \ar[r, "\iota \times \id"] \ar[d, "\sigma"] & \Y_1
      \underset{\Y_0}{\times} \Y_1 \ar[d, "\kappa"] \\ \Y_0 \ar[r,
      "\epsilon"] & \Y_1
    \end{tikzcd}
    \quad
    \begin{tikzcd}
      \Y_1 \ar[r, "\id \times \iota"] \ar[d, "\tau"] & \Y_1
      \underset{\Y_0}{\times} \Y_1 \ar[d, "\kappa"] \\ \Y_0 \ar[r,
      "\epsilon"] & \Y_1
    \end{tikzcd}
    \]
    commute. A \emph{functor} $\phi \c \Y \to \Y'$ of internal
    categories in $\C$ consists of morphisms $\phi_0 \c \Y_0 \to
    \Y'_0$ and $\phi_1 \c \Y_1 \to \Y'_1$ in $\C$ making the diagrams
    \[
    \begin{tikzcd}
      \Y_1 \ar[r, "\phi_1"] \ar[d, "\sigma"] & \Y'_1 \ar[d, "\sigma'"]
      \\ \Y_0 \ar[r, "\phi_0"] & \Y'_0
    \end{tikzcd}
    \quad
    \begin{tikzcd}
      \Y_1 \ar[r, "\phi_1"] \ar[d, "\tau"] & \Y'_1 \ar[d, "\tau'"]
      \\ \Y_0 \ar[r, "\phi_0"] & \Y'_0
    \end{tikzcd}
    \quad
    \begin{tikzcd}
      \Y_0 \ar[r, "\phi_0"] \ar[d, "\epsilon"] & \Y'_0 \ar[d,
      "\epsilon'"] \\ \Y_1 \ar[r, "\phi_1"] & \Y'_1
    \end{tikzcd}
    \quad
    \begin{tikzcd}
      \Y_1 \underset{\Y_0}{\times} \Y_1 \ar[r, "\phi_1 \times \phi_1"]
      \ar[d,
      "\kappa"] & \Y'_1 \underset{\Y'_0}{\times} \Y'_1 \ar[d, "\kappa'"] \\
      \Y_1 \ar[r, "\phi_1"] & \Y'_1
    \end{tikzcd}
    \]
    commute. The functor $\phi$ is called \emph{fully faithful} if
    \[
    \begin{tikzcd}
      \Y_1 \ar[r, "\phi_1"] \ar[d, "\sigma \times \tau"] & \Y'_1
      \ar[d, "\sigma' \times \tau'"] \\ \Y_0 \times \Y_0 \ar[r, "\phi_0
      \times \phi_0"] & \Y'_0 \times \Y'_0
    \end{tikzcd}
    \]
    is a pullback square.
  \end{definitions}

  \begin{definitions}
    \label{anafunctor}
    Assume $\C$ is equipped with a collection of morphisms, called
    \emph{covers}, such that:
    \begin{enumerate}
    \item \label{cover-iso} any isomorphism is a cover;
    \item \label{cover-stab} compositions and pullbacks of covers are
      covers;
    \item \label{cover-eff} every cover is an effective epimorphism.
    \end{enumerate}
    Then a functor $\phi \c \Y \to \Y'$ of internal categories in $\C$
    is called \emph{covering} if the morphism $\phi_0 \c \Y_0 \to
    \Y'_0$ is a cover.

    An \emph{anafunctor} $\Y \to \Y'$ of internal categories in $\C$
    is a diagram of functors
    \[
    \begin{tikzcd}
      \Z \ar[r, "\phi'"] \ar[d, "\phi"] & \Y' \\ \Y &
    \end{tikzcd}
    \]
    where $\phi$ is fully faithful and covering. Using assumptions
    \pref{cover-iso} and \pref{cover-stab} we can define identity and
    composition of anafunctors, and hence obtain:
    \begin{itemize}
    \item \emph{the category $\Cat(\C)$ of internal categories and
        anafunctors in $\C$};
    \item \emph{the category $\Gpd(\C)$ of internal groupoids and
        anafunctors in $\C$}.
    \end{itemize}

  \end{definitions}

  \begin{remark}
    Assumption \pref{cover-eff} is used to define natural
    transformations of anafunctors, turning $\Cat(\C)$ and $\Gpd(\C)$
    into $2$-categories, but we do not use this in these notes.
  \end{remark}

  \begin{convention}
    When $\C$ is a topos, we always take the collection of
    (effective) epimorphisms as our covers in \pref{anafunctor}.
  \end{convention}

  %%%%%%%%%%%%%%%%%%%%%%%%%%%%%%%%%%%%%%%%%%%%%%%%%%%%%%%%%%%%%%%%%%%%

  \section{Glueing sheaves}

  \begin{notation}
    In this section, $\X$ is a site.
  \end{notation}

  \begin{lemma}
    \label{glueing}
    Suppose we are given:
    \begin{itemize}
    \item a covering $C$ of $\X$;
    \item for each $U \in \X$, sheaves $\E_i$ on $\X_{/U_i}$ for $i
      \in C(U)$;
    \item for each morphism $V \to U$ in $\X$, isomorphisms
      \[
      \Phi_{i,j} \c \E_i|_{U_i \times_U V_j} \to \E_j|_{U_i \times_U
        V_j} \quad \text{for } i \in C(U), j \in C(V)
      \]
      such that given another morphism $W \to V$, the diagram
      \begin{equation}
        \label{glue-cocycle}
        \begin{tikzcd}
          \E_i|_{U_i \times_U V_j \times_V W_k}
          \ar[rr, "\Phi_{i,k}"] \ar[dr, "\Phi_{i,j}"'] & &
          \E_k|_{U_i \times_U V_j \times_V W_k}. \\
          & \E_j|_{U_i \times_U V_j \times_V W_k}
          \ar[ur, "\Phi_{j,k}"'] &
        \end{tikzcd}
      \end{equation}
      is commutative for each $k \in C(W)$.
    \end{itemize}
    Then there is a sheaf $\E$ on $\X$ and for each $U \in \X$,
    isomorphisms $\Psi_i \c \E|_{U_i} \isoto \E_i$ for $i \in
    C(U)$ such that for any morphism $V \to U$ in $\X$, the diagram
    \[
    \begin{tikzcd}
      \E_i|_{U_i \times_U V_j}
      \ar[d, "\Psi_j^{-1}"] \ar[rr, "\Phi_{i,j}"] & &
      \E_j|_{U_i \times_U V_j} \\
      (\E|_{U_i})|_{U_i \times_U V_j} \ar[r, "\sim"] &
      \E|_{U_i \times_U V_j} \ar[r, "\sim"] &
      (\E|_{V_j})|_{U_i \times_U V_j} \ar[u, "\Psi_i"]
    \end{tikzcd}
    \]
    is commutative for $i \in C(U), j \in C(V)$.
  \end{lemma}

  \begin{proof}
    For $U \in \X$ we define
    \[
    \E(U) \ce \l\{ (s_i) \in \prod_{i \in C(U)} \E_i(U_i) :
    \Phi_{i,j}(s_i) = s_j \text{ for } i,j \in C(U) \r\}.
    \]
    For a morphism $V \to U$ we define the restriction map $\E(U) \to
    \E(V)$ by sending $(s_i)_{i \in C(U)}$ to $(t_j)_{j \in C(V)}$,
    where $t_j \in \E_j(V_j)$ is defined by
    \[
    t_j|_{U_i \cap_U V_j} = s_i|_{U_i \cap_U V_j} \quad\text{for } i
    \in C(U);
    \]
    note, $t_j$ exists and is unique by the hypothesis that
    \pref{glue-cocycle} commutes and by the sheaf condition on $\E_j$.
  \end{proof}
\end{appendices}

%%%%%%%%%%%%%%%%%%%%%%%%%%%%%%%%%%%%%%%%%%%%%%%%%%%%%%%%%%%%%%%%%%%%%%

\nocite{stacks}
\bibliographystyle{amsalpha}
\bibliography{refs}

\end{document}
