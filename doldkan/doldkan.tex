%%%%%%%%%%%%%%%%%%%%%%%%%%%%%%%%%%%%%%%%%%%%%%%%%%%%%%%%%%%%%%%%%%%%%%

\newcommand{\ob}{\oper{ob}}
\renewcommand{\hom}{\oper{hom}}
\newcommand{\id}{\oper{id}}
\newcommand{\im}{\oper{im}}
\newcommand{\op}{\oper{op}}

\newcommand{\Top}{\oper{Top}}
\newcommand{\Set}{\oper{Set}}
\newcommand{\Ab}{\oper{Ab}}
\newcommand{\Grp}{\oper{Grp}}
\newcommand{\Mod}{\oper{Mod}}
\newcommand{\Simplex}{\Delta}
\newcommand{\s}{\oper{s}}
\newcommand{\Ch}{\oper{Ch}}

\newcommand{\Sing}{\oper{Sing}}
\renewcommand{\H}{\mathrm{H}}

%%%%%%%%%%%%%%%%%%%%%%%%%%%%%%%%%%%%%%%%%%%%%%%%%%%%%%%%%%%%%%%%%%%%%%


\title{The Dold-Kan correspondence}
\author{Arpon Raksit}
\date{October 26, 2013}

\begin{document}
\maketitle
\thispagestyle{fancy}

%%%%%%%%%%%%%%%%%%%%%%%%%%%%%%%%%%%%%%%%%%%%%%%%%%%%%%%%%%%%%%%%%%%%%%

\renewcommand{\C}{\mathcal{C}}
\newcommand{\Top}{\mathrm{Top}}
\newcommand{\Set}{\mathrm{Set}}
\newcommand{\Ab}{\mathrm{Ab}}
\newcommand{\Grp}{\mathrm{Grp}}
\newcommand{\Mod}{\mathrm{Mod}}
\newcommand{\Simplex}{\Delta}
\newcommand{\s}{\mathrm{s}}
\newcommand{\Ch}{\mathrm{Ch}}
\newcommand{\Sing}{\operatorname{Sing}}

%%%%%%%%%%%%%%%%%%%%%%%%%%%%%%%%%%%%%%%%%%%%%%%%%%%%%%%%%%%%%%%%%%%%%%

\section{Introduction}

\begin{definition}
  A \textit{simplicial object} in a category $\C$ is a contravariant
  functor from the simplex category $\Simplex$ to $\C$. We denote the
  category $\C^{\Simplex^\op}$ of simplicial objects in $\C$ by
  $\s\C$. E.g., $\s\Set$ is the category of \textit{simplicial sets}
  and $\s\Ab$ is the category of \textit{simplicial abelian groups}.
\end{definition}

\begin{nothing}
  Recall we have a functor $\Sing \c \Top \to \s\Set$, sending $X
  \mapsto \hom_\Top(|\Delta^\bullet|, X)$. Lately we've been talking
  about $\Sing$ for two reasons:
  \begin{enumerate}
  \item It's a right adjoint to geometric realisation $|{-}| \c \s\Set
    \to \Top$.
  \item $\Sing(X)$ is a Kan complex for all $X \in \Top$---this was
    the start of the slogan ``Kan complexes are like spaces''.
  \end{enumerate}
  But this isn't the first place one sees $\Sing$, probably. Indeed,
  the singular homology functors are essentially defined by a
  composition
  \[
  \begin{tikzcd}[column sep = large]
    H_n(-;\Z) \c \Top \rar{\Sing} & \s\Set \rar{\Z} & \s\Ab \rar{\sum
      (-1)^i d_i} & \Ch_{\ge 0} \rar{H_n} & \Ab.
  \end{tikzcd}
  \]
  Here $\Z$ denotes the functor which takes free abelian groups
  level-wise, from which we get the \textit{singular chain complex} by
  letting the boundary map be given by $\del \ce \sum (-1)^i
  d_i$.
\end{nothing}

This was just to remind us that we've seen a natural functor $\s\Ab
\to \Ch$ relating simplicial abelian groups and chain complexes. We'll
look at it a bit more carefully in a second, and develop this
relationship much further.

%%%%%%%%%%%%%%%%%%%%%%%%%%%%%%%%%%%%%%%%%%%%%%%%%%%%%%%%%%%%%%%%%%%%%%

\renewcommand{\A}{\mathcal{A}}

\section{Stating the correspondence}

We fix $\A$ any abelian category---but we'll probably be imagining $\A
= \Ab$, or more generally $\A = \Mod(R)$ for some commutative
ring $R$.\footnote{In a couple of arguments, we won't be totally
  categorical and will use elements here to make things clearer. But
  it's not all that hard to see how one would get rid of them, and in
  any case we can appeal to Freyd-Mitchell.}

\begin{notation}
  We denote the category of nonnegatively graded chain complexes in
  $\A$ (and chain maps) by $\Ch_{\ge 0}(\A)$.
\end{notation}

Let's now make precise the $\del \ce \sum (-1)^i d_i$ business
with which we started this discussion.

\begin{definition}
  Let $A \in \s\A$ a simplicial object in $\A$ (e.g., a simplicial
  abelian group). We define the \textit{associated chain complex}
  $C(A) \in \Ch_{\ge 0}(\A)$ by
  \[
  \textstyle{C_n(A) \ce A_n, \quad \del_n \ce \sum_{i=0}^n
    (-1)^i d_i \c C_n(A) \to C_{n-1}(A)}
  \]
  for $n \ge 0$. Note that the simplicial identities clearly imply
  $\del^2 = 0$, so $C(A)$ is indeed a chain complex. Moreover, this
  evidently defines a functor $C \c \s\A \to \Ch_{\ge 0}(\A)$.
\end{definition}

This is perhaps the most natural---or familiar, at least---functor
$\s\A \to \Ch_{\ge 0}(\A)$, but it turns out not to be the cleanest to
use when discussing the relationship between the two categories. In
fact, we will want to use the following alternative.

\begin{definition}
  Again let $A \in \s\A$ a simplicial object in $\A$. We define the
  \textit{normalised chain complex} $N(A) \in \Ch_{\ge 0}(\A)$ by
  $N_0(A) \ce A_0$ and
  \[
  \textstyle{N_n(A) \ce
    \bigcap_{i=0}^{n-1} \ker(d_i) \subseteq A_n, \quad \del_n
    \ce (-1)^n d_n \c N_n(A) \to N_{n-1}(A)}
  \]
  for $n \ge 1$. The simplicial identities imply both that
  $d_n(N_n(A)) \subseteq N_{n-1}(A)$, assumed in the above definition
  of $\del_n$, and that $\del^2 = 0$. Again this gives a functor $N \c
  \s\A \to \Ch_{\ge 0}(\A)$.
\end{definition}

What is this unmotivated nonsense? Well, let's at least see an example.

\begin{example}
  \label{BG}
  Recall there is a functor $B \c \Ab \to \s\Ab$ which associates to an
  abelian group $G$ it's ``classifying space'' $BG$, which is
  constructed as the nerve of the groupoid with one object and
  morphisms $G$. In particular, we have
  \begin{itemize}
  \item $BG_n \simeq G^n$ for $n \ge 0$;
  \item the face map $d_i \c BG_n \to BG_{n-1}$ sends
    \[
    (g_1,\ldots,g_n) \mapsto (g_1,\ldots,g_{i-1}, g_i + g_{i+1},
    g_{i+2}, \ldots, g_n),
    \]
    where we have let $g_0 \ce 0$ and $g_{n+1} \ce 0$.
  \end{itemize}
  Let's compute the normalised chain complex $N(BG)$.
  \begin{itemize}
  \item Of course $N_0(BG) \simeq BG_0 \simeq 0$ is the trivial group.
  \item Thus $N_1(BG) = \ker(d_0 \c BG_1 \to BG_0) = BG_1 \simeq G$.
  \item Let $n \ge 2$. Let $g \ce (g_1, \ldots, g_n) \in
    BG_n$. Observe that by definition $g \in \ker(d_0) \implies g_2 =
    \cdots = g_n = 0$ and thus $g \in \ker(d_1) \implies g_1 + g_2 = 0
    \implies g_1 = 0$. So then $N_n(BG) \simeq 0$.
  \end{itemize}
  It follows also of course that the homology of $N(BG)$ is just $G$
  concentrated in degree $1$. (Perhaps this reminds you of the
  homotopy groups of $BG$! We will see why this is so in \textsection
  4.)
\end{example}

Ok that's an example, but maybe the definition of $N$ still seems
crazy. Have no fear, for $C$ and $N$ are intimately related! For
instance we can note immediately from the definitions that the natural
inclusion $N(A) \to C(A)$ is in fact a chain map for $A \in \s\A$. But
there's more!

\begin{definition}
  Let $A \in \s\A$. We define the \textit{degenerate subcomplex}
  $D(A)$ of $C(A)$ by $D_0(A) \ce 0$ and $D_n(A) \ce
  \sum_{i=0}^{n-1} \im(s_i)$ for $n \ge 1$. That is, $D(A)$ is
  generated by the images of the degeneracy maps. Note that by the
  simplicial identities
  \[
  \textstyle{\del s_j = \sum_{i=0}^n (-1)^i d_i s_j = \sum_{i=0}^{j-1}
    (-1)^i s_{j-1} d_i + \sum_{i=j+2}^n (-1)^i s_j d_i,}
  \]
  so $D(A)$ is indeed a subcomplex.
\end{definition}

\begin{proposition}
  \label{splitting}
  Let $A \in \s\A$. For $n \ge 0$ the natural map
  \[
  \phi \c N_n(A) \oplus D_n(A) \to A_n = C_n(A)
  \]
  induced by the inclusions is an isomorphism. Therefore we have a
  natural isomorphism $N(A) \simeq C(A)/D(A)$. Furthermore, the
  inclusion $N(A) \to C(A)$ is a natural chain homotopy equivalence.
\end{proposition}

\begin{proof}
  When $n=0$ we have by definition that $D_0(A) \simeq 0$ and $N_0(A)
  \inj A_0$ an isomorphism, so the claim is tautological. So fix $n
  \ge 1$. We first show $N_n(A) \cap D_n(A) \simeq 0$ (we use elements
  here). Suppose $y \in N_n(A) \cap D_n(A)$. We will show inductively
  that we can write $y = \sum_{j=i}^{n-1} s_j(x_j)$. The base case is
  the assumption $y \in D_n(A)$ and when we reach $i=n$ we will have
  $y = 0$, as desired. Assume the claim holds for $i$. We first reduce
  to the case that $d_ix_j = 0$ for $j > i$. This reduction is
  tautological if $n=1$. If $n \ge 2$ we have a canonical splitting
  $A_{n-1} \simeq \ker(d_i) \oplus \im(s_i)$ as a consequence of the
  simplicial identity $d_is_i = \id_{A_{n-2}}$. Writing $x_j
  = u_j + v_j$ in this splitting, with $v_j = s_iw_j$, we have by the
  simplicial identities that
  \[
  s_jx_j = s_ju_j + s_js_iw_j = s_ju_j + s_is_jw_j.
  \]
  Moving all the $s_is_jw_j$ into the $s_ix_i$ term proves the
  reduction step. Now since $d_ix_j = 0$ for $j > i$ and $y \in
  \ker(d_i) \subset N_n(A)$, the simplicial
  identities give
  \[
  0 = d_iy = d_is_ix_i + \sum_{j>i} d_is_jx_j = x_i + \sum_{j > i}
  s_jd_ix_j = x_i,
  \]
  which proves the induction step.

  Now we're just left to show that $\phi$ is surjective. We prove by
  downward induction on $0 \le j \le n-1$ that
  \[
  \textstyle{\im(\phi) \supseteq N_j \ce\bigcap_{i=0}^j
    \ker(d_i).}
  \]
  The base case $j = n-1$ is tautological and the final case $j = 0$
  will proved the desired splitting. Now consider the map $\psi
  \ce \id_{A_n} - s_jd_j \c A_n \to A_n$. Observe by the
  simplicial identities that
  \[
  d_j\psi = d_j - d_js_jd_j = d_j - d_j = 0 \quad\text{and}\quad
  d_i\psi = d_i - d_is_jd_j = d_i - s_{j-1}d_{j-1}d_i
  \]
  for $i < j$, implying that $\psi(N_j) \subseteq N_{j+1}$. By
  induction $\im(\phi) \supseteq N_{j+1}$, and since $\im(s_jd_j)
  \supseteq D_n(A)$ it follows that $\im(\phi) \supseteq N_j$.

  The proof of the last statement regarding the chain homotopy
  equivalence is omitted here; see \cite{goerssjardine} or
  \cite{weibel}.
\end{proof}

So there's the relationship between $C$ and $N$: the normalised chain
complex somehow tells us the nondegenerate information of the
associated chain complex, and moreover loses no homological
information. With these definitions in hand, we can now state our main
goal, the \textit{Dold-Kan correspondence}.

\begin{theorem}[Dold-Kan]
  \label{doldkan}
  The functor $N \c \s\A \to \Ch_{\ge 0}(\A)$ is an equivalence of
  categories.
\end{theorem}

%%%%%%%%%%%%%%%%%%%%%%%%%%%%%%%%%%%%%%%%%%%%%%%%%%%%%%%%%%%%%%%%%%%%%%

\section{Proving the correspondence}

\begin{definition}
  Let $C \in \Ch_{\ge 0}(\A)$. Define
  \[
  \Gamma_n(C) \ce \bigoplus_{[n] \surj [k]} C_k,
  \]
  where the direct sum is over all surjections $\sigma \c [n] \surj
  [k]$ in the category $\Simplex$.

  Let $\nu \c [m] \to [n]$ a morphism in $\Simplex$. Let $\tau \c [n]
  \surj [k]$ an indexing surjection. We can factor $\tau\nu$ as a
  composition $[m] \surj [j] \inj [k]$ of a surjection $\sigma$ and an
  injection $\iota$. Then we define a map\footnote{This definition is
    not so random: compare it with our definition of the normalised
    chain complex $N$.}
  \[
  C_k \to C_j \quad\text{as}\quad
  \begin{cases}
    \id_{C_n}, & \text{if } j = k; \\
    (-1)^n \del_n, & \text{if } j = k-1 \text{ and } \iota=d^k; \\
    0, & \text{otherwise}.
  \end{cases}
  \]
  Then composition with the inclusion $C_j \to \Gamma_m(C)$ of the
  factor with index $\sigma \c [m] \surj [j]$ gives a map $C_k \to
  \Gamma_m(C)$. Finally, the direct sum of these maps gives us an
  induced morphism $\nu^* \c \Gamma_n(C) \to \Gamma_m(C)$.

  Suppose $\mu \c [l] \to [m]$ is another morphism in
  $\Simplex$. Factoring $\sigma\mu$ as $\rho\theta \c [l] \surj [i]
  \inj [j]$, we have a commutative diagram
  \[
  \begin{tikzcd}
    {[l]} \rar{\mu} \ar[twoheadrightarrow]{d}{\rho} & {[m]} \rar{\nu}
    \ar[twoheadrightarrow]{d}{\sigma} & {[n]}
    \ar[twoheadrightarrow]{d}{\tau} \\ {[i]}
    \ar[hookrightarrow]{r}{\theta} & {[j]}
    \ar[hookrightarrow]{r}{\iota} & {[k]}.
  \end{tikzcd}
  \]
  It's easy to see then that $(\nu\mu)_* = \mu_*\nu_*$.

  It is also evident that a chain map $C \to D$ in $\Ch_{\ge 0}(\A)$
  gives rise to a simplicial map $\Gamma(C) \to \Gamma(D)$ in $\s\A$
  via factor-wise application. So finally we have constructed a
  functor
  \[
  \Gamma \c \Ch_{\ge 0}(\A) \to \s\A,
  \]
  which to each chain complex in $\A$ gives an \textit{associated
    simplicial object in $\A$}.
\end{definition}

To prove Dold-Kan, we're going to show that $\Gamma$ is a
quasi-inverse to $N \c \s\A \to \Ch_{\ge 0}(\A)$.

\begin{definition}
  Observe that there is
  a natural transformation $\Phi \c \Gamma \circ N \to \id_{\s\A}$
  defined by the maps
  \[
  \Phi_n(A) \c \Gamma_n(N(A)) = \bigoplus_{[n] \surj [k]} N_k(A) \to A_n
  \]
  for $A \in \s\A$ and $n \ge 0$, which restrict to the factor indexed
  by $\sigma \c [n] \surj [k]$ as the composition
  \[
  \begin{tikzcd}
    N_k(A) \ar[hookrightarrow]{r} & A_k \rar{\sigma^*} & A_n.
  \end{tikzcd}
  \]
  (It is clear this defines a simplicial map $\Gamma(N(A)) \to A$ which
  is natural in $A$.)
\end{definition}

\begin{lemma}
  \label{gamma-n}
  In fact $\Phi \c \Gamma \circ N \to \id_{\s\A}$, defined above, is a
  natural isomorphism.
\end{lemma}

\begin{proof}
  Fix $A \in \s\A$. We will prove by induction on $n \ge 0$ that
  \[
  \Phi_n(A) \c \Gamma_n(N(A)) \to A_n
  \]
  is an isomorphism, and then we will be done. Since the only
  surjection $[0] \surj [k]$ in $\Simplex$ is $\id_{[0]}$, and the
  inclusion $N_0(A) \inj A_0$ is an isomorphism, the base case $n=0$
  is tautological.

  First, surjectivity. Recall from (\ref{splitting}) that we have a
  splitting $A_n \simeq N_n(A) \oplus D_n(A)$. From the factor
  $\id_{[n]} \c [n] \surj [n]$ of $\Gamma_n(N(A))$ we $\im(\Phi_n(A))
  \supseteq N_n(A)$. By induction $\Phi_{n-1}(A)$ is surjective, so we
  must have $\im(\Phi_n(A)) \supseteq D_n(A)$ by definition. Hence
  $\Phi_n(A)$ is surjective.

  Next, injectivity (we use elements here). Assume we have $(x_\sigma)
  \in \ker(\Phi_n(A))$.  Fix $0 \le k < n$. Observe that to each
  surjection $\sigma \c [n] \surj [k]$ we can assign a section of
  $\sigma$,
  \[
  \nu_\sigma \c [k] \inj [n], \quad \nu_\sigma(i) \ce \max\{j \in
     [n] \mid \sigma(j) = i\}.
  \]
  If we have $\sigma, \sigma' \c [n] \surj [k]$, we say
  \[
  \sigma \le \sigma' \iff \nu_\sigma(i) \le \nu_{\sigma'}(i) \text{
    for all } i \in [k].
  \]
  In particular, $\sigma'\nu_\sigma = \id_{[k]} \implies \sigma \le
  \sigma'$. If there exists $\tau \c [n] \surj [k]$ such that $x_\tau
  \ne 0$, choose a \textit{maximal} such $\tau$ (with respect to the
  ordering just defined). By definition of the simplicial structure on
  $\Gamma(N(A))$, it follows that the component of
  $\nu_\tau^*(x_\sigma)$ in the factor of $\Gamma_k(N(A))$ indexed by
  $\id_{[k]}$ is precisely $x_\tau$. But then, by induction,
  \[
  (x_\sigma) \in \ker(\Phi_n(A)) \implies \nu_\tau^*(x_\sigma) \in
  \ker(\Phi_k(A)) \implies x_\tau = 0,
  \]
  contradiction.

  So we must have $x_\sigma = 0$ for all $\sigma \ne \id_{[n]}$. But
  the restriction of $\Phi_n(A)$ to the factor indexed by $\id_{[n]}$
  is just the inclusion $N_n(A) \inj A_n$. So then $x_{\id_{[n]}} = 0$
  too, and hence $\Phi_n(A)$ is injective.
\end{proof}

\begin{lemma}
  \label{n-gamma}
  Let $C \in \Ch_{\ge 0}(\A)$. For $n \ge 0$, the natural inclusion
  \[
  \Psi_n(C) \c N_n(\Gamma(C)) \inj C_n(\Gamma(C)) = \Gamma_n(C) =
  \bigoplus_{[n] \surj [k]} C_k
  \]
  has image the factor $C_n$ indexed by $\id_{[n]}$. This of course
  gives a natural isomorphism
  \[
  \Psi \c N \circ \Gamma \to \id_{\Ch_{\ge 0}(\A)}.
  \]
\end{lemma}

\begin{proof}
  By definition of the simplicial structure on $\Gamma(C)$ we have
  \[
  \textstyle{C_n \subseteq \bigcap_{i=0}^{n-1} \ker(d_i) =
    \im(\Psi_n(C))}.
  \]
  Now note for $\sigma \c [n] \surj [k]$ with $k < n$, we must have a
  factorisation of $\sigma$ as a composition
  \[
  \begin{tikzcd}
    {[n]} \ar[twoheadrightarrow]{r}{s^i} & {[n-1]}
    \ar[twoheadrightarrow]{r} & {[k]},
  \end{tikzcd}
  \]
  so it follows that the factor of $\Gamma_n(C)$ indexed by $\sigma$
  lies in the image of the degeneracies $D_n(\Gamma(C))$. Then we're
  done, since by (\ref{splitting}) we have a splitting
  \[
  \Gamma_n(C) \simeq N_n(\Gamma(C)) \oplus D_n(\Gamma(C)). \qedhere
  \]
\end{proof}

\begin{proof}[Proof of (\ref{doldkan})]
  The natural isomorphisms $\Phi \c \Gamma \circ N \to \id_{\s\A}$ and
  $\Psi \c N \circ \Gamma \to \id_{\Ch_{\ge 0}(\A)}$ of Lemmas
  \ref{gamma-n} and \ref{n-gamma} exhibit $N$ (and $\Gamma$) as an
  equivalence of categories, thus proving the Dold-Kan correspondence.
\end{proof}

%%%%%%%%%%%%%%%%%%%%%%%%%%%%%%%%%%%%%%%%%%%%%%%%%%%%%%%%%%%%%%%%%%%%%%

\section{Applying the correspondence}

\begin{notation}
  For the remainder we will fix some commutative ring $R$ and actually
  set $\A \ce \Mod(R)$.
\end{notation}

\begin{nothing}
  Recall that if $X \in \s\Set$ is \textit{fibrant} (i.e., a
  \textit{Kan complex}) with basepoint $v \in X_0$ then we can define
  its homotopy groups for $n \ge 0$,
  \[
  \pi_n(X,v) \ce [(\Delta^n, \del\Delta^n), (X,v)],
  \]
  that is, homotopy classes $[\alpha]$ of maps $\alpha \c \Delta^n \to
  X$ which fit in the commutative diagram
  \[
  \begin{tikzcd}
    \Delta^n \rar{\alpha} & X \\ \del\Delta^n \ar[hookrightarrow]{u}
    \rar & \Delta^0 \uar{v}.
  \end{tikzcd}
  \]
\end{nothing}

We now state a couple of facts about the homotopy groups.

\begin{lemma}
  \label{null-homotopic}
  Let $X \in \s\Set$ fibrant, $v \in X_0$, and $n \ge 0$. Denote all
  degeneracies of $v$ also by $v$, so that $[v] \in \pi_n(X,v)$ is the
  identity element. Let $[\alpha] \in \pi_n(X,v)$ represented by
  $\alpha \c \Delta^n \to X$. Then $[\alpha] = [v]$ if and only if
  there exists $\omega \in X_{n+1}$ such that
  \[
  d_{n+1}\omega = \alpha \quad\text{and}\quad d_i\omega = v\text{ for
  } 0 \le i \le n.
  \]
\end{lemma}

\begin{proof}
  Omitted.
\end{proof}

\begin{lemma}
  \label{s-rmod-homotopy}
  Let $G \in \s\Grp$ a simplical group and $v \in G_0$. Then the
  following hold.
  \begin{enumerate}
  \item The simplicial set underlying $G$ is fibrant, so the homotopy
    groups $\pi_n(G,v)$ are well-defined.
  \item The group structure on $G_n$ induces a natural group structure
    on $\pi_n(G,v)$ which agrees with the homotopy group structure.
  \end{enumerate}
\end{lemma}

\begin{proof}
  Omitted. See \cite{goerssjardine} for (1), and (2) can be proved by
  an Eckmann-Hilton argument.
\end{proof}

We can now state a relationship between the homotopy theory of $\s\A$
and that of $\Ch_{\ge 0}(\A)$.

\begin{proposition}
  \label{homotopy-homology}
  Let $A \in \s\A$ a simplicial $R$-module and $0 \in A_0$ the
  identity element. Then for $n \ge 0$ we have natural isomorphisms
  \[
  \pi_n(A,0) \simeq H_n(N(A)) \simeq H_n(C(A)).
  \]
  In particular, the functors $N$ and $\Gamma$ correspond
  quasi-isomorphisms in $\Ch_{\ge 0}(\A)$ with weak equivalences in
  $\s\A$.
\end{proposition}

\begin{proof}
  The second isomorphism is immediate from (\ref{splitting}), so we
  just prove the first. Let $x \in N_n(A) \subseteq A_n$. For $n \ge
  1$ we have by definition that
  \[
  \textstyle{x \in \ker(\del_n) \iff x \in \bigcap_{i=0}^n \ker(d_n)},
  \]
  which says precisely that $x$ represents an element $[x] \in
  \pi_n(A,0)$. And by (ref{null-homotopic}) we have for all $n \ge 0$
  that $[x] = 0 \in \pi_n(A,0)$ precisely when $x \in
  \im(\del_{n+1})$. Finally by (\ref{s-rmod-homotopy}) we can take the
  group structure on $\pi_n(A,0)$ to be the one induced by the
  $R$-module structure of $A_n$, whence the above discussion
  immediately gives us a (manifestly natural) isomorphism $\pi_n(A,0)
  \simeq H_n(N(A))$.
\end{proof}

\begin{remark}
  Recall that in (\ref{BG}) we computed the homology of $N(BG)$ for an
  abelian group $G$ to be $G$ concentrated in degree $1$. Then
  (\ref{homotopy-homology}) gives, as expected,
  \[
  \pi_n(BG,0) \simeq
  \begin{cases}
    G, & \text{if } n = 1; \\
    0, & \text{otherwise}.
  \end{cases}
  \]
  The same observation motivates more generally the following
  definition.
\end{remark}

\begin{definition}
  Let $A$ an $R$-module and $n \ge 0$. Define $A[n] \in \Ch_{\ge
    0}(\A)$ the complex with $A$ concentrated in degree $n$. Define
  the \textit{Eilenberg-Maclane space} $K(A,n) \ce \Gamma(A[n])
  \in \s\A$.
\end{definition}

\begin{remark}
  If one remembers that geometric realisation preserves homotopy
  groups, in the sense that $\pi_n(|X|,v) \simeq \pi_n(X,v)$ for
  fibrant $X \in \s\Set$, then $|K(A,n)|$ with the definition above
  really is an Eilenberg-Maclane space, in the ordinary topological
  sense.
\end{remark}

We'll end with an interesting consequence of
(\ref{homotopy-homology}). Assume that $R$ is a PID.

\begin{lemma}
  \label{quasi-homology}
  Let $C \in \Ch_{\ge 0}(\A)$. Then there is a quasi-isomorphism
  \[
  C \simeq \prod_{n \ge 0} H_n(C)[n].
  \]
\end{lemma}

\begin{proof}
  Let $n \ge 0$. Let $Z_n \ce \ker(\del_n)$ and $B_n \ce
  \im(\del_{n+1})$ the cycles and boundaries, respectively, in
  $C_n$. Let $F_n$ a free $R$-module with a surjection $F_n \surj
  Z_n$. Let
  \[
  F'_n \ce \ker(F_n \surj Z_n \surj H_n(C)),
  \]
  which is free since it is a submodule of a free $R$-module and $R$
  is a PID. There is then an induced map $F'_n \to B_n$, which lifts
  to a map $F'_n \to C_{n+1}$ since $F'_n$ is free and $\del_{n+1} \c
  C_{n+1} \to B_n$ is surjective. Define $F_n(C) \in \Ch_{\ge 0}(\A)$
  the complex with just $F_n$ in degree $n$ and $F'_n$ in degree
  $n+1$. The maps $F_n \to Z_n \inj C_n$ and $F'_n \to C_{n+1}$ give a
  chain map $F_n(C) \to C$ which induces an isomorphism $H_n(F_n(C))
  \simeq F_n/F'_n \simeq H_n(C)$. And the map $F_n \to H_n(C)$
  induces a quasi-isomorphism $F_n(C) \simeq H_n(C)[n]$.

  Since homology commutes with products, there are then
  quasi-isomorphisms
  \[
  C \simeq \prod_{n \ge 0} \F_n(C) \simeq \prod_{n \ge 0}
  H_n(C)[n]. \qedhere
  \]
\end{proof}

\begin{proposition}
  Let $A \in \s\A$ a simplicial $R$-module. Then $A$ is weakly
  equivalent to a product of Eilenberg-Maclane spaces:
  \[
  A \simeq \prod_{n \ge 0} K(\pi_n(A,0),n).
  \]
\end{proposition}

\begin{proof}
  In light of (\ref{homotopy-homology}) and the fact that $\Gamma$
  preserves products (as an equivalence of categories), this is
  immediate from applying $\Gamma$ to (\ref{quasi-homology}) with $C
  \ce N(A)$.
\end{proof}

%%%%%%%%%%%%%%%%%%%%%%%%%%%%%%%%%%%%%%%%%%%%%%%%%%%%%%%%%%%%%%%%%%%%%%

\nocite{goerssjardine, riehl-ssets, mathew-doldkan, weibel}
\bibliographystyle{amsalpha}
\bibliography{refs}

\end{document}
