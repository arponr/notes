%%%%%%%%%%%%%%%%%%%%%%%%%%%%%%%%%%%%%%%%%%%%%%%%%%%%%%%%%%%%%%%%%%%%%%

\newcommand{\ob}{\oper{ob}}
\renewcommand{\hom}{\oper{hom}}
\newcommand{\id}{\oper{id}}
\newcommand{\im}{\oper{im}}
\newcommand{\op}{\oper{op}}

\newcommand{\Top}{\oper{Top}}
\newcommand{\Set}{\oper{Set}}
\newcommand{\Ab}{\oper{Ab}}
\newcommand{\Grp}{\oper{Grp}}
\newcommand{\Mod}{\oper{Mod}}
\newcommand{\Simplex}{\Delta}
\newcommand{\s}{\oper{s}}
\newcommand{\Ch}{\oper{Ch}}

\newcommand{\Sing}{\oper{Sing}}
\renewcommand{\H}{\mathrm{H}}

%%%%%%%%%%%%%%%%%%%%%%%%%%%%%%%%%%%%%%%%%%%%%%%%%%%%%%%%%%%%%%%%%%%%%%

%%%%%%%%%%%%%%%%%%%%%%%%%%%%%%%%%%%%%%%%%%%%%%%%%%%%%%%%%%%%%%%%%%%%%%

\newcommand{\ob}{\oper{ob}}
\renewcommand{\hom}{\oper{hom}}
\newcommand{\id}{\oper{id}}
\newcommand{\im}{\oper{im}}
\newcommand{\op}{\oper{op}}

\newcommand{\Top}{\oper{Top}}
\newcommand{\Set}{\oper{Set}}
\newcommand{\Ab}{\oper{Ab}}
\newcommand{\Grp}{\oper{Grp}}
\newcommand{\Mod}{\oper{Mod}}
\newcommand{\Simplex}{\Delta}
\newcommand{\s}{\oper{s}}
\newcommand{\Ch}{\oper{Ch}}

\newcommand{\Sing}{\oper{Sing}}
\renewcommand{\H}{\mathrm{H}}

%%%%%%%%%%%%%%%%%%%%%%%%%%%%%%%%%%%%%%%%%%%%%%%%%%%%%%%%%%%%%%%%%%%%%%


\title{Exactness of functors}
\author{Arpon Raksit}
\date{November 17, 2013 (original); \today\ (last edit)}

\begin{document}
\maketitle
\thispagestyle{fancy}

%%%%%%%%%%%%%%%%%%%%%%%%%%%%%%%%%%%%%%%%%%%%%%%%%%%%%%%%%%%%%%%%%%%%%%

\renewcommand{\A}{\mathcal{A}}
\renewcommand{\B}{\mathcal{B}}

%%%%%%%%%%%%%%%%%%%%%%%%%%%%%%%%%%%%%%%%%%%%%%%%%%%%%%%%%%%%%%%%%%%%%%

\section{Products and coproducts}

\begin{definition}
  \label{biproduct}
  Let $\A$ a category and $\{A_i\}_{i \in I} \subseteq \ob(\A)$. A
  \emph{biproduct} of the $A_i$ is an object $\bigoplus A_i \in
  \ob(\A)$ equipped with morphisms
  \[
  \textstyle{A_i \to \bigoplus A_i \quad\text{and}\quad \bigoplus A_i
    \to A_i \quad \text{for }i \in I,}
  \]
  making $\bigoplus A_i$ simultaneously a product $\prod A_i$ and
  coproduct $\coprod A_i$. By the usual abuse of notation, to say we
  have such data attached to an object $B \in \ob(\A)$, we will simply
  write $B \simeq \bigoplus A_i$.
\end{definition}

\begin{lemma}
  \label{biproduct-formulae}
  Let $\A$ an additive category and $A,B,C \in \ob(\A)$. If $C$ is a
  product or coproduct of $A$ and $B$, then there exist morphisms
  \[
  i : A \to C, \quad j : B \to C, \quad p : C \to A, \quad q : C \to B
  \]
  satisfying the following identities:
  \[
  pi = \id_A, \quad qj = \id_B, \quad pj = 0, \quad qi = 0, \quad
  ip+jq = \id_C.
  \]
  In this case, $C$ equipped with $i,j,p,q$ is in fact a biproduct of
  $A$ and $B$.
\end{lemma}

\begin{proof}
  Suppose we have a coproduct $A \overset{i}{\longto} C
  \overset{j}{\longfrom} B$. The identity and zero morphisms induce $p
  : C \to A$ and $q : C \to B$ with
  \[
  pi = \id_A, \quad qj = \id_B, \quad pj = 0, \quad qi = 0.
  \]
  Consider the morphism $ip + jq : C \to C$. From the identities above
  and the universal property of the coproduct we get that
  \[
  (ip + jq)i = i \quad\text{and}\quad (ip+jq)j = j \implies ip+jq =
  \id_C.
  \]

  \medskip
  Now suppose we are given morphisms $f : D \to A$ and $g : D \to
  B$. By the above, we have a morphism $if + jg : D \to C$ with
  $p(if+jg) = f$ and $q(if+jg) = g$. Conversely, for any morphism $h :
  D \to C$ with $ph = f$ and $qh = g$ we must have
  \[
  h = (ip+jq)h = i(ph) + j(qh) = if + jg.
  \]
  So the induced morphism $h$ is indeed unique, implying $A
  \overset{p}{\longfrom} C \overset{q}{\longto} B$ is a product, and
  hence $C \simeq A \oplus B$.

  \medskip
  If we start with the product instead of the coproduct, we just
  dualise the above argument (which I learned from
  \cite{mo-additive-functor-direct-sum}).
\end{proof}


\begin{notation}
  In light of (\ref{biproduct-formulae}), in an additive category we
  will use finite products, coproducts, and biproducts
  interchangeably, and denote all of these with $\oplus$.
\end{notation}

\begin{lemma}
  \label{additive-functor-preserves-products}
  Let $\A$ and $\B$ additive categories and $F : \A \to \B$ an
  additive functor. Then $F$ preserves finite products.
\end{lemma}

\begin{proof}
  Let $A,B \in \ob(\A)$. Let $A \oplus B$ their biproduct, and the
  morphisms $i$, $j$, $p$, and $q$ as in
  (\ref{biproduct-formulae}). Since $F$ is an additive functor we get
  the same identities for $F(i)$, $F(j)$, $F(p)$, and $F(q)$.  But
  then the argument from (\ref{biproduct-formulae}) implies that $F(A
  \oplus B) \simeq F(A) \oplus F(B)$.
\end{proof}

\begin{lemma}
  \label{sum-factor}
  Let $\A$ an additive category and $A,B \in \ob(\A)$. Let $f,g : A
  \to B$ two morphisms. Then $f + g : A \to B$ can be factored as
  \[
  \begin{tikzcd}[column sep = large]
    A \rar{f \times g} & B \oplus B \rar{\id_B \amalg \id_B} & B,
  \end{tikzcd}
  \]
  where $f \times g$ is the morphism induced to the product by $f$ and
  $g$, and $\id_B \amalg \id_B$ is the morphism induced from the
  coproduct by $\id_B$ and $\id_B$.
\end{lemma}

\begin{proof}
  In the notation of (\ref{biproduct-formulae}), we have $f \times g =
  if + jg$ and $\id_B \amalg \id_B = p + q$, so
  \[
  (\id_B \amalg \id_B)(f \times g) = (p+q)(if+jg) = f+g. \qedhere
  \]
\end{proof}

\section{Left and right exact functors}

\begin{definition}
  \label{exactness}
  Let $\A$ and $\B$ categories. We say a functor $F : \A \to \B$ is:
  \begin{enumerate}
  \item \emph{left exact} if $F$ preserves finite limits;
  \item \emph{right exact} if $F$ preserves finite colimits;
  \item \emph{exact} if $F$ is both left and right exact.
  \end{enumerate}
\end{definition}

\begin{example}
  Left (resp. right) adjoints will be left (resp. right) exact since
  they preserve \emph{all} limits (resp. colimits). This will seem
  nicer when we actually characterise exactness of functors in terms
  of exactness of sequences below.
\end{example}

\begin{lemma}
  \label{exact-additive}
  Let $\A$ and $\B$ additive categories and $F : \A \to \B$ a
  functor. If $F$ is left or right exact then $F$ is additive.
\end{lemma}

\begin{proof}
  Let $A,B \in \ob(\A)$ and $f,g : A \to B$ two morphisms. Since if
  $F$ preserves limits or colimits then $F$ preserves biproducts, we
  have the commutative diagram
  \[
  \begin{tikzcd}[column sep = large]
    F(A) \rar{F(f \times g)} \ar{dr}[swap]{F(f) \times F(g)} & F(B
    \oplus B) \rar{F(\id_B \amalg \id_B)} & F(B), \\ & F(B) \oplus
    F(B) \ar{u}[anchor=center,rotate=90,yshift=3pt]{\sim}
    \ar{ur}[swap]{\id_{F(B)} \amalg \id_{F(B)}} &
  \end{tikzcd}
  \]
  notation as in (\ref{sum-factor}). Then we're done by
  (\ref{sum-factor}).
\end{proof}

\begin{lemma}
  \label{left-exact-equiv}
  Let $\A$ and $\B$ abelian categories and $F : \A \to \B$ an additive
  functor. The following are equivalent.
  \begin{enumerate}
  \item For every exact sequence $0 \to A \to B \to C$ in $\A$ the
    induced sequence $0 \to F(A) \to F(B) \to F(C)$ in $\B$ is exact.
  \item For every exact sequence $0 \to A \to B \to C \to 0$ in $\A$
    the induced sequence $0 \to F(A) \to F(B) \to F(C)$ in $\B$ is
    exact.
  \item $F$ preserves kernels.
  \item $F$ is left exact.
  \end{enumerate}
\end{lemma}

\begin{proof}
  (1 $\Rightarrow$ 2) Tautology.

  \medskip
  (2 $\Rightarrow$ 3) Let $\phi : A \to B$ a morphism in $\A$. Then we
  have short exact sequences
  \[
  0 \to \ker(\phi) \to A \to \im(\phi) \to 0, \quad 0 \to \im(\phi)
  \to B \to \coker(\phi) \to 0,
  \]
  which by assumption give rise to exact sequences
  \[
  0 \to F(\ker(\phi)) \to F(A) \to F(\im(\phi)), \quad 0 \to
  F(\im(\phi)) \to F(B) \to F(\coker(\phi)).
  \]
  Since $F(\phi)$ factors as $F(A) \to F(\im(\phi)) \to F(B)$, the
  exactness above implies
  \[
  \ker(F(\phi)) \simeq \ker(F(A) \to F(\im(\phi)) \simeq
  F(\ker(\phi)).
  \]

  \medskip
  (3 $\Rightarrow$ 4) By (\ref{additive-functor-preserves-products}),
  $F$ preserves finite products. And equalisers can be expressed as
  kernels in additive categories. Thus this is immediate from the fact
  that all (finite) limits can be expressed in terms of (finite)
  products and equalisers.

  \medskip
  (4 $\Rightarrow$ 1) To say $0 \to A \to B \to C$ is exact is just to
  say that $A \simeq \ker(B \to C)$. Thus this follows from the fact
  that kernels are finite limits.
\end{proof}

Of course we can dualise (\ref{left-exact-equiv}) to obtain the
analogous charactersiations of right exact functors.

%%%%%%%%%%%%%%%%%%%%%%%%%%%%%%%%%%%%%%%%%%%%%%%%%%%%%%%%%%%%%%%%%%%%%%

\nocite{weibel, stacks}
\bibliographystyle{amsplain}
\bibliography{refs}

\end{document}
