%%%%%%%%%%%%%%%%%%%%%%%%%%%%%%%%%%%%%%%%%%%%%%%%%%%%%%%%%%%%%%%%%%%%%%

\renewcommand{\A}{\mathbb{A}}
\renewcommand{\O}{\mathcal{O}}

\renewcommand{\a}{\mathfrak{a}}
\newcommand{\p}{\mathfrak{p}}
\newcommand{\q}{\mathfrak{q}}

\newcommand{\height}{\operatorname{ht}}

%%%%%%%%%%%%%%%%%%%%%%%%%%%%%%%%%%%%%%%%%%%%%%%%%%%%%%%%%%%%%%%%%%%%%%

%%%%%%%%%%%%%%%%%%%%%%%%%%%%%%%%%%%%%%%%%%%%%%%%%%%%%%%%%%%%%%%%%%%%%%

\renewcommand{\A}{\mathbb{A}}
\renewcommand{\O}{\mathcal{O}}

\renewcommand{\a}{\mathfrak{a}}
\newcommand{\p}{\mathfrak{p}}
\newcommand{\q}{\mathfrak{q}}

\newcommand{\height}{\operatorname{ht}}

%%%%%%%%%%%%%%%%%%%%%%%%%%%%%%%%%%%%%%%%%%%%%%%%%%%%%%%%%%%%%%%%%%%%%%


\title{Exactness of functors}
\author{Arpon Raksit}
\date{November 16, 2013 (original); \today\ (last edit)}

\begin{document}
\maketitle
\thispagestyle{fancy}

%%%%%%%%%%%%%%%%%%%%%%%%%%%%%%%%%%%%%%%%%%%%%%%%%%%%%%%%%%%%%%%%%%%%%%

\renewcommand{\A}{\mathcal{A}}
\renewcommand{\B}{\mathcal{B}}
\renewcommand{\C}{\mathcal{C}}

%%%%%%%%%%%%%%%%%%%%%%%%%%%%%%%%%%%%%%%%%%%%%%%%%%%%%%%%%%%%%%%%%%%%%%

\section{Products and coproducts}

\begin{definition}
  \label{biproduct}
  Let $\C$ a category and $\{A_i\}_{i \in I} \subseteq \ob(\C)$. A
  \emph{biproduct} of the $A_i$ is an object $\bigoplus A_i \in
  \ob(\C)$ and morphisms
  \[
  \textstyle{A_i \to \bigoplus A_i \quad\text{and}\quad \bigoplus A_i
    \to A_i \quad \text{for }i \in I,}
  \]
  making $\bigoplus A_i$ simultaneously a product $\prod A_i$ and
  coproduct $\coprod A_i$. By the usual abuse of notation, to say we
  have such data attached to an object $B \in \ob(\C)$, we will simply
  write $B \simeq \bigoplus A_i$.
\end{definition}

\begin{lemma}
  \label{coproduct-biproduct}
  Let $\A$ an additive category and $A,B \in \ob(\A)$. Then $A \amalg
  B \simeq A \times B$, i.e., finite products and coproducts agree,
  and thus are automatically biproducts.
\end{lemma}

\begin{proof}
  Suppose we have a coproduct $A \overset{i}{\longto} A \amalg B
  \overset{j}{\longfrom} B$. The identity and zero morphisms induce $p
  : A \amalg B \to A$ and $q : A \amalg B \to B$ with
  \begin{equation}
    \label{coproduct-formulae}
    pi = \id_A, \quad qj = \id_b, \quad pj = 0, \quad\text{and}\quad
    qi = 0.
  \end{equation}
  Consider the morphism $ip + jq : C \to C$. From the identities above
  and the universal property of the coproduct we get that
  \begin{equation}
    \label{biproduct-formula}
    (ip + jq)i = i \quad\text{and}\quad (ip+jq)j = j \implies ip+jq =
    \id_C.
  \end{equation}

  \medskip
  Now suppose we are given morphisms $f : D \to A$ and $g : D \to
  B$. By the above, we have a morphism $if + jg : D \to C$ with
  $p(if+jg) = f$ and $q(if+jg) = g$. Conversely, for any morphism $h :
  D \to C$ with $ph = f$ and $qh = g$ we must have
  \[
  h = (ip+jq)h = i(ph) + j(qh) = if + jg.
  \]
  So the induced morphism $h$ is indeed unique, implying $A \amalg B
  \simeq A \times B$.

  \medskip
  If we start with the product instead of the coproduct, we just
  dualise the above argument (which I learned from
  \cite{mo-additive-functor-direct-sum}).
\end{proof}

\begin{notation}
  In light of (\ref{coproduct-biproduct}), in an additive category we
  will use finite products, coproducts, and biproducts
  interchangeably, and denote all of these with $\oplus$.
\end{notation}

\begin{lemma}
  \label{additive-functor-preserves-products}
  Let $\A$ and $\B$ additive categories and $F : \A \to \B$ an
  additive functor. Then $F$ preserves finite products.
\end{lemma}

\begin{proof}
  This is a consequence of the proof of
  (\ref{coproduct-biproduct}). Let $A,B \in \ob(\A)$. Let $A \oplus B$
  their biproduct, and the morphisms $i,j,p,q$ as in
  (\ref{coproduct-formulae}, \ref{biproduct-formula}). Since $F$ is a
  functor, (\ref{coproduct-formulae}) implies
  \[
  F(p)F(i) = \id_{F(A)}, \quad F(q)F(j) = \id_{F(b)}, \quad F(p)F(j) =
  0, \quad F(q)F(i) = 0,
  \]
  and since $F$ is additive, (\ref{biproduct-formula}) implies
  $F(i)F(p) + F(j)F(q) = \id_{F(A \oplus B)}$. But then the argument
  from (\ref{coproduct-biproduct}) implies that $F(A \oplus B) \simeq
  F(A) \oplus F(B)$.
\end{proof}

\section{\{Left, right\} exact functors}

\begin{definition}
  \label{exactness}
  Let $\A$ and $\B$ abelian categories and $F : \A \to \B$ an
  additive functor. If for every exact sequence $0 \to A \to B \to C
  \to 0$ in $\A$, the induced sequence in $\B$
  \begin{itemize}
  \item $0 \to F(A) \to F(B) \to F(C)$ is exact, we say $F$ is
    \emph{left exact};
  \item $F(A) \to F(B) \to F(C) \to 0$ is exact, we say $F$ is
    \emph{right exact};
  \item $0 \to F(A) \to F(B) \to F(C) \to 0$ is exact, we say $F$ is
    \emph{exact}.
  \end{itemize}
  Note $F$ is exact if and only if it is left exact and right exact.
\end{definition}

\begin{lemma}
  \label{left-exact-equiv}
  Let $\A$ and $\B$ abelian categories and $F : \A \to \B$ an additive
  functor. The following are equivalent.
  \begin{enumerate}
  \item $F$ is left exact.
  \item $F$ preserves kernels.
  \item $F$ preserves finite limits.
  \item For every exact sequence $0 \to A \to B \to C$ in $\A$ the
    induced sequence $0 \to F(A) \to F(B) \to F(C)$ in $\B$ is exact.
  \end{enumerate}
\end{lemma}

\begin{proof}
  $(1 \Rightarrow 2)$ Assume $F$ is left exact. Let $\phi : A \to B$ a
  morphism in $\A$. Then we have short exact sequences
  \[
  0 \to \ker(\phi) \to A \to \im(\phi) \to 0, \quad 0 \to \im(\phi)
  \to B \to \coker(\phi) \to 0,
  \]
  giving rise to exact sequences
  \[
  0 \to F(\ker(\phi)) \to F(A) \to F(\im(\phi)), \quad 0 \to
  F(\im(\phi)) \to F(B) \to F(\coker(\phi)).
  \]
  Since $F(\phi)$ factors as $F(A) \to F(\im(\phi)) \to F(B)$, the
  exactness above implies
  \[
  \ker(F(\phi)) \simeq \ker(F(A) \to F(\im(\phi)) \simeq
  F(\ker(\phi)).
  \]

  \medskip
  $(2 \Rightarrow 3)$ By (\ref{additive-functor-preserves-products}),
  $F$ preserves finite products. And equalisers can be expressed as
  kernels in additive categories. Thus this is immediate from the fact
  that all (finite) limits can be expressed in terms of (finite)
  products and equalisers.

  \medskip
  $(3 \Rightarrow 4)$ To say $0 \to A \to B \to C$ is exact is just to
  say that $A \simeq \ker(B \to C)$. Thus this follows from the fact
  that kernels are finite limits.

  \medskip
  $(4 \Rightarrow 1)$ Tautology.
\end{proof}

Of course we can dualise (\ref{left-exact-equiv}) to obtain the
analogous charactersiations of right exact functors.

\begin{lemma}
  \label{adjoint-exact}
  Let $\A$ and $\B$ abelian categories and $F : \A \to \B$ an additive
  functor. If $F$ is a left (resp. right) adjoint then $F$ is left
  (resp. right) exact.
\end{lemma}

\begin{proof}
  Left (resp. right) adjoints preserve \emph{all} limits
  (resp. colimits), so this follows from (\ref{left-exact-equiv}) and
  its dual.
\end{proof}

\end{situation}




%%%%%%%%%%%%%%%%%%%%%%%%%%%%%%%%%%%%%%%%%%%%%%%%%%%%%%%%%%%%%%%%%%%%%%

\nocite{weibel}
\bibliographystyle{amsplain}
\bibliography{refs}

\end{document}
