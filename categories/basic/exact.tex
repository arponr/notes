%%%%%%%%%%%%%%%%%%%%%%%%%%%%%%%%%%%%%%%%%%%%%%%%%%%%%%%%%%%%%%%%%%%%%%

\renewcommand{\A}{\mathbb{A}}
\renewcommand{\O}{\mathcal{O}}

\renewcommand{\a}{\mathfrak{a}}
\newcommand{\p}{\mathfrak{p}}
\newcommand{\q}{\mathfrak{q}}

\newcommand{\height}{\operatorname{ht}}

%%%%%%%%%%%%%%%%%%%%%%%%%%%%%%%%%%%%%%%%%%%%%%%%%%%%%%%%%%%%%%%%%%%%%%

%%%%%%%%%%%%%%%%%%%%%%%%%%%%%%%%%%%%%%%%%%%%%%%%%%%%%%%%%%%%%%%%%%%%%%

\renewcommand{\A}{\mathbb{A}}
\renewcommand{\O}{\mathcal{O}}

\renewcommand{\a}{\mathfrak{a}}
\newcommand{\p}{\mathfrak{p}}
\newcommand{\q}{\mathfrak{q}}

\newcommand{\height}{\operatorname{ht}}

%%%%%%%%%%%%%%%%%%%%%%%%%%%%%%%%%%%%%%%%%%%%%%%%%%%%%%%%%%%%%%%%%%%%%%


\title{Exactness of functors}
\author{Arpon Raksit}
\date{November 16, 2013 (original); \today\ (last edit)}

\begin{document}
\maketitle
\thispagestyle{fancy}

%%%%%%%%%%%%%%%%%%%%%%%%%%%%%%%%%%%%%%%%%%%%%%%%%%%%%%%%%%%%%%%%%%%%%%

\renewcommand{\A}{\mathcal{A}}
\renewcommand{\B}{\mathcal{B}}
\renewcommand{\C}{\mathcal{C}}

%%%%%%%%%%%%%%%%%%%%%%%%%%%%%%%%%%%%%%%%%%%%%%%%%%%%%%%%%%%%%%%%%%%%%%

\section{Products and coproducts}

\begin{definition}
  \label{biproduct}
  Let $\C$ a category and $\{A_i\}_{i \in I} \subseteq \ob(\C)$. A
  \textit{biproduct} of the $A_i$ is an object $\bigoplus A_i \in
  \ob(\C)$ and morphisms
  \[
  \textstyle{A_i \to \bigoplus A_i \quad\text{and}\quad \bigoplus A_i
    \to A_i \quad \text{for }i \in I,}
  \]
  making $\bigoplus A_i$ simultaneously a product $\prod A_i$ and
  coproduct $\coprod A_i$. By the usual abuse of notation, to say we
  have such data attached to an object $B \in \ob(\C)$, we will simply
  write $B \simeq \bigoplus A_i$.
\end{definition}

\begin{lemma}
  \label{coproduct-biproduct}
  Let $\A$ an additive category and $A,B \in \ob(\A)$. Then $A \amalg
  B \simeq A \times B$, i.e., finite products and coproducts agree,
  and thus are automatically biproducts.
\end{lemma}

\begin{proof}
  Suppose we have a coproduct $A \overset{i}{\longto} A \amalg B
  \overset{j}{\longfrom} B$. The identity and zero morphisms induce $p
  : A \amalg B \to A$ and $q : A \amalg B \to B$ with
  \begin{equation}
    \label{coproduct-formulae}
    pi = \id_A, \quad qj = \id_b, \quad pj = 0, \quad\text{and}\quad
    qi = 0.
  \end{equation}
  Consider the morphism $ip + jq : C \to C$. From the identities above
  we get that $(ip + jq)i = i$ and $(ip+jq)j = j$. By the universal
  property of the coproduct it follows that $ip+jq = \id_C$.

  Now suppose we are given morphisms $f : D \to A$ and $g : D \to
  B$. By the above, we have a morphism $if + jg : D \to C$ with
  $p(if+jg) = f$ and $q(if+jg) = g$. Conversely, for any morphism $h :
  D \to C$ with $ph = f$ and $qh = g$ we must have
  \[
  h = (ip+jq)h = i(ph) + j(qh) = if + jg.
  \]
  So the induced morphism $h$ is indeed unique, implying $A \amalg B
  \simeq A \times B$.

  If we start with the product instead of the coproduct, we just
  dualise the above argument (which I learned from
  \cite{mo-additive-functor-direct-sum}).
\end{proof}

\begin{notation}
  In light of (\ref{coproduct-biproduct}), in an additive category we
  will denote finite products, coproducts, and biproducts all with the
  symbol $\oplus$.
\end{notation}

\begin{lemma}
  \label{additive-functor-preserves-products}
  Let $\A$ and $\B$ additive categories. Let $F : \A \to \B$ an
  additive functor.\ref{coproduct-formulae}
\end{lemma}

\section{Left- and right-exact functors}


\end{situation}




%%%%%%%%%%%%%%%%%%%%%%%%%%%%%%%%%%%%%%%%%%%%%%%%%%%%%%%%%%%%%%%%%%%%%%

\nocite{weibel}
\bibliographystyle{amsplain}
\bibliography{refs}

\end{document}
