%%%%%%%%%%%%%%%%%%%%%%%%%%%%%%%%%%%%%%%%%%%%%%%%%%%%%%%%%%%%%%%%%%%%%%

\newcommand{\ob}{\oper{ob}}
\renewcommand{\hom}{\oper{hom}}
\newcommand{\id}{\oper{id}}
\newcommand{\im}{\oper{im}}
\newcommand{\op}{\oper{op}}

\newcommand{\Top}{\oper{Top}}
\newcommand{\Set}{\oper{Set}}
\newcommand{\Ab}{\oper{Ab}}
\newcommand{\Grp}{\oper{Grp}}
\newcommand{\Mod}{\oper{Mod}}
\newcommand{\Simplex}{\Delta}
\newcommand{\s}{\oper{s}}
\newcommand{\Ch}{\oper{Ch}}

\newcommand{\Sing}{\oper{Sing}}
\renewcommand{\H}{\mathrm{H}}

%%%%%%%%%%%%%%%%%%%%%%%%%%%%%%%%%%%%%%%%%%%%%%%%%%%%%%%%%%%%%%%%%%%%%%

%%%%%%%%%%%%%%%%%%%%%%%%%%%%%%%%%%%%%%%%%%%%%%%%%%%%%%%%%%%%%%%%%%%%%%

\newcommand{\ob}{\oper{ob}}
\renewcommand{\hom}{\oper{hom}}
\newcommand{\id}{\oper{id}}
\newcommand{\im}{\oper{im}}
\newcommand{\op}{\oper{op}}

\newcommand{\Top}{\oper{Top}}
\newcommand{\Set}{\oper{Set}}
\newcommand{\Ab}{\oper{Ab}}
\newcommand{\Grp}{\oper{Grp}}
\newcommand{\Mod}{\oper{Mod}}
\newcommand{\Simplex}{\Delta}
\newcommand{\s}{\oper{s}}
\newcommand{\Ch}{\oper{Ch}}

\newcommand{\Sing}{\oper{Sing}}
\renewcommand{\H}{\mathrm{H}}

%%%%%%%%%%%%%%%%%%%%%%%%%%%%%%%%%%%%%%%%%%%%%%%%%%%%%%%%%%%%%%%%%%%%%%


\title{Defining the cotangent complex}
\author{Arpon Raksit}
\date{December 16, 2013 (original); \today\ (last edit)}

\begin{document}
\maketitle
\thispagestyle{fancy}

\numberwithin{equation}{subsection}

%%%%%%%%%%%%%%%%%%%%%%%%%%%%%%%%%%%%%%%%%%%%%%%%%%%%%%%%%%%%%%%%%%%%%%

\renewcommand{\C}{\mathcal{C}}

\section*{Opening nonsense}

I've recently learned a bit about the general framework of model
categories. It's sort of amazing, in the sense that seemingly a lot of
the basic homotopy theory of spaces which I've seen can be axiomatised
in a totally formal and categorical way. This includes something like
Whitehead's theorem (and its converse)---that a map of CW complexes is
a homotopy equivalence if and only if it's a weak homotopy
equivalence---which I never would have thought of as a categorical
fact.

Anyway, this axiomatisation and generalisation is beautiful and all,
but of course one hopes that it leads to new homotopy theory, in
addition to providing a nice framework for existing homotopy theory to
live in. Obviously it did. My goal in writing these notes was to learn
about one of the earliest examples of this, the development of a
(co)homology theory of commutative algebras by Andr\'{e}
\cite{andre-cotangent} and Quillen \cite{quillen-cotangent}.

%%%%%%%%%%%%%%%%%%%%%%%%%%%%%%%%%%%%%%%%%%%%%%%%%%%%%%%%%%%%%%%%%%%%%%

\newcommand{\ab}{\operatorname{ab}}

\section{Abelianisation and K\"ahler differentials}

The goal of this section is to motivate---on the level of analogy at
least, following \cite{goerss-modelsimplicial}---the content of
\textsection 2 and the construction of the cotangent complex in
\textsection 3.

\subsection{Singular homology}

We first ask: how can we view singular homology in a sort of
homotopyish way? Perhaps the most immediate answer is the
following. Suppose $X$ is a topological space. We can choose a CW
approximation, that is, a CW complex $Y$ and a weak equivalence $Y \to
X$. Let $\ab(Y)$ denote the free topological abelian group on
$Y$. Then one version of the Dold-Thom theorem \cite{mccord-doldthom}
states that
\[
\pi_*(\ab(Y)) \simeq \H_*(Y) \simeq \H_*(X).
\]
Thus we should view $\ab(Y)$ as the object which gives, via its
homotopy, singular homology. Note that to apply Dold-Thom we had to
choose a CW approximation to $X$ before applying $\ab$. From the point
of view of model categories, this is precisely choosing a cofibrant
replacement (in the standard model structure on topological
spaces). So finally we should view singular homology as the derived
functor of $\ab$.

Another answer to the question goes through simplicial sets rather
than topological spaces. Suppose $K \in \s\Set$, perhaps thinking $K =
\Sing(X)$ for some topological space $X$. Let $\ab(K) \coloneqq \Z
K\in \s\Mod(\Z)$ the free simplicial abelian group on $K$.  Then the
Dold-Kan correspondence\footnote{Many thanks to Albrecht Dold in this
  section!} tells us that
\[
\pi_*(\ab(K)) \simeq \H_*(K) \simeq \H_*(X).
\]
So again we view $\ab(K)$ as giving homology via its homotopy. Here we
did not have to choose a cofibrant replacement, but this is simply
because all $K \in \s\Set$ are cofibrant!

These two homotopic points of view on singular homology motivate the
following general philosophy due to Quillen.

%%%%%%%%%%%%%%%%%%%%%%%%%%%%%%%%%%%%%%%%%%%%%%%%%%%%%%%%%%%%%%%%%%%%%%

\section{Simplicial model categories}

Our first step will be to give the category of simplicial commutative
algebras over a ring $R$ the structure of a simplicial model
category. (Technically, we will only need the model category structure
to define the cotangent complex and Andr\'e-Quillen (co)homology, but
let's just be thorough about things.) We will do this much more
generally, following \cite[Ch. II]{goerssjardine}.

\subsection{The simplicial structure}

We first recall what we mean by a simplicial category in this context,
and how to recognise that we have one.

\begin{notation}
  If $\C$ is a category then we will denote by $\s\C$ the category of
  simplicial objects in $\C$, i.e., functors $\Delta^\op \to \C$ where
  $\Delta$ is the ordinal number category.
\end{notation}

\begin{definition}
  \label{simplcat}
  A \emph{simplicial category} is a category $\C$ equipped with
  functors
  \[
  -\otimes- : \C \times \s\Set \to \C, \quad \map_l : \C^\op \times \C
  \to \s\Set, \quad \map_r : \s\Set^\op \times \C \to \C
  \]
  and the following isomorphisms, each of which is natural is all of
  its variables:
  \begin{enumerate}
  \item \label{simplcat-twovaradj} $\hom_{\s\Set}(K, \map_l(A,B)) \simeq
    \hom_\C(A \otimes K, B) \simeq \hom_\C(A, \map_r(K,B))$,
  \item \label{simplcat-assoc} $A \otimes (K \times L) \simeq (A
    \otimes K) \otimes L$,
  \item \label{simplcat-agree} $\map_l(A,B)_0 \simeq \hom_\C(A,B)$.
  \end{enumerate}
\end{definition}

\begin{lemma}
  Let $\C$ be a category equipped with a functor $-\otimes- : \C
  \times \s\Set \to \C$ satisfying the following:
  \begin{enumerate}
  \item the functor $- \otimes K : \C \to \C$ has a right adjoint
    $\map_r(K,-) : \C \to \C$ for each $K \in \s\Set$,
  \item the functor $A \otimes - : \s\Set \to \C$ preserves colimits
    for each $A \in \C$,
  \item $A \otimes * \simeq A$ naturally in $A \in \C$,
  \item the natural isomorphism \ref{simplcat-assoc}.
  \end{enumerate}
  Define $\map_l : \C^\op \times \C \to \s\Set$ by $\map_l(A,B)_n
  \coloneqq \hom_\C(A \otimes \Delta^n, B)$ for $n \ge 0$. Then the
  above data gives $\C$ the structure of a simplicial category.
\end{lemma}

\begin{proof}
  Omitted---this is an exercise in adjunctions.
\end{proof}

\begin{construction}
  \label{coprod-tensor}
  Let $\C$ be a category with coproducts. We define a functor $-
  \otimes - : \s\C \times \s\Set \to \s\C$ as follows. Let $A \in
  \s\C$ and $K \in \s\Set$. For $n \ge 0$ define
  \[
  (A \otimes K)_n \coloneqq \coprod_{k \in K_n} A_n.
  \]
  For $\phi : [m] \to [n]$ a morphism in the category $\Delta$ define
  the induced morphism $\phi^* : (A \otimes K)_n \to (A \otimes K)_m$
  such that the following commutes for $k \in K_n$:
  \[
  \begin{tikzcd}
    A_n \rar{\phi^*} \dar{\incl_k} & A_m \dar{\incl_{\phi^*(k)}}
    \\ \coprod_{l \in K_n} A_n \rar{\phi^*} &\coprod_{l \in K_m} A_m.
  \end{tikzcd}
  \]
\end{construction}

\begin{examples}
  If $\C = \Set$ then we simply recover $A \otimes K = A \times K$. If
  $\C = \Set_*$ (pointed sets) then we get $A \otimes K = A \times K /
  * \times K$, which is just the smash product $A \wedge K_+$ with
  $K_+$, denoting $K$ with an added disjoint basepoint. Thus the
  construction of (\ref{coprod-tensor}) is not so crazy. We state
  things in such generality so that we can later apply the same
  construction to algebraic categories, like modules or rings.
\end{examples}

\begin{proposition}
  \label{coprod-simplcat}
  Let $\C$ be a category which is complete and has coproducts. With $-
  \otimes - : \s\C \times \s\Set \to \s\C$ defined as in
  (\ref{coprod-tensor}), $\s\C$ is a simplicial category.
\end{proposition}

This gives us a simplicial category structure on simplicial groups,
simplicial $R$-modules, simplicial $R$-algebras, etc., since all of
the underlying algebraic categories are complete and cocomplete. Thus
we can move on.

%%%%%%%%%%%%%%%%%%%%%%%%%%%%%%%%%%%%%%%%%%%%%%%%%%%%%%%%%%%%%%%%%%%%%%

\subsection*{The model structure}



%%%%%%%%%%%%%%%%%%%%%%%%%%%%%%%%%%%%%%%%%%%%%%%%%%%%%%%%%%%%%%%%%%%%%%

\section{The cotangent complex}

%%%%%%%%%%%%%%%%%%%%%%%%%%%%%%%%%%%%%%%%%%%%%%%%%%%%%%%%%%%%%%%%%%%%%%

\bibliographystyle{amsalpha}
\bibliography{refs}

\end{document}
