%%%%%%%%%%%%%%%%%%%%%%%%%%%%%%%%%%%%%%%%%%%%%%%%%%%%%%%%%%%%%%%%%%%%%%

\renewcommand{\A}{\mathbb{A}}
\renewcommand{\O}{\mathcal{O}}

\renewcommand{\a}{\mathfrak{a}}
\newcommand{\p}{\mathfrak{p}}
\newcommand{\q}{\mathfrak{q}}

\newcommand{\height}{\operatorname{ht}}

%%%%%%%%%%%%%%%%%%%%%%%%%%%%%%%%%%%%%%%%%%%%%%%%%%%%%%%%%%%%%%%%%%%%%%

%%%%%%%%%%%%%%%%%%%%%%%%%%%%%%%%%%%%%%%%%%%%%%%%%%%%%%%%%%%%%%%%%%%%%%

\renewcommand{\A}{\mathbb{A}}
\renewcommand{\O}{\mathcal{O}}

\renewcommand{\a}{\mathfrak{a}}
\newcommand{\p}{\mathfrak{p}}
\newcommand{\q}{\mathfrak{q}}

\newcommand{\height}{\operatorname{ht}}

%%%%%%%%%%%%%%%%%%%%%%%%%%%%%%%%%%%%%%%%%%%%%%%%%%%%%%%%%%%%%%%%%%%%%%


\numberwithin{equation}{subsection}

\renewcommand{\C}{\mathcal{C}}
\renewcommand{\L}{\mathbb{L}}
\newcommand{\ab}{\operatorname{ab}}
\newcommand{\Der}{\operatorname{Der}}

%%%%%%%%%%%%%%%%%%%%%%%%%%%%%%%%%%%%%%%%%%%%%%%%%%%%%%%%%%%%%%%%%%%%%%

\title{Defining the cotangent complex}
\author{Arpon Raksit}
\date{December 16, 2013 (original); \today\ (last edit)}

\begin{document}
\maketitle
\thispagestyle{fancy}

%%%%%%%%%%%%%%%%%%%%%%%%%%%%%%%%%%%%%%%%%%%%%%%%%%%%%%%%%%%%%%%%%%%%%%


\section*{Opening nonsense}

I've recently learned a bit about the general framework of model
categories. It's sort of amazing, in the sense that seemingly a lot of
the basic homotopy theory of spaces which I've seen can be axiomatised
in a totally formal and categorical way. This includes something like
Whitehead's theorem (and its converse)---that a map of CW complexes is
a homotopy equivalence if and only if it's a weak homotopy
equivalence---which I never would have thought of as a categorical
fact.

Anyway, this axiomatisation and generalisation is beautiful and all,
but of course one hopes that it leads to new homotopy theory, in
addition to providing a nice framework for existing homotopy theory to
live in. Obviously it did. My goal in writing these notes was to learn
about one of the earliest examples of this, the development of a
(co)homology theory of commutative algebras by Andr\'{e}
\cite{andre-cotangent} and Quillen \cite{quillen-cotangent}.

\begin{notation-nonum}
  \begin{itemize}[leftmargin=*]
  \item Denote by $\Set$ the category of sets.
  \item All rings (and algebras) are commutative.
  \item For a ring $R$ denote by $\Mod(R)$ the category of $R$-modules
    and $\Alg(R)$ the category of $R$-algebras.
  \item For a category $\C$ denote by $\s\C$ the category of
    simplicial objects in $\C$.
  \end{itemize}
\end{notation-nonum}

%%%%%%%%%%%%%%%%%%%%%%%%%%%%%%%%%%%%%%%%%%%%%%%%%%%%%%%%%%%%%%%%%%%%%%

\section{Abelianisation and K\"ahler differentials}

The goal of this section is to motivate---on the level of analogy at
least, following \cite{goerss-modelsimplicial}---the content of
\textsection 2 and the construction of the cotangent complex in
\textsection 3.

\subsection{Singular homology}
\label{singhom}

In hopes of generalising the notion of homology via the theory of
model categories, we first ask the following.

\begin{question}
  How can we view singular homology in a homotopyish way?
\end{question}

\begin{nothing}
  \label{sing-doldthom}
  Perhaps the most immediate answer is the following. Suppose $X$ is a
  topological space. We can choose a CW approximation, that is, a CW
  complex $Y$ and a weak equivalence $Y \to X$. Let $\ab(Y)$ denote
  the free topological abelian group on $Y$. Then one version of the
  Dold-Thom theorem \cite{mccord-doldthom} states that
  \[
  \pi_*(\ab(Y)) \simeq \H_*(Y) \simeq \H_*(X).
  \]
  Thus we should view $\ab(Y)$ as the object which gives, via its
  homotopy, singular homology. Note that to apply Dold-Thom we had to
  choose a CW approximation to $X$ before applying $\ab$. From the
  point of view of model categories, this is precisely choosing a
  cofibrant replacement (in the standard model structure on
  topological spaces). So finally we should view singular homology as
  the derived functor of $\ab$.
\end{nothing}

\begin{nothing}
  \label{sing-doldkan}
  Another answer to the question goes through simplicial sets rather
  than topological spaces. Suppose $K \in \s\Set$, perhaps thinking $K =
  \Sing(X)$ for some topological space $X$. Let $\ab(K) \coloneqq \Z
  K\in \s\Mod(\Z)$ the free simplicial abelian group on $K$.  Then the
  Dold-Kan correspondence\footnote{Many thanks to Albrecht Dold in this
    section!} tells us that
  \[
  \pi_*(\ab(K)) \simeq \H_*(K) \simeq \H_*(X).
  \]
  So again we view $\ab(K)$ as giving homology via its homotopy. Here
  we did not have to choose a cofibrant replacement, but this is
  simply because (in the standard model structure) all $K \in \s\Set$
  are cofibrant!
\end{nothing}

\subsection{Quillen homology}
\label{quillenhom}

The two points of view on singular homology given in
(\ref{sing-doldthom}) and (\ref{sing-doldkan}) motivate the following
general philosophy, due to Quillen.

\begin{definitions}
  \begin{enumerate}[leftmargin=*]
  \item An \emph{abelian group object} in a category $\C$ is an object $A
    \in \C$ such that the functor $\hom_\C(-,A) : \C^\op \to \Set$
    factors through the forgetful functor $\Mod(\Z) \to \Set$, i.e.,
    $\hom_\C(B,A)$ is naturally an abelian group.
  \item A \emph{morphism of abelian group objects} $A,A' \in \C$ is a
    morphism $A \to A'$ in $\C$ for which the induced map $\hom_\C(B,A)
    \to \hom_\C(B,A')$ is a group homomorphism for $B \in \C$.
  \item Denote the resulting \emph{subcategory of abelian group
    objects} in by $\C_{\ab}$.
  \end{enumerate}
\end{definitions}

\begin{nothing}
  \label{abgpob-mult}
  One can check that if $\C$ has a terminal object $*$ and binary
  products, then $A \in \C$ is an abelian group object if and only if
  there exist morphisms
  \begin{itemize}
  \item $\mu : A \times A \to A$ (multiplication),
  \item $\epsilon : * \to A$ (identity),
  \item $\iota : A \to A$ (inverse),
  \end{itemize}
  such that the usual diagrams encoding abelian group structure
  commute, and a morphism of abelian group objects is one which
  respects multiplication.
\end{nothing}

\begin{nothing}
  \label{simplab}
  Let $\C$ be a category with a terminal object and binary
  products. It is immediate from (\ref{abgpob-mult}) that
  $(\s\C)_{\ab} \simeq \s(\C_{\ab})$ That is, the two subcategories of
  $\s\C$ consisting of:
  \begin{itemize}
  \item abelian group objects
  \item objects which are degree-wise abelian group objects and whose
    face and degeneracy maps are morphisms of abelian group objects
  \end{itemize}
  can be identified in the natural manner. We won't distinguish
  between the two, and will denote them by $\s\C_{\ab}$.

  Suppose moreover that the inclusion $\C_{\ab} \to \C$ has a left
  adjoint $\ab : \C \to \C_{\ab}$. It is easy to see then that the
  inclusion $\s\C_{\ab} \to \s\C$ has left adjoint $\s\C \to
  \s\C_{\ab}$ given by applying $\ab$ degree-wise.
\end{nothing}

\begin{definition}
  \label{derivedab}
  Let $\C$ be a model category. Suppose there is a model structure on
  $\C_{\ab}$ such that the inclusion $\C_{\ab} \to \C$ is a right
  Quillen functor, with left adjoint $\ab : \C \to \C_{\ab}$, called
  \emph{abelianisation}. Then \emph{homology} is the total left
  derived functor $\L \ab$, i.e., the homology of $X \in \C$ is the
  object $\L \ab(X) \in \C_{\ab}$.
\end{definition}

\begin{examples}
  This philosophy is of course a generalisation of the discussion in
  \textsection\ref{singhom}. If $\C = \Top$ then $\C_{\ab}$ is the
  subcategory of topological abelian groups and if $\C = \s\Set$ then
  $\C_{\ab}$ is the subcategory of simplicial abelian groups. In both
  cases the natural model structure on $\C_{\ab}$ is defined precisely
  such that $\C_{\ab} \to \C$ is right Quillen, with left adjoints
  $\ab : \Top \to \Top_{\ab}$ and $\ab : \s\Set \to \s\Set_{\ab}
  \simeq \s\Mod(\Z)$ the free topological and simplicial abelian group
  functors, respectively.
\end{examples}

\subsection{Abelianisation of algebras}
\label{abelianalgebra}

Of course we must then ask: to what other categories $\C$ can we
fruitfully apply this notion of homology to? As stated earlier, the
goal here is to describe what happens when we consider the category of
commutative rings or algebras. So first we need to understand what
abelianisation is in this context.

\begin{notation}
  We fix a ring $R$.
\end{notation}

\begin{proposition}
  The only abelian group object in $\Alg(R)$ is the zero ring.
\end{proposition}

\begin{proof}
  The terminal object in $\Alg(R)$ is the zero ring $0$, and the only
  $A \in \Alg(R)$ for which there exists a morphism $0 \to A$ is $A =
  0$ itself.
\end{proof}

\begin{nothing}
  So we've already run into a subtlety in this category. We'll resolve
  this issue by working in a slightly different category. Fix an
  $R$-algebra $A$. Let
  \[
  \C \coloneqq \Alg(R)_{/A}
  \]
  be the overcategory of $R$-algebras over $A$. I.e., an object of
  $\C$ is a sequence of ring morphisms $R \to B \to A$ (which we of
  course abusively refer to as $B$) whose composition is the structure
  morphism $R \to A$.
\end{nothing}

Before continuing the discussion on abelianisation, we review
derivations and K\"ahler differentials.

\begin{definitions}
  \label{derivations}
  Let $B$ be an $R$-algebra and $M$ a $B$-module.
  \begin{enumerate}[leftmargin=*]
  \item An \emph{$R$-derivation} $d : B \to M$ is a morphism of
    $R$-modules satisfying the Leibniz rule, $d(xy) = x\,d(y) +
    y\,d(x)$ for $x,y \in B$.
  \item The set $\Der_R(B,M)$ of $R$-derivations has a natural
    $B$-module (in particular abelian group) structure.
  \item \label{kahdiff} The $B$-module $\Omega_{B/R}$ of
    \emph{K\"ahler differentials} is defined by the universal property
    \[
    \hom_{\Mod(B)}(\Omega_{B/R}, M) \simeq \Der_R(B,M).
    \]
    In fact, if we let $\mu : B \otimes_R B \to B$ be the product map
    and $I \coloneqq \ker(\mu)$, then there is an $R$-derivation $d :
    B \to I/I^2$ sending $x \mapsto 1 \otimes x - x \otimes 1$ such
    that the map
    \[
    \hom_{\Mod(B)}(I/I^2, M) \to \Der_R(B,M), \quad \phi \mapsto \phi
    \circ d
    \]
    is an isomorphism, so $\Omega_{B/R} \simeq I/I^2$. We omit the
    verification of this construction.
  \end{enumerate}
\end{definitions}

\begin{nothing}
  \label{square-zero}
  We define a functor
  \[
  \Mod(A) \to \C, \quad M \mapsto A \oplus M,
  \]
  where the ring structure on $A \oplus M$ is given by $(a,x)(b,y) =
  (ab,ay+bx)$. Then the structure morphism $R \to A$ and projection $A
  \oplus M \to A$ determine $R \to A \oplus M \to A$ as an object in
  $\C$. If we have a morphism $\phi : M \to N$ in $\Mod(A)$ then
  $\id_A \oplus \phi : A \oplus M \to A \oplus N$ evidently gives a
  morphism in $\C$.
\end{nothing}

\begin{proposition}
  \label{modules-abelian}
  The functor $\Mod(A) \to \C$ defined in (\ref{square-zero}) factors
  through the inclusion $\C_{\ab} \to \C$, and furthermore induces an
  equivalence of categories
  \[
  \Mod(A) \simeq \C_{\ab}.
  \]
\end{proposition}

\begin{proof}
  The first claim is that $A \oplus M$ is an abelian group object for
  any $M \in \Mod(A)$. Let $B \in \C$. By definition of $\C$ and $A
  \oplus M$, giving a morphism $B \to A \oplus M$ is equivalent to
  giving an $R$-linear map $d : B \to M$ satisfying $d(xy) = x\,d(y) +
  y\,d(x)$ for $x,y \in B$. (Here we have viewed $M$ as a module over
  $R$ and $B$ via restriction of scalars.) I.e., we have a natural
  isomorphism
  \begin{equation}
    \label{sqzero-der}
    \hom_{\C}(B, A \oplus M) \simeq \Der_R(B,M),
  \end{equation}
  which is an abelian group, giving our claim.

  We next show that if $B \in \C_{\ab}$ is an abelian group object
  then it is isomorphic to $A \oplus M$ for some $M \in \Mod(A)$. Let
  $\eta : B \to A$ be the map over $A$. Note the terminal object of
  $\C$ is $A$ and the product is given by fibred product $B \times_A
  B$. To have an abelian group object we in particular need morphisms
  $\mu : B \times_A B \to B$ and $\epsilon : A \to B$.  Since
  $\epsilon$ is a morphism over $A$, it must be a section of
  $\eta$. Thus if we set $M \coloneqq \ker(\eta)$ we have a splitting
  $B \simeq A \oplus M$ as $A$-modules. Next, $\epsilon$ being a
  ``two-sided identity'' means
  \[
  \mu \circ (\epsilon \eta, \id_B) = \id_B = \mu \circ (\id_B,
  \epsilon \eta),
  \]
  and hence $\mu(x,0) = x = \mu(0,x)$ for $x \in M$. This implies $xy
  = \mu(x,0)\mu(0,y) = \mu(0,0) = 0$ for $x,y \in M$, and hence for
  $a,b \in A$ we have
  \[
  (\epsilon(a)+x)(\epsilon(b)+y) = \epsilon(ab) + \epsilon(a)y +
  \epsilon(b)x
  \]
  It follows that in fact $B \simeq A \oplus M$ as objects of $\C$.

  Finally we must show the functor is fully faithful. Faithfulness is
  clear. A morphism $\psi : A \oplus M \to A \oplus N$ over $A$ must
  be of the form $\id_A \oplus \phi$. And since
  \[
  (1,x)(1,y) = (1,x+y), \quad (a,0)(1,x) = (1,ax)
  \]
  for $a \in A$ and $x,y \in M$ or $x,y \in N$, if $\psi$ is a
  morphism of abelian group objects, then $\phi$ is $A$-linear.
\end{proof}

\begin{proposition}
  \label{adjointkahler}
  The assignment $B \mapsto \Omega_{B/R} \otimes_B A$ defines a
  functor $\C \to \Mod(A)$ which is left adjoint to inclusion $\Mod(A)
  \simeq \C_{\ab} \to \C$.
\end{proposition}

\begin{proof}
  Functoriality is clear from the definition
  (\ref{derivations}.\ref{kahdiff}) of $\Omega_{B/R}$ and the Yoneda
  lemma, as one can pull back derivations. Then adjointness follows
  from (\ref{sqzero-der}):
  \begin{align*}
  \hom_\C(B, A \oplus M) &\simeq \Der_R(B,M) \\ &\simeq
  \hom_{\Mod(B)}(\Omega_{B/R}, M) \\ &\simeq
  \hom_{\Mod(A)}(\Omega_{B/R} \otimes_B A, M). \qedhere
  \end{align*}
\end{proof}

\begin{nothing}
  We now have a convenient (and pretty intruiging, actually) notion of
  abelianisation for the category $\C = \Alg(R)_{/A}$. As per the
  philosophy of \textsection \ref{quillenhom}, we should now take the
  derived functor. But wait! We don't even have a model structure on
  $\C$! And why would we expect to? Indeed, we should extend the above
  discussion (\ref{square-zero})--(\ref{adjointkahler}) to
  \emph{simplicial} objects via (\ref{simplab}) to get an adjunction
  \[
  \begin{tikzcd}
    \Omega_{-/R} \otimes_- A \ : \ \s\Alg(R)_{/A} \rar[yshift=2pt] &
    \s\Mod(A) \ :\ A \oplus -. \lar[yshift=-2pt]
  \end{tikzcd}
  \]
  It seems much more natural to give model structures to these
  categories, and in fact this is what we will do in the following
  section.
\end{nothing}

%%%%%%%%%%%%%%%%%%%%%%%%%%%%%%%%%%%%%%%%%%%%%%%%%%%%%%%%%%%%%%%%%%%%%%

\section{Simplicial model categories}

Here we describe the simplicial model category structure on a variety
of (simplicial) algebraic categories, including the relevant ones
discussed above. (Technically, we will only need the model category
structure to define the cotangent complex and Andr\'e-Quillen
(co)homology, but let's just be thorough about things.)

\subsection{The simplicial structure}

We first recall what we mean by a simplicial category in this context,
and how to recognise that we have one.

\begin{definition}
  \label{simplcat}
  A \emph{simplicial category} is a category $\C$ equipped with
  functors
  \[
  -\otimes- : \C \times \s\Set \to \C, \quad \map_l : \C^\op \times \C
  \to \s\Set, \quad \map_r : \s\Set^\op \times \C \to \C
  \]
  and the following isomorphisms, each of which is natural is all of
  its variables:
  \begin{enumerate}
  \item \label{twovaradj} $\hom_{\s\Set}(K, \map_l(A,B)) \simeq
    \hom_\C(A \otimes K, B) \simeq \hom_\C(A, \map_r(K,B))$,
  \item \label{assoc} $A \otimes (K \times L) \simeq (A
    \otimes K) \otimes L$,
  \item \label{agree} $\map_l(A,B)_0 \simeq \hom_\C(A,B)$.
  \end{enumerate}
\end{definition}

\begin{lemma}
  Suppose $\C$ is equipped with a functor $-\otimes- : \C \times
  \s\Set \to \C$ satisfying the following:
  \begin{enumerate}
  \item the functor $- \otimes K : \C \to \C$ has a right adjoint
    $\map_r(K,-) : \C \to \C$ for each $K \in \s\Set$,
  \item the functor $A \otimes - : \s\Set \to \C$ preserves colimits
    for each $A \in \C$,
  \item $A \otimes * \simeq A$ naturally in $A \in \C$,
  \item the natural isomorphism (\ref{simplcat}.\ref{assoc}).
  \end{enumerate}
  Define $\map_l : \C^\op \times \C \to \s\Set$ by $\map_l(A,B)_n
  \coloneqq \hom_\C(A \otimes \Delta^n, B)$ for $n \ge 0$. Then the
  above data gives $\C$ the structure of a simplicial category.
\end{lemma}

\begin{proof}
  Omitted---this is an exercise in adjunctions.
\end{proof}

\begin{nothing}
  \label{coprod-tensor}
  Let $\C$ be a category with coproducts. We define a functor $-
  \otimes - : \s\C \times \s\Set \to \s\C$ as follows. Let $A \in
  \s\C$ and $K \in \s\Set$. For $n \ge 0$ define
  \[
  (A \otimes K)_n \coloneqq \coprod_{k \in K_n} A_n.
  \]
  For $\phi : [m] \to [n]$ a morphism in the category $\Delta$ define
  the induced morphism $\phi^* : (A \otimes K)_n \to (A \otimes K)_m$
  such that the following commutes for $k \in K_n$:
  \[
  \begin{tikzcd}
    A_n \rar{\phi^*} \dar{\incl_k} & A_m \dar{\incl_{\phi^*(k)}}
    \\ \coprod_{l \in K_n} A_n \rar{\phi^*} &\coprod_{l \in K_m} A_m.
  \end{tikzcd}
  \]
\end{nothing}

\begin{examples}
  If $\C = \Set$ then we simply recover $A \otimes K = A \times K$. If
  $\C = \Set_*$ (pointed sets) then we get $A \otimes K = A \times K /
  * \times K$, which is just the smash product $A \wedge K_+$ with
  $K_+$, denoting $K$ with an added disjoint basepoint. Thus the
  construction of (\ref{coprod-tensor}) is not so crazy. We state
  things in such generality so that we can later apply the same
  construction to algebraic categories, like modules or rings.
\end{examples}

\begin{proposition}
  \label{coprod-simplcat}
  Let $\C$ be a category which is complete and has coproducts. With $-
  \otimes - : \s\C \times \s\Set \to \s\C$ defined as in
  (\ref{coprod-tensor}), $\s\C$ is a simplicial category.
\end{proposition}

This gives us a simplicial category structure on simplicial groups,
simplicial $R$-modules, simplicial $R$-algebras, etc., since all of
the underlying algebraic categories are complete and cocomplete. Thus
we can move on.

%%%%%%%%%%%%%%%%%%%%%%%%%%%%%%%%%%%%%%%%%%%%%%%%%%%%%%%%%%%%%%%%%%%%%%

\subsection{The model structure}

Yo dude.

%%%%%%%%%%%%%%%%%%%%%%%%%%%%%%%%%%%%%%%%%%%%%%%%%%%%%%%%%%%%%%%%%%%%%%

\section{The cotangent complex}

We've now equipped a substantial number of interesting categories with
simplicial model structures. As a happy coincidence (not really) we've
managed to do this for the categories $\s\Alg(R)_{/A}$ and $\s\Mod(A)$
we were discussing back in \textsection \ref{abelianalgebra}. (Recall
$R$ is a ring and $A$ an $R$-algebra.)

\subsection{The definition}

The next step is to apply (\ref{derivedab}) to get our ``homology
object''. We recall that in the current setting we have an adjunction
\[
\begin{tikzcd}
  \Omega_{-/R} \otimes_- A \ : \ \s\Alg(R)_{/A} \rar[yshift=2pt] &
  \s\Mod(A) \ :\ A \oplus -, \lar[yshift=-2pt]
\end{tikzcd}
\]
where the functor $A \otimes -$ is the inclusion of abelian group
objects and the K\"ahler differentials $\Omega_{-/R} \otimes_- A$ is
the adjoint abelianisation functor. But to be able to derive
abelianisation as in (\ref{derivedab}), we need to check something.

\begin{lemma}
  The inclusion functor $\s\Mod(A) \to \s\Alg(R)_{/A}$ is right
  Quillen (and hence abelianisation is left Quillen).
\end{lemma}

\begin{proof}
  
\end{proof}

%%%%%%%%%%%%%%%%%%%%%%%%%%%%%%%%%%%%%%%%%%%%%%%%%%%%%%%%%%%%%%%%%%%%%%

\nocite{iyengar-cotangent, goerssjardine}
\bibliographystyle{amsalpha}
\bibliography{refs}

\end{document}
