\section{Character theory}
\label{char}

We finally arrive at the main course of this thesis. In
\cref{char-loops} we define our $p$-adic loop space functor, and in
\cref{char-main}, everything comes together at once to prove our
generalized character theory.

\begin{notation}
  \label{char-fixp}
  Throughout this section we work at a fixed prime $p$.
\end{notation}

%%%%%%%%%%%%%%%%%%%%%%%%%%%%%%%%%%%%%%%%%%%%%%%%%%%%%%%%%%%%%%%%%%%%%%

\subsection{Loops}
\label{char-loops}

\begin{notation}
  \label{char-inthom}
  There is an internal hom in the $\infty$-category $\Space_\Glo$,
  which we denote $[-,-]$. It satisfies the adjunction
  \[
  \Map_{\Space_\Glo}(X \times Y, Z) \iso
  \Map_{\Space_\Glo}(X, [Y,Z]),
  \]
  which means it can be computed via the formula
  \[
  [Y,Z](\lB G) \iso
  \Map_{\Space_\Glo}(\lB G, [Y,Z]) \iso
  \Map_{\Space_\Glo}(\lB G \times Y, Z).
  \]
  In particular, for $Y = \lB H$ we have
  \begin{align}
    \label{char-inthom-case}
    [\lB H,Z](\lB G) &\iso
    \Map_{\Space_\Glo}(\lB G \times \lB H, Z) \\ &\iso
    \Map_{\Space_\Glo}(\lB (G \times H), Z) \nonumber \\ &\iso
    Z(\lB (G \times H)) \nonumber
  \end{align}
\end{notation}

\begin{definition}
  \label{char-loopdfn}
  We define the \emph{($p$-adic) loop space} functor:
  \[
  \cL \c \Space_\Glo \to \Space_\Glo, \quad
  X \mapsto \colim_k {[\lB\lZ/p^k, X]}.
  \]
  One may want to think of this as an internal hom $[\lB\lZ_p, -]$
  from the pro-object $\lB\lZ_p \ce \lim_k \lB\lZ/p^k$ in
  $\Space_\Glo$.
\end{definition}

\begin{lemma}
  \label{char-loop-colim}
  The loop space functor $\cL$ preserves colimits.
\end{lemma}

\begin{proof}
  Since colimits commute with other colimits, it suffices to show for
  each $k$ that $[\lB\lZ/p^k, -]$ preserves colimits, which is easy to
  see: by \cref{char-inthom-case},
  \begin{align*}
  [\lB\lZ/p^k, \colim_i X_i](\lB G) &\iso
  \l(\colim_i X_i\r)(\lB(G \times \lZ/p^k)) \\ &\iso
  \colim_i X_i(\lB(G \times \lZ/p^k)) \\&\iso
  \colim_i {[\lB\lZ/p^k, X_i](\lB G)}. \qedhere
  \end{align*}
\end{proof}

\begin{lemma}
  \label{char-loop-gspace}
  Let $G$ be a finite group and let $X$ be a $G$-space, which we will
  abusively identify with its associated global space via the functor
  $\delta_G \c G\dash\Space \to \Space_\Glo$ defined in
  \cref{global-gspace-adj-unwrap}. Then its loop space $\cL X$ is the
  global space associated to the $G$-space
  \[
  Y \ce \coprod_{\alpha \in \Hom(\lZ_p,G)} X^{\im(\alpha)}.
  \]
\end{lemma}

\begin{proof}
  It suffices to prove the claim after replacing $\cL X$ with
  $[\lB\lZ/p^k, X]$ and $Y$ with
  $Y_k \ce \coprod_{\alpha \in \Hom(\lZ/p^k,G)} X^{\im(\alpha)}$,
  since $G$ being finite implies $Y \iso Y_k$ for sufficiently large
  $k$. When we view $Y_k$ as a global space, $Y_k(\lB H)$ is the
  classifying space of the action groupoid
  \[
  \l( \coprod_{\beta \in \Hom(H,G)} Y_k^{\im(\beta)} \r) \sslash G.
  \]
  Now, for $\alpha \in \Hom(\lZ/p^k, G)$ and $\beta \in \Hom(H,G)$,
  $x \in X^{\im(\alpha)} \subseteq Y_k$ is fixed by $\im(\beta)$ if
  and only if $\im(\alpha)$ and $\im(\beta)$ commute within $G$ and
  $x$ is a fixed point of the group $\im(\alpha)\im(\beta)$ they
  generate. So this action groupoid can be rewritten as
  \[
  \l(\coprod_{\gamma \in \Hom(H \times \lZ/p^k)} X^{\im(\gamma)}\r)
  \sslash G.
  \]
  The classifying space of this is by definition equivalent to
  $X(\lB(H \times \lZ/p^k))$, which by \cref{char-inthom-case} is
  equivalent to $[\lB\lZ/p^k, X](\lB H)$. So we conclude
  $Y_k \iso [\lB\lZ/p^k, X]$, as desired.
\end{proof}

\begin{lemma}
  \label{char-loop-bg}
  Let $G$ be a finite group. We have
  \[
  \cL \lB G \iso \coprod_{[\alpha]} \lB C(\alpha),
  \]
  where $[\alpha]$ runs over the conjugacy classes of homomorphisms
  $\lZ/p^k \to G$ for some sufficiently large $k$ (the largest
  $p$-power torsion in $G$) and $C(\alpha) \subseteq G$ denotes the
  centralizer of $\alpha$. In particular, for $G = A$ abelian there is
  a natural equivalence
  \[
  \cL \lB A \iso \coprod_{a \in A[p^\infty]} \lB A,
  \]
  where $A[p^\infty] \subseteq A$ is the subgroup of $p$-power
  torsion.
\end{lemma}

\begin{proof}
  This follows immediately from \cref{char-loop-gspace}.
\end{proof}

\begin{lemma}
  \label{char-loop-spect}
  Let $E$ be a global spectrum. Then:
  \begin{enumerate}
  \item \label{char-loop-spect-is} The composite
    $E' \ce E \circ \cL \c \Space_\Glo^\op \to \Spect$ is also a
    global spectrum, i.e. takes colimits of global spaces to limits of
    spectra.
  \item \label{char-loop-spect-abdesc} If $E$ satisfies abelian
    descent, then so does $E'$.
  \end{enumerate}
\end{lemma}

\begin{proof}
  Statement \cref{char-loop-spect-is} is immediate from $\cL$ being
  colimit-preserving \cref{char-loop-colim}. For the second statement,
  we would like to prove that $E'(\lB G) \to E'(\lB^\ab G)$ is an
  equivalence for all $G$, assuming this holds for $E$. We first
  observe that for $X$ a global space, there's a natural equivalence
  of global spaces $(\cL X)^\ab \iso \cL X^\ab$:
  \begin{align*}
  (\cL X)^\ab(\lB H) &\iso
  \cL X(\lB H^\ab) \\ &\iso
  \colim_k X(\lB(H^\ab \times \lZ/p^k)) \\ &\iso
  \colim_k X(\lB(H \times \lZ/p^k)^\ab) \\ &\iso
  \cL X^\ab(\lB H),
  \end{align*}
  where we have applied
  \cref{abdesc-abelian-descent-abelianization,char-inthom-case}.
  Combining this with \cref{char-loop-bg}, we see that
  \[
  \cL(\lB^\ab G) \iso
  (\cL \lB G)^\ab \iso
  \l( \coprod_{[\alpha]} \lB C(\alpha) \r)^\ab,
  \]
  and by \cref{abdesc-abelian-descent-abelianization} the last of
  these is equivalent to $\coprod_{[\alpha]} \lB^\ab
  C(\alpha)$.
  Finally, after chasing these equivalences we see that the map
  $E'(\lB G) \to E'(\lB^\ab G)$ is the map
  \[
  E(\cL \lB G) \iso
  E \l( \coprod_{[\alpha]} \lB C(\alpha) \r) \to
  E \l( \coprod_{[\alpha]} \lB^\ab C(\alpha) \r) \iso
  E(\cL \lB^\ab G),
  \]
  induced by the coproduct of the canonical maps
  $\lB^\ab C(\alpha) \to \lB C(\alpha)$. Since $E$ takes coproducts to
  products, it follows that if $E$ satisfies abelian descent then so
  does $E'$.
\end{proof}

%%%%%%%%%%%%%%%%%%%%%%%%%%%%%%%%%%%%%%%%%%%%%%%%%%%%%%%%%%%%%%%%%%%%%%

\subsection{The main theorem}
\label{char-main}

We are now ready to state and prove our version of character theory.

\begin{notation}
  \label{char-finalntn}
  Let $E$ be a Morava E-theory of height $n$, and let
  $L_t \ce L_{K(t)}E$. Let $C_t$ be the flat $L_t$-algebra constructed
  in \cref{pdiv-einftyct}. We will denote the associated Borel global
  spectra also by $E$ and $L_t$.
\end{notation}

\begin{nothing}
  \label{char-tensloop}
  By \cref{char-loop-spect} we have a global spectrum
  $L_t \circ \cL^{n-t}$, which by \cref{char-loop-bg} is given by the
  formula
  \begin{equation}
    \label{char-loopsum}
    \lB G \mapsto
    L_t\l( \coprod_{[\alpha]} \rB C(\alpha) \r) \iso
    \bigoplus_{[\alpha]} L_t(\rB C(\alpha)),
  \end{equation}
  where now $[\alpha]$ runs over conjugacy classes of homomorphisms
  $(\lZ/p^k)^{n-t} \to G$ (for sufficiently large $k$). So
  $L_t \circ \cL^{n-t}$ again takes values in $L_t$-modules, and thus
  we may define $C_t \otimes (L_t \circ \cL^{n-t})$. And since tensor
  product commutes with direct sum, this is equivalent to the global
  spectrum $(C_t \otimes L_t) \circ \cL^{n-t}$.
\end{nothing}


\begin{theorem}
  \label{char-char}
  There is an equivalence of global spectra
  \[
  C_t \otimes E \iso C_t \otimes (L_t \circ \cL^{n-t}).
  \]
\end{theorem}

\begin{proof}
  By \cref{abdesc-main} the global spectra $C_t \otimes E$ and
  $C_t \otimes L_t$ satisfy abelian descent. By \cref{char-loop-spect}
  this implies
  $C_t \otimes (L_t \circ \cL^{n-t}) \iso (C_t \otimes L_t) \circ
  \cL^{n-t}$
  also satisfies abelian descent. Thus is suffices to give the desired
  equivalence after restricting from $\Glo$ to $\Alo$. For finite
  abelian groups $A$, the formula \cref{char-loopsum}, coming from
  \cref{char-loop-bg}, gives a natural equivalence
  \[
  (C_t \otimes (L_t \circ \cL^{n-t}))(\lB A) \iso
  C_t \otimes_{L_t} \bigoplus_{(A[p^\infty])^{n-t}} L_t(\rB A),
  \]
  which is naturally equivalent to
  $(C_t \otimes E)(\lB A) \iso C_t \otimes_A E(\rB A)$ by our analysis
  of $p$-divisible groups \cref{pdiv-main}.
\end{proof}

\begin{remark}
  Since $C_t$ is flat over $L_t$ and $E$ by \cref{pdiv-einfty-flat},
  our theorem gives for each finite group $G$ an equivalence of
  $G$-equivariant cohomology theories
  \[
  C_t^* \otimes_{E^*} E_G(X) \iso
  C_t^* \otimes_{L_t^*} (L_t)_G^*(\cL^{n-t}X),
  \]
  for finite $G$-spaces $X$. So we indeed recover isomorphisms of the
  form proven in \cite{hkr-char,stapleton-tgcm}. Note however that we
  did not explicitly construct this isomorphism when $G$ is
  nonabelian.
\end{remark}

And with that we have completed our mission! To conclude, let me just
summarize the ideas that got us here. Complex-oriented descent allowed
us to show that the the global spectra $C_t \otimes E$ and
$C_t \otimes (L_t \circ \cL^{n-t})$ are determined by their values on
abelian groups, and hence in essence by the $p$-divisible groups
$\lG_E$ and $\lG_{L_t}$ associated to Morava E-theory and its
localization. Then the fact that we could identify the two sides of
character theory simply boiled down to the behavior of these
$p$-divisible groups: when $\lG_E$ is base changed to $L_t$, it loses
$n-t$ infinitesimal height, and gains $n-t$ \'etale height; then when
base changed to $C_t$, this \'etale part becomes constant, which
translates into the $(n-t)$-fold looping on the right-hand side.
