\section{Chromatic homotopy theory}
\label{chrom}

\newcommand{\CP}{\lC\lP}

In this section we lay out some of the fundamental ideas in chromatic
homotopy theory, enough that we can somewhat safely speak about the
generalizations of character theory alluded to in \cref{intro}. As
this section is intended primarily as background, it's far less formal
than the others and quite light on proofs. The lecture notes
\cite{hopkins-coctalos,lurie-chromatic} are nice places to read about
these ideas in more depth and detail (and my debt to these sources in
this exposition will be clear).

\subsection{Formal group laws \& Chern classes}

Our starting point is the theory of Chern classes associated to
complex vector bundles. Actually let's just focus on (complex) line
bundles $L \to X$, in which case all that's of interest is the first
Chern class $c_1(L) \in \rH^2(X) \ce \rH^2(X;\lZ)$. Recall two facts
about this situation:
\begin{itemize}
\item Chern classes are natural: the Chern class of the pullback of a
  line bundle $L \to X$ in a map $f \c Y \to X$ is just the pullback
  $f^*(c_1(L)) \in \rH^2(Y)$ of the Chern class of $L$.
\item There is a universal line bundle: letting $\cO(1)$ denote the
  tautological line bundle over the space $\CP^\infty \iso \rB\rU(1)$,
  there is a natural bijection between the homotopy classes of maps
  $f \c X \to \CP^\infty$ and isomorphism classes of line bundles on
  $X$, given by associating such a map $f$ to the pullback
  $f^*(\cO(1))$ of the tautological bundle. That is, the functor
  associating to a space $X$ the set of isomorphism classes of line
  bundles on it is representable (in the homotopy category of spaces)
  by $\CP^\infty$, and the tautological bundle $\cO(1)$ is truly
  tautological: it corresponds to the identity map
  $\id \c \CP^\infty \to \CP^\infty$.
\end{itemize}
Of course Chern classes are also isomorphism-invariant, so it follows
from these facts that the Chern classes of all line bundles $L$ are
determined by a choice of Chern class
$c_1(\cO(1)) \in \rH^2(\CP^\infty)$. Now, recall that the cohomology
ring of $\CP^\infty$ is given by\footnote{It may be more common to
  write $\rH^*(\CP^\infty) \iso \lZ[t]$, i.e. to use the polynomial
  ring rather than the power series ring. This is just a convention:
  does one want to assemble the graded pieces of $\rH^*(X)$ by taking
  their direct product or their direct sum? In other words, does one
  want to consider infinite or just finite sums of homogenous elements
  in the ring $\rH^*(X)$? In just a bit we'll be considering
  cohomology theories $E$ other than singular cohomology, in
  particular ones in which the coefficient ring $E^* \ce E^*(\pt)$ is
  not concentrated in degree $0$, but may in fact be nonzero in
  infinitely many degrees. But even in this more general situation,
  we'd like to think of $E^*$ as some sort of base ring of
  coefficients, in which case we may have infinite sums in the ring
  $E^*(X)$ still living in some finite graded piece. It will thus be
  more natural to consider the power series ring $E^*[[t]]$, and so we
  begin with this convention here.}
\[
\rH^*(\CP^\infty) \iso \lZ[[t]],
\]
where the generator lives in degree $2$, i.e.
$t \in \rH^2(\CP^\infty)$. So of course then the interesting choice to
make is $c_1(\cO(1)) \ce t$. But note that $t$ is really only
well-defined up to a unit in $\lZ$, i.e. up to sign! So to be precise
we need to fix such a generator and then declare this to be
$c_1(\cO(1))$.

Now, the set of isomorphism classes of line bundles on a space $X$ is
often referred to as its Picard group; the name isn't so important
right now, but it is important that it's a group! Recall that the
operation is given by tensor product of line bundles. So it's natural
to ask: given two line bundles $L$ and $L'$ on $X$, can we express
$c_1(L \otimes L')$ in terms of $c_1(L)$ and $c_1(L')$? In fact one
can do so quite easily:
\begin{equation}
  \label{chrom-cherntensor}
    c_1(L \otimes L') = c_1(L) + c_1(L') \in \rH^2(X).
\end{equation}

Since there is universal line bundle $\cO(1)$, proving this boils down
to proving it for the universal example. The universal \emph{pair} of
line bundles is given by the bundles $\pi_1^*\cO(1),\pi_2^*\cO(1)$ on
$\CP^\infty \times \CP^\infty$, where
$\pi_1,\pi_2 \c \CP^\infty \times \CP^\infty \to \CP^\infty$ denote
the projections, and the formula \cref{chrom-cherntensor} holds if and
only if it holds for this universal example, that is if
\[
c_1(\pi_1^*\cO(1) \otimes \pi_2^*\cO(1)) =
c_1(\pi_1^*\cO(1)) + c_1(\pi_2^*\cO(1)) \in
\rH^2(\CP^\infty \times \CP^\infty).
\]
To see that this is true, we observe that the canonical map
$\iota \c \CP^\infty \vee \CP^\infty \to \CP^\infty \times \CP^\infty$
induces an isomorphism in $\rH^2$,
\begin{equation}
\label{chrom-cherntensorpf}
\iota^* \c \rH^2(\CP^\infty \times \CP^\infty) \isoto
\rH^2(\CP^\infty \vee \CP^\infty),
\end{equation}
and that the pullback bundle
$\iota^*(\pi_1^*\cO(1) \otimes \pi_2^*\cO(1))$ is just a copy of
$\cO(1)$ over each summand of $\CP^\infty \vee \CP^\infty$.

We can alternatively view this universal example as follows: by the
Yoneda lemma, the tensor product operation on line bundles corresponds
to some kind of multiplication map
$\mu \c \CP^\infty \times \CP^\infty \to \CP^\infty$. In cohomology
this gives a map
\[
\lZ[[t]] \iso
\rH^*(\CP^\infty) \lblto{\mu^*}
\rH^*(\CP^\infty \times \CP^\infty) \iso
\lZ[[x,y]],
\]
using the Kunneth isomorphism on the right-hand side, so
$x = \pi_1^*(t)$ and $y = \pi_2^*(t)$. By definition we'll have
$\mu^*\cO(1) \iso \pi_1^*\cO(1) \otimes \pi_2^*\cO(1)$.  So we end up
with
\begin{align*}
\mu^*(t) &=
\mu^*(c_1(\cO(1))) =
c_1(\mu^*\cO(1)) =
c_1(\pi_1^*\cO(1) \otimes \pi_2^*\cO(1)) =
c_1(\pi_1^*\cO(1) \otimes \pi_2^*\cO(1)), \\
x + y &=
\pi_1^*(t) +  \pi_2^*(t) =
\pi_1^*(c_1(\cO(1))) +  \pi_2^*(c_1(\cO(1))) =
c_1(\pi_1^*\cO(1)) + c_1(\pi_2^*\cO(1)).
\end{align*}
And therefore we conclude that the formula \cref{chrom-cherntensor} is
equivalent to the formula $\mu^*(t) = x + y$.

Somebody interested in stable homotopy theory might now ask: what
precisely did we use about singular cohomology in the above? We
observe that the essential facts were the computations of
$\rH^*(\CP^\infty)$ and $\rH^*(\CP^\infty \times \CP^\infty)$ as power
series rings. This motivates the following definition.\footnote{One
  can also give alternative, somewhat more elementary (but equivalent)
  definitions of complex-orientability than the one given here, but
  this is the one which most easily and succinctly fits into the
  motivation being given here.}

\begin{definition}
  \label{chrom-complor}
  A multiplicative cohomology theory, i.e. a (homotopy) ring spectrum,
  $E$ is said to be \emph{complex-orientable} if the Atiyah-Hirzebruch
  spectral sequence
  \[
  \rH^p(\CP^\infty;E^q(\pt)) \Rightarrow E^{p+q}(\CP^\infty)
  \]
  degenerates at its second page. By the standard computations in
  singular cohomology, this condition implies that there is an
  isomorphism $E^*(\CP^\infty) \iso E^*[[t]]$, where
  $E^* \ce E^*(\pt)$. A choice of such an isomorphism, i.e. a choice
  of generator $t$ (which again is well-defined only up to the units
  of $E^0$) is referred to as a \emph{complex-orientation} of $E$. We
  say $E$ is \emph{complex-oriented} if it is equipped with a
  complex-orientation.

  In fact the degeneration of the spectral sequence for $\CP^\infty$
  forces the analagous degeneration for
  $\CP^\infty \times \CP^\infty$, whence a complex-orientation on $E$
  similarly determines an isomorphism
  $E^*(\CP^\infty \times \CP^\infty) \iso E^*[[x,y]]$.
\end{definition}

\begin{examples}
  \label{chrom-coex}
  We give the two most basic examples of complex-orientable cohomology
  theories, which are essentially the two motivating examples of
  cohomology theories in the first place:
  \begin{enumerate}
  \item Obviously singular cohomology $\rH$ is complex-orientable.
  \item One can show that complex K-theory $\rK$ is complex-orientable
    as well.
  \end{enumerate}
\end{examples}

So we can create a theory of Chern classes with values in any
complex-oriented cohomology theory $E$. But now there is no reason for
formula \cref{chrom-cherntensor} to hold; its validity in singular
cohomology relies on the fact \cref{chrom-cherntensorpf}, which relies
on the coefficient ring $\rH^* \iso \lZ$ being concentrated in degree
$0$, a situation we don't in general expect to be in (e.g. for complex
K-theory, Bott periodicity tells us $\rK^* \iso \lZ[\beta,\beta^{-1}]$
with $\beta$ in degree $-2$). However, we can still say something
about the Chern class of a tensor product with values in $E^*$: it is
still determined by the universal example, which is still determined
by the map in cohomology
\begin{equation}
\label{chrom-cofgl}
E^*[[t]] \iso
E^*(\CP^\infty) \lblto{\mu^*}
E^*(\CP^\infty \times \CP^\infty) \iso
E^*[[x,y]],
\end{equation}
and again the formula for the Chern class of a tensor product will be
given by the power series $f \ce \mu^*(t) \in E^*[[x,y]]$. The fact
that the tensor product of line bundles is unital, commutative, and
associative (all up to isomorphism) implies that $f$ lies in some
special class of power series in two variables, which we should think
of as some kind of infinitesimal version of a group operation. We give
a name to this special class now, as it is quite important.

\begin{definition}
  \label{chrom-fgl}
  A \emph{formal group law} over a commutative ring $R$ is a power
  series $f \in R[[x,y]]$ satisfying:
  \begin{enumerate}
  \item $f(x,0) = x$ and $f(0,y) = y$;
  \item $f(x,y) = f(y,x)$;
  \item $f(f(x,y),z) = f(x,f(y,z))$.
  \end{enumerate}
\end{definition}

\begin{remark}
  \label{chrom-grcomm}
  Note that we'll always speak about formal group laws over
  \emph{commutative} rings, but the coefficient ring $E^*$ of a
  multiplicative cohomology theory is \emph{graded-commutative}. This
  is no problem here: since $\mu^*$ respects grading and the classes
  $t,x,y$ in \cref{chrom-cofgl} are in degree $2$, the power series $f
  \ce \mu^*(t)$ must have coefficients in the even degrees $E^{2*}$,
  which of course do form a commutative ring.
\end{remark}

\begin{examples}
  \label{chrom-fglex}
  We give the two most basic examples (over any commutative ring $R$):
  \begin{enumerate}
  \item First we have the \emph{additive formal group law} $f(x,y) =
    x+y$. We saw above that this arises is stable homotopy theory via
    singular cohomology.
  \item Second we have the \emph{multiplicative formal group law}
    $f(x,y) = x+y+xy = (1+x)(1+y)-1$. One should think of this really
    as multiplication $xy$, except with the identity shifted to $0$
    rather than $1$. One can show without much trouble that this
    arises as the formal group law associated to complex K-theory.
  \end{enumerate}
\end{examples}

It turns out that this assignment
\[
\l\{ \text{complex-oriented cohomology theories} \r\} \to
\l\{ \text{formal group laws} \r\}
\]
defined by \cref{chrom-cofgl} is an absurdly interesting one to
consider. Maybe one is already astounded by the fact that the examples
in \cref{chrom-coex} and \cref{chrom-fglex} are entirely parallel,
i.e. the two simplest formal group laws correspond to the two simplest
(in some sense, maybe historical or pedagogical) cohomology theories;
I certainly still am! This construction is some sort of approximation
of topological objects---cohomology theories, i.e. spectra---by
algebraic objects---formal group laws. This is precisely the sort of
machine that algebraic topology is all about, but this is a case of
the algebraic approximation retaining a miraculous amount of
information about the topology. Chromatic homotopy theory essentially
refers to the business of understanding just how good this
approximation is, and we'll review some of the key aspects of this
theory in the following two subsections. Just as this one is, the next
two subsections have titles of the form ``$A$ \& $B$''; in all three
cases $A$ is some concept in the theory of formal groups and $B$ the
avatar of $A$ in stable homotopy theory.

\subsection{The Lazard ring \& complex bordism}
\label{chrom-lazmu}

The first important observation to make about formal group laws is
that there's a universal one.

\begin{construction}
  \label{chrom-univfgl}
  There 
\end{construction}

\subsection{Stratifying by height \& Morava K- and E-theories}
\label{chrom-strat}