\section{Chromatic homotopy theory}
\label{chrom}

\newcommand{\CP}{\lC\lP}

In this section we lay out some of the fundamental ideas in chromatic
homotopy theory, enough that we can somewhat safely speak about the
generalizations of character theory alluded to in \cref{intro}. As
this section is intended primarily as background, it's essentially
void of proofs. The lecture notes
\cite{hopkins-coctalos,lurie-chromatic} are nice places to read about
these ideas in more depth and detail (and my debt to these sources in
this exposition will be clear).

\subsection{Formal group laws \& Chern classes}

Our starting point is the theory of Chern classes associated to
complex vector bundles. Actually, let's just focus on (complex) line
bundles $L \to X$, in which case all that's of interest is the first
Chern class $c_1(L) \in \rH^2(X) \ce \rH^2(X;\lZ)$. Recall two facts
about this situation:
\begin{itemize}
\item Chern classes are natural: the Chern class $c_1(f^*L)$ of the
  pullback of a line bundle $L \to X$ in a map $f \c Y \to X$ is the
  pullback $f^*(c_1(L)) \in \rH^2(Y)$ of the Chern class of $L$.
\item There is a universal line bundle: letting $\cO(1)$ denote the
  tautological line bundle over the space $\CP^\infty \iso \rB\rU(1)$,
  there is a natural bijection between homotopy classes of maps
  $f \c X \to \CP^\infty$ and isomorphism classes of line bundles on
  $X$, given by associating to such a map $f$ the pullback
  $f^*(\cO(1))$ of the tautological bundle. That is, the functor
  associating to a space $X$ the set of isomorphism classes of line
  bundles on it is representable (in the homotopy category of spaces)
  by $\CP^\infty$, and the tautological bundle $\cO(1)$ is truly
  tautological: it corresponds to the identity map
  $\id \c \CP^\infty \to \CP^\infty$.
\end{itemize}
Of course Chern classes are also isomorphism-invariant, so it follows
from these facts that the Chern classes of \emph{any} line bundles $L$
is determined by a choice of Chern class
$c_1(\cO(1)) \in \rH^2(\CP^\infty)$. Now, recall that the cohomology
ring of $\CP^\infty$ is given by\footnote{It may be more common to
  write $\rH^*(\CP^\infty) \iso \lZ[t]$, i.e. to use the polynomial
  ring rather than the power series ring. This is just a convention:
  does one want to assemble the graded pieces of $\rH^*(X)$ by taking
  their direct product or their direct sum? In other words, does one
  want to consider infinite or just finite sums of homogenous elements
  in the ring $\rH^*(X)$? In just a bit we'll be considering
  cohomology theories $E$ other than singular cohomology, in
  particular ones in which the coefficient ring $E^* \ce E^*(\pt)$ is
  not concentrated in degree $0$, but may in fact be nonzero in
  infinitely many degrees. But even in this more general situation,
  we'd like to think of $E^*$ as some sort of base ring of
  coefficients, in which case we may have infinite sums in the ring
  $E^*(X)$ still living in some finite graded piece. It will thus be
  more natural to consider the power series ring $E^* \ldb t \rdb$,
  and so we begin with this convention here.}
\[
\rH^*(\CP^\infty) \iso \lZ \ldb t \rdb,
\]
where the generator lives in degree $2$, i.e.
$t \in \rH^2(\CP^\infty)$. So of course then the interesting choice to
make is $c_1(\cO(1)) \ce t$. But note that $t$ is really only
well-defined up to a unit in $\lZ$, i.e. up to sign! So to be precise
we need to fix such a generator and then declare this to be
$c_1(\cO(1))$.

Now, the set of isomorphism classes of line bundles on a space $X$ is
often referred to as its Picard group; the name isn't so important
right now, but it is important that it's a group. Recall that the
operation is given by tensor product of line bundles. So it's natural
to ask: given two line bundles $L$ and $L'$ on $X$, can we express
$c_1(L \otimes L')$ in terms of $c_1(L)$ and $c_1(L')$? In fact one
can do so quite easily:
\begin{equation}
  \label{chrom-cherntensor}
    c_1(L \otimes L') = c_1(L) + c_1(L') \in \rH^2(X).
\end{equation}

Since there is universal line bundle $\cO(1)$, proving this boils down
to proving it for the universal example. The universal \emph{pair} of
line bundles is given by the bundles $\pi_1^*\cO(1),\pi_2^*\cO(1)$ on
$\CP^\infty \times \CP^\infty$, where
$\pi_1,\pi_2 \c \CP^\infty \times \CP^\infty \to \CP^\infty$ denote
the projections. So the formula \cref{chrom-cherntensor} holds if and
only if it holds for this universal example, that is if
\[
c_1(\pi_1^*\cO(1) \otimes \pi_2^*\cO(1)) =
c_1(\pi_1^*\cO(1)) + c_1(\pi_2^*\cO(1)) \in
\rH^2(\CP^\infty \times \CP^\infty).
\]
To see that this is true, we observe that the canonical map
$\iota \c \CP^\infty \vee \CP^\infty \to \CP^\infty \times \CP^\infty$
induces an isomorphism in $\rH^2$,
\begin{equation}
\label{chrom-cherntensorpf}
\iota^* \c \rH^2(\CP^\infty \times \CP^\infty) \isoto
\rH^2(\CP^\infty \vee \CP^\infty),
\end{equation}
and that the pullback bundle
$\iota^*(\pi_1^*\cO(1) \otimes \pi_2^*\cO(1))$ is just a copy of
$\cO(1)$ over each summand of $\CP^\infty \vee \CP^\infty$.

We can alternatively view this universal example as follows: by the
Yoneda lemma, the tensor product operation on line bundles corresponds
to some kind of multiplication map
$\mu \c \CP^\infty \times \CP^\infty \to \CP^\infty$. In cohomology
this gives a map
\[
\lZ \ldb t \rdb \iso
\rH^*(\CP^\infty) \lblto{\mu^*}
\rH^*(\CP^\infty \times \CP^\infty) \iso
\lZ \ldb x,y \rdb,
\]
using the Kunneth isomorphism on the right-hand side, so
$x = \pi_1^*(t)$ and $y = \pi_2^*(t)$. By definition we'll have
$\mu^*\cO(1) \iso \pi_1^*\cO(1) \otimes \pi_2^*\cO(1)$.  So we end up
with
\[
\mu^*(t) =
\mu^*(c_1(\cO(1))) =
c_1(\mu^*\cO(1)) =
c_1(\pi_1^*\cO(1) \otimes \pi_2^*\cO(1))
\]
and
\begin{align*}
x + y &=
\pi_1^*(t) +  \pi_2^*(t) =
\pi_1^*(c_1(\cO(1))) +  \pi_2^*(c_1(\cO(1))) \\ &=
c_1(\pi_1^*\cO(1)) + c_1(\pi_2^*\cO(1)).
\end{align*}
Therefore we conclude that the formula \cref{chrom-cherntensor} is
equivalent to the formula $\mu^*(t) = x + y$.

Somebody interested in stable homotopy theory might now ask: what
precisely did we use about singular cohomology in the above? We
observe that the essential facts were the computations of
$\rH^*(\CP^\infty)$ and $\rH^*(\CP^\infty \times \CP^\infty)$ as power
series rings. This motivates the following definition.\footnote{One
  can also give alternative, somewhat more elementary (but equivalent)
  definitions of complex-orientability than the one given here, but
  this is the one which most easily and succinctly fits into the
  motivation being given here.}

\begin{definition}
  \label{chrom-complor}
  A multiplicative cohomology theory, i.e. a (homotopy) ring spectrum,
  $E$ is said to be \emph{complex-orientable} if the Atiyah-Hirzebruch
  spectral sequence
  \[
  \rH^p(\CP^\infty;E^q(\pt)) \Rightarrow E^{p+q}(\CP^\infty)
  \]
  degenerates at its second page. By the standard computations in
  singular cohomology, this condition implies that there is an
  isomorphism $E^*(\CP^\infty) \iso E^* \ldb t \rdb$, where
  $E^* \ce E^*(\pt)$. A choice of such an isomorphism, i.e. a choice
  of generator $t$ (which again is well-defined only up to the units
  of $E^0$) is referred to as a \emph{complex-orientation} of $E$. We
  say $E$ is \emph{complex-oriented} if it is equipped with a
  complex-orientation.

  In fact the degeneration of the spectral sequence for $\CP^\infty$
  forces the analagous degeneration for
  $\CP^\infty \times \CP^\infty$, whence a complex-orientation on $E$
  similarly determines an isomorphism
  $E^*(\CP^\infty \times \CP^\infty) \iso E^* \ldb x,y \rdb$.
\end{definition}

\begin{examples}
  \label{chrom-coex}
  We give the two most basic examples of complex-orientable cohomology
  theories, which are essentially the two motivating examples of
  cohomology theories in the first place:
  \begin{enumerate}
  \item Obviously singular cohomology $\rH$ is complex-orientable.
  \item One can show that complex K-theory $\rK$ is complex-orientable
    as well.
  \end{enumerate}
\end{examples}

So we can create a theory of Chern classes with values in any
complex-oriented cohomology theory $E$. But now there is no reason for
formula \cref{chrom-cherntensor} to hold in general. Its validity in
singular cohomology relies on the fact \cref{chrom-cherntensorpf},
which relies on the coefficient ring $\rH^* \iso \lZ$ being
concentrated in degree $0$; this of course is not true in general,
e.g. for complex K-theory, Bott periodicity tells us
$\rK^* \iso \lZ[\beta,\beta^{-1}]$ with $\beta$ in degree
$-2$. However, we can still say something about the Chern class of a
tensor product with values in $E^*$: it is still determined by the
universal example, which is still determined by the map in cohomology
\begin{equation}
\label{chrom-cofgl}
E^* \ldb t \rdb \iso
E^*(\CP^\infty) \lblto{\mu^*}
E^*(\CP^\infty \times \CP^\infty) \iso
E^* \ldb x,y \rdb,
\end{equation}
and again the formula for the Chern class of a tensor product will be
given by the power series $f(x,y) \ce \mu^*(t) \in E^* \ldb x,y \rdb$.
The fact that the tensor product of line bundles is unital,
commutative, and associative (all up to isomorphism) implies that $f$
is a special kind of power series in two variables, which we can think
of as an infinitesimal version of a group operation.

\begin{definition}
  \label{chrom-fgl}
  A \emph{formal group law} over a commutative ring $R$ is a power
  series $f \in R \ldb x,y \rdb$ satisfying:
  \begin{enumerate}
  \item \label{chrom-fgl-u} $f(x,0) = x$ and $f(0,y) = y$;
  \item \label{chrom-fgl-s} $f(x,y) = f(y,x)$;
  \item \label{chrom-fgl-a} $f(f(x,y),z) = f(x,f(y,z))$.
  \end{enumerate}
\end{definition}

\begin{remark}
  \label{chrom-grcomm}
  Note that we'll always speak about formal group laws over
  \emph{commutative} rings, but the coefficient ring $E^*$ of a
  multiplicative cohomology theory is \emph{graded-commutative}. This
  is no problem here: since $\mu^*$ respects grading and the classes
  $t,x,y$ in \cref{chrom-cofgl} are in degree $2$, the power series $f
  \ce \mu^*(t)$ must have coefficients in the even degrees $E^{2*}$,
  which of course do form a commutative ring.
\end{remark}

\begin{examples}
  \label{chrom-fglex}
  We give the two most basic examples (over any commutative ring $R$):
  \begin{enumerate}
  \item First we have the \emph{additive formal group law},
    $f(x,y) = x+y$. We saw above that this arises in stable homotopy
    theory via singular cohomology.
  \item Second we have the \emph{multiplicative formal group law},
    $f(x,y) = x+y+xy = (1+x)(1+y)-1$. One should think of this really
    as multiplication $xy$, except with the identity shifted to $0$
    rather than $1$. One can show without much trouble that this
    arises as the formal group law associated to complex K-theory.
  \end{enumerate}
\end{examples}

It turns out that this assignment
\[
\l\{ \text{complex-oriented cohomology theories} \r\} \to
\l\{ \text{formal group laws} \r\}
\]
defined by \cref{chrom-cofgl} is an absurdly interesting one to
consider. Maybe one is already astounded by the fact that the examples
in \cref{chrom-coex} and \cref{chrom-fglex} are entirely parallel,
i.e. the two simplest formal group laws correspond to the two simplest
(in some sense, maybe historical or pedagogical) cohomology theories;
I certainly still am! This construction is some sort of approximation
of topological objects---cohomology theories, i.e. spectra---by
algebraic objects---formal group laws. This is precisely the sort of
machine that algebraic topology is all about, but this is a case of
the algebraic approximation retaining a miraculous amount of
information about the topology. Chromatic homotopy theory essentially
refers to the business of understanding just how good this
approximation is, and we'll review some of the key aspects of this
theory in the following two subsections. Just as this one is, the next
two subsections have titles of the form ``$A$ \& $B$''; in all three
cases $A$ is some concept in the theory of formal groups and $B$ is
the avatar of $A$ in stable homotopy theory.

\subsection{The Lazard ring \& complex bordism}
\label{chrom-lazmu}

\begin{nothing}
  \label{chrom-univfgl}
  The first important observation to make about formal group laws is
  that there's a universal one. A formal group law is a power series
  \[
  f(x,y) = \sum_{i,j} a_{i,j} x^i y^j
  \]
  with coefficients in a ring $R$, satisfiying three conditions. The
  power series is of course formally determined by the coefficients
  $a_{i,j}$, and the conditions can be expressed purely in terms of
  the coefficients as well:
  \begin{enumerate}
  \item that $f(x,0) = x$ is equivalent to $a_{i,0}$ being $1$ for
    $i = 1$ and $0$ otherwise, and similarly for $f(0,y) = y$;
  \item that $f(x,y) = f(y,x)$ is equivalent to $a_{i,j} = a_{j,i}$;
  \item that $f(x,f(y,z)) = f(f(x,y),z)$ is again equivalent to
    certain integer polynomial relations among the coefficients
    $a_{i,j}$, but these are more complicated and omitted here.
  \end{enumerate}
  We conclude that there is some ideal $I \subseteq \lZ[a_{i,j}]$ such
  that specifying a formal group law over a ring $R$ is equivalent to
  specifying a morphism of rings $\lZ[a_{i,j}]/I \to R$. That is, the
  formal group law $\sum_{i,j} a_{i,j} x^i y^j$ over the ring
  $\rL \ce \lZ[a_{i,j}]/I$ is the universal example of a formal group
  law. For example, we can rephrase the discussion of the previous
  section by saying that a complex-orientation on a ring spectrum $E$
  determines a morphism of rings $\rL \to E^*$. In fact, there is a
  natural (even) grading on $\rL$ so that this is a morphism of graded
  rings. The (graded) ring $\rL$ is known as the \emph{Lazard ring}.
\end{nothing}

\begin{nothing}
  \label{chrom-mu}
  It turns out that there is also a universal example of a
  complex-oriented cohomology theory, known as \emph{complex bordism}
  and denoted $\MU$. (As the name suggests, $\MU$ is connected to
  (co)bordisms of manifolds, but we have no time to discuss its
  origins or construction.) More precisely, there is a bijection
  between complex-orientations of a ring spectrum $E$ and homotopy
  classes of maps of ring spectra $\MU \to E$. In particular, there is
  a \emph{canonical} complex-orientation of $\MU$, corresponding to
  the identity map $\id \c \MU \to \MU$.
\end{nothing}

We have now given two universal examples: a complex-orientation of a
ring spectrum $E$ canonically determines morphisms $\MU \to E$ and
$\rL \to E^*$. It is hopefully natural to guess the following result
then.

\begin{theorem}[Quillen]
  \label{chrom-quillen}
  The map $L \to \MU^*$ determined by the canonical
  complex-orientation of $\MU$ is an isomorphism. In other words, the
  formal group law associated to $\MU$ is the universal one, and the
  map $\rL \iso \MU^* \to E^*$ induced by a complex-orientation
  $\MU \to E$ of a ring spectrum $E$ is precisely the map classifying
  the formal group law on $E^*$.
\end{theorem}

\begin{remark}
  \label{chrom-quillenhard}
  Note that we said it was natural to guess the previous result, but
  then labeled it a theorem and attributed it to Quillen. Indeed, the
  result isn't formal or easy, but an incredibly deep computation, and
  sort of miraculous!  Again, we have no time here to delve into the
  details.
\end{remark}

One reason to get really excited about \cref{chrom-quillen} is that it
inspires a method for inverting the process of extracting formal group
laws from complex-oriented cohomology theories. That is, if we are
given a formal group law over some ring $R$, classified by a map
$\MU^* \iso \rL \to R$, we could try to define a complex-oriented
cohomology theory $E$ with exactly this associated formal group law by
defining
\[
E^*(X) \ce \MU^*(X) \otimes_\rL R
\]
(for finite CW complexes $X$). But of course this won't always define
a cohomology theory. A priori, to retain the necessary exact sequences
we would need to assume the map $\rL \to R$ is flat. However, this is
quite a stringent condition. Another important theorem tells us that
the Lazard ring $\rL$ is isomorphic to an infinite polynomial ring
$\lZ[b_1,b_2,\ldots]$, and flatness over such a large ring is indeed a
rather limiting hypothesis (maybe this is easier to understand
geometrically: imagine trying to be flat over an infinite-dimensional
affine space!).

Luckily, it turns out that we can get away with a significantly weaker
flatness hypothesis to obtain a cohomology theory. To state this
correctly we should introduce formal groups, the more invariant,
coordinate-free object underlying formal group laws.

\begin{notation}
  \label{chrom-algr}
  If $R$ is a ring, let $\Alg_R$ denote the category of $R$-algebras.
\end{notation}

Suppose we are given a formal group law $f \in R \ldb x,y \rdb$. We
are supposed to think of this as some kind of group operation, but
it's a power series so it doesn't quite make sense to actually apply
the operation to elements of $R$. However, we can apply it to
nilpotent elements of $R$. More precisely, if
$\Nil \c \Alg_R \to \Set$ denotes the functor sending an $R$-algebra
to its set of nilpotent elements, then $f$ defines a lift of $\Nil$ to
a functor to abelian groups, $\lG_f \c \Alg_R \to \Ab$. Note that
$\Nil$ is the colimit of the functors corepresented by the
$R$-algebras $R[t]/(t^n)$. So, algebro-geometrically $\Nil$ just
corresponds to the formal scheme
$\Spf(R\ldb t \rdb) \iso \colim_n \Spec(R[t]/(t^n))$, and the formal
group law $f$ determines a group sctructure on this formal
scheme. More generally, we make the following definition.

\begin{definition}
  \label{chrom-fg}
  A \emph{formal group} over a ring $R$ is a functor
  $\lG \c \Alg_R \to \Ab$ which is:
  \begin{enumerate}
  \item a sheaf with respect to the Zariski topology on $\Alg_R$;
  \item Zariski-locally isomorphic to functors of the form $\lG_f$,
    for $f$ a formal group law.
  \end{enumerate}
\end{definition}

Perhaps this definition is a bit opaque, so let's elaborate for a
little while. We said that formal groups are a more invariant notion
than formal group laws, but how exactly? In other words, given two
formal group laws $f,f'$ over a ring $R$, when are the associated
formal groups $\lG_f, \lG_{f'}$ isomorphic? One can show that the
formal groups are isomorphic precisely when there is a
``change-of-variable'' relating $f$ and $f'$, i.e. an invertible power
series $g \in R \ldb t \rdb$ such that $f(g(x),g(y)) = g(f'(x,y))$. So
we can think alternatively think of a formal group over $R$ as the
following data:
\begin{itemize}
\item an open covering of $R$, i.e. elements $r_1,\ldots,r_n \in R$
  such that $(r_1,\ldots,r_n) = (1) = R$;
\item formal group laws $f_i$ over each localization $R[r_i^{-1}]$;
\item changes-of-variable $g_{i,j} \in R[(r_ir_j)^{-1}] \ldb t \rdb$
  relating $f_i$ and $f_j$, i.e. such that
  \[
  g_{i,j}f_i(g_{i,j}^{-1}(x),g_{i,j}^{-1}(y)) = f_j(x,y) \in
  R[(r_ir_j)^{-1}] \ldb x,y \rdb
  \]
\item which are coherent in that there are identifications
  $g_{i,k} = g_{j,k} \circ g_{i,j}$ over
  $R[(r_ir_jr_k)^{-1}] \ldb t \rdb$.
\end{itemize}

This is just the natural way to glue together formal group laws along
changes-of-variable. This can also be thought of naturally in the
language of stacks. We have seen that there is a ``moduli stack of
formal group laws'' $\cM_\FGL$, i.e. an algebro-geometric object such
that maps $\Spec(R) \to \cM_\FGL$ are in bijection with formal group
laws over $R$: it is just the affine scheme $\Spec(\rL)$ associated to
the Lazard ring. Then change-of-variable can be encoded as the action
of a group scheme $G$ (paramterizing invertible power series
$g(t) = b_1t + b_2t^2 + \cdots$) on the scheme $\Spec(L)$
(paramterizing formal group laws), the action sending a formal group
law $f(x,y)$ to the formal group law $g(f(g^{-1}(x),g^{-1}(y))$. Hence
the ``moduli stack of formal groups'' should be defined to be the
quotient stack $\cM_{\FG} \ce \Spec(\rL) / G$. And indeed when one
unwraps all of the definitions, one now has a groupoid of maps
$\Spec(R) \to \cM_\FG$, which is precisely the groupoid of formal
groups over $R$, as defined above. In this language, the quotient map
$\Spec(\rL) \to \cM_\FG$ corresponds to taking the formal group
underlying a formal group law.

Let us now return to the problem of defining a cohomology theory
$E^*(X) \ce \MU^*(X) \otimes_\rL R$ given a formal group law over
$R$. As we said above, one might expect that we need flatness of
$\rL \to R$, i.e. of $\Spec(R) \to \Spec(\rL)$, to do so. But in fact
one can show that it suffices that the composite
$\Spec(R) \to \Spec(\rL) \to \cM_\FG$ be flat. This motivates the
following definition.

\begin{definition}
  \label{chrom-landexact}
  We say a formal group over a ring $R$ is \emph{Landweber-exact} if
  the classifying map $\Spec(R) \to \cM_\FG$ is flat. We say a formal
  group law over $R$ is Landweber-exact if its underlying formal group
  is.
\end{definition}

\begin{remark}
  \label{chrom-left}
  The terminology of Landweber-exactness originates from a (very
  useful) theorem of Landweber, the \emph{Landweber exact functor
    theorem}, which gives a simple algebraic (necessary and
  sufficient) criterion for a formal group law to be Landweber-exact.
\end{remark}

With more work, one can refine our two constructions extracting formal
group laws from complex-oriented cohomology theories and building
complex-oriented cohomology theories from Landweber exact formal group
laws as follows.

\begin{definition}
  \label{chrom-evenper}
  We say a ring spectrum $E$ is \emph{weakly even periodic} if
  \begin{enumerate}
  \item $E^i \iso 0$ for $i$ odd;
  \item the multiplication map $E^2 \otimes_{E^0} E^{-2} \to E^0$ is
    an isomorphism.
  \end{enumerate}
  The second condition implies that $E^2$ is an invertible
  $E^0$-module and that $E^{2k} \iso (E^2)^{\otimes k}$ for
  $k \in \lZ$. We say $E$ is \emph{even periodic} if it satisfies the
  following stronger condition:
  \begin{enumerate}
  \item[(b$'$)] there is an invertible element $\beta \in E^{-2}$, so
    that multiplication by $\beta$ determines an isomorphism
    $E^k \iso E^{k-2}$ for all $k \in \lZ$.
  \end{enumerate}
\end{definition}

\begin{remark}
  \label{chrom-evenperfacts}
  Any weakly even periodic ring spectrum $E$ is complex-orientable. If
  $E$ is even periodic, then it's easy to see that the associated
  formal group law over $E^*$ can be viewed simply as a formal group
  law over $E^0$. In fact, even if $E$ is just weakly even periodic,
  there is a formal group over $E^0$, i.e. a map
  $\Spec(E^0) \to \cM_\FG$, which when pulled back to $E^*$ is just
  the formal group $\Spec(E^*) \to \Spec(E^0) \to \cM_\FG$ underlying
  the formal group law over $E^*$.
\end{remark}

\begin{proposition}
  \label{chrom-evenperfg}
  Consider the category of Landweber-exact formal groups, i.e. the
  category of flat maps $\lG \c \Spec(R) \to \cM_\FG$. There is a
  functor $\lG \mapsto E_\lG$ from this category to the homotopy
  category of weakly even periodic ring spectra. Moreover, this
  functor is an equivalence of categories, with the inverse functor
  sending $E$ to the formal group over $E^0$ discussed in
  \cref{chrom-evenperfacts}.
\end{proposition}

%%%%%%%%%%%%%%%%%%%%%%%%%%%%%%%%%%%%%%%%%%%%%%%%%%%%%%%%%%%%%%%%%%%%%%

\subsection{Stratifying by height \& Morava K- and E-theories}
\label{chrom-strat}

In this subsection, we strengthen even further the intimate
relationship between stable homotopy theory and formal groups by
discussing some geometric structure of the moduli stack of formal
groups $\cM_\FG$, and how this geometric structure is reflected in
stable homotopy theory.

\begin{definitions}
  \label{chrom-nseries}
  Let $f$ be a formal group law over a ring $R$.
  \begin{enumerate}[leftmargin=*]
  \item We inductively define its \emph{$n$-series}
    $[n] \in R \ldb t \rdb$ of $f$ by $[0]_f(t) = 0$ and
    $[n](t) = f([n-1](t), t)$ for $n \in \lZ_{> 0}$.
  \item One can show that for a prime $p$, if $p=0$ in $R$ then the
    $p$-series of $f$ is either zero or has the form
    $[p](t) = rt^{p^n} + O(t^{p^n+1})$ for some nonzero $r \in R$.
    With a fixed prime $p$ understood, we denote the coefficient of
    $t^{p^n}$ in $[p](t)$ by $v_n$ for $n \ge 0$. We say $f$ has
    \emph{height $\ge n$} if $v_k = 0$ for $k < n$, and has
    \emph{height (exactly) $n$} if furthermore $v_n$ is invertible in
    $R$.
  \item One can show that height of a formal group law is invariant
    under change-of-variable, i.e. is actually a property of the
    underlying formal group. In fact, at each prime $p$ we may define
    a stratification of $\cM_\FG \times \Spec(\lZ_{(p)})$ by defining
    the closed substacks $\cM_\FG^{\ge n}$ parameterizing formal
    groups of height $\ge n$, whose locally closed strata
    $\cM_\FG^n \ce \cM_\FG^{\ge n} - \cM_\FG^{\ge n-1}$ are precisely
    the substacks parameterizing formal groups of height exactly $n$.
  \end{enumerate}
\end{definitions}

\begin{examples}
  \label{chrom-htex}
  \begin{enumerate}[leftmargin=*]
  \item It's easy to see that for any formal group law $f$ over a ring
    $R$, the coefficient of $t$ in $[n](t)$ is just $n$. In
    particular, $f$ has height $0$ if and only if $p$ is invertible in
    $R$, and $f$ has height $\ge 1$ if and only if $p = 0$ in $R$. In
    other words, $\cM_\FG^0 \iso \cM_\FG \times \Spec(\lQ)$ and
    $\cM_\FG^{\ge 1} \iso \cM_\FG \times \Spec(\lF_p)$.
  \item Consider the additive formal group law $f(x,y) = x + y$ over a
    ring $R$ with $p=0$. Its $p$-series is evidently $0$, so we say
    $f$ has \emph{infinite height}.
  \item Consider the multiplicative formal group law
    $f(x,y) = x + y + xy = (1+x)(1+y) - 1$ over a ring $R$ with
    $p=0$. Its $p$-series is $(1+t)^p - 1 = t^p$, so $f$ has height
    $1$.
  \end{enumerate}
\end{examples}

Now how does the notion of height appear in stable homotopy theory? We
can't give a complete answer to this question, which is central to
chromatic homotopy theory, but we can say a bit.

It is often fruitful to think of abelian groups as ``living over''
(the spectrum of) the integers, and to study them by localizing or
completing at one prime $p$ at a time. We can do the same thing for
spectra, but now there is more structure to consider than just the
integer primes. We have seen that the coefficient ring of the spectrum
$\MU$ is given by the Lazard ring $\rL$. Thus, for any the spectrum
$X$, the $\MU$-homology of $X$ is a module $\MU_*(X)$ over $\rL$,
i.e. a quasicoherent sheaf on $\Spec(\rL)$. However, more is true: the
action of the group scheme $G$ on $\Spec(\rL)$ described above in
\cref{chrom-lazmu} lifts naturally to an action on the sheaf
$\MU_*(X)$. Thus we may view $\MU_*(X)$ as a quasicoherent sheaf on
the quotient stack $\cM_\FG \iso \Spec(\rL) / G$. I.e. we may think of
$\MU$-homology as a functor from the stable homotopy category to the
category of quasicoherent sheaves on $\cM_\FG$, and therefore think of
the stable homotopy category as ``living over'' $\cM_\FG$.

So we should ask: after localizing at a prime $p$, is there a way of
further localizing or completing the stable homotopy category with
respect to the strata of $\cM_\FG$? Indeed there is, and describing
this allows us to introduce some central objects in stable homotopy
theory: the Morava K-theories and E-theories.

\begin{definition}
  \label{chrom-def}
  Let $\lG_0$ be a formal group over a field $\kappa$. A
  \emph{deformation} of $\lG_0$ is a local artin ring $A$ with residue
  field $\kappa$, together with a formal group $\lG$ over $A$ which
  restricts to $\lG_0$ in the quotient map $A \to \kappa$. (Note that
  since $\kappa, A$ are both local, the formal groups $\lG_0,\lG$ are
  just the formal groups underlying certain formal group laws
  $f_0,f$.)
\end{definition}

\begin{proposition}
  \label{chrom-univdef}
  Let $\lG_0$ be a formal group of height $n$ over a perfect field
  $\kappa$ of characteristic $p$. Let $\rW(\kappa)$ denote the ring of
  Witt vectors of $\kappa$. Then there is a universal deformation
  $\lG$ of $\lG_0$ over the local artin ring
  $A \ce \rW(\kappa) \ldb v_1, \ldots, v_{n-1} \rdb$ with residue
  field $\kappa$. That is, for another local artin ring $A'$ with
  residue field $\kappa$, there is a natural bijection between the set
  of isomorphism classes of deformations of $\lG_0$ over $A'$ and the
  set $\Hom_{/\kappa}(A,A')$ of ring maps $A \to A'$ over $\kappa$.
\end{proposition}

\begin{nothing}
  \label{chrom-moravathy}
  As noted in \cref{chrom-def}, in \cref{chrom-univdef} one can think
  of the formal group $\lG_0$ simply as a formal group law $f_0$ of
  height $n$ over $\kappa$, and the universal deformation as a formal
  group law $f$ over
  $A \ce \rW(\kappa) \ldb v_1, \ldots, v_{n-1} \rdb$.  Essentially by
  construction, one can show that the universal deformation $f$ will
  be Landweber-exact. Thus by \cref{chrom-evenperfg} we have an even
  periodic\footnote{It is automatically even periodic, not just weakly
    even periodic, since the base ring $A$ is local.} spectrum $E$
  whose associated formal group law is $f$, which we call the
  \emph{Morava E-theory} associated to $\lG_0$. We can say much more
  though: a theorem of Goerss-Hopkins-Miller\footnote{Disclaimer: I
    have not studied this theorem at all, yet.} tells us that there is
  an essentially unique $\rE_\infty$-ring spectrum structure on
  $E$. For the remainder, we will always think of Morava E-theory as
  an $\rE_\infty$-ring spectrum.

  On the other hand, the orginal formal group law $f_0$ will not be
  Landweber exact. Nevertheless, we may associate a spectrum to it via
  Morava E-theory. Namely, we let $v_0 \ce p \in A$, and for
  $0 \le i < n$ define $M(i)$ to be the cofiber of the map of
  $E$-module spectra $E \to E$ given by multiplication by
  $v_i \in A \iso E^0$. We can then consider the tensor product
  $K \ce \bigotimes_{i=0}^{n-1} M(i)$ of $E$-module spectra, which is
  independent of the precise choice of generators
  $v_1,\ldots,v_n \in A$, and has coefficient ring
  $K^* \iso \kappa[\beta,\beta^{-1}]$, with $\beta \in K^{-2}$. (So,
  just as $\kappa$ is the quotient of the maximal ideal
  $(p,v_1,\ldots,v_{n-1}) \subseteq A$, we should think of $K$ as the
  quotient of $E$ by this maximal ideal, in some sense.) We call this
  spectrum $K$ the \emph{Morava K-theory} associated to $\lG_0$.
\end{nothing}

We now give a definition of localizing the stable homotopy category
with respect to a spectrum; localizing with respect to Morava K-theory
and E-theory will realize the localization/completion to the strata of
$\cM_\FG$ we were asking for above.

\begin{definition}
  \label{chrom-localization}
  Let $E$ be a spectrum. We say:
  \begin{itemize}
  \item a spectrum $X$ is \emph{$E$-acyclic} if the $E$-homology of
    $X$ vanishes, i.e. if $E_*(X) \iso 0$ or equivalently if the smash
    product $E \otimes X$ is $0$;
  \item a spectrum $Y$ is \emph{$E$-local} if any map $X \to Y$ is
    nullhomotopic when $X$ is an $E$-acyclic spectrum.
  \end{itemize}
  One can define an \emph{$E$-localization functor}
  $L_E \c \Spect \to \Spect$ on the $\infty$-category of spectra with
  the following properties:
  \begin{itemize}
  \item For any the spectrum $X$, the spectrum $L_E X$ is $E$-local.
  \item There is a natural transformation from the identity functor on
    $\Spect$ to $L_E$, i.e. for any spectrum $X$ there is a natural
    localization map $X \to L_E X$. Moreover, this map is an
    isomorphism in $E$-homology.
  \end{itemize}
\end{definition}

\begin{examples}
  Let $\lG_0$ be a formal group of height $n$ over a perfect field
  $\kappa$ of characteristic $p$. Let $E$ and $K$ denote the
  associated Morava E-theory and K-theory. Then the $E$-localization
  functor $L_E$ should be thought of as restriction to the open
  substack $\cM_\FG^{\le n}$ complementary to $\cM_\FG^{\ge n+1}$. And
  the $K$-localization functor $L_K$ should be thought of as
  completiong along the locally closed substack $\cM_\FG^n$. In fact,
  one can show that the localization functor $L_K$ really only depends
  on our chosen prime $p$ and the height $n$, and for the remainder we
  will simply denote this functor by $L_{K(n)}$. It is also common to
  abusively just call these spectra \emph{Morava E-theory and K-theory
    of height $n$}, and denote them $E(n)$ and $K(n)$, without
  specifying the field $\kappa$ or formal group $\lG_0$ (this is what
  we did in the introduction).
\end{examples}

Here ends our discussion of the prerequisite ideas from chromatic
homotopy theory that will be needed in what follows. We have certainly
omitted a great deal, but hopefully the reader at this point has
enough of a picture in mind that they can appreciate how our main
objects of study arise.

