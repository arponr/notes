\section{Abelian descent}
\label{abdesc}

In many cases of interest, a global spectrum is determined by its
values on abelian groups. This section is devoted to understanding
this phenomenon, which can be encoded rigorously as follows.

\begin{notation}
  \label{abdesc-alo}
  Let $\Alo$ denote the full subcategory of $\Glo$ spanned by the
  (finite) abelian groups.
\end{notation}

\begin{definition}
  \label{abdesc-abelian-descent}
  We say a global spectrum $E \in \Spect_\Glo$ satisfies \emph{abelian
    descent} if it is the right Kan extension of its restriction to
  $\Alo$, i.e. if the canonical map
  \[
  E(\lB G) \to \lim_{\lB A \in \Alo_{/\lB G}} E(\lB A)
  \]
  is an equivalence for all finite groups $G$.
\end{definition}

\subsection{Alternate characterizations}
\label{abdesc-alternate}

In this short subsection we unpack the definition
\cref{abdesc-abelian-descent} of abelian descent by giving a
tautological reformulation, and then giving (again fairly
tautological) reformulations of this reformulation. But all this
tautology is worth something! It will allow us to give (in the next
subsection) a more concrete way to check that a global spectrum
satisfies abelian descent, so just stay tuned for a moment.

\newcommand{\ab}{\mathrm{ab}}

\begin{definition}
  \label{abdesc-abelianization}
  For $X \in \Space_\Glo$ we define the \emph{abelianization} $X^\ab$
  of $X$ by
  \[
  X^\ab \ce \colim_{\lB A \in \Alo_{/X}} \lB A \in \Space_\Glo.
  \]
  Note that there is a canonical map $X^\ab \to X$. To save some
  parentheses, we'll write $\lB^\ab G$ in place of $(\lB G)^\ab$ for
  $G \in \Glo$.
\end{definition}

\begin{lemma}
  \label{abdesc-abelian-descent-abelianization}
  A global spectrum $E \in \Spect_\Glo$ satisfies abelian descent if
  and only if the canonical map $E(\lB G) \to E(\lB^\ab G)$ is an
  equivalence for all $G \in \Glo$.
\end{lemma}

\begin{proof}
  This is immediate from the definitions
  \cref{abdesc-abelian-descent,abdesc-abelianization}, as $E$ takes
  colimits in $\Space_\Glo$ to limits in $\Spect$.
\end{proof}

\begin{lemma}
  \label{abdesc-abelianization-alternate}
  Let $\ab \c \Glo \to \Alo$ be the functor sending a group $\lB H$ to
  its group-theoretic abelianization\footnote{Recall the
    abelianization of a group $H$ is the quotient by its commutator
    subgroup $H/[H,H]$.}  $\lB H^\ab$. For $X \in \Space_\Glo$ we have
  $X^\ab \iso X \circ \ab$, i.e. there are functorial equivalences
  $X^\ab(\lB H) \iso X(\lB H^\ab)$ for $H \in \Glo$.
\end{lemma}

\begin{proof}
  Let $X \in \Space_\Glo$ and $\lB H \in \Glo$. Observe that for any
  $\lB A \in \Alo$ the canonical map
  $\lB A(\lB H^\ab) \to \lB A(\lB H)$ is an equivalence. It follows
  that we have an equivalence
  \[
  X^\ab(\lB H) \iso
  \colim_{\lB A \in \Alo_{/X}} \lB A(\lB H) \iso
  \colim_{\lB A \in \Alo_{/X}} \lB A(\lB H^\ab),
  \]
  functorial in $\lB H$. But now the canonical map
  \[
  \colim_{\lB A \in \Alo_{/X}} \lB A(\lB H^\ab) \to X(\lB H^\ab)
  \]
  is clearly an equivalence since $\lB H^\ab \in \Alo$.
\end{proof}

\begin{lemma}
  \label{abdesc-abelianization-as-Gspace}
  Let $G$ be a finite group. Let $\Delta_G \dashv \Gamma_G$ be the
  adjunction constructed in \cref{global-gspace-adj}.
  \begin{enumerate}
  \item \label{abdesc-abelianization-gamma} For a subgroup
    $H \subseteq G$, the fixed-point space $\Gamma_G(\lB^\ab G)^H$ is
    contractible if $H$ is abelian and empty if $H$ is nonabelian.
  \item \label{abdesc-abelianization-delta} Conversely, if $X$ is a
    $G$-space such that $X^H$ is contractible if $H$ is abelian and
    empty if $H$ is nonabelian, then $\Delta_G(X)$ is equivalent to
    $\lB^\ab G \to \lB G$ in $(\Space_\Glo)_{/\lB G}$.
  \end{enumerate}
\end{lemma}

\begin{proof}
  Take a subgroup $H \subseteq G$. By
  \cref{global-gspace-adj-unwrap}\cref{global-gspace-adj-unwrap-right},
  $\Gamma_G(\lB^\ab G)^H$ is given by
  \[
  \lB^\ab G(\lB H) \times_{\lB G(\lB H)} \{i_H\} \iso
  \lB G(\lB H^\ab) \times_{\lB G(\lB H)} \{i_H\},
  \]
  where $i_H \in \lB G(H)$ corresponds to the inclusion $H \inj G$;
  this clearly is contractible when $H$ is abelian and empty when $H$
  is nonabelian, proving \cref{abdesc-abelianization-gamma}.

  Now let $X$ be a $G$-space with this same property: $X^H$ is
  contractible for $H$ abelian and empty for $H$ nonabelian. Recall
  $\Delta_G(X)$ is the canonical map
  $\delta_G(X) \to \delta_G(*) \iso \lB G$, where for a $G$-space $Y$,
  $\delta_G(Y)(\lB H)$ is the homotopy quotient
  \[
  \l(\coprod_{\phi \in \Hom(H,G)} Y^{\im(\phi)}\r)_{\rh G} \iso 
  \rB \l(\l(\coprod_{\phi \in \Hom(H,G)} Y^{\im(\phi)}\r)
  \sslash G \r)
  \]
  for $\lB H \in \Glo$. Since $X^{\im(\phi)}$ is only nonempty when
  $\im(\phi)$ is abelian, i.e. when $\phi$ factors through $H^\ab$, we
  have
  \[
  \delta_G(X)(H) \iso
  \l(\coprod_{\phi \in \Hom(H^\ab,G)} X^{\im(\phi)}\r)_{\rh G}.
  \]
  It follows from \cref{abdesc-abelianization-alternate} that
  $\delta_G(X) \to \lB G$ factors through the map
  $\lB^\ab G \to \lB G$; for $\lB H \in \Glo$ the resulting map
  $\delta_G(X) \to \lB^\ab G$ looks like
  \[
  \l(\coprod_{\phi \in \Hom(H^\ab,G)} X^{\im(\phi)}\r)_{\rh G} \iso
  \delta_G(X)(H) \to
  \lB^\ab G(H) \iso
  \l(\coprod_{\phi \in \Hom(H^\ab,G)} \pt \r)_{\rh G},
  \]
  induced by the unique maps $X^{\im(\phi)} \to \pt$. But these maps
  are all equivalences by our contracibility hypothesis, whence
  $\delta_G(X) \to \lB^\ab G$ is an equivalence (over $\lB G$), as
  desired.
\end{proof}

% \begin{remark}
%   \label{abdesc-abelianization-faithful}
%   Note that \cref{abdesc-abelianization-as-Gspace} in particular
%   implies $\Delta_G\Gamma_G(\lB^\ab G) \iso \lB^\ab G$, i.e. that
%   $\lB^\ab G \to\lB G$ is faithful. One can also show this more
%   directly. Let $H \in \Glo$ and $N \subseteq H$ a normal subgroup. We
%   want to show that the induced diagram
%   \[
%   \begin{tikzcd}
%     \lB^\ab G(H/N) \ar[r] \ar[d] &
%     \lB^\ab G(H) \ar[d] \\
%     \lB G(H/N) \ar[r] &
%     \lB G(H)
%   \end{tikzcd}
%   \]
%   is a pullback square. By \cref{abdesc-abelianization-alternate} we
%   can identify this with the diagram of 1-groupoids
%   \[
%   \begin{tikzcd}
%     \Hom((H/N)^\ab,G) \sslash G \ar[r] \ar[d] &
%     \Hom(H^\ab,G) \sslash G \ar[d] \\
%     \Hom(H/N,G) \sslash G \ar[r] &
%     \Hom(H,G) \sslash G.
%   \end{tikzcd}
%   \]
%   It is enough to that check this is a pullback square of 1-groupoids,
%   and this is straightforward given the explicit description of
%   pullback squares of 1-groupoids (see
%   e.g. \cite[\S6.3]{strickland-k(n)-duality}).
% \end{remark}

%%%%%%%%%%%%%%%%%%%%%%%%%%%%%%%%%%%%%%%%%%%%%%%%%%%%%%%%%%%%%%%%%%%%%%

\subsection{Complex-oriented descent}
\label{abdesc-codesc}

The takeaway of \cref{abdesc-alternate} is that whether or not a
global spectrum satisfies abelian descent comes down to comparing, for
each finite group $G$, what it assigns to the trivial $G$-space and
what it assigns to a $G$-space whose $H$-fixed point space is
contractible for $H \subseteq G$ abelian and empty for $H \subseteq G$
nonabelian. In this subsection we construct a more explicit model for
the latter, and use this to describe a practical method for proving
that various global spectra associated to complex-oriented cohomology
theories satisfy abelian descent. This idea is originally due to
Quillen \cite{quillen-equivariant-i}, and is key to our approach to
character theory.

\begin{notation}
  \label{abdesc-codesc-ntn}
  For the remainder of this subsection:
  \begin{enumerate}
  \item Fix a finite group $G$.
  \item Fix a faithful complex representation\footnote{Recall that
      such a thing always exists, e.g. the regular representation
      $\lC[G]$ always works.}  $\rho \c G \inj \Aut(V)$.
  \item Choosing a $G$-equivariant hermitian inner
    product\footnote{Recall that such a thing always exists, e.g. by
      averaging any ordinary hermitian inner product over $G$.} on and
    a basis of $V$, this determines an embedding into the unitary
    group $i \c G \inj \rU(k)$ for $k \ce \dim(V)$.
  \item \label{abdesc-flag-variety} Let $T \subseteq \rU(k)$ be a
    maximal torus. Let $F \ce \rU(k)/T$, which is a $G$-space via
    $i$. Recall that $F$ is the space of complete flags in $V$ (if you
    imagine $T$ as the subspace of unitary diagonal matrices, then $F$
    is evidently the space of $k$-tuples of orthogonal lines in $V$,
    which is the same as the space of complete flags).
  \end{enumerate}
\end{notation}

\begin{lemma}
  \label{abdesc-flag-abstab}
  For a subgroup $H \subseteq G$, the fixed-point space $F^H$ is
  nonempty if and only if $H$ is abelian.
\end{lemma}

\begin{proof}
  This is immediate from the theory of maximal tori. It's clear that
  $F^H$ is nonempty if and only if some conjugage $uHu^{-1}$ is
  contained in our maximal torus $T$, which is true if and only if $H$
  is abelian.
\end{proof}

\begin{notation}
  \label{abdesc-Espace}
  If $X$ is a space, let $\rE X$ denote the geometric realization of
  the simplicial space
  \[
  \simpl{X}{X \times X}{X \times X \times X}.
  \]
\end{notation}

\begin{lemma}
  \label{abdesc-Espace-type}
  Let $X$ be a space. If $X$ is empty then $\rE X$ is empty; if $X$ is
  nonempty then $\rE X$ is contractible.
\end{lemma}

\begin{proof}
  The claim is obvious for $X$ empty. For $X$ nonempty there is a
  standard way to construct contracting homotopies (see
  e.g. \cite[3.14]{dugger-primer}).
\end{proof}

\begin{lemma}
  \label{abdesc-BabG-construction}
  For a subgroup $H \subseteq G$, the fixed-point space $(\rE F)^H$ is
  contractible for $H$ abelian and empty for $H$ nonabelian. In other
  words, by \cref{abdesc-abelianization-as-Gspace}, $\rE F$ is a model
  for $\lB^\ab G$.
\end{lemma}

\begin{proof}
  Since geometric realization of simplicial spaces commutes with
  finite limits, in particular taking $H$-fixed-points, this is
  immediate from \cref{abdesc-flag-abstab,abdesc-Espace-type}.
\end{proof}

\begin{lemma}
  \label{abdesc-flag-faithflat}
  Let $E_G$ be a ring $G$-spectrum. Suppose $E_G^* \to E_G^*(F)$ is a
  faithfully flat map of rings, and that the canonical map
  $E_G^*(F)^{\otimes n} \to E_G^*(F^{\times n})$ is an isomorphism for
  all $n \ge 1$. Then the canonical map $E_G^* \to E_G^*(\rE F)$ is an
  isomorphism.
\end{lemma}

\begin{proof}
  Consider the function $G$-spectrum $E_G(\rE F)$. Since $\rE F$ is
  defined by a geometric realization, this function spectrum is the
  the totalization of the cosimplicial $G$-spectrum
  \[
  \cosimpl{E_G(F)}{E_G(F \times F)}{E_G(F \times F \times F)}.
  \]
  Now, the cohomology ring $E_G^*(\rE F)$ is given by the homotopy
  groups of the $G$-fixed points $E_G(\rE F)^G$ of this function
  spectrum. Viewing $G$-spectra as presheaves of spectra on $\Orb(G)$,
  taking $G$-fixed points amounts to evaluating on the orbit $G$, and
  so we see that $E_G(\rE F)^G$ is the totalization of the
  cosimplicial spectrum
  \[
  \cosimpl{E_G(F)^G}{E_G(F \times F)^G}{E_G(F \times F \times F)^G}.
  \]  
  We apply the Bousfield-Kan spectral sequence arising from this
  cosimplicial spectrum, which will converge to the homotopy groups of
  the totalization, i.e. $E_G^*(\rE F)$, and which has second page
  given by the complex
  \[
  \cosimpl{E_G^*(F)}{E_G^*(F \times F))}{E_G^*(F \times F \times F)}.
  \]
  By hypothesis we have $E_G^*(F^n) \iso E_G^*(F)^{\otimes n}$, so
  this is just the Amitsur complex of the map $E_G^* \to E_G^*(F)$.
  But since this map is faithfully flat, the homology of this complex
  is just $E_G^*$ concentrated in degree $0$. It follows that the
  spectral sequence degenerates and gives the desired isomorphism
  $E_G^* \isoto E_G^*(\rE F)$.
\end{proof}

\begin{lemma}[{\cite[\S2.2--2.3]{hkr-char}}]
  \label{abdesc-co-flag-faithflat}
  Let $E$ be a complex-orientable ring spectrum.
  \begin{enumerate}
  \item \label{abdesc-co-flag-bundle} Let $\cV \to X$ be a vector
    bundle, and $\cF \to X$ the associated bundle of complete flags in
    $\cV$. Then $E^*(\cF)$ is a finite free module over
    $E^*(X)$.
  \item \label{abdesc-co-flag-pullback} Moreover, suppose we have a
    map $Y \to X$. Let $\cV_Y$ be the pullback of $\cV$ to $Y$ and
    $\cF_Y$ the bundle of complete flags in $\cV_Y$. Then the
    canonical map
    \[
    E^*(Y) \otimes_{E^*(X)} E^*(\cF) \to E^*(\cF_Y)
    \]
    is an isomorphism.
  \item \label{abdesc-co-flag-interest} Let $E_G$ be the ring
    $G$-spectrum representing the Borel $G$-equivariant cohomology
    associated to $E$. Then $E_G^*(F)$ is a finite free module over
    $E_G^*$, and the canonical map
    $E_G^*(F)^{\otimes n} \to E_G^*(F^{\times n})$ is an isomorphism
    for all $n \ge 1$.
  \end{enumerate}
\end{lemma}

\begin{proof}[Sketch of proof]
  Recall that $E_G$ is defined such that
  $E_G^*(F) \iso E^*(\rE G \times_G F)$ and $E_G^* \iso E^*(\rB G)$.
  Noting that $\rE G \times_G F \to BG$ is the bundle of complete
  flags in the vector bundle $EG \times_G V \to BG$, we see that the
  first assertion of \cref{abdesc-co-flag-interest} follows from
  \cref{abdesc-co-flag-bundle}. The second then follows from
  \cref{abdesc-co-flag-pullback}, where we inductively pull back via
  the map
  $\rE G \times_G F^{\times n} \to \rE G \times_G F^{\times(n-1)}$.

  We prove \cref{abdesc-co-flag-bundle} by induction on the rank of
  the vector bundle $\cV \to X$. Of course the claim is trivial when
  the rank is $0$, giving us our base case, so assume the rank is
  $n \ge 1$. Consider the associated projective bundle $\lP\cV \to X$.
  The pullback bundle $\cV' \ce \lP\cV \times_X \cV$ over $\lP V$
  contains a canonical line sub-bundle $\cL$. Contemplating the
  quotient bundle $\cW \ce \cV'/\cL$, we see that the space of
  complete flags in $\cW$ can be identified with the space $\cF$ of
  complete flags in $\cV$. Thus the map $E^*(X) \to E^*(\cF)$ factors
  as the composition $E^*(X) \to E^*(\lP\cV) \to E^*(\cF)$. Since
  $\cW \to \lP\cV$ has rank $n-1$, we know by induction that
  $E^*(\cF)$ is a finite free $E^*(\lP\cV)$-module. So we just need
  that $E^*(\lP\cV)$ is a finite free $E^*(X)$-module. This is a
  standard application of the Leray-Hirsch theorem, the point being
  that $\lP V \to X$ is a bundle with fiber $\lC\lP^n$ and a complex
  orientation of $E$ induces an isomorphism
  $E^*(\lC\lP^n) \iso E^*[x]/(x^n)$, which of course is finite
  free. Statement \cref{abdesc-co-flag-pullback} will then follow from
  the naturality of this argument.
\end{proof}

Our work in this section culminates in the following result.

\begin{proposition}
  \label{abdesc-main}
  Let $E$ be a complex-orientable $\rE_\infty$-ring spectrum. Let $E
  \to C$ be a flat map of $\rE_\infty$-rings. Then the global
  spectrum $C \otimes E \in \Spect_\Glo$ given by the presheaf $\lB G
  \mapsto C \otimes_E E(\rB G)$ satisfies abelian descent.
\end{proposition}

\begin{proof}
  By \cref{abdesc-abelian-descent-abelianization} it suffices to show
  $(C \otimes E)(\lB G) \to (C \otimes E)(\lB^\ab G)$ is an
  equivalence for any fixed finite group $G$. By
  \cref{abdesc-abelianization-as-Gspace,abdesc-BabG-construction} it
  suffice to show that the map
  $(C \otimes E)_G^* \to (C \otimes E)_G^*(\rE F)$ is an
  isomorphism. By \cref{abdesc-flag-faithflat} it suffices to show
  that $(C \otimes E)_G^* \to (C \otimes E)_G^*(F)$ is faithfully flat
  and that
  $(C \otimes E)_G^*(F)^{\otimes n} \to (C \otimes E)_G^*(F^{\times
    n})$
  is an isomorphism for all $n \ge 1$. Since $F$ is a compact
  $G$-manifold, hence equivalent to a finite $G$-space, and $C$ is
  flat over $E$, these maps are just given by
  $C^* \otimes_{E^*} E_G^* \to C^* \otimes_{E^*} E_G^*(F)$ and
  $C_* \otimes_{E^*} E_G^*(F)^{\otimes n} \to C^* \otimes_{E^*}
  E_G^*(F^{\times n})$,
  respectively. We are then done by
  \cref{abdesc-co-flag-faithflat}\cref{abdesc-co-flag-interest}.
\end{proof}
