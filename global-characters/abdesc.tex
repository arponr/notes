\section{Abelian descent}
\label{abdesc}

In many cases of interest, a global spectrum is determined by its
values on abelian groups. This section is devoted to understanding
this phenomenon, which can be encoded rigorously as follows.

\begin{notation}
  \label{abdesc-alo}
  Let $\Alo$ denote the full subcategory of $\Glo$ spanned by the
  abelian groups.
\end{notation}

\begin{definition}
  \label{abdesc-abelian-descent}
  We say a global spectrum $E \in \Sp_\Glo$ satisfies \emph{abelian
    descent} if it is the right Kan extension of its restriction to
  $\Alo$, i.e. if the canonical map
  \[
  E(\lB G) \to \lim_{A \in \Alo_{/\lB G}} E(\lB A)
  \]
  is an equivalence for all $G \in \Glo$.
\end{definition}

\subsection{Alternate characterizations}
\label{abdesc-alternate}

In this short subsection we unpack the definition
\cref{abdesc-abelian-descent} of abelian descent by giving a
tautological reformulation, and then giving (again fairly
tautological) reformulations of this reformulation. But all this
tautology is worth something! It will allow us to give (in the next
subsection) more concrete ways to check that a global spectrum
satisfies abelian descent, so just stay tuned for a moment.

\newcommand{\ab}{\mathrm{ab}}

\begin{definition}
  \label{abdesc-abelianization}
  For $X \in \Top_\Glo$ we define the \emph{abelianization} $X^\ab$ of
  $X$ to be the left Kan extension of its restriction to $\Alo$,
  i.e.
  \[
  X^\ab \ce \colim_{A \in \Alo_{/X}} \lB A \in \Top_\Glo.
  \]
  Note that there is a canonical map $X^\ab \to X$. To save some
  parentheses, we'll write $\lB^\ab G$ in place of $(\lB G)^\ab$ for
  $G \in \Glo$.
\end{definition}

\begin{lemma}
  \label{abdesc-abelian-descent-abelianization}
  A global spectrum $E \in \Sp_\Glo$ satisfies abelian descent if and
  only if the canonical map $E(\lB G) \to E(\lB^\ab G)$ is an
  equivalence for all $G \in \Glo$.
\end{lemma}

\begin{proof}
  This is immediate from the definitions
  \cref{abdesc-abelian-descent,abdesc-abelianization}, as $E$ takes
  colimits in $\Top_\Glo$ to limits in $\Sp$.
\end{proof}

\begin{lemma}
  \label{abdesc-abelianization-alternate}
  Let $\ab \c \Glo \to \Alo$ be the functor sending a group $H$ to
  its group-theoretic abelianization\footnote{Recall the
    abelianization of a group $H$ is the quotient by its commutator
    subgroup $H/[H,H]$.}  $H^\ab$. For $X \in \Top_\Glo$ we have
  $X^\ab \iso X \circ \ab$, i.e. there are functorial equivalences
  $X^\ab(H) \iso X(H^\ab)$ for $H \in \Glo$.
\end{lemma}

\begin{proof}
  Let $X \in \Top_\Glo$ and $H \in \Glo$. Observe that for any
  $A \in \Alo$ the canonical map $\lB A(H^\ab) \to \lB A(H)$ is an
  equivalence. It follows that we have an equivalence
  \[
  X^\ab(H) \iso
  \colim_{A \in \Alo_{/X}} \lB A(H) \iso
  \colim_{A \in \Alo_{/X}} \lB A(H^\ab),
  \]
  functorial in $H$. But now the canonical map
  \[
  \colim_{A \in \Alo_{/X}} \lB A(H^\ab) \to X(H^\ab)
  \]
  is clearly an equivalence since $H^\ab \in \Alo$.
\end{proof}

\begin{lemma}
  \label{abdesc-abelianization-as-Gspace}
  Let $G \in \Glo$. Let $\Gamma_G \c (\Top_\Glo)_{/\lB G} \to \Top_G$
  be the functor described in \cref{global-homotopy}, right adjoint to
  the embedding $\Delta_G \c \Top_G \to (\Top_\Glo)_{/\lB G}$.
  \begin{enumerate}
  \item \label{abdesc-abelianization-gamma} For a subgroup
    $H \subseteq G$, the fixed-point space $\Gamma_G(\lB^\ab G)^H$ is
    contractible if $H$ is abelian and empty if $H$ is nonabelian.
  \item \label{abdesc-abelianization-delta} Conversely, if $X$ is a
    $G$-space such that $X^H$ is contractible if $H$ is abelian and
    empty if $H$ is nonabelian, then $\Delta_G(X)$ is equivalent to
    $\lB^\ab G \to \lB G$ in $(\Top_\Glo)_{/\lB G}$.
  \end{enumerate}
\end{lemma}

\begin{proof}
  Take a subgroup $H \subseteq G$. By definition, $\Gamma_G(\lB^\ab
  G)^H$ is given by
  \[
  (\Top_\Glo)_{/\lB G}(\lB H, \lB^\ab G) \iso
  \lB^\ab G(H) \times_{\lB G(H)} \{i_H\} \iso
  \lB G(H^\ab) \times_{\lB G(H)} \{i_H\},
  \]
  where $i_H \in \lB G(H)$ corresponds to the inclusion $H \inj G$;
  this clearly is contractible when $H$ is abelian and empty when $H$
  is nonabelian, proving \cref{abdesc-abelianization-gamma}.

  Now let $X$ be a $G$-space with this same property: $X^H$ is
  contractible for $H$ abelian and empty for $H$ nonabelian. Recall
  $\Delta_G(X)$ is the canonical map
  $\delta_G(X) \to \delta_G(*) \iso \lB G$, where for a $G$-space $Y$
  we have
  \[
  \delta_G(Y)(H) \iso
  \l(\coprod_{\phi \in \Grp(H,G)} Y^{\im(\phi)}\r) \sslash G
  \]
  for $H \in \Glo$. Since $X^{\im(\phi)}$ is only nonempty when
  $\im(\phi)$ is abelian, i.e. when $\phi$ factors through $H^\ab$, we
  have
  \[
  \delta_G(X)(H) \iso
  \l(\coprod_{\phi \in \Grp(H^\ab,G)} X^{\im(\phi)}\r) \sslash G.
  \]
  It follows from \cref{abdesc-abelianization-alternate} that
  $\delta_G(X) \to \lB G$ factors through the map
  $\lB^\ab G \to \lB G$: for $H \in \Glo$ the resulting map
  $\delta_G(X) \to \lB^\ab G$ looks like
  \[
  \l(\coprod_{\phi \in \Grp(H^\ab,G)} X^{\im(\phi)}\r) \sslash G \iso
  \delta_G(X)(H) \to
  \lB^\ab G(H) \iso
  \l(\coprod_{\phi \in \Grp(H^\ab,G)} * \r) \sslash G,
  \]
  induced by the unique maps $X^{\im(\phi)} \to *$. But these maps are
  all equivalences by our contracibility hypothesis, whence
  $\delta_G(X) \to \lB^\ab(G)$ is an equivalence (over $\lB G$), as
  desired.
\end{proof}

\begin{remark}
  \label{abdesc-abelianization-faithful}
  Note that \cref{abdesc-abelianization-as-Gspace} in particular
  implies $\Delta_G\Gamma_G(\lB^\ab G) \iso \lB^\ab G$, i.e. that
  $\lB^\ab G \to\lB G$ is faithful. One can also show this more
  directly. Let $H \in \Glo$ and $N \subseteq H$ a normal subgroup. We
  want to show that the induced diagram
  \[
  \begin{tikzcd}
    \lB^\ab G(H/N) \ar[r] \ar[d] &
    \lB^\ab G(H) \ar[d] \\
    \lB G(H/N) \ar[r] &
    \lB G(H)
  \end{tikzcd}
  \]
  is a pullback square. By \cref{abdesc-abelianization-alternate} we
  can identify this with the diagram of 1-groupoids
  \[
  \begin{tikzcd}
    \Grp((H/N)^\ab,G) \sslash G \ar[r] \ar[d] &
    \Grp(H^\ab,G) \sslash G \ar[d] \\
    \Grp(H/N,G) \sslash G \ar[r] &
    \Grp(H,G) \sslash G.
  \end{tikzcd}
  \]
  It is enough to that check this is a pullback square of 1-groupoids,
  and this is straightforward given the explicit description of
  pullback squares of 1-groupoids (see
  e.g. \cite[\S6.3]{strickland-k(n)-duality}).
\end{remark}

\subsection{Complex-oriented descent}
\label{abdesc-codesc}

The takeaway of \cref{abdesc-alternate} is that whether or not a
global spectrum satisfies abelian descent comes down to comparing, for
each $G \in \Glo$, what it assigns to the trivial $G$-space and what
it assigns to a $G$-space whose $H$-fixed point space is contractible
for $H \subseteq G$ abelian and empty for $H \subseteq G$
nonabelian. In this subsection we construct a more explicit model for
the latter, and use this to describe a practical method for proving
that various global spectra associated to complex-oriented cohomology
theories satisfy abelian descent. This idea is originally due to
Quillen \cite{quillen-equivariant-i}, and will be key in proving later
on that our generalized character maps for Morava E-theory are
isomorphisms.

\begin{notation}
  \label{abdesc-codesc-ntn}
  For the remainder of this subsection:
  \begin{enumerate}
  \item Fix a group $G \in \Glo$.
  \item Fix a faithful complex representation\footnote{Recall that
      such a thing always exists, e.g. the regular representation
      $\lC[G]$ always works.}  $\rho \c G \inj \Aut(V)$.
  \item Choosing a $G$-equivariant hermitian inner
    product\footnote{Recall that such a thing always exists, e.g. by
      averaging any ordinary hermitian inner product over $G$.} on and
    a basis of $V$, this determines an embedding into the unitary
    group $i \c G \inj \rU(k)$ for $k \ce \dim(V)$.
  \item \label{abdesc-flag-variety} Let $T \subseteq \rU(k)$ be a
    maximal torus. Let $F \ce \rU(k)/T$, which is a $G$-space via
    $i$. Recall that $F$ is the space of complete flags in $V$ (if you
    imagine $T$ as the subspace of unitary diagonal matrices, then $F$
    is evidently the space of $k$-tuples of orthogonal lines in $V$,
    which is the same as the space of complete flags).
  \end{enumerate}
\end{notation}

\begin{lemma}
  \label{abdesc-flag-abstab}
  For a subgroup $H \subseteq G$, the fixed-point space $F^H$ is
  nonempty if and only if $H$ is abelian.
\end{lemma}

\begin{proof}
  Maximal torus blah blah
\end{proof}

\begin{notation}
  \label{abdesc-Espace}
  If $X$ is a space, let $\rE X$ denote the geometric realization of
  the simplicial space
  \[
  \simpl{X}{X \times X}{X \times X \times X}.
  \]
\end{notation}

\begin{lemma}
  \label{abdesc-Espace-type}
  Let $X$ be a space. If $X$ is empty then $\rE X$ is empty; if $X$ is
  nonempty then $\rE X$ is contractible.
\end{lemma}

\begin{proof}
  The claim is obvious for $X$ empty. For $X$ nonempty there is a
  standard way to construct contracting homotopies (see
  e.g. \cite[3.14]{dugger-primer}).
\end{proof}

\begin{lemma}
  \label{abdesc-BabG-construction}
  For a subgroup $H \subseteq G$, the fixed-point space $(\rE F)^H$ is
  contractible for $H$ abelian and empty for $H$ nonabelian. In other
  words, by \cref{abdesc-abelianization-as-Gspace}, $\rE F$ is a model
  for $\lB^\ab G$.
\end{lemma}

\begin{proof}
  Since geometric realization of simplicial spaces commutes with
  finite limits, in particular taking $H$-fixed-points, this is
  immediate from \cref{abdesc-flag-abstab,abdesc-Espace-type}.
\end{proof}

\begin{lemma}
  \label{abdesc-flag-faithflat}
  Let $E$ be a ring $G$-spectrum. Suppose $E^* \to E^*(F)$ is
  faithfully flat and for all (finite?) $G$-spaces $X$ the canonical
  map $E^*(X) \otimes_{E^*} E^*(F) \to E^*(X \times F)$ is an
  isomorphism. Then $E^* \to E^*(EF)$ is an isomorphism.
\end{lemma}

\begin{proof}
  As $\rE F$ is defined as a geometric realization, the function
  spectrum $E(\rE F)$ is given by the corresponding totalization,
  namely of the cosimplicial $G$-spectrum
  \[
  \cosimpl{E(F)}{E(F \times F)}{E(F \times F \times F)}.
  \]
\end{proof}

\begin{lemma}
  \label{abdesc-co-flag-faithflat}
  Let $E$ be a complex-orientable spectrum.
  \begin{enumerate}
  \item \label{abdesc-co-flag-bundle} Let $\cV \to X$ be a vector
    bundle, and $\cF \to X$ the associated bundle of complete flags in
    $\cV$. Then $E^*(\cF)$ is a finite free module over $E^*(X)$.
  \item \label{abdesc-co-flag-interest} $(E^{\Bor}_G)^*(F)$ is a
    finite free module over $(E^\Bor_G)^*$, and for $X$ a $G$-space
    the canonical map $E^*(X) \otimes_{E^*} E^*(F) \to E^*(X \times
    F)$ is an isomorphism.
  \end{enumerate}
\end{lemma}

\begin{proof}
  Recall that $E^\Bor_G$ is defined such that
  $(E^\Bor_G)^*(F) \iso E^*(EG \times_G F)$ and
  $(E^\Bor_G)^* \iso E^*(BG)$. Noting that $EG \times_G F \to BG$ is
  the bundle of complete flags in the vector bundle
  $EG \times_G V \to BG$, we see that \cref{abdesc-co-flag-interest}
  follows from \cref{abdesc-co-flag-bundle}.

  We prove \cref{abdesc-co-flag-bundle} by induction on the rank of
  the vector bundle $\cV \to X$. Of course the claim is trivial when
  the rank is $0$, giving us our base case, so assume the rank is
  $n \ge 1$. Consider the associated projective bundle $\lP\cV \to X$.
  The pullback bundle $\cV' \ce \lP\cV \times_X \cV$ over $\lP V$
  contains a canonical line sub-bundle $\cL$. Contemplating the
  quotient bundle $\cW \ce \cV'/\cL$, we see that the space of
  complete flags in $\cW$ can be identified with the space $\cF$ of
  complete flags in $\cV$. Thus the map $E^*(X) \to E^*(\cF)$ factors
  as the composition $E^*(X) \to E^*(\lP\cV) \to E^*(\cF)$. Since
  $\cW \to \lP\cV$ has rank $n-1$, we know by induction that
  $E^*(\cF)$ is a finite free $E^*(\lP\cV)$-module. So we just need
  that $E^*(\lP\cV)$ is a finite free $E^*(X)$-module. This is a
  standard application of the Leray-Hirsch theorem, the point being
  that $\lP V \to X$ is a bundle with fiber $\lC\lP^n$ and a complex
  orientation of $E$ induces an isomorphism
  $E^*(\lC\lP^n) \iso E^*[x]/(x^n)$, which of course is finite free.
\end{proof}