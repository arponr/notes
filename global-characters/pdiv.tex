\section{$p$-divisible groups}
\label{pdiv}

We saw in \cref{chrom} that there is a remarkable connection between
stable homotopy theory and the theory of formal groups. In particular
we saw that, at a fixed prime $p$, the stratification of formal groups
by height is reflected in homotopy theory through these cohomology
theories known at Morava K- and E-theory. It turns out that, to
understand how these different heights interact in stable homotopy
theory, it is extremely useful not just to consider the formal groups
at hand, but to consider the $p$-power torsion of these formal groups
as what are known as $p$-divisible groups. Indeed the main theorem in
this thesis, i.e. the character theory presented in \cref{char}, is
precisely an expression of this philosophy. This section reviews the
bare minimum of what we'll need to know about $p$-divisible groups.

%%%%%%%%%%%%%%%%%%%%%%%%%%%%%%%%%%%%%%%%%%%%%%%%%%%%%%%%%%%%%%%%%%%%%%

\subsection{Generalities}
\label{pdiv-gen}

\begin{definition}
  \label{pdiv-pdivgrp}
  Let $A$ be a commutative ring. A \emph{$p$-divisible group} $\lG$ of
  \emph{height $n$} over $A$ is a system
  $(\lG_k, i_k)_{k \in \lZ_{\ge 0}}$, where for each
  $k \in \lZ_{\ge 0}$:
  \begin{itemize}
  \item $\lG_k$ is a finite free\footnote{Usually one says ``locally
      free'' or ``flat'' rather than restricting to ``free''
      here. However, we will only deal with $p$-divisible groups for
      which these finite parts are in fact free, and we will indeed
      need to use this freeness hypothesis in \cref{pdiv-split}.}
    commutative group scheme over $A$ of rank $p^{nk}$ (note that
    finite morphisms are affine by definition, so $\lG_k$ is affine);
  \item $i_k \c \lG_k \to \lG_{k+1}$ is a morphism of group schemes
    over $A$ such that the sequence
    \[
    0 \to \lG_k \lblto{i_k} \lG_{k+1} \lblto{(p^k)} \lG_{k+1}
    \]
    is exact, where $(p^k)$ denotes multiplication by $p^k$.
  \end{itemize}
  It's easy to see that these morphisms identify $\lG_k$ as the
  $p^k$-torsion in $\lG_{k+l}$ for all $l > 0$. Thus we will more
  often than not think of $\lG$ not as the inductive system
  $(\lG_k,i_k)$, but as the colimit of this system, and denote $\lG_k$
  by $\lG[p^k]$.

  For $p$-divisible groups $\lG = (\lG_k,i_k)$ and $\lG' =
  (\lG'_k,i'_k)$ over $A$, a morphism $\phi \c \lG \to \lG'$ is a
  sequence of morphisms $\phi_k \c \lG_k \to \lG'_k$ for $k \in
  \lZ_{\ge 0}$ making the following diagram commute
  \[
  \begin{tikzcd}
    \cdots \ar[r] &
    \lG_k \ar[r, "i_k"] \ar[d, "\phi_k"] &
    \lG_{k+1} \ar[r] \ar[d, "\phi_{k+1}"] &
    \cdots \\
    \cdots \ar[r] &
    \lG'_k \ar[r, "i'_k"] &
    \lG'_{k+1} \ar[r] &
    \cdots
  \end{tikzcd}
  \]
\end{definition}

\begin{notation}
  If $A \to B$ is a morphism of rings and we have a (group) scheme
  $\lG$ over $A$, then we will denote the base change
  $\Spec(B) \times_{\Spec(A)} \lG$ by $\lG_B$. Similarly if we have an
  ordinary group $\lG$, then we will denote the associated constant
  group scheme over a ring $A$ by $\lG_A$.
\end{notation}

\begin{example}
  Suppose we had not a group scheme but an ordinary group $\lG$
  satisfying the axioms of a $p$-divisible group. Then $\lG[p]$ would
  a be a finite $p$-torsion group of order $p^n$, hence would
  necessarily be isomorphic to $(\lZ/p)^n$. Then $\lG[p^2]$ would be a
  finite $p^2$-torsion group of order $p^{2n}$ whose $p$-torsion was
  given by $(\lZ/p)^n$, hence would necessarily be isomorphic to
  $(\lZ/p^2)^n$. And inductively we see that the only possibility is
  that $\lG \iso \colim_k (\lZ/p^k)^n \iso (\lQ_p/\lZ_p)^n$.

  We deduce then that any constant $p$-divisible group of height $n$
  over a ring $A$ is isomorphic to $(\lQ_p/\lZ_p)^n_A$.
\end{example}



\begin{proposition}
  \label{pdiv-connet}
  Infinitesimal-\'etale sequence.
\end{proposition}

\begin{proposition}
  \label{pdiv-connet-split}
  Let $\lG$ be a $p$-divisible group which has an connected-\'etale
  sequence. Let $r$ be the height of $\lG_\et$. The following are
  equivalent:
  \begin{enumerate}
  \item the connected-\'etale sequence splits, i.e.
    $\lG \iso \lG_\inf \times \lG_\et$, and the \'etale part of $\lG$
    is constant, i.e. $\lG_\et \iso (\lQ_p/\lZ_p)^r$.
  \item there is a map $(\lQ_p/\lZ_p)^r \to \lG$ such that the
    composite $(\lQ_p/\lZ_p)^r \to \lG_\et$ is an isomorphism.
  \end{enumerate}
\end{proposition}

\begin{proof}
  5-lemma.
\end{proof}

\begin{proposition}
  \label{pdiv-algcl}
  Any $p$-divisible group $\lG$ over an algebraically closed field $K$
  satisfies the equivalent conditions of \cref{pdiv-connet-split}.
\end{proposition}

\subsection{Splitting the connected-\'etale sequence}
\label{pdiv-split}

In this subsection we construct the universal extension of a ring over
which a given $p$-divisible group splits as the product of its
infinitesimal part and a constant \'etale part. We essentially follow
Stapleton \cite[\S2.8]{stapleton-tgcm}, who essentially follows
Hopkins-Kuhn-Ravenel \cite[\S\S6.1--6.2]{hkr-char}. However, rather
than restrict to the specific example which will be relevant to
character theory, here we distill this part of the argument which
applies more generally; I think this generality actually clarifies the
exposition slightly.

\begin{notation}
  \label{pdiv-split-ntn}
  Throughout this subsection we let $\lG$ be a $p$-divisible group of
  height $n$ over a ring $A$. Assume that we have a connected-\'etale
  sequence
  \[
  0 \to \lG_\inf \to \lG \to \lG_\et \to 0
  \]
  where $\lG_\inf$ has height $t$ and therefore $\lG_\et$ has height
  $n-t$.

  Let $\Lambda \ce (\lQ_p/\lZ_p)^{n-t}$.\footnote{Warning: our
    $\Lambda$ is denoted $\Lambda^\vee$ in
    \cite{hkr-char,stapleton-tgcm}, and thus our $\Lambda^\vee$
    (appearing in \cref{pdiv-chromex}) is their $\Lambda$.} For
  $k \in \lZ_{\ge 0}$ we may choose generators
  $\lambda^k_1, \ldots, \lambda^k_{n-t}$ of
  $\Lambda[p^k] \iso (\lZ/p^k)^{n-t}$ which are coherent in the sense
  that $\lambda^k_i = p\lambda^{k+1}_i$ for $1 \le i \le n-t$.
\end{notation}

\begin{lemma}
  \label{pdiv-split-hom}
  For $k \in \lZ_{\ge 0}$, the functor $\Alg_A \to \Set$ assigning to
  an $A$-algebra $B$ the set $\Hom_B(\Lambda[p^k]_B, \lG[p^k]_B)$ of
  morphisms of group schemes $\Lambda[p^k]_B \to \lG[p^k]_B$ over $B$
  is corepresented by the $A$-algebra
  \[
  A_{t,k} \ce (\cO_{\lG[p^k]})^{\otimes(n-t)} \ce \cO_{\lG[p^k]}
  \otimes_A \cdots \otimes_A \cO_{\lG[p^k]};
  \]
  i.e. any morphism $\Lambda[p^k]_B \to \lG[p^k]_B$ is the base change
  of a universal morphism
  $\phi_\univ \c \Lambda[p^k]_{A_{t,k}} \to \lG[p^k]_{A_{t,k}}$ in a
  unique map $A_{t,k} \to B$.

  The natural transformation
  \[
  \Hom_B(\Lambda[p^{k+1}]_B,\lG[p^{k+1}]_B) \to
  \Hom_B(\Lambda[p^k]_B,\lG[p^k]_B)
  \]
  given by restriction corresponds via the Yoneda lemma to the
  morphism $A_{t,k} \to A_{t,k+1}$ induced by the
  multiplication-by-$p$ map $\lG[p^{k+1}] \to \lG[p^k]$.
\end{lemma}

\begin{proof}
  Let $B \in \Alg_A$. Our chosen generators
  $\lambda^k_1, \ldots, \lambda^k_{n-t}$ of $\Lambda[p^k]$ determine
  an isomorphism $\Lambda[p^k] \iso (\lZ/p^k)^{n-t}$, so specifying a
  morphism of group schemes $\Lambda[p^k]_B \to \lG[p^k]_B$ over $B$
  is equivalent to specifying $n-t$ morphisms of schemes
  $\Spec(B) \to \lG[p^k]_B$ over $B$, which are equivalent to
  morphisms $\Spec(B) \to \lG[p^k]$ over $A$. Since $\lG[p^k]$ is
  affine, this is equivalent to specifying $n-t$ morphisms of
  $A$-algebras $\cO_{\lG[p^k]} \to B$, which is equivalent to
  specifying a single morphism of $A$-algebras
  $(\cO_{\lG[p^k]})^{\otimes(n-t)} \to B$. We have now described a
  bijection
  $\Hom_B(\Lambda[p^k]_B, \lG[p^k]_B) \iso \Hom_{\Alg_A}(A_{t,k},B)$
  which clearly is natural in $B$, proving the claim. Note that the
  universal morphism
  $\phi_\univ \c \Lambda[p^k]_{A_{t,k}} \to \lG[p^k]_{A_{t,k}}$
  corresponds to the identity $\id \c A_{t,k} \to A_{t,k}$ under this
  bijection.

  The final statement about the induced map $A_{t,k} \to A_{t,k+1}$ is
  immediate from the coherence of the generators $\lambda^k_1, \ldots,
  \lambda^k_{n-t}$ fixed in \cref{pdiv-split-ntn}.
\end{proof}

\begin{lemma}
  \label{pdiv-split-iso}
  For $k \in \lZ_{\ge 0}$, the functor $\Alg_A \to \Set$ assigning to
  an $A$-algebra $B$ the subset
  $\Iso_B(\Lambda[p^k]_B, \lG_\et[p^k]_B) \subseteq
  \Hom_B(\Lambda[p^k]_B, \lG[p^k]_B)$
  consisting of morphisms $\Lambda[p^k]_B \to \lG[p^k]_B$ for which
  the composite morphism
  \[
  \Lambda[p^k]_B \to \lG[p^k]_B \to \lG_\et[p^k]_B
  \]
  is an isomorphism is corepresented by the localization
  $A_{t,k}[\Delta_{t,k}^{-1}]$ for some element
  $\Delta_{t,k} \in A_{t,k}$. More precisely we have a commutative
  diagram of natural transformations
  \begin{equation}
    \label{pdiv-split-iso-dgm}
    \begin{tikzcd}
      \Iso_B(\Lambda[p^k]_B, \lG_\et[p^k]_B) \ar[r, hook]
      \ar[d, "\sim"{rotate=90, yshift=-4pt, xshift=-4pt}] &
      \Hom_B(\Lambda[p^k]_B, \lG[p^k]_B)
      \ar[d, "\sim"{rotate=90, yshift=-4pt, xshift=-4pt}] \\
      \Hom_A(A_{t,k}[\Delta^{-1}_{t,k}],B) \ar[r, hook] &
      \Hom_A(A_{t,k},B),
    \end{tikzcd}
  \end{equation}
  where the bottom map is the canonical inclusion and the right map is
  the isomorphism found in \cref{pdiv-split-hom}. Finally,
  $A_{t,k}[\Delta^{-1}_{t,k}]$ is faithfully flat over $A$.
\end{lemma}

\begin{proof}
  Suppose we have a morphism $\phi \c \Lambda[p^k]_B \to \lG[p^k]_B$.
  By \cref{pdiv-split-hom} this determines a unique map
  $\alpha \c A_{t,k} \to B$ in which $\phi$ is the base change of
  $\phi_\univ$. Consider the composite
  \begin{equation}
    \label{pdiv-split-iso-geo}
    \Lambda[p^k]_{A_{t,k}} \lblto{\phi_\univ}
    \lG[p^k]_{A_{t,k}} \to
    \lG_\et[p^k]_{A_{t,k}},
  \end{equation}
  which corresponds to a morphism of $A_{t,k}$-algebras
  \begin{equation}
    \label{pdiv-split-iso-alg}
    \cO_{\lG_\et[p^k]_{A_{t,k}}} \to \cO_{\Lambda[p^k]_{A_{t,k}}}.
  \end{equation}
  By hypothesis and definition, both algebras are finite free of the
  same rank, so we may consider its determinant
  $\Delta_{t,k} \in A_{t,k}$. Now, everything in sight is affine, so
  the base change $\Lambda[p^k]_B \to \lG_\et[p^k]_B$ of
  \cref{pdiv-split-iso-geo} in $\alpha$, which by definition of
  $\alpha$ is the composite
  \[
  \Lambda[p^k]_B \lblto{\phi} \lG[p^k]_B \to \lG_\et[p^k]_B,
  \]
  is an isomorphism if and only if the base change
  $\cO_{\lG_\et[p^k]_B} \to \cO_{\Lambda[p^k]_B}$ of
  \cref{pdiv-split-iso-alg} in $\alpha$ is an isomorphism. But
  of course this is true if and only if $\alpha(\Delta_{t,k})$ is a
  unit in $B$, i.e. if $\alpha$ factors through the localization
  $A_{t,k}[\Delta_{t,k}^{-1}]$. This proves the existence and
  commutativity of the diagram \cref{pdiv-split-iso-dgm}.

  Finally we prove faithful flatness. Since $\cO_{\lG[p^k]}$ is finite
  free over $A$, so is $A_{t,k}$. And localization is flat so this
  implies $A_{t,k}[\Delta_{t,k}^{-1}]$ is flat over $A$. So to prove
  faithful flatness we just need to show that
  $\Spec(A_{t,k}[\Delta_{t,k}^{-1}]) \to \Spec(A)$ is surjective. Let
  $\kp \in \Spec(A)$; let $K$ be the algebraic closure of the fraction
  field of the domain $A/\kp$, so we have a map $\beta \c A \to K$
  with kernel $\kp$. By \cref{pdiv-algcl} there is a map
  $\Lambda[p^k]_K \to \lG[p^k]_K$ such that the composite
  $\Lambda[p^k]_K \to \lG[p^k]_K \to \lG_\et[p^k]_K$ is an
  isomorphism. So by the above, $\beta$ factors through a map
  $\gamma \c A_{t,k}[\Delta_{t,k}^{-1}] \to K$. If we set
  $\kq \ce \ker(\gamma) \in \Spec(A_{t,k}[\Delta_{t,k}^{-1}])$ then
  $\kq$ restricts to $\kp$ in $\Spec(A)$. Since $\kp$ was arbitrary,
  we have the desired surjectivity.
\end{proof}

\begin{proposition}
  \label{pdiv-split-main}
  The functor $\Alg_A \to \Set$ assigning to an $A$-algebra $B$ the
  subset
  $\Iso_B(\Lambda_B, (\lG_\et)_B) \subseteq
  \Hom_B(\Lambda_B, \lG_B)$
  consisting of morphisms $\Lambda_B \to \lG_B$ for which
  the composite morphism
  \[
  \Lambda_B \to \lG_B \to (\lG_\et)_B
  \]
  is an isomorphism is corepresented by
  $A_t \ce \colim_k A_{t,k}[\Delta_{t,k}^{-1}]$, where
  $A_{t,k}, \Delta_{t,k}$ as in \cref{pdiv-split-hom,pdiv-split-iso}.
\end{proposition}

\begin{proof}
  By definition of a morphism of $p$-divisible groups,
  \begin{align*}
    \Iso_B(\Lambda_B, (\lG_\et)_B) &\iso
    \lim_k \Iso_B(\Lambda[p^k]_B, \lG_\et[p^k]_B) \\ &\iso
    \lim_k \Hom_A(A_{t,k}[\Delta_{t,k}^{-1}], B) \\ &\iso
    \Hom_A(\colim_k A_{t,k}[\Delta_{t,k}^{-1}], B),
  \end{align*}
  as desired.
\end{proof}

%%%%%%%%%%%%%%%%%%%%%%%%%%%%%%%%%%%%%%%%%%%%%%%%%%%%%%%%%%%%%%%%%%%%%%

\subsection{Chromatic examples}
\label{pdiv-chromex}