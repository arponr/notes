\section{\texorpdfstring{$p$}{p}-divisible groups}
\label{pdiv}

The takeway of \cref{chrom} is that there is a remarkable connection
between stable homotopy theory and the theory of formal groups. In
particular we saw that, at a fixed prime $p$, the stratification of
formal groups by height is reflected in homotopy theory through
certain spectra known at Morava K- and E-theory. It turns out that, to
understand how these different heights interact in stable homotopy
theory, it is extremely useful not just to consider the formal groups
at hand, but to consider the $p$-power torsion of these formal groups
as what are known as $p$-divisible groups. This section reviews the
bare minimum of what we'll need to know about $p$-divisible groups,
and then analyzes the $p$-divisible groups in chromatic homotopy
theory that we care about.

\begin{notation}
  \label{pdiv-fixp}
  Throughout this section we fix a prime $p$.
\end{notation}

%%%%%%%%%%%%%%%%%%%%%%%%%%%%%%%%%%%%%%%%%%%%%%%%%%%%%%%%%%%%%%%%%%%%%%

\subsection{Generalities}
\label{pdiv-gen}

\begin{notation}
  \label{pdiv-grpschemes}
  We recall some basic terminology and notation regarding
  (commutative) group schemes:
  \begin{enumerate}
  \item A sequence $0 \to \lG' \lblto{i} \lG \lblto{j} \lG'' \to 0$ of
    finite group schemes is called \emph{short exact} if $j$ is
    faithfully flat and $i$ is a closed immersion which identifies
    $\lG'$ with the (category-theoretic) kernel of $j$. If we have
    such a sequence such that $\lG'$ and $\lG''$ are finite of ranks
    $a$ and $b$, then $\lG$ is finite of rank $ab$.
  \item If we have a group scheme $\lG$ over a ring $A$ and a ring
    extension $A \to B$, we denote the base change $\lG
    \times_{\Spec(A)} \Spec(B)$ by $\lG_B$.
  \item Similarly, if we have an ordinary group $\lG$, we denote the
    associated constant group over a ring $A$ by $\lG_A$.
  \item Note that finite morphisms are by definition affine, so if
    $\lG$ is a finite group scheme over a ring $A$, it is in fact an
    affine group scheme. In this case we denote the corresponding ring
    (that is, the global sections of $\lG$) by $\cO_\lG$.
  \end{enumerate}
\end{notation}

\begin{definition}
  \label{pdiv-pdivgrp}
  Let $A$ be a commutative ring. A \emph{$p$-divisible group} $\lG$ of
  \emph{height $n$} over $A$ is a system
  $(\lG_k, i_k)_{k \in \lZ_{\ge 0}}$, where for each
  $k \in \lZ_{\ge 0}$:
  \begin{itemize}
  \item $\lG_k$ is a finite free\footnote{Usually one says ``locally
      free'' or ``flat'' rather than restricting to ``free''
      here. However, we will only deal with $p$-divisible groups for
      which these finite parts are in fact free, and we will indeed
      need to use this freeness hypothesis in \cref{pdiv-split}.}
    commutative group scheme over $A$ of rank $p^{nk}$;
  \item $i_k \c \lG_k \to \lG_{k+1}$ is a morphism of group schemes
    over $A$ such that the sequence
    \[
    0 \to \lG_k \lblto{i_k} \lG_{k+1} \lblto{(p^k)} \lG_{k+1}
    \]
    is exact, where $(p^k)$ denotes multiplication by $p^k$;
    i.e. $i_k$ identifies $\lG_k$ as the $p^k$-torsion in
    $\lG_{k+1}$.
  \end{itemize}
  It's easy to see that these morphisms in fact identify $\lG_k$ as
  the $p^k$-torsion in $\lG_{k+l}$ for all $l > 0$. Thus we will
  basically always think of $\lG$ not as the inductive system
  $(\lG_k,i_k)$, but as the colimit of this system, and denote $\lG_k$
  by $\lG[p^k]$.

  Morphisms, short exact sequences, base changes, direct sums, etc. of
  $p$-divisible groups are all defined in the obvious way in terms of
  their definitions for finite group schemes. Note that because rank
  of finite group schemes is multiplicative in short exact sequences,
  height of $p$-divisible groups is additive in short exact sequences.
\end{definition}

\begin{example}
  \label{pdiv-constant}
  Suppose we had not a group scheme but an ordinary group $\lG$
  satisfying the axioms of a $p$-divisible group. Then $\lG[p]$ would
  a be a finite $p$-torsion group of order $p^n$, hence would
  necessarily be isomorphic to $(\lZ/p)^n$. Next, $\lG[p^2]$ would be
  a finite $p^2$-torsion group of order $p^{2n}$, whose $p$-torsion
  was given by $(\lZ/p)^n$, hence would necessarily be isomorphic to
  $(\lZ/p^2)^n$. And inductively we see that the only possibility is
  that $\lG \iso \colim_k (\lZ/p^k)^n \iso (\lQ_p/\lZ_p)^n$.

  We deduce then that any constant $p$-divisible group of height $n$
  over a ring $A$ is isomorphic to $(\lQ_p/\lZ_p)^n_A$.
\end{example}

\begin{example}
  \label{pdiv-formal}
  Suppose we have a formal group law $f$ over a commutative ring $A$.
  Let $\lG_f \ce \Spf(A \ldb t \rdb)$ denote the associated formal
  group. We would like to say that the $p$-power torsion
  $\lG_f[p^\infty] \ce \colim_k \lG_f[p^k]$ of $\lG_f$ is a
  $p$-divisible group, where
  $\lG_f[p^k] \iso \Spec(A \ldb t \rdb / ([p^k](t)))$ with
  $[p^k](t) \in A \ldb t \rdb$ the $p^k$-series of $f$. A form of the
  Weierstrass preparation theorem \cite[5.1--5.2]{hkr-char} tells us
  that if $A$ is complete with respect to an ideal $I$, and the
  $p$-series of $f$ satisfies
  $[p](t) \equiv ut^{p^n} \mod (I, t^{p^n+1})$ for a unit
  $u \in A^\times$ and $n \ge 1$, then the $A$-algebra
  $A \ldb t \rdb / ([p^k](t))$ is a free $A$-module with basis
  $\{1, t, \ldots, t^{p^{nk} -1}\}$ for all $k \ge 0$. So under these
  hypotheses, the $p$-power torsion $\lG_f[p^\infty]$ will indeed be a
  $p$-divisible group, of height $n$.
\end{example}

Often it is possible to decompose an arbitrary $p$-divisible group
$\lG$ into ones which look like the two examples above. Namely, one
often has a short exact sequence
\begin{equation}
  \label{pdiv-infet}
  0 \to \lG_\inf \to \lG \to \lG_\et \to 0,
\end{equation}
where $\lG_\inf$ is an ``infinitesimal'' $p$-divisible group arising
from some formal group law, and $\lG_\et$ is \'etale (hence fairly
close to being constant). In \cref{pdiv-split} we will analyze the
behavior of such short exact sequences, specifically how we can extend
our base ring such that $\lG_\et$ actually becomes constant and such
that the sequence splits. The following elementary observation will be
useful in this analysis.

\begin{proposition}
  \label{pdiv-infet-split}
  Let $\lG$ be a $p$-divisible group over a ring $A$. Suppose we have
  a short exact sequence \cref{pdiv-infet}. Let $r$ be the height of
  $\lG_\et$. The following are equivalent:
  \begin{enumerate}
  \item \label{pdiv-infet-split-yeah} the exact sequence splits and
    the \'etale part of $\lG$ is constant, i.e.
    $\lG \iso \lG_\inf \oplus \lG_\et$ and
    $\lG_\et \iso (\lQ_p/\lZ_p)^r_A$;
  \item \label{pdiv-infet-split-map} there is a map
    $(\lQ_p/\lZ_p)^r_A \to \lG$ such that the composite
    $(\lQ_p/\lZ_p)^r_A \to \lG_\et$ is an isomorphism.
  \end{enumerate}
\end{proposition}

\begin{proof}
  Clearly \cref{pdiv-infet-split-yeah}
  implies \cref{pdiv-infet-split-map}. Conversely, assuming
  \cref{pdiv-infet-split-map}, we get a map of short exact sequences
  \[
  \begin{tikzcd}
    0 \ar[r] &
    \lG_\inf \ar[r] \ar[d] &
    \lG_\inf \oplus (\lQ_p/\lZ_p)_A^r \ar[r] \ar[d] &
    (\lQ_p/\lZ_p)_A^r \ar[r] \ar[d] &
    0 \\
    0 \ar[r] &
    \lG_\inf \ar[r] &
    \lG \ar[r] &
    \lG_\et \ar[r] &
    0.
  \end{tikzcd}
  \]
  The left vertical map is just the identity, and the right vertical
  map is by hypothesis an isomorphism. We conclude by applying the
  $5$-lemma.
\end{proof}

%%%%%%%%%%%%%%%%%%%%%%%%%%%%%%%%%%%%%%%%%%%%%%%%%%%%%%%%%%%%%%%%%%%%%%

\subsection{Splitting the connected-\'etale sequence}
\label{pdiv-split}

In this subsection we construct the universal extension of a ring over
which a given $p$-divisible group splits as the direct sum of its
infinitesimal part and a constant \'etale part. We essentially follow
Stapleton \cite[\S2.8]{stapleton-tgcm}, who essentially follows
Hopkins-Kuhn-Ravenel \cite[\S\S6.1--6.2]{hkr-char}. However, rather
than restrict to the specific example which will be relevant to
character theory, here we isolate this part of the argument which
applies more generally; I think this generality actually clarifies the
exposition slightly.

\begin{notation}
  \label{pdiv-split-ntn}
  Throughout this subsection we let $\lG$ be a $p$-divisible group of
  height $n$ over a ring $A$. Assume that we have a short exact
  sequence of $p$-divisible groups
  \[
  0 \to \lG_\inf \to \lG \to \lG_\et \to 0
  \]
  where $\lG_\inf$ has height $t$ and $\lG_\et$ is \'etale of height
  $n-t$.

  Let $\Lambda \ce (\lQ_p/\lZ_p)^{n-t}$.\footnote{Warning: our
    $\Lambda$ is denoted $\Lambda^\vee$ in
    \cite{hkr-char,stapleton-tgcm}, and thus our $\Lambda^\vee$
    (appearing in \cref{pdiv-chromex}) is their $\Lambda$.} For all
  $k \ge 0$ we may choose generators
  $\lambda^k_1, \ldots, \lambda^k_{n-t}$ of
  $\Lambda[p^k] \iso (\lZ/p^k)^{n-t}$ which are coherent in the sense
  that $\lambda^k_i = p\lambda^{k+1}_i$ for $1 \le i \le n-t$.
\end{notation}

\begin{lemma}
  \label{pdiv-split-hom}
  For $k \ge 0$, the functor $\Alg_A \to \Set$ assigning to an
  $A$-algebra $B$ the set $\Hom_B(\Lambda[p^k]_B, \lG[p^k]_B)$ of
  morphisms of group schemes $\Lambda[p^k]_B \to \lG[p^k]_B$ over $B$
  is corepresented by the $A$-algebra
  \[
  C_k \ce (\cO_{\lG[p^k]})^{\otimes(n-t)} \ce \cO_{\lG[p^k]}
  \otimes_A \cdots \otimes_A \cO_{\lG[p^k]};
  \]
  i.e. any morphism $\Lambda[p^k]_B \to \lG[p^k]_B$ is the base change
  of a universal morphism
  $\phi_\univ \c \Lambda[p^k]_{C_k} \to \lG[p^k]_{C_k}$ in a
  unique map $C_k \to B$.

  The natural transformation
  \[
  \Hom_B(\Lambda[p^{k+1}]_B,\lG[p^{k+1}]_B) \to
  \Hom_B(\Lambda[p^k]_B,\lG[p^k]_B)
  \]
  given by restriction corresponds via the Yoneda lemma to the
  morphism $C_k \to C_{k+1}$ induced by the multiplication-by-$p$ map
  $\lG[p^{k+1}] \to \lG[p^k]$.
\end{lemma}

\begin{proof}
  Let $B \in \Alg_A$. Our chosen generators
  $\lambda^k_1, \ldots, \lambda^k_{n-t}$ of $\Lambda[p^k]$ determine
  an isomorphism $\Lambda[p^k] \iso (\lZ/p^k)^{n-t}$. Therefore
  restricting to the factors in the constant group scheme
  \[
  \Lambda[p^k]_B \iso \coprod_{\lambda \in \Lambda[p^k]} \Spec(B)
  \]
  indexed by these generators determines a bijection between morphisms
  of group schemes $\Lambda[p^k]_B \to \lG[p^k]_B$ over $B$ and
  $(n-t)$-tuples of morphisms of schemes $\Spec(B) \to \lG[p^k]_B$
  over $B$. The latter are equivalent to morphisms
  $\Spec(B) \to \lG[p^k]$ over $A$. Since $\lG[p^k]$ is affine, this
  equivalent to specifying a morphism of $A$-algebras
  $(\cO_{\lG[p^k]})^{\otimes(n-t)} \to B$. We have now described a
  bijection
  $\Hom_B(\Lambda[p^k]_B, \lG[p^k]_B) \iso \Hom_{\Alg_A}(C_k,B)$ which
  clearly is natural in $B$, proving the claim. Note that the
  universal morphism
  $\phi_\univ \c \Lambda[p^k]_{C_k} \to \lG[p^k]_{C_k}$ corresponds to
  the identity $\id \c C_k \to C_k$ under this bijection.

  The final statement about the induced map $C_k \to C_{k+1}$ is
  immediate from the coherence of the generators $\lambda^k_1, \ldots,
  \lambda^k_{n-t}$ fixed in \cref{pdiv-split-ntn}.
\end{proof}

\begin{lemma}
  \label{pdiv-split-iso}
  For $k \ge 0$, the functor $\Alg_A \to \Set$ assigning to an
  $A$-algebra $B$ the subset
  $\Iso_B(\Lambda[p^k]_B, \lG_\et[p^k]_B) \subseteq
  \Hom_B(\Lambda[p^k]_B, \lG[p^k]_B)$
  consisting of morphisms $\Lambda[p^k]_B \to \lG[p^k]_B$ for which
  the composite morphism
  \[
  \Lambda[p^k]_B \to \lG[p^k]_B \to \lG_\et[p^k]_B
  \]
  is an isomorphism is corepresented by the localization
  $C_k[\Delta_k^{-1}]$, for some element $\Delta_k \in C_k$. More
  precisely, we have a commutative diagram of natural transformations
  \begin{equation}
    \label{pdiv-split-iso-dgm}
    \begin{tikzcd}
      \Iso_B(\Lambda[p^k]_B, \lG_\et[p^k]_B) \ar[r, hook]
      \ar[d, "\sim"{rotate=90, yshift=-4pt, xshift=-4pt}] &
      \Hom_B(\Lambda[p^k]_B, \lG[p^k]_B)
      \ar[d, "\sim"{rotate=90, yshift=-4pt, xshift=-4pt}] \\
      \Hom_A(C_k[\Delta^{-1}_{t,k}],B) \ar[r, hook] &
      \Hom_A(C_k,B),
    \end{tikzcd}
  \end{equation}
  where the bottom map is the canonical inclusion and the right map is
  the isomorphism found in \cref{pdiv-split-hom}. Finally,
  $C_k[\Delta^{-1}_{t,k}]$ is faithfully flat over $A$.
\end{lemma}

\begin{proof}
  Suppose we have a morphism $\phi \c \Lambda[p^k]_B \to \lG[p^k]_B$.
  By \cref{pdiv-split-hom} this determines a unique map
  $\alpha \c C_k \to B$ in which $\phi$ is the base change of
  $\phi_\univ$. Consider the composite
  \begin{equation}
    \label{pdiv-split-iso-geo}
    \Lambda[p^k]_{C_k} \lblto{\phi_\univ}
    \lG[p^k]_{C_k} \to
    \lG_\et[p^k]_{C_k},
  \end{equation}
  which corresponds to a morphism of $C_k$-algebras
  \begin{equation}
    \label{pdiv-split-iso-alg}
    \cO_{\lG_\et[p^k]_{C_k}} \to \cO_{\Lambda[p^k]_{C_k}}.
  \end{equation}
  By hypothesis and definition, both algebras are finite free of the
  same rank, so we may consider its determinant $\Delta_k \in C_k$.
  Now, everything in sight is affine, so the base change
  $\Lambda[p^k]_B \to \lG_\et[p^k]_B$ of \cref{pdiv-split-iso-geo} in
  $\alpha$, which by definition of $\alpha$ is the composite
  \[
  \Lambda[p^k]_B \lblto{\phi} \lG[p^k]_B \to \lG_\et[p^k]_B,
  \]
  is an isomorphism if and only if the base change
  $\cO_{\lG_\et[p^k]_B} \to \cO_{\Lambda[p^k]_B}$ of
  \cref{pdiv-split-iso-alg} in $\alpha$ is an isomorphism. But of
  course this is true if and only if $\alpha(\Delta_k)$ is a unit in
  $B$, i.e. if $\alpha$ factors through the localization
  $C_k[\Delta_k^{-1}]$. This proves the existence and commutativity of
  the diagram \cref{pdiv-split-iso-dgm}.

  Finally we prove faithful flatness. Since $\cO_{\lG[p^k]}$ is finite
  free over $A$, so is $C_k$. And localization is flat so this implies
  $C_k[\Delta_k^{-1}]$ is flat over $A$. So to prove faithful flatness
  we just need to show that $\Spec(C_k[\Delta_k^{-1}]) \to \Spec(A)$
  is surjective. Let $\kp \in \Spec(A)$; let $K$ be the algebraic
  closure of the fraction field of the domain $A/\kp$, so we have a
  map $\beta \c A \to K$ with kernel $\kp$. Since $\lG_\et$ is
  \'etale, it must be constant when base-changed to $K$. So by
  \cref{pdiv-constant} there must be an isomorphism
  $\Lambda[p^k]_K \isoto \lG_\et[p^k]_K$. Since $K$ is algebraically
  closed, there is necessarily a map $\Lambda[p^k]_K \to \lG[p^k]_K$
  lifting this isomorphism.\footnote{This is a fact which I read in
    \cite[p. 32]{demazure-pdiv}, and which is proved in \cite[III,
    3.7.6]{demazure-groupes}.} So by the above, $\beta$ must factor
  through a map $\gamma \c C_k[\Delta_k^{-1}] \to K$. If we set
  $\kq \ce \ker(\gamma) \in \Spec(C_k[\Delta_k^{-1}])$ then $\kq$
  restricts to $\kp$ in $\Spec(A)$.  Since $\kp$ was arbitrary, we
  have the desired surjectivity.
\end{proof}

\begin{remark}
  \label{pdiv-localmaps}
  By the commutativity of \cref{pdiv-split-iso-dgm}, the maps
  $C_k \to C_{k+1}$ defined in \cref{pdiv-split-hom} also induce maps
  $C_k[\Delta_k^{-1}] \to C_{k+1}[\Delta_{k+1}^{-1}]$ on the
  localizations defined in \cref{pdiv-split-iso}.
\end{remark}

\begin{proposition}
  \label{pdiv-split-main}
  The functor $\Alg_A \to \Set$ assigning to an $A$-algebra $B$ the
  subset
  $\Iso_B(\Lambda_B, (\lG_\et)_B) \subseteq
  \Hom_B(\Lambda_B, \lG_B)$
  consisting of morphisms $\Lambda_B \to \lG_B$ for which
  the composite morphism
  \[
  \Lambda_B \to \lG_B \to (\lG_\et)_B
  \]
  is an isomorphism is corepresented by
  $C \ce \colim_k C_k[\Delta_k^{-1}]$, where $C_k$ and $\Delta_k$ are
  as in \cref{pdiv-split-hom,pdiv-split-iso}.
\end{proposition}

\begin{proof}
  By definition of a morphism of $p$-divisible groups,
  \begin{align*}
    \Iso_B(\Lambda_B, (\lG_\et)_B) &\iso
    \lim_k \Iso_B(\Lambda[p^k]_B, \lG_\et[p^k]_B) \\ &\iso
    \lim_k \Hom_A(C_k[\Delta_k^{-1}], B) \\ &\iso
    \Hom_A(\colim_k C_k[\Delta_k^{-1}], B),
  \end{align*}
  as desired.
\end{proof}

%%%%%%%%%%%%%%%%%%%%%%%%%%%%%%%%%%%%%%%%%%%%%%%%%%%%%%%%%%%%%%%%%%%%%%

\subsection{Chromatic examples}
\label{pdiv-chromex}

We now discuss the $p$-divisible groups associated to Morava E-theory
and its $K(t)$-localization. Some proofs and computations in this
section are omitted/cited.\footnote{These omissions are unfortunate,
  but this thesis has a due date, which is just as unfortunate.}

\begin{nothing}
  \label{pdiv-ethy}
  Let $E$ be the Morava E-theory associated to a formal group $\lG_0$
  of height $n$ over a perfect field $\kappa$ of characteristic
  $p$. As discussed in \cref{chrom-moravathy}, $E$ is even periodic
  with coefficient ring given by
  \[
  E^0 \iso \rW(\kappa) \ldb v_1, \ldots, v_{n-1} \rdb.
  \]
  As shown in \cite{rezk-hopkinsmiller}, the formal group law (over
  $E^0$) associated to $E$ can be taken such that its $p$-series
  satisfies the following: for $0 \le t < n$,
  \begin{equation}
    \label{pdiv-epseries-t}
    [p](x) \equiv v_tx^{p^t} \mod (I_t, x^{p^t+1}),
  \end{equation}
  where $I_t$ denotes the ideal $(p,v_1,\ldots,v_{t-1})$. For $t =
  n$, the assumption that $E$ is of height $n$ implies that
  \begin{equation}
    \label{pdiv-epseries-n}
    [p](x) \equiv v_nx^{p^n} \mod (I_n, t^{p^n+1}),
  \end{equation}
  where $v_n \in E^0$ is a unit, and now $I_n$ is the maximal ideal
  $(p,v_1,\ldots,v_{n-1})$ in the local ring $E^0$. Since $E^0$ is
  complete with respect to $I_n$, \cref{pdiv-formal} implies that the
  $p$-power torsion of the formal group of $E$ (the universal
  deformation of $\lG_0$) is a $p$-divisible group of height $n$. We
  will denote this $p$-divisible group by $\lG_E$.
\end{nothing}

\begin{nothing}
  \label{pdiv-ktlocal}
  Now, fix $0 \le t < n$. We consider the spectrum
  $L_t \ce L_{K(t)}E(n)$, which has the structure of an
  $\rE_\infty$-ring because $E$ does. One can show that $L_t$ is also
  an even periodic spectrum whose coefficient ring is obtained from
  $E^0$ by inverting $v_t$ and then completing with respect to the
  ideal $I_t$; that is,
  \[
  L_t^0 \iso \rW(\kappa) \ldb v_1, \ldots, v_{n-1} \rdb
  [v_t^{-1}]_{I_t},
  \]
  and the localization map $E \to L_t$ induces the canonical map of
  coefficient rings $E^0 \to L_t^0$. We let $I_t$ denote the ideal
  $(p,v_1,\ldots,v_{t-1})$ in $L_t^0$ as well (which should not be too
  confusing). The formal group law associated to the spectrum $L_t$ is
  obtained simply by applying the canonical map $E^0 \to L_t^0$ to the
  coefficients of the formal group law associated to $E$. In
  particular, the congruence \cref{pdiv-epseries-t} still holds for
  the formal group law of $L_t$. Since $v_t$ is invertible in $L_t^0$
  and $L_t^0$ is complete with respect to $I_t$, \cref{pdiv-formal}
  implies that the $p$-power torsion of the formal group of $L_t$ is a
  $p$-divisible group of height $t$. We will denote this $p$-divisible
  group by $\lG_{L_t}$.
\end{nothing}

\begin{remark}
  \label{pdiv-noqtor}
  Maybe it seems abusive to denote the $p$-divisible groups above by
  $\lG_E$ and $\lG_{L_t}$, since it looks like they denote the entire
  formal group. One reason this isn't so bad is that \emph{all} the
  torsion in these formal groups is necessarily $p$-power torsion. As
  mentioned earlier, for $s > 0$ the $s$-series of a formal group law
  is always of the form $[s](t) = st + O(t^2)$. If $s$ is coprime to
  $p$, then it is invertible in the rings $E^0$ and $L_t^0$, so $[s]$
  is an invertible power series, and hence the formal group can have
  no $s$-torsion.
\end{remark}

The next result states that torsion of the formal groups associated to
$E$ and $L_t$, i.e. the $p$-divisible groups $\lG_E$ and $\lG_{L_t}$,
can be expressed purely in terms of cohomology.

\begin{proposition}[{\cite[\S5.4]{hkr-char}}]
  Let $F$ denote either the spectrum $E$ or $L_t$.
  \label{pdiv-cohom}
  \begin{enumerate}
  \item \label{pdiv-cohom-m} Let $m > 0$. Then
    \[
    F^*(\rB\lZ/m) \iso F^* \ldb t \rdb / ([m](t)),
    \]
    where $[m]$ denotes the $m$-series of the formal group law
    associated to $F$. In particular, for $m = p^k$ we have
    $F^0(\rB\lZ/p^k) \iso \cO_{\lG_F[p^k]}$,
    i.e. $\Spec(F^0(\rB\lZ/p^k)) \iso \lG_F[p^k]$.
  \item \label{pdiv-cohom-pk} Moreover, if $m = sp^k$ with $s$ coprime
    to $p$, then the map
    \[
    F^* \ldb t \rdb / ([m](t)) \iso F^*(\rB\lZ/m) \to
    F^*(\rB\lZ/p^k) \iso F^* \ldb t \rdb / ([p^k](t))
    \]
    is an isomorphism.
  \item \label{pdiv-cohom-gen} Most generally, if $A$ is a finite
    abelian group whose subgroup of $p$-torsion is given by
    $A[p^\infty] \iso \bigoplus_i \lZ/p^{k_i}$, then
    \[
    F^*(\rB A) \iso F^*(\rB A[p^\infty]) \iso
    \bigotimes_i F^*(\rB\lZ/p^{k_i}),
    \]
    implying $F^0(\rB A) \iso \bigotimes_i \cO_{\lG_F[p^{k_i}]}$.
  \end{enumerate}
\end{proposition}

\begin{proof}[Idea of proof]
  What \cref{pdiv-cohom-m} says intuitively is that the fact that
  $\lZ/m$ is the $m$-torsion in $\rS^1$ is still visible after
  applying $F$-cohomology, most clearly when the cohomology rings are
  viewed algebro-geometrically. More precisely, the exact sequence
  $\lZ/m \to \rS^1 \to \rS^1$ gives us a fiber sequence
  $\rB\lZ/m \to \rB\rS^1 \to \rB\rS^1$, and the above result tells us
  that this sequence identifies $\Spec(F^*(\rB\lZ/m))$ with the
  $m$-torsion in the formal group
  $\Spf(F^*(\rB\rS^1)) \iso \Spf(F^*(\CP^\infty)) \iso \Spf(F^* \ldb t
  \rdb)$ associated to $F$.

  Then \cref{pdiv-cohom-pk} just follows from the fact that the formal
  groups we are considering only have $p$-power torsion, as discussed
  in \cref{pdiv-noqtor}. Finally, \cref{pdiv-cohom-gen} follows from a
  Kunneth isomorphism, which exists because we know the rings
  $\cO_{\lG_F[p^k]}$ are finite free over $F^0$.
\end{proof}

\begin{nothing}
  \label{pdiv-eltmap}
  The map of ring spectra $E \to L_t$ gives natural maps in cohomology
  $E^0(\lB\lZ/p^k) \to L_t(\lB\lZ/p^k)$ for each $k$, which induce
  maps
  \[
  L_t \otimes_{E^0} E^0(\lB\lZ/p^k) \to L_t(\lB\lZ/p^k).
  \]
  Thus, if we let $\lG \ce (\lG_E)_{L_t^0}$ be the base change of
  $\lG_E$ along $E^0 \to L_t^0$, \cref{pdiv-cohom} implies that these
  maps determine a map of $p$-divisible groups
  $i \c \lG_{L_t} \to \lG$ over $L_t^0$.
\end{nothing}

\begin{proposition}[{\cite[\S2.1]{stapleton-tgcm}}]
  \label{pdiv-eltexact}
  The map $i \c \lG_{L_t} \to \lG$ is injective with \'etale
  quotient. That is, we have an exact sequence
  \[
  0 \to \lG_{L_t} \lblto{i} \lG \to \lG_\et \to 0
  \]
  of $p$-divisible groups over $L_t^0$, with $\lG_\et$ \'etale of
  height $n-t$.
\end{proposition}

\begin{nothing}
  \label{pdiv-eltsplit}
  We can now apply the general analysis carried out in
  \cref{pdiv-split} to the exact sequence given by
  \cref{pdiv-eltexact}. Let
  $C_{t,k}^0 \ce (\cO_{\lG[p^k]})^{\otimes(n-t)}$, and
  $\Delta_{t,k} \in C_{t,k}^0$ the determinant inverted in
  \cref{pdiv-split-iso}. Then \cref{pdiv-split-main} tells us that
  $\lG$ splits into a direct sum of $\lG_{L_t}$ and the constant group
  $\Lambda \ce (\lQ_p/\lZ_p)^{n-t}$ over
  $C_t^0 \ce \colim_k C_{t,k}^0[\Delta_{t,k}^{-1}]$. More precisely,
  there is a canonical isomorphism of $p$-divisible groups
  \begin{equation}
    \label{pdiv-eltsplit-geo}
    (\lG_{L_t})_{C_t^0} \oplus \Lambda_{C_t^0} \isoto \lG_{C_t^0}.
  \end{equation}
  Algebraically this is an isomorphism of Hopf algebras
  \[
  C_t^0 \otimes_{L_t^0} \cO_{\lG[p^k]} \isoto
  \l(C_t^0 \otimes_{L_t^0} \cO_{\lG_{L_t}[p^k]}) \otimes_{C_t^0}
  \cO_{\Lambda[p^k]_{C_t^0}} \r)
  \]
  for all $k$. Let's unwrap a few definitions:
  \begin{itemize}
  \item
    $\cO_{\lG[p^k]} \iso L_t^0 \otimes_{E^0} \cO_{\lG_E[p^k]} \iso
    L_t^0 \otimes_{E^0} E^0(\rB\lZ/p^k)$,
    the second isomorphism coming from \cref{pdiv-cohom};
  \item $\cO_{\lG_{L_t}[p^k]} \iso L_t^0(\rB\lZ/p^k)$, again by
    \cref{pdiv-cohom};
  \item $\cO_{\Lambda[p^k]_{C_t^0}} \iso \bigoplus_{\lambda \in
      \Lambda[p^k]} C_t^0$.
  \end{itemize}
  So we may rewrite the above isomorphism of Hopf algebras as
  \begin{align}
    \label{pdiv-eltsplit-alg}
    C_t^0 \otimes_{E^0} E^0(\rB\lZ/p^k) &\isoto
    \l( C_t^0 \otimes_{L_t^0} L_t^0(\rB \lZ/p^k) \r) \otimes_{C_t^0}
    \bigoplus_{\lambda \in \Lambda[p^k]} C_t^0 \\ &\iso
    C_t^0 \otimes_{L_t^0} \bigoplus_{\lambda \in \Lambda[p^k]}
    L_t^0(\rB\lZ/p^k).
  \end{align}
  Note finally that by \cref{pdiv-cohom}, there is an isomorphism
  $C_{t,k}^0 \iso L_t^0(\rB \Lambda^\dual_k)$, where $\Lambda^\dual_k
  \ce (\lZ/p^k)^{n-t}$. The maps $C_{t,k}^0 \to C_{t,k+1}^0$ in the
  colimit defining $C_t$ are then just induced by the canonical maps
  $\Lambda^\dual_{k+1} \to \Lambda^\dual_k$.

  Let's now understand the above isomorphisms in terms of
  cohomology. We have two factors:
  \begin{itemize}
  \item The first is given by the map
    $(\lG_{L_t})_{C_t^0} \to \lG_{C_t^0}$ in \cref{pdiv-eltsplit-geo},
    which algebraically is given by the map
    $C_t^0 \otimes_{E^0} E^0(\rB\lZ/p^k) \to C_t^0 \otimes_{L_t^0}
    L_t^0(\rB\lZ/p^k)$
    in \cref{pdiv-eltsplit-alg}. This is just the base change of our
    inclusion $\lG_{L_t} \inj \lG$, which by definition is
    algebraically just induced by the canonical map
    $E^0(\rB\lZ/p^k) \to L_t^0(\rB\lZ/p^k)$.
  \item The second is given by the map
    $\Lambda_{C_t^0} \to \lG_{C_t^0}$ in \cref{pdiv-eltsplit-geo},
    which algebraically is given by the map
    $C_t^0 \otimes_{E^0} E^0(\rB\lZ/p^k) \to \bigoplus_{\lambda \in
      \Lambda[p^k]} C_t^0$
    in \cref{pdiv-eltsplit-alg}. This is determined by maps
    $\phi_\lambda \c E^0(\rB\lZ/p^k) \to C_t^0$ for each
    $\lambda \in \Lambda[p^k]$. The definition of $\phi_\lambda$ comes
    from the definition of $C_t^0$. We had a sequence of fixed
    generators
    $\lambda^k_1, \ldots, \lambda^k_{n-t} \in \Lambda[p^k]$. So first
    suppose $\lambda = \lambda^k_i$. Then $\phi_\lambda$ is the
    composition
    \[
    E^0(\rB\lZ/p^k) \to
    L_t^0(\rB\lZ/p^k) \lblto{\psi_\lambda}
    L_t^0(\rB\lZ/p^k)^{\otimes(n-t)} \iso C_{t,k} \to
    C_{t,k}[\Delta_{t,k}^{-1}] \to C_t^0
    \]
    where $\psi_\lambda$ is the inclusion of the $i$-th factor of the
    tensor product. More generally, if
    $\lambda = \sum_i a_i \lambda^k_i$ then $\psi_\lambda$ will be the
    corresponding linear combination of inclusions
    $\sum_i a_i \psi_{\lambda^k_i}$, where now summation refers to the
    formal group operation on
    $L_t^0(\rB\lZ/p^k) \iso \cO_{\lG_{L_t}[p^k]}$. We can rephrase
    this using our identification
    \[
    C_{t,k}^0 \iso
    L_t^0(\rB\lZ/p^k)^{\otimes(n-t)} \iso
    L_t^0(\rB\Lambda^\dual_k).
    \]
    Then $\psi_\lambda$ is simply induced by the homomorphism
    $\Lambda^\dual_k \to \lZ/p^k$ which sends
    $(x_1,\ldots,x_{n-t}) \mapsto \sum a_i x_i$.
  \end{itemize}
\end{nothing}

Ok, that was was a fairly long block of algebra and
isomorphism-chasing, but we can cleanly record (a slight
generalization of) our conclusion as follows.

\begin{proposition}
  \label{pdiv-iso-cohom}
  Let $A$ be a finite abelian group, and let $A[p^\infty] \subseteq A$
  denote the subgroup of $p$-torsion. Since $A$ is finite, we can in
  fact pick a finite $k$ for which $A[p^\infty] = A[p^k]$. Then to
  each tuple $a = (a_1,\ldots,a_{n-t}) \in A[p^\infty]^{n-t}$ we can
  associate the homomorphism
  $\Lambda^\dual_k \iso (\lZ/p^k)^{n-t} \to A$ which sends
  $(x_1,\ldots,x_{n-1}) \mapsto \sum_i a_ix_i$. This determines a map
  \[
  E^0(\rB A) \to
  E^0(\rB\Lambda^\dual_k) \to
  L_t^0(\rB\Lambda^\dual_k) \iso
  C_{t,k}^0 \to
  C_t^0,
  \]
  which is independent of our choice of $k$. These maps, along with
  the canonical map $E^0(\rB A) \to L_t^0(\rB A)$, determine an
  isomorphism
  \begin{align}
    \label{pdiv-iso-cohom-eqn}
    C_t^0 \otimes_{E^0} E^0(\rB A) &\isoto
    \l( C_t^0 \otimes_{L_t^0} L_t^0(\rB A) \r) \otimes_{C_t^0}
    \bigoplus_{a \in A[p^\infty]^{n-t}} C_t^0 \\ &\iso
    C_t^0 \otimes_{L_t^0} \bigoplus_{a \in A[p^\infty]^{n-t}}
  L_t^0(\rB A), \nonumber
  \end{align}
  natural in $A$.
\end{proposition}

\begin{proof}
  The naturality of the defined map is clear. To prove it is an
  isomorphism we can pick an isomorphism
  $A[p^\infty] \iso \bigoplus \lZ/p^{k_i}$, and using
  \cref{pdiv-cohom} reduce to the case $A = \lZ/p^k$. Then the claim
  is precisely what was shown above in \cref{pdiv-eltsplit}.
\end{proof}

We now show that we can lift this result from cohomology to an
equivalence at the level of $\rE_\infty$-ring spectra.

\begin{construction}
  \label{pdiv-einftyct}
  For each $k$ we may define an $\rE_\infty$-ring
  \[
  C_{t,k} \ce L_t(\rB\Lambda^\dual_k),
  \]
  the function spectrum whose homotopy groups give the cohomology ring
  $L_t^*(\rB\Lambda^\dual_k)$. In particular, its $\pi_0$ agrees with
  what we have already been calling $C_{t,k}^0$. It is then possible
  to construct an $\rE_\infty$-ring $C_{t,k}[\Delta_{t,k}^{-1}]$ whose
  $\pi_0$ is precisely $C_{t,k}^0[\Delta_{t,k}^{-1}]$. Taking the
  colimit over the canonical maps
  $\rB\Lambda^\dual_{k+1} \to \rB\Lambda^\dual_k$, we may define
  \[
  C_t \ce \colim_k C_{t,k}[\Delta_{t,k}^{-1}]
  \]
  to obtain an $\rE_\infty$-ring whose $\pi_0$ is precisely $C_t^0$
  from above.
\end{construction}

\begin{lemma}
  \label{pdiv-einfty-flat}
  The $\rE_\infty$-ring $C_t$ is even periodic, and the
  $\rE_\infty$-ring maps $E \to L_t \to C_t$ are flat.
\end{lemma}

\begin{proof}
  To see that $C_t$ is even periodic it suffices to show that each
  $C_{t,k}$ is even periodic, which is immediate from the computation
  \cref{pdiv-cohom} of $L_t^*(\rB\Lambda^\dual_k)$. Since $E$ and
  $L_t$ are also even periodic, it suffices to check flatness in
  $\pi_0$. But this we know already: $L_t^0$ is obtained by from $E^0$
  by localization and then completion (of a noetherian ring), hence
  flat; and $C_t^0$ is flat over $L_t^0$ since each $C_{t,k}^0$ is
  (faithfully) flat over $L_t^0$, as proved in \cref{pdiv-split-iso}.
\end{proof}

\begin{proposition}
  \label{pdiv-main}
  The isomorphism \cref{pdiv-iso-cohom-eqn} can be lifted from $\pi_0$
  to an equivalence of $\rE_\infty$-rings
  \[
  C_t \otimes_E E(\rB A) \iso C_t \otimes_{L_t} \bigoplus_{a \in
    A[p^\infty]} L_t(\rB A),
  \]
  natural in $A$.
\end{proposition}

\begin{proof}
  First note that $\pi_0$ of the left- and right-hand sides actually
  recover the left- and right-hand sides of \cref{pdiv-eltsplit-alg}
  because everything in sight is flat. Now, the isomorphism
  \cref{pdiv-iso-cohom-eqn} was determined by maps in cohomology,
  induced by maps between classifying spaces. Since the localization
  map $E \to L_t$ is a map of $\rE_\infty$-rings, it is evident that
  these maps can be lifted to $\rE_\infty$-maps on the associated
  function spaces. So we automatically get the desired map of
  $\rE_\infty$-rings, which by construction is an isomorphism on
  $\pi_0$. But by \cref{pdiv-cohom,pdiv-einfty-flat} both sides are
  even periodic, so it must then be an equivalence.
\end{proof}
