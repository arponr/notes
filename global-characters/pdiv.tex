\section{$p$-divisible groups}
\label{pdiv}

We saw in \cref{chrom} that there is a remarkable connection between
stable homotopy theory and the theory of formal groups. In particular
we saw that, at a fixed prime $p$, the stratification of formal groups
by height is reflected in homotopy theory through these cohomology
theories known at Morava K- and E-theory. It turns out that, to
understand how these different heights interact in stable homotopy
theory, it is extremely useful not just to consider the formal groups
at hand, but to consider the $p$-power torsion of these formal groups
as what are known as $p$-divisible groups. Indeed the main theorem in
this thesis, i.e. the character theory presented in \cref{char}, is
precisely an expression of this philosophy. This sections contains
what we'll need to know there about $p$-divisible groups.

%%%%%%%%%%%%%%%%%%%%%%%%%%%%%%%%%%%%%%%%%%%%%%%%%%%%%%%%%%%%%%%%%%%%%%

\subsection{Generalities}
\label{pdiv-gen}

\begin{definition}
  \label{pdiv-pdivgrp}
  Let $R$ be a commutative ring. A \emph{$p$-divisible group} $\lG$ of
  \emph{height $n$} over $R$ is a system
  $(\lG_k, i_k)_{k \in \lZ_{\ge 0}}$, where for each
  $k \in \lZ_{\ge 0}$:
  \begin{itemize}
  \item $\lG_k$ is a finite flat commutative group scheme over $R$ of
    rank $p^{nk}$;
  \item $i_k \c \lG_k \to \lG_{k+1}$ is a morphism of group schemes
    over $R$ such that the sequence
    \[
    0 \to \lG_k \lblto{i_k} \lG_{k+1} \lblto{(p^k)} \lG_{k+1}
    \]
    is exact, where $(p^k)$ denotes multiplication by $p^k$;
    i.e. $i_k$ identifies $\lG_k$ as the $p^k$-torsion in
    $\lG_{k+1}$.
  \end{itemize}
\end{definition}

\begin{remark}
  Finite morphisms are affine by definition!
\end{remark}

\begin{example}
  The only constant $p$-divisible group of height $n$ is
  $(\lQ_p/\lZ_p)^n$.
\end{example}

\begin{example}
  The $p$-torsion of an abelian scheme over $R$.
\end{example}

\subsection{Splitting the connected-\'etale sequence}
\label{pdiv-split}

In this subsection we construct the universal extension of a ring over
which a given $p$-divisible group splits as the direct sum of its
infinitesimal part and a constant \'etale part. We essentially follow
Stapleton \cite[\S2.8]{stapleton-tgcm}, but we distill this part of
the argument and work in the general situation, rather than the
specific example which will be relevant to character theory; I think
this generality actually clarifies the exposition slightly.

\begin{notation}
  \label{pdiv-split-ntn}
  Throughout this subsection we let $\lG$ be a $p$-divisible group
  over a commutative ring $A$. We let
  \[
  0 \to \lG^\inf \to \lG \to \lG^\et \to 0
  \]
  denote the standard connected-\'etale sequence of $\lG$. We let $n$
  denote the height of $\lG$, $t$ the height of $\lG^\inf$, and
  therefore $n-t$ the height of $\lG^\et$.

  Let $\Lambda \ce (\lZ_p)^{n-t}$ so
  $\Lambda^\dual \iso (\lQ_p/\lZ_p)^{n-t}$.
\end{notation}

To reiterate, armed now with notation, our goal is to find the
universal extension $A_t$ of $A$ over which there is an isomorphism
\begin{equation}
\label{pdiv-split-ad}
\lG^\inf_{A_t} \oplus \Lambda^\dual_{A_t} \isoto
\lG_{A_t}
\end{equation}
of $p$-divisible groups over $A_t$. We'll construct $A_t$ in two
steps, the first of which will be to find the universal extension
$A'_t$ of $R$ over which there just is a map
\begin{equation}
\label{pdiv-split1-ad}
\Lambda^\dual_{A'_t} \to \lG_{A'_t},
\end{equation}
i.e. compatible maps $\Lambda^\dual[p^k]_{A'_t} \to \lG[p^k]_{A'_t}$
for all $k \in \lZ_{\ge 0}$.

\begin{proposition}
  \label{pdiv-split-step1}
  \begin{enumerate}[leftmargin=*]
  \item \label{pdiv-split-step1-finite} For $k \in \lZ_{\ge 0}$, the
    functor
    \[
    F_k \c \Alg_A \to \Set, \quad
    B \mapsto \Hom_B(\Lambda^\dual[p^k]_B, \lG[p^k]_B)
    \]
    is corepresented by the $A$-algebra
    $R'_{t,k} \ce (\cO_{\lG[p^k]})^{\otimes(n-t)}$.
  \item \label{pdiv-split-step1-limit} The functor
    \[
    F \c \Alg_A \to \Set, \quad
    B \mapsto \Hom_B(\Lambda^\dual_B, \lG_B)
    \]
    is corepresented by the $A$-algebra $A_t \ce \colim_k A_{t,k}$,
    where the morphisms $A_{t,k} \to A_{t,k+1}$ defining the colimit
    are induced by \cref{pdiv-split-step1-finite} via natural
    transformations $F_{k+1} \to F_k$.
  \end{enumerate}
\end{proposition}

\begin{proof}
  We first prove \cref{pdiv-split-step1-finite}, so fix
  $k \in \lZ_{\ge 0}$. Let $B \in \Alg_A$. Since $\Lambda^\dual[p^k]$
  and $\lG[p^k]$ are affine, specifying a homomorphism of group
  schemes $\Lambda^\dual[p^k]_B \to \lG[p^k]_B$ over $B$ is equivalent
  to specifying a homomorphism of Hopf algebras
  \[
  B \otimes_A \cO_{\lG[p^k]} \iso
  \cO_{\lG[p^k]_B} \to
  \cO_{\Lambda^\dual[p^k]_B} \iso
  \prod_{x \in \Lambda^\vee[p^k]} B
  \]
  under $B$. By restricting to the factors on the right-hand side
  corresponding to our chosen $(n-t)$ generators of
  $\Lambda^\vee[p^k]$, this is furthermore equivalent to specifying
  $(n-t)$ homomorphisms of $B$-algebras
  $B \otimes_A \cO_{\lG[p^k]} \to B$. Of course this is equivalent to
  specifying $(n-t)$ homomorphisms of $A$-algebras
  $\cO_{\lG[p^k]} \to B$, which is equivalent to specifying a single
  homomorphism of $A$-algebras
  $(\cO_{\lG[p^k]})^{\otimes(n-t)} \to B$. We have now described a
  bijection $F(B) \iso \Hom_{\Alg_A}(R'_{t,k},B)$ which clearly is
  natural in $B$, proving the claim.

  We now prove \cref{pdiv-split-step1-limit}.
\end{proof}

%%%%%%%%%%%%%%%%%%%%%%%%%%%%%%%%%%%%%%%%%%%%%%%%%%%%%%%%%%%%%%%%%%%%%%

\subsection{Chromatic examples}
\label{pdiv-chromex}