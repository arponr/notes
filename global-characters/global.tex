\section{Global equivariant homotopy theory}
\label{global}

In \cref{intro} we translated the character theory of a finite group
$G$ into a statement in $G$-equivariant homotopy theory. If we want to
take this perspective seriously as a strategy for giving a
\emph{proof} of character theory---well, more interestingly, for
giving proofs of generalizations of character theory---it will turn
out to be crucial that we can make this statement for \emph{all}
finite groups $G$, and that there is moreover some relationship among
the statements for all these groups. This idea should be familiar from
the classical setting of representation theory: one often tries to get
a handle on the representations of a group $G$ by considering the
representations induced from subgroups $H \subseteq G$.

To systematically handle such phenomena, we want some kind of homotopy
theory in which we can study equivariance for all finite\footnote{We
  restrict ourselves to finite groups because these are the groups
  which will be relevant to our discussion later on. In other
  situations one might be interested in, say, all compact Lie
  groups. The framework we set up in this section should carry over
  without trouble in such situations.}  groups simultaneously. This
idea goes by the name of \emph{global equivariant homotopy
  theory}. This section is devoted to setting up the basic framework
of global equivariant homotopy theory.\footnote{The first time my eyes
  saw the global perspective was when reading J. Lurie's survey
  \cite{lurie-elliptic-survey}, but the ideas are left ever so
  slightly implicit there and thus my brain didn't process what was
  really happening for quite some time. What shed light on this
  picture for me was a note of C. Rezk \cite{rezk-global-cohesion},
  from which I have learned much of what is included in this section
  (and the presentation here pretty clearly reflects that).}

%%%%%%%%%%%%%%%%%%%%%%%%%%%%%%%%%%%%%%%%%%%%%%%%%%%%%%%%%%%%%%%%%%%%%%

\subsection{The global indexing category}
\label{global-indexcat}

To guide our definition of global equivariant homotopy theory, we
recall the setup of (unstable) $G$-equivariant homotopy theory for a
fixed group $G$. This is summarized by the following theorem, often
referred to as Elmendorf's theorem. We currently don't have any time
to discuss its proof, but one can see \cite{nlab-equivhomotopy} for
more, including a detailed list of references.

\begin{theorem}[Elmendorf]
  \label{global-elmendorf}
  Let $G$ be a finite group. The following homotopy theories are all
  equivalent:
  \begin{enumerate}
  \item \label{global-elmendorf-gtop} $G\dash\Top$: the relative
    category of topological spaces equipped with a $G$-action, with
    weak equivalences defined to be the ($G$-equivariant) maps
    $X \to Y$ which induce weak equivalences (of topological spaces)
    on the fixed-point spaces $X^H \to Y^H$ for all subgroups
    $H \subseteq G$. (In fact there is a standard model structure on
    this relative category.)
  \item \label{global-elmendorf-gcw} $G\dash\CW$: the relative
    category of $G$-CW-complexes, with weak equivalences defined to be
    the ($G$-equivariant) homotopy equivalences. (In fact this is just
    the subcategory of fibrant-cofibrant objects in the model
    structure on $G\dash\Top$ mentioned above.)
  \item \label{global-elmendorf-presheaf} $\Space_{\Orb(G)}$: the
    homotopy theory of presheaves of spaces on the category of
    $G$-orbits $\Orb(G)$, i.e. the $\infty$-category of functors
    $\Orb(G)^\op \to \Space$.
  \end{enumerate}
  The equivalence of \cref{global-elmendorf-gtop} and
  \cref{global-elmendorf-presheaf} is exhibited by the functor
  $G\dash\Top \to \Space_{\Orb(G)}$ sending $X \in G\dash\Top$ to the
  presheaf $O \mapsto \Map_{G\dash\Top}(O,X)$, or in other words
  $G/H \mapsto X^H$.
\end{theorem}

\begin{notation}
  \label{global-gspace}
  We'll denote the equivalent homotopy theories described in
  \cref{global-elmendorf} by $G\dash\Space$ when we want to make a
  statement without evoking any particular model.
\end{notation}

As far as I understand, it is formulation
\cref{global-elmendorf}\cref{global-elmendorf-gcw} of $G\dash\Space$
which makes it interesting and useful: it turns out that
$G$-CW-complexes up to homotopy equivalence effectively model the
spaces of interest to us, e.g. smooth $G$-manifolds (see
\cite{mo-equivhmptygrps} for some discussion about this, and
\cref{abdesc-flag-faithflat} for an example of us critically using
this fact). However, the presheaf formulation
\cref{global-elmendorf}\cref{global-elmendorf-presheaf} is quite
convenient from a categorical perspective: $\Space_{\Orb(G)}$ comes to
us immediately as an $\infty$-category, in terms of the
$\infty$-category of spaces, allowing us to speak very easily about
limits and colimits for example. Such a framework will also be
convenient in the global setting (evidence abounds in this section and
the next), inspiring us to formulate global equivariant homotopy
theory also as the homotopy theory of presheaves of spaces on some
indexing category $\Glo$. Before giving the definition of $\Glo$, we
give some intuition guiding what the definition should be.

\begin{intuition}
  \label{global-indexint}
  \begin{enumerate}[leftmargin=*]
  \item \label{global-indexint-unstable} Unstably, we want to somehow
    assemble the homotopy theories $G\dash\Space$ for all $G$. So a
    natural starting point is to ask: how are all these homotopy
    theories related? Of course, whenever we have a group homomorphism
    $\phi \c H \to G$, we can pull back an action of $G$ on a space to
    an action of $H$, giving us a restriction functor
    $\phi^* \c G\dash\Space \to H\dash\Space$. If this were all there
    was to the answer, then we could just define $\Glo$ to be the
    category of groups. However there is something else to remember:
    whenever $\psi \c H \to G$ is a $G$-conjugate $g\phi g^{-1}$ of
    $\phi \c H \to G$, there is a natural equivalence
    $\phi^* \isoto \psi^*$, given on a $G$-space $X$ by the action
    $g \c \phi^*(X) \to \psi^*(X)$.

    We can summarize the above discussion as follows. If we view our
    groups $G$ as groupoids in the usual manner, then $\Space_G$ is
    functorial not only in category of groups, but in the
    \emph{$2$-category} of such groupoids (where the $1$-morphisms are
    functors, corresponding to group homomorphisms, and the
    $2$-morphisms are natural transformations, corresponding to
    conjugation of group homomorphisms). We will spell this out in
    detail in \cref{global-topgpds,global-indexcat-dfn} to give our
    definition of $\Glo$.

  \item \label{global-indexint-stable} Stably, intuition comes from
    the most basic example of global equivariant cohomology theory
    that we want to account for: the Borel equivariant cohomology
    $E^\Bor$ associated to any non-equivariant cohomology theory
    $E$. Recall that Borel $G$-equivariant cohomology is defined so
    that for a $G$-space $X$,
    \[
    E^\Bor_G(X) \iso E(X_{\rh G}) \iso E(\rE G \times_G X).
    \]
    In particular, for $X = \pt$ the trivial $G$-space,
    $E^\Bor_G(\pt) \iso E(\rB G)$. Thus we might imagine $E^\Bor$ as
    something functorial with respect to the classifying space
    $\rB G$.  We will show in \cref{global-stable} that the definition
    motivated by \cref{global-indexint-unstable} will realize this
    intuition as well.
  \end{enumerate}
\end{intuition}

\begin{notation}
  \label{global-topgpds}
  At various points in our discussion of global equivariant homotopy
  theory, it will be convenient to work with \emph{topological
    groupoids}, i.e. groupoids internal to the ordinary category of
  topological spaces. These of course include ordinary discrete
  groupoids. So let's just quickly establish some notation and
  terminology for these things:
  \begin{enumerate}
  \item If $\cG$ is a topological groupoid, the usual \emph{nerve}
    construction gives us a simplicial space $\rN\cG$, and we define
    the \emph{classifying space} $\rB\cG$ to be the geometric
    realization $|\rN\cG|$ of the nerve.
  \item The most relevant class of examples for us are the
    \emph{action groupoids}: if $X$ is a topological space with a
    $G$-action, then there is a topological groupoid $X \sslash G$
    whose object space is $X$ and morphism space is $X \times G$, with
    the structure maps $X \times G \doubto X$ being the projection and
    the action. In this case the classifying space $\rB(X \sslash G)$
    is equivalent to the homotopy quotient
    $X_{\rh G} \iso \rE G \times_G X$. A particularly relevant example
    in this class comes from the trivial $G$-space $\pt$, with
    $\rB(\pt \sslash G)$ being the classifying space
    $\rB G \iso \rE G/G$ of the group $G$. When there shouldn't be a
    possibility of confusion, we denote the groupoid $\pt \sslash G$
    simply by $G$.
  \item If $\cG,\cH$ are topological groupoids then there is a
    topological groupoid $\Fun(\cG,\cH)$ of functors $\cG \to \cH$.
    Set-theoretically $\Fun(\cG,\cH)$ has functors as objects and
    natural transformations as morphisms; the topologies on these are
    induced by the relevant mapping space topologies.
  \item We say a (discrete) groupoid $\cG$ is \emph{connected} if for
    any objects $x,y \in \cG$ there is a morphism $x \to y$ in
    $\cG$. We say $\cG$ is \emph{finite} if for any object $x \in \cG$
    the automorphism group $\Aut_\cG(x)$ is finite.
  \end{enumerate}
\end{notation}

\begin{definition}
  \label{global-indexcat-dfn}
  The \emph{global indexing category} $\Glo$ is the $2$-category of
  finite connected (discrete) groupoids; the $1$-morphisms are given
  by functors of groupoids and the $2$-morphisms by natural
  transformations.

  Recall that whenever we choose an object of a connected groupoid
  $\cG$ we get an isomorphism $\cG$ with the group $\Aut_\cG(x)$. So,
  more concretely one can think of objects in $\Glo$ simply as finite
  groups $G$, in which case $\Glo$ is precisely the $2$-category of
  finite groups described in
  \cref{global-indexint}\cref{global-indexint-unstable}. We defined
  $\Glo$ with more general objects because it is convenient to not
  have to choose basepoints.
\end{definition}

We conclude this subsection by showing how to relate the orbit
category $\Orb(G)$ of a fixed group $G$ with the global indexing
category $\Glo$. This will allow us in
\cref{global-unstable,global-stable} to relate each fixed
$G$-equivariant homotopy theory with global equivariant homotopy
theory.

\begin{construction}
  \label{global-orbtoglo}
  Let $G$ be a finite group. We construct a fully faithful embedding
  $\Theta_G$ of $\Orb(G)$ into the overcategory $\Glo_{/G}$ as
  follows. First define a functor $\theta_G \c \Orb(G) \to \Glo$ by
  sending $T \in \Orb(G)$ to $\theta_G(T) \ce T \sslash G$, which is
  connected by definition of an orbit and finite since $G$ is; this is
  evidently functorial. (If one chooses basepoints, then $\theta_G$
  can alternatively be thought of as sending $G/H \mapsto H$.)  Now
  note that for $\pt \in \Orb(G)$ the trivial orbit we have
  $\theta_G(\pt) \iso G$. Since $\pt$ is a final object in $\Orb(G)$,
  the functor $\theta_G$ determines a functor
  \[
  \Orb(G) \iso
  \Orb(G)_{/\pt} \lblto{\theta_G}
  \Glo_{/\theta_G(\pt)} \iso
  \Glo_{/G},
  \]
  which we denote $\Theta_G$. That is, $\Theta_G$ assigns to $T$ the
  map $\theta_G(T \to \pt)$, which is a map
  $\theta_G(T) \to \theta_G(\pt) \iso G$. Seeing that $\Theta_G$ is
  fully faithful is a fairly simple exercise, which I'll fill in
  later.
\end{construction}

%%%%%%%%%%%%%%%%%%%%%%%%%%%%%%%%%%%%%%%%%%%%%%%%%%%%%%%%%%%%%%%%%%%%%%

\subsection{The unstable side: global spaces}
\label{global-unstable}

Now that we've defined the global indexing category, the definition of
``space'' in global equivariant homotopy theory is immediate.

\begin{definition}
  \label{global-space}
  A \emph{global space} is a presheaf of spaces on $\Glo$, i.e. a
  functor $\Glo^\op \to \Space$. We denote the $\infty$-category of
  global spaces by $\Space_\Glo$.
\end{definition}

\begin{notation}
  \label{global-yoneda}
  We denote the Yoneda embedding $\Glo \to \Space_\Glo$ by $\lB$. So
  for each finite group $G$ we have a global space $\lB G$.
\end{notation}

We claimed in \cref{global-indexint}\cref{global-indexint-unstable}
that this definition of global equivariant homotopy theory would be a
useful way to assemble the homotopy theories of $G$-spaces for all
$G$. We justify this claim now, by embedding, for each $G \in \Glo$,
the homotopy theory of $G$-spaces into the homotopy theory of global
spaces over $\lB G$.

\begin{proposition}
  \label{global-gspace-adj}
  Let $G \in \Glo$. There is an adjunction
  \[
  \Delta_G \c G\dash\Space \fromto
  (\Space_{\Glo})_{/\lB G} \lc \Gamma_G,
  \]
  where $(\Space_\Glo)_{/\lB G}$ denotes the overcategory of maps of
  global spaces $X \to \lB G$. Moreover the left adjoint $\Delta_G$ is
  fully faithful.
\end{proposition}

\begin{proof}
  In \cref{global-orbtoglo} we constructed a fully faithful embedding
  $\Theta_G \c \Orb(G) \inj \Glo_{/G}$. From this we obtain an
  adjunction of presheaf categories
  \[
  (\Theta_G)_! \c \Space_{\Orb(G)} \fromto
  \Space_{\Glo_{/G}} \lc (\Theta_G)^*,
  \]
  where $(\Theta_G)^*$ is restriction along $\Theta_G$, and its left
  adjoint $(\Theta_G)_!$ is left Kan extension along $\Theta_G$ (this
  adjunction is essentially the definition of left Kan
  extension). Now, by a general fact about the interaction of presheaf
  categories and overcategories, there is an equivalence
  $\Space_{\Glo_{/G}} \iso (\Space_\Glo)_{/\lB G}$. Applying this and
  the fact/definition \cref{global-elmendorf} that
  $G\dash\Space \iso \Space_{\Orb(G)}$, we obtain the claimed
  adjunction $\Delta_G \dashv \Gamma_G$. Finally, $\Theta_G$ being
  fully faithful implies immediately that left Kan extension
  $(\Theta_G)_!$ is fully faithful, and hence $\Delta_G$ is fully
  faithful.
\end{proof}

\begin{remark}
  \label{global-gspace-adj-unwrap}
  It's a straightforward exercise in definition-chasing to unwrap the
  categorical formalities used in the proof of
  \cref{global-gspace-adj} and obtain the following more explicit
  formulae:
  \begin{enumerate}
  \item \label{global-gspace-adj-unwrap-right} Viewing $\Gamma_G$ as a
    functor $(\Space_\Glo)_{/\lB G} \to \Space_{\Orb(G)}$, it assigns
    to a map of global spaces $X \to \lB G$ the presheaf
    $T \mapsto X(T \sslash G) \times_{\lB G(T \sslash G)} \{\phi\}$,
    where $\phi \c T \sslash G \to G$ is the canonical map of
    groupoids. After choosing basepoints, the presheaf could also be
    described as $G/H \mapsto X(H) \times_{\lB G(H)} \{\phi\}$, where
    $\phi \c H \to G$ is the inclusion of a subgroup.
  \item \label{global-gspace-adj-unwrap-left} There is a functor
    $\delta_G \c G\dash\Top \to \Space_\Glo$ which to
    $X \in G\dash\Top$ assigns the presheaf
    $\cH \mapsto \rB \Fun(\cH, X \sslash G)$. In the case $\cH$ is a
    group $H$, note that $\Fun(H, X \sslash G)$ is simply the action
    groupoid
    \[
    \l( \coprod_{\phi \in \Hom(H,G)} X^{\im(\phi)} \r) \sslash G,
    \]
    e.g. $\delta_G(\pt) \iso \lB G$. Then, viewing
    $\Delta_G$ as a functor $G\dash\Top \to (\Space_\Glo)_{/\lB G}$,
    it assigns to $X \in G\dash\Top$ the map $\delta_G(X \to \pt)$,
    which is a map $\delta_G(X) \to \delta_G(\pt) \iso \lB G$.
  \end{enumerate}
\end{remark}

That's all we have to say about the general theory of global spaces,
since that's all that will be relevant to our study of character
theory. See \cite{rezk-global-cohesion} for some more interesting
phenomena in this setting.

%%%%%%%%%%%%%%%%%%%%%%%%%%%%%%%%%%%%%%%%%%%%%%%%%%%%%%%%%%%%%%%%%%%%%%

\subsection{The stable side: global spectra}
\label{global-stable}

Unless I work something out in the next couple of days, we'll be
taking a na\"ive approach to global spectra.

\begin{definition}
  \label{global-spectrum}
  A \emph{global spectrum} is a presheaf of spectra on $\Glo$, i.e. a
  functor $\Glo^\op \to \Spect$. We denote the $\infty$-category of
  global spaces by $\Spect_\Glo$.
\end{definition}

\begin{remark}
  Since $\Space_\Glo$ is the free cocompletion of $\Glo$, one can
  equivalently view a global spectrum as a functor
  $(\Space_\Glo)^\op \to \Spect$ which takes colimits of global spaces
  to limits of spectra.
\end{remark}