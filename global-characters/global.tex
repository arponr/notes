\section{Global equivariant homotopy theory}
\label{global}

In \cref{intro} we translated the character theory of a finite group
$G$ into a statement in $G$-equivariant homotopy theory. If we want to
take this perspective seriously as a strategy for giving a
\emph{proof} of character theory---well, more interestingly, for
giving proofs of generalizations of character theory---it will turn
out to be crucial that we can make this statement for \emph{all}
(finite groups) $G$, and that there is moreover some relationship
among the statements for all these groups. This idea should be
familiar from the classical setting of representation theory: one
often tries to get a handle on the representations of a group $G$ by
considering the representations induced from subgroups
$H \subseteq G$.

To systematically handle such phenomena, we want some kind of homotopy
theory in which we can study equivariance for all (finite) groups
simultaneously. This idea goes by the name of \emph{global equivariant
  homotopy theory}. This section is devoted to setting up the basic
framework of global equivariant homotopy theory.\footnote{The first
  time my eyes saw the global perspective was when reading J. Lurie's
  survey \cite{lurie-elliptic-survey}, but the ideas are left ever so
  slightly implicit there and thus my brain didn't process what was
  really happening for quite some time. What shed light on this
  picture for me was a note of C. Rezk \cite{rezk-global-cohesion},
  from which I have learned much of what is included in this section
  (and the presentation here pretty clearly reflects that).}

%%%%%%%%%%%%%%%%%%%%%%%%%%%%%%%%%%%%%%%%%%%%%%%%%%%%%%%%%%%%%%%%%%%%%%

\subsection{The global indexing category}
\label{global-indexcat}

To guide our definition of global equivariant homotopy theory, we
recall the setup of $G$-equivariant homotopy theory for a fixed group
$G$.

\begin{theorem}[Elmendorf]
  \label{global-elmendorf}
  Let $G$ be a compact Lie group. The following homotopy theories are
  all equivalent:
  \begin{enumerate}
  \item \label{global-elmendorf-gcw} $G\dash\CW$: the category of
    $G$-CW-complexes, with weak equivalences defined to be the
    ($G$-equivariant) homotopy equivalences.
  \item \label{global-elmendorf-gtop} $G\dash\Top$: the category of
    topological spaces equipped with a $G$-action, with weak
    equivalences defined to be the ($G$-equivariant) maps $X \to Y$
    which induce weak equivalences (of topological spaces) on the
    fixed-point spaces $X^H \to Y^H$, for all closed subgroups
    $H \subseteq G$.
  \item \label{global-elmendorf-presheaf} $\Space_{\cO(G)}$: the
    homotopy theory of presheaves of spaces on the category of
    $G$-orbits $\cO(G)$. Abstractly this is the $\infty$-category of
    $\infty$-functors $\cO(G)^\op \to \Space$. Concretely this is
    presented by the category of ordinary functors
    $\cO(G)^\op \to \Top$, with weak equivalences defined object-wise.
  \end{enumerate}
  The equivalence of \cref{global-elmendorf-gtop} and
  \cref{global-elmendorf-presheaf} is exhibited by the functor
  $G\dash\Top \to \Space_{\cO(G)}$ sending a $G$-space $X$ to the
  presheaf $O \mapsto \Map_{G\dash\Top}(O,X)$, or in other words
  $G/H \mapsto X^H$.
\end{theorem}

\begin{proof}
  Omitted. See \cite{nlab-equivhomotopy} for more, including a list of
  references.
\end{proof}

\begin{notation}
  \label{global-gspace}
  When we don't want to specify any particular model, we denote the
  equivalent homotopy theories given in \cref{global-elmendorf} by
  $G\dash\Space$, and refer to it as the homotopy theory of
  \emph{$G$-spaces} or \emph{$G$-equivariant homotopy theory}.
\end{notation}

As far as I understand, it is formulation
\cref{global-elmendorf}\cref{global-elmendorf-gcw} which is the
motivating definition of $G\dash\Space$: it turns out that
$G$-CW-complexes up to homotopy equivalence effectively model the
spaces of interest to us, e.g. smooth $G$-manifolds (see
\cite{mo-equivhmptygrps} for some discussion about this, and \ref{}
for an example of us critically using this fact). However, it the
presheaf formulation $\Space_{\cO(G)}$ which will inspire our
formulation of global equivariant homotopy theory. (Note that this
presheaf formulation is convenient from a higher-categorical
perspective, since $\Space_{\cO(G)}$ comes to us immediately as an
$\infty$-category, in terms of the $\infty$-category of spaces.)

Namely, we will define global equivariant homotopy theory also as the
homotopy theory of presheaves of spaces on some indexing category
$\Glo$. Before giving the definition of $\Glo$, we give some intuition
guiding what the definition should be:
\begin{itemize}
\item Unstably, we want to somehow assemble the homotopy theories
  $G\dash\Space$ for all $G$. So a natural starting point is to ask:
  how are all these homotopy theories related? The first half of the
  answer to this is clear: whenever we have a group homomorphism
  $\phi \c H \to G$, we can pull back an action of $G$ on a space to
  an action of $H$, and this gives us a restriction functor
  $\phi^* \c G\dash\Space \to H\dash\Space$. If this were all there
  was to the answer, then we could just define $\Glo$ to be the
  category of relevant groups. But there is indeed a second half to
  this answer: we should remember when two homomorphisms $\phi,\psi$
  determine equivalent restriction functors $\phi^*,\psi^*$.
\item Stably, intuition comes from the most basic example of global
  equivariant cohomology theory  that we want to account for: the
  Borel-equivariant cohomology $E^\Bor$ associated to an ordinary
  non-equivariant cohomology theory $E$.
\end{itemize}

\begin{notation}
  \label{global-topgpds}
  Throughout this section it will be convenient to work with
  \emph{topological groupoids}, i.e. groupoids internal to the
  ordinary category of topological spaces. So let's just quickly
  establish some notation for these things:
  \begin{enumerate}
  \item If $\cG$ is a topological groupoid, the usual \emph{nerve}
    construction gives us a simplicial space $\rN\cG$, and we define
    the \emph{classifying space} $\rB\cG$ to be the geometric
    realization $|\rN\cG|$ of the nerve.
  \item The most relevant class of examples for us are the
    \emph{action groupoids}: if $X$ is a topological space with a
    $G$-action, then there is a topological groupoid $X \sslash G$
    whose object space is $X$ and morphism space is $X \times G$, with
    the structure maps $X \times G \doubto X$ being the projection and
    the action. In this case the classifying space $\rB(X \sslash G)$
    is equivalent to the homotopy quotient
    $X_{hG} \iso (\rE G \times X)/G$. A particularly relevant example
    in this class comes from the trivial $G$-space $\pt$, with
    $\rB(\pt \sslash G)$ being the classifying space
    $\rB G \iso \rE G/G$ of the group $G$.
  \item If $\cG,\cH$ are topological groupoids then there is a
    topological groupoid $\Fun(\cG,\cH)$ of functors $\cG \to \cH$.
    Set-theoretically $\Fun(\cG,\cH)$ has functors as objects and
    natural transformations as morphisms; the topologies on these are
    induced by the relevant mapping space topologies.
  \end{enumerate}
\end{notation}

\begin{definition}
  \label{global-indexcat-dfn}
  The \emph{global indexing category} $\Glo$ is the
  topologically-enriched category whose objects are finite\footnote{We
    restrict ourselves to finite groups because these are the groups
    which will be relevant to our discussion later on. In other
    situations one might be interested in, say, all compact Lie
    groups. The framework we set up in this section should carry over
    without trouble in such situations.} groups $G$, with the mapping
  space $\Map_{\Glo}(H,G)$ defined to be the classifying space
  $\rB \Fun(\pt \sslash H, \pt \sslash G)$ defined in
  \cref{global-topgpds}; composition is defined in the obvious manner.
\end{definition}

\subsection{The unstable side: global spaces}
\label{global-unstable}

Hi

\subsection{The stable side: global spectra}
\label{global-stable}

Hello