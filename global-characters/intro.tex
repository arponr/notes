\section{Introduction}
\label{intro}

\newcommand{\Rep}{\operatorname{Rep}}
\newcommand{\Cl}{\operatorname{Cl}}
\newcommand{\tr}{\operatorname{tr}}

We set out from someplace quite classical: the representation theory
of finite groups, in particular the theory of characters
therein. Let's briefly recall how this story goes. Let $G$ be a finite
group. The goal of character theory is to understand a representation
$\rho \c G \to \Aut(V)$ of $G$ on a complex vector space $V$ via its
associated \emph{character}: the function
\begin{equation}
  \label{intro-classical-character}
  \chi_\rho \ce {\tr} \circ \rho \c G \to \lC,
\end{equation}
where $\tr$ denotes the trace map. The beauty of character theory is
that it accomplishes this goal quite thoroughly. This fact can be
articulated succinctly as follows.

\begin{definitions}
  \label{intro-classical-character-dfns}
  \begin{enumerate}[leftmargin=*]
  \item Consider the collection of isomorphism classes of complex
    representations of $G$, which forms a commutative semiring under
    the operations of direct sum and tensor product; the Grothendieck
    ring of this semiring, i.e. the commutative ring obtained by
    formally adjoining additive inverses, is denoted $\Rep(G)$ and
    referred to as the \emph{representation ring} of $G$.
  \item A function on $G$ is called a \emph{class function} if it is
    conjugation-invariant. Let $\Cl(G;\lC)$ denote the $\lC$-algebra
    of class functions $G \to \lC$.
  \end{enumerate}
\end{definitions}

\begin{theorem}[Classical character theory]
  \label{intro-classical-thm}
  The assignment $\rho \mapsto \chi_\rho$ defined in
  \cref{intro-classical-character} determines a ring homomorphism
  $\Rep(G) \to \Cl(G;\lC)$, which furthermore induces an isomorphism
  \[
  \lC \otimes_\lZ \Rep(G) \isoto \Cl(G;\lC)
  \]
  of $\lC$-algebras.
\end{theorem}

This theorem is just a restatement of the main facts one usually hears
when learning character theory:
\begin{itemize}
\item characters behave well with respect to direct sums and tensor
  products of representations (because traces do);
\item a representation is determined up to isomorphism by its
  character;
\item the characters of representations span the vector space of class
  functions.
\end{itemize}

The goal of the this thesis is to explain a generalization of this
classical character theory, which suggests itself when we view
\cref{intro-classical-thm} with the eyes of a (stable) homotopy
theorist.

%%%%%%%%%%%%%%%%%%%%%%%%%%%%%%%%%%%%%%%%%%%%%%%%%%%%%%%%%%%%%%%%%%%%%%

\subsection{Translating}
\label{intro-trans}

We will translate \cref{intro-classical-thm} into the language of
homotopy theory via (topological, complex) K-theory. Recall that
K-theory is a (generalized) cohomology theory arising from the theory
of complex vector bundles. However, for the purposes of this
introduction, why don't we just focus on the degree $0$ term of this
cohomology theory. For a space $X$ this is denoted
$\rK(X) = \rK^0(X)$, and is the Grothendieck ring associated to the
semiring of isomorphism classes of (complex) vector bundles on $X$
(the operations in this semiring again coming from direct sum and
tensor product). For example, over the trivial space $X = \pt$, a
vector bundle is simply a vector space, determined up to isomorphism
by its dimension; $\rK(\pt)$ is therefore the Grothendieck ring of
$\lZ_{\ge 0}$, which of course is just $\lZ$.

There are two natural \emph{equivariant} analogues of K-theory. That
is, there are two candidates which might replace K-theory when we
would like to study not just spaces but spaces equipped with a
$G$-action for some finite group $G$. The first is a na\"ive, formal
construction which one can make whenever one wants a $G$-equivariant
version of a cohomology theory, known as the associated \emph{Borel
  equivariant cohomology theory}. Given a $G$-space $X$, we may always
replace it with a homotopy equivalent space $Y$ on which $G$ acts
freely: there is a contractible space $\rE G$ with a free $G$ action,
so we may take $Y \ce \rE G \times X$. When $G$ acts freely on $Y$, it
is natural to expect that the $G$-equivariant version of a cohomology
theory applied to $Y$ recovers the original cohomology theory's value
on the quotient space $Y/G$. This motivates the Borel construction
(here applied to K-theory), given by defining
\[
\rK^\Bor_G(X) \ce \rK((\rE G \times X)/G).
\]
In particular, when $X$ is the trivial $G$-space $\pt$ we have
$\rK^\Bor_G(\pt) \iso \rK(\rB G)$, where $\rB G \iso \rE G / G$ is the
classifying space of $G$.

But for $\rK$-theory there is another, more geometric candidate for
its equivariant analogue. Namely, there is a notion of a
\emph{$G$-equivariant (complex) vector bundle} over a $G$-space
$X$. So we may analagously define a $G$-equivariant cohomology theory
which in degree $0$ is given by the Grothendieck ring $\rK_G(X)$ of
$G$-equivariant vector bundles on $X$. Now, a $G$-equivariant vector
bundle over the trivial $G$-space $\pt$ is simply a representation of
$G$, so by definition we have $\rK_G(\pt) \iso \Rep(G)$. It is with
this tautological renaming of $\Rep(G)$ that we look to begin viewing
character theory through the lens of homotopy theory.

The first non-tautological step we take is to compare these
equivariant theories $\rK^\Bor_G$ and $\rK_G$. The theory $\rK_G$
still has the expected property that when $G$ acts freely on $Y$ we
have $\rK_G(Y) \iso \rK(Y/G)$. Thus for any $G$-space $X$ we have a
natural comparison map
\[
\rK_G(X) \to
\rK_G(\rE G \times X) \iso
\rK((\rE G \times X)/G) \iso
\rK^\Bor_G(X),
\]
where the first map is induced by the projection
$\rE G \times X \to X$. Of course this map is not an isomorphism, but
a theorem of Atiyah-Segal tells us it's not terribly far from being an
isomorphism. This is easiest to state for $X = \pt$, in which case the
above gives a map\footnote{This map can alternatively be described as
  follows: a representation of $G$ is a map
  $\pi_1(\rB G) \iso G \to \Aut(V)$, which determines a local system
  with fiber $V$ on $\rB G$, which determines a vector bundle with
  fiber $V$ on $\rB G$.}
\begin{equation}
  \label{intro-atsegmap}
  \Rep(G) \iso \rK_G(\pt) \to \rK^\Bor_G(\pt) \iso \rK(\rB G).
\end{equation}

\begin{definition}
  \label{intro-aug}
  To a representation $V$ of $G$ we can associate its dimension
  $\dim(V)$. This extends to a ring morphism
  $\dim \c \Rep(G) \to \lZ$. We call the kernel of this morphism the
  \emph{augmentation ideal} of $\Rep(G)$, and denote it $\Aug(G)$.
\end{definition}

\begin{theorem}[Atiyah-Segal completion]
  \label{intro-atseg}
  The map \cref{intro-atsegmap} exhibits the ring $\rK(\rB G)$ as the
  completion of the ring $\Rep(G)$ at the augmentation ideal
  $\Aug(G)$.
\end{theorem}

We now have a precise relationship between $\rK(\rB G)$ and
$\rK_G(\pt)$, and an identification of $\rK_G(\pt)$ with
$\Rep(G)$. Thus we might hope to rephrase the isomorphism of character
theory \cref{intro-classical-thm} in terms of $\rK(\rB G)$. We take
one further step in order to accomplish this.

\begin{notation}
  \label{intro-fixp}
  For the remainder of this subsection and the next, fix a prime $p$.
\end{notation}

\begin{lemma}
  \label{intro-ideals}
  Suppose $G$ is a $p$-group. Then for some $n \in \lZ_{> 0}$, the
  ideal $\Aug(G)^n$ is contained in the ideal $p \cdot \Rep(G)$.
\end{lemma}

\begin{proof}
  The ideal $\Aug(G)$ is additively generated by elements of the form
  $[V] - d$, where $V$ is an irreducible representation of $G$ and
  $d \ce \dim(V)$ is the class of the $\dim(V)$-dimensional trivial
  representation of $G$. Since $G$ has only finitely many irreducible
  representations $V$, it suffices to show that, fixing such a $V$,
  the element $([V] - d)^n$ is in the ideal $p \cdot \Rep(G)$ for some
  $n \in \lZ_{> 0}$.

  Now, Brauer's induction theorem tells us that $[V]-d$ is a linear
  combination of elements of the form $\Ind_H^G([c]-1)$ where
  $H \subseteq G$ is a subgroup, $c$ is a one-dimensional
  representation (i.e. linear character) on $H$, and $\Ind_H^G$
  denotes induction from $H$ to $G$. So it now suffices to show for
  any such $H,c$ that $\Ind_H^G([c]-1)^n \in p \cdot \Rep(G)$ for some
  $n \in \lZ_{> 0}$. We now have two cases:
  \begin{itemize}
  \item Suppose $H$ is a proper subgroup of $G$. The push-pull formula
    tells us that
    \[
    \Ind_H^G([c]-1)^n = \Ind_H^G\l(([c]-1) \cdot
    (\Ind_H^G([c]-1)|_H)^{n-1}\r)
    \]
    where $|_H$ denotes restriction from $G$ to $H$. By induction (the
    other kind now) on the order of our group,
    $(\Ind_H^G([c]-1)|_H)^{n-1} \in p \cdot \Rep(H)$ for some
    $n \in \lZ_{> 0}$, implying
    $\Ind_H^G([c]-1)^n \in p \cdot \Rep(G)$, as desired.
  \item Else $H = G$, in which case $G$ being a $p$-group implies
    $[c]^{p^m} = 1$ for some $m \in \lZ_{> 0}$. But then the binomial
    theorem implies $[c-1]^{p^m} \in p \cdot \Rep(G)$, as
    desired. \qedhere
  \end{itemize}
\end{proof}

\begin{lemma}
  \label{intro-completions}
  If $G$ is a $p$-group, the Atiyah-Segal map \cref{intro-atsegmap},
  $\Rep(G) \to \rK(\rB G)$, is an isomorphism after $p$-completion,
  i.e. after tensoring with the $p$-adic integers $\lZ_p$.
\end{lemma}

\begin{proof}
  This follows immediately from \cref{intro-atseg,intro-ideals}, the
  latter implying that completing $\Rep(G)$ at $\Aug(G)$ is subsumed
  by completing at $p$. That is, completing at $\Aug(G)$ and then
  completing at $p$ is equivalent to just completing at $p$.
\end{proof}

This is nice. If we pick an embedding $\lZ_p \to \lC$ then we can now
rewrite one side of character theory \cref{intro-classical-thm}
completely in terms of ordinary K-theory when $G$ is a $p$-group:
\[
\lC \otimes_\lZ \Rep(G) \iso
\lC \otimes_{\lZ_p} (\lZ_p \otimes_\lZ \Rep(G)) \iso
\lC \otimes_{\lZ_p} \hat\rK_p(\rB G),
\]
where $\hat\rK_p(\rB G) \ce \lZ_p \otimes_\lZ \rK(\rB G)$ is the
$p$-completion of the K-theory of $\rB G$.

The other side of character theory can be rewritten in terms of the
classifying space $\rB G$ as well. If
$\Omega\rB G \ce \Map_*(\rS^1,\rB G)$ denotes the pointed loop space,
then its connected components are given by
\[
\pi_0(\Omega\rB G) \iso \pi_1(\rB G) \iso G.
\]
So the singular cohomology with complex coefficients
$\rH^0(\Omega\rB G;\lC)$ in degree $0$ is just the vector space of
functions $G \to \lC$. Intuitively, changing the basepoint in $\rB G$
corresponds to conjugation in $G$, and indeed it's not hard to show
that if we instead take the unpointed, or \emph{free}, loop space
$\cL \rB G \ce \Map(\rS^1,\rB G)$, then its connected components are
given by $G$ modulo conjugation. And thus
\[
\rH^0(\cL\rB G;\lC) \iso \Cl(G;\lC).
\]

So finally we have completely translated character theory into the
language of homotopy theory: for $G$ a $p$-group we have an
isomorphism
\begin{equation}
  \label{intro-translated}
  \lC \otimes_{\lZ_p} \hat\rK_p(\rB G) \iso \rH^0(\cL\rB G; \lC).
\end{equation}

%%%%%%%%%%%%%%%%%%%%%%%%%%%%%%%%%%%%%%%%%%%%%%%%%%%%%%%%%%%%%%%%%%%%%%

\subsection{Generalizing}
\label{intro-general}

We now seek to explain the generalization of \cref{intro-translated}
that will be proved in this thesis. There are many new notions which
need to be introduced to state this generalization, of which we can
only give a vague summary here. Indeed a significant portion of this
thesis is devoted to introducing and developing these notions, and the
truly interested reader is kindly invited to continue reading after
the nebulous haze of intuition which follows. Nevertheless, we will
try to describe this generalization here, one step at a time, in
efforts to continue our story in a coherent manner:

\begin{itemize}[leftmargin=*]
\item By replacing the free loop space functor with a sort of $p$-adic
  loop space functor $\cL$, we may obtain the isomorphism
  \cref{intro-translated} not only for $p$-groups, but for all finite
  groups $G$.

\item One can $p$-adically complete K-theory as a cohomology theory,
  and thus obtain a full cohomology theory $\hat\rK_p$. One can then
  upgrade \cref{intro-translated} into an isomorphism in all degrees
  by replacing the right-hand side with a periodic version of singular
  cohomology:
  \begin{equation}
    \label{intro-gen-alldeg}
      \lC \otimes_{\lZ_p} \hat\rK_p^*(\rB G) \iso
      \prod_{k \in \lZ} \rH^{2k+*}(\cL\rB G;\lC).
  \end{equation}

\item Given a cohomology theory $E$, there is a process, analogous to
  the situation for abelian groups, of obtaining a rationalized
  cohomology theory $\lQ \otimes E$. In the case of K-theory,
  rationalizing leaves us simply with periodic rational singular
  cohomology (the phrase ``Chern character'' is relevant here). In
  more precise language, in our situation we have a natural
  isomorphism of cohomology theories
  \[
  (\lQ \otimes \hat\rK_p)^*(-) \iso
  \prod_{k \in \lZ} \rH^{2k+*}(-;\lQ_p).
  \]
  Therefore, we may restate \cref{intro-gen-alldeg} as
  \begin{equation}
    \label{intro-gen-ratl}
    \lC \otimes_{\lZ_p} \hat\rK_p^*(\rB G) \iso
    \lC \otimes_{\lQ_p} (\lQ \otimes \hat\rK_p)^*(\cL\rB G).
  \end{equation}

\item It is not in fact necessary to extend coefficients all the way
  to $\lC$. One may recall that even in the statement of classical
  character theory \cref{intro-classical-thm}, it suffices to extend
  only to a field extension of $\lQ$ containing all of the roots of
  unity.

\item The $p$-adic completion of K-theory is the first member of a
  naturally occurring family of cohomology theories, $E(n)$ for
  $n \in \lZ_{> 0}$, known as the Morava E-theories. We will show that
  character theory generalizes to these cohomology theories as
  follows:
  \begin{equation}
    \label{intro-gen-ethy}
    C^* \otimes_{E(n)^*} E(n)^*(\rB G) \iso
    C^* \otimes_{\lQ \otimes E(n)^*}
    (\lQ \otimes E(n))^*(\cL^n\rB G).
  \end{equation}
  Here (and for the remainder) $E^*$ denotes the coefficient ring
  $E^*(\pt)$ of a cohomology theory $E$, $C^*$ denotes some (nonzero!)
  ring extension of $\lQ \otimes E(n)^*$, and $\cL^n$ denotes the
  $n$-fold composition of the ($p$-adic) loop space functor.

\item Again in analogy to the situation for abelian groups, we should
  think of rationalization as some kind of localization process. In
  fact, rationalization is the zeroth member of a naturally occurring
  family of localization processes on cohomology theories, $L_{K(t)}$
  for $t \in \lZ_{\ge 0}$. We will prove more generally for
  $0 \le t < n$ that
  \begin{equation}
    \label{intro-gen-ethy-kthy}
    C^* \otimes_{E(n)^*} E(n)^*(\rB G) \iso
    C^* \otimes_{L_{K(t)}E(n)^*}
    L_{K(t)}E(n)^*(\cL^{n-t}\rB G)
  \end{equation}
  for some (nonzero) extension $C^*$ of $L_{K(t)}E(n)^*$ (note that
  $L_{K(t)}E(n)^*$ is itself an extension of $E(n)^*$, just as
  localization usually works, and just as $\lQ \otimes E(n)^*$ is an
  extension of $E(n)^*$).

\item We may view $L_{K(t)}E(n)^*(\cL^{n-t}-)$ as a cohomology theory
  itself, and thus view \cref{intro-gen-ethy-kthy} as a relationship
  between two cohomology theories evaluated on $\rB G$. We discussed
  in \cref{intro-trans} that we may think of this as the associated
  Borel-equivariant cohomology theories evaluated on the trivial
  $G$-space. Indeed the two expressions
  \[
  C^* \otimes_{E(n)^*} (E(n)^\Bor_G)^*(X), \quad
  C^* \otimes_{L_{K(t)}E(n)^*}
  (L_{K(t)}E(n)^\Bor_G)^*(\cL^{n-t}X)
  \]
  determine $G$-equivariant cohomology theories on finite $G$-spaces
  $X$, and we will generalize \cref{intro-gen-ethy-kthy} by proving
  that these two $G$-equivariant cohomology theories are naturally
  isomorphic.

\item Finally, observe that we are making statements about
  $G$-equivariant cohomology theories uniformly over all finite groups
  $G$. The study of these sorts of objects which in some sense are
  equivariant with respect to all (finite) groups simultaneously is
  referred to as \emph{global equivariant homotopy theory}. We will
  take this perspective very seriously, and critically use that the
  isomorphisms for the various finite groups $G$ are related amongst
  one other. One can summarize this by saying that above
  generalization of character theory is in fact given by an
  isomorphism between certain ``global equivariant cohomology
  theories'', and this is precisely what we will prove.
\end{itemize}

%%%%%%%%%%%%%%%%%%%%%%%%%%%%%%%%%%%%%%%%%%%%%%%%%%%%%%%%%%%%%%%%%%%%%%

\subsection{Overview}
\label{intro-overview}

\usetikzlibrary{positioning}
\begin{center}
  \begin{tikzpicture}[%
    ->,>=stealth',shorten >=1pt,auto,node distance=4cm, thick,
    main node/.style={%
      rectangle,draw,font=\normalfont,minimum width=6em,
      execute at begin node={\begin{varwidth}{6em}\centering},
      execute at end node={\end{varwidth}}}]

    \node[main node] (chrom)
      {\cref{chrom}, Chromatic homotopy theory};
    \node[main node] (global) [right=2cm of chrom]
      {\cref{global}, Global equivariant homotopy theory};
    \node[main node] (pdiv) [below=1cm of chrom]
      {\cref{pdiv}, $p$-divisible groups};
    \node[main node] (abdesc) [right=2cm of pdiv]
      {\cref{abdesc}, Abelian descent};
    \node[main node] (char) [below right=1cm and 0cm of pdiv]
      {\cref{char}, Character theory};

    \path
      (chrom) edge (pdiv)
              edge (abdesc)
      (global) edge (abdesc)
      (pdiv) edge (char)
      (abdesc) edge (char);
  \end{tikzpicture}
\end{center}

%%%%%%%%%%%%%%%%%%%%%%%%%%%%%%%%%%%%%%%%%%%%%%%%%%%%%%%%%%%%%%%%%%%%%%

\subsection{Mathematical and notational conventions}
\label{intro-conventions}

%%%%%%%%%%%%%%%%%%%%%%%%%%%%%%%%%%%%%%%%%%%%%%%%%%%%%%%%%%%%%%%%%%%%%%