\section{Introduction}
\label{intro}

We set out from someplace quite classical: the representation theory
of finite groups, in particular the theory of characters
therein. Let's briefly recall how this story goes. Let $G$ be a finite
group. The goal of character theory is to understand a representation
$\rho \c G \to \Aut(V)$ of $G$ on a complex vector space $V$ via its
associated \emph{character}: the function
\begin{equation}
  \label{intro-classical-character}
  \chi_\rho \ce {\tr} \circ \rho \c G \to \lC,
\end{equation}
where $\tr$ denotes the trace map. The beauty of character theory is
that it accomplishes its goal quite thoroughly, which can be
articulated succinctly as follows.

\begin{definitions}
  \label{intro-classical-character-dfns}
  \begin{enumerate}[leftmargin=*]
  \item Consider the collection of isomorphism classes of complex
    representations of $G$, which forms a commutative semiring under
    the operations of direct sum and tensor product; the Grothendieck
    ring of this semiring, i.e. the commutative ring obtained by
    formally adjoining additive inverses, is denoted $\Rep(G)$ and
    referred to as the \emph{representation ring} of $G$.
  \item A function on $G$ is called a \emph{class function} if it is
    conjugation-invariant. Let $\Cl(G;\lC)$ denote the $\lC$-algebra
    of class functions $G \to \lC$.
  \end{enumerate}
\end{definitions}

\begin{theorem}[Classical character theory]
  \label{intro-classical-thm}
  The assignment $\rho \mapsto \chi_\rho$ defined in
  \cref{intro-classical-character} determines a ring homomorphism
  $\Rep(G) \to \Cl(G;\lC)$, which furthermore induces an isomorphism
  \[
  \lC \otimes_\lZ \Rep(G) \isoto \Cl(G;\lC)
  \]
  of $\lC$-algebras.
\end{theorem}

This theorem is just a restatement of the main facts one usually hears
when learning character theory:
\begin{itemize}
\item characters behave well with respect to direct sums and tensor
  products of representations (because traces do);
\item a representation is determined up to isomorphism by its
  character;
\item the characters of representations span the $\lC$-vector space of
  class functions.
\end{itemize}

The aim of this thesis is to explain a generalization of this
classical character theory to the setting of stable homotopy theory,
due to Hopkins-Kuhn-Ravenel \cite{hkr-char} and Stapleton
\cite{stapleton-tgcm}. This generalization might be seen as a bridge
between representation theory and stable homotopy theory, but is
intimately related to certain ideas in algebraic geometry as well. The
confluence of these many areas of mathematics is precisely why I find
this subject so fascinating, and perhaps my real aim here is just to
give a sense of the many intertwining ideas at play, with
(generalized) character theory as a central, motivating objective.

For now though, in this introduction, we simply seek to explain how
one adapts this classical story in representation theory to a story in
stable homotopy theory.

%%%%%%%%%%%%%%%%%%%%%%%%%%%%%%%%%%%%%%%%%%%%%%%%%%%%%%%%%%%%%%%%%%%%%%

\subsection{Translating}
\label{intro-trans}

We will translate \cref{intro-classical-thm} into the language of
homotopy theory via (topological, complex) K-theory. Recall that
K-theory is a cohomology theory arising from the theory of complex
vector bundles. However, for the purposes of this introduction, why
don't we just focus on the degree $0$ term of this cohomology
theory. For a space $X$ this is denoted $\rK(X) = \rK^0(X)$, and is
the Grothendieck ring associated to the semiring of isomorphism
classes of (complex) vector bundles on $X$ (the operations in this
semiring again coming from direct sum and tensor product). For
example, over the trivial one-point space $X = \pt$, a vector bundle
is simply a vector space, determined up to isomorphism by its
dimension; $\rK(\pt)$ is therefore the Grothendieck ring of
$\lZ_{\ge 0}$, which of course is just $\lZ$.

There are two natural \emph{equivariant} analogues of K-theory. That
is, there are two candidates which might replace K-theory when we
would like to study not just spaces but spaces equipped with a
$G$-action for some finite group $G$. The first is a na\"ive, formal
construction which one can make whenever one wants a $G$-equivariant
version of a cohomology theory, known as the associated \emph{Borel
  equivariant cohomology theory}. Given a $G$-space $X$, we may always
replace it with a homotopy equivalent space $Y$ on which $G$ acts
freely: there is a contractible space $\rE G$ with a free $G$ action,
so we may take $Y \ce \rE G \times X$ with $G$ acting diagonally. When
$G$ acts freely on $Y$, it is natural to expect that the
$G$-equivariant version of a cohomology theory applied to $Y$ recovers
the original cohomology theory's value on the quotient space
$Y/G$. This motivates the Borel construction (here applied to
K-theory), given by defining
\[
\rK^\Bor_G(X) \ce \rK((\rE G \times X)/G).
\]
In particular, when $X$ is the trivial $G$-space $\pt$ we have
$\rK^\Bor_G(\pt) \iso \rK(\rB G)$, where $\rB G \iso \rE G / G$ is the
classifying space of $G$.

But for $\rK$-theory there is another, more geometric candidate for
its equivariant analogue. Namely, there is a notion of a
\emph{$G$-equivariant (complex) vector bundle} over a $G$-space
$X$. So, analagously to K-theory, we may define a $G$-equivariant
cohomology theory which in degree $0$ is given by the Grothendieck
ring $\rK_G(X)$ of $G$-equivariant vector bundles on $X$. Now, just as
a vector bundle over the trivial space is just a vector space, a
$G$-equivariant vector bundle over the trivial $G$-space $\pt$ is
simply a representation of $G$. Thus we by definition have
$\rK_G(\pt) \iso \Rep(G)$. It is with this tautological renaming of
$\Rep(G)$ that we look to begin viewing character theory through the
lens of homotopy theory.

The first non-tautological step we take is to compare these
equivariant theories $\rK^\Bor_G$ and $\rK_G$. The theory $\rK_G$
still has the expected property that $\rK_G(Y) \iso \rK(Y/G)$ when $G$
acts freely on $Y$. Thus, for any $G$-space $X$ we have a natural
comparison map
\[
\rK_G(X) \to
\rK_G(\rE G \times X) \iso
\rK((\rE G \times X)/G) \iso
\rK^\Bor_G(X),
\]
where the first map is induced by the projection
$\rE G \times X \to X$. Of course this map is not an isomorphism, but
a theorem of Atiyah-Segal tells us it's not terribly far from being an
isomorphism. This is easiest to state for $X = \pt$, in which case the
above gives a map\footnote{This map can alternatively be described as
  follows: a representation of $G$ is a map
  $\pi_1(\rB G) \iso G \to \Aut(V)$, which determines a local system
  on $\rB G$ with value $V$, which determines a vector bundle on
  $\rB G$ with fiber $V$.}
\begin{equation}
  \label{intro-atsegmap}
  \Rep(G) \iso \rK_G(\pt) \to \rK^\Bor_G(\pt) \iso \rK(\rB G).
\end{equation}

\begin{definition}
  \label{intro-aug}
  To a representation $V$ of $G$ we can associate its dimension
  $\dim(V)$. This extends to a ring morphism
  $\dim \c \Rep(G) \to \lZ$. We call the kernel of this morphism the
  \emph{augmentation ideal} of $\Rep(G)$, and denote it $\Aug(G)$.
\end{definition}

\begin{theorem}[Atiyah-Segal completion]
  \label{intro-atseg}
  The map \cref{intro-atsegmap} exhibits the ring $\rK(\rB G)$ as the
  completion of the ring $\Rep(G)$ at the augmentation ideal
  $\Aug(G)$.
\end{theorem}

We now have a precise relationship between $\rK(\rB G)$ and
$\rK_G(\pt)$, and an identification of $\rK_G(\pt)$ with
$\Rep(G)$. Thus we might hope to rephrase the isomorphism of character
theory \cref{intro-classical-thm} in terms of $\rK(\rB G)$. We make
one further algebraic modification in order to accomplish this.

\begin{notation}
  \label{intro-fixp}
  For the remainder of this subsection and the next, fix a prime $p$.
\end{notation}

\begin{lemma}
  \label{intro-ideals}
  Suppose $G$ is a $p$-group. Then for some $n \in \lZ_{> 0}$, the
  ideal $\Aug(G)^n$ is contained in the ideal $p \cdot \Rep(G)$.
\end{lemma}

The proof is relegated to a footnote.\footnote{ I quite like this
  proof, but couldn't justify interrupting our story with a little
  fact from representation theory. In any case, it's still here for
  the curious:

  \begin{proof}[Proof of \cref{intro-ideals}]
    Any element of $\Aug(G)$ is a linear combination of elements of
    the form $[V] - d$, where $V$ is an irreducible representation of
    $G$ and $d \ce \dim(V)$ is the class of the $\dim(V)$-dimensional
    trivial representation of $G$. Since $G$ has only finitely many
    irreducible representations $V$, it suffices to fix such a $V$ and
    show that the element $([V] - d)^n$ is in the ideal
    $p \cdot \Rep(G)$ for some $n \in \lZ_{> 0}$.

    Now, Brauer's induction theorem tells us that $[V]-d$ is a linear
    combination of elements of the form $\Ind_H^G([c]-1)$ where
    $H \subseteq G$ is a subgroup, $c$ is a one-dimensional
    representation (i.e. linear character) on $H$, and $\Ind_H^G$
    denotes induction from $H$ to $G$. So it now suffices to show for
    any such $H$ and $c$ that $\Ind_H^G([c]-1)^n \in p \cdot \Rep(G)$
    for some $n \in \lZ_{> 0}$. There are two cases:
    \begin{itemize}
    \item Suppose $H$ is a proper subgroup of $G$. The push-pull
      formula tells us that
      \[
      \Ind_H^G([c]-1)^n = \Ind_H^G\l(([c]-1) \cdot
      (\Ind_H^G([c]-1)|_H)^{n-1}\r),
      \]
      where $|_H$ denotes restriction from $G$ to $H$. By induction
      (the other kind now) on the order of our group,
      $(\Ind_H^G([c]-1)|_H)^{n-1} \in p \cdot \Rep(H)$ for some
      $n \in \lZ_{> 0}$, implying
      $\Ind_H^G([c]-1)^n \in p \cdot \Rep(G)$, as desired.
    \item Else $H = G$, in which case $G$ being a $p$-group implies
      $[c]^{p^m} = 1$ for some $m \in \lZ_{> 0}$. But then the
      binomial theorem implies $[c-1]^{p^m} \in p \cdot \Rep(G)$, as
      desired. \qedhere
    \end{itemize}
  \end{proof}
}

\begin{lemma}
  \label{intro-completions}
  If $G$ is a $p$-group, the Atiyah-Segal map \cref{intro-atsegmap} is
  an isomorphism after $p$-completion. So
  $\lZ_p \otimes_\lZ \Rep(G) \iso \lZ_p \otimes_\lZ \rK(\rB G)$.
\end{lemma}

\begin{proof}
  This follows immediately from \cref{intro-atseg,intro-ideals}, the
  latter implying that completing $\Rep(G)$ at $\Aug(G)$ is subsumed
  by completing at $p$. That is, completing at $\Aug(G)$ and then
  completing at $p$ is equivalent to just completing at $p$.
\end{proof}

This is nice. If we pick an embedding $\lZ_p \to \lC$, we can now
rewrite one side of character theory \cref{intro-classical-thm}
completely in terms of ordinary K-theory when $G$ is a $p$-group:
\[
\lC \otimes_\lZ \Rep(G) \iso
\lC \otimes_{\lZ_p} (\lZ_p \otimes_\lZ \Rep(G)) \iso
\lC \otimes_{\lZ_p} \hat\rK_p(\rB G),
\]
where $\hat\rK_p(\rB G) \ce \lZ_p \otimes_\lZ \rK(\rB G)$ is the
$p$-completion of the K-theory of $\rB G$.

The other side of character theory can be rewritten in terms of the
classifying space $\rB G$ as well. If
$\Omega\rB G \ce \Map_*(\rS^1,\rB G)$ denotes the pointed loop space,
then its connected components are given by
\[
\pi_0(\Omega\rB G) \iso \pi_1(\rB G) \iso G.
\]
(In fact $\Omega\rB G$ is equivalent to the discrete space $G$.) So
the singular cohomology with complex coefficients
$\rH(\Omega\rB G;\lC) \ce \rH^0(\Omega\rB G;\lC)$ in degree $0$ is the
$\lC$-vector space of functions $G \to \lC$. Intuitively, changing the
basepoint in $\rB G$ corresponds to conjugation in $G$, and indeed
it's not hard to show that if we instead take the unpointed, or
\emph{free}, loop space $\cL \rB G \ce \Map(\rS^1,\rB G)$, then its
connected components are given by $G$ modulo conjugation. And thus
\[
\rH(\cL\rB G;\lC) \iso \Cl(G;\lC).
\]

So finally we have completely translated character theory into the
language of homotopy theory: for $G$ a $p$-group we have an
isomorphism
\begin{equation}
  \label{intro-translated}
  \lC \otimes_{\lZ_p} \hat\rK_p(\rB G) \iso \rH(\cL\rB G; \lC).
\end{equation}

%%%%%%%%%%%%%%%%%%%%%%%%%%%%%%%%%%%%%%%%%%%%%%%%%%%%%%%%%%%%%%%%%%%%%%

\subsection{Generalizing}
\label{intro-gen}

We now seek to describe the generalizations of \cref{intro-translated}
that were found by Hopkins-Kuhn-Ravenel and Stapleton, and which will
be proved in this thesis. There are many new notions which need to be
introduced to state these generalizations, of which we can only give a
vague summary here. Indeed, a significant portion of this thesis is
devoted to introducing and developing these notions, and the truly
interested reader is kindly invited to continue reading after the
nebulous haze of intuition which follows. Nevertheless, we must
complete the story we set out to tell in this introduction; hopefully
the following step-by-step path of generalization from
\cref{intro-translated} to the main theorem \cref{intro-main-informal}
serves as a coherent and useful way to do so.

\begin{nothing}
  \label{intro-genpath}
  Recall we are still working at a fixed prime $p$. Also, if $E$ is a
  cohomology theory, we denote its \emph{coefficient ring} $E^*(\pt)$
  by $E^*$.
  \begin{enumerate}[leftmargin=*]
  \item We may modify the free loop space functor $\cL$ (into a sort
    of ``$p$-adic loop space functor'') so that the isomorphism
    \cref{intro-translated} holds not only for $p$-groups but for all
    finite groups $G$.

  \item Instead of just $p$-adically completing the group
    $\rK(\rB G)$, one can $p$-adically complete K-theory as a
    cohomology theory, and thus obtain a full cohomology theory
    $\hat\rK_p$. One can then upgrade \cref{intro-translated} into an
    isomorphism in all degrees by replacing the right-hand side with a
    periodic version of singular cohomology:
    \begin{equation}
      \label{intro-gen-alldeg}
      \lC \otimes_{\lZ_p} \hat\rK_p^*(\rB G) \iso
      \prod_{k \in \lZ} \rH^{2k+*}(\cL\rB G;\lC).
    \end{equation}

  \item Given a cohomology theory $E$, there is a process of obtaining
    a ``rationalized cohomology theory'', denoted $\lQ \otimes E$. As
    the notation suggests, this is analogous to the operation of
    tensoring abelian groups with $\lQ$. The coefficient ring
    $(\lQ \otimes E)^*$ is the tensor product $\lQ \otimes_\lZ E^*$,
    but for a general space $X$ it is not true that
    $(\lQ \otimes E)^*(X) \iso \lQ \otimes_\lZ E^*(X)$. Indeed the
    right-hand side isn't even a cohomology theory, for two (possible)
    reasons:
    \begin{itemize}
    \item Tensoring with $\lQ$ may not preserve exact sequences.
    \item Cohomology theories should take disjoint unions to
      products. For finite disjoint unions the right-hand side has no
      problem, since tensor product commutes with direct sums, but for
      infinite disjoint unions we run into trouble. So the above
      \emph{really} isn't true when we talk about spaces $X$ which are
      not finite CW-complexes.
    \end{itemize}

    In the case of K-theory, rationalizing simply leaves us with
    periodic rational singular cohomology.\footnote{The phrase ``Chern
      character'' is relevant here.} Stated more precisely for
    $p$-adically completed K-theory, there is a natural isomorphism of
    cohomology theories
    \[
    (\lQ \otimes \hat\rK_p)^*(-) \iso \prod_{k \in \lZ}
    \rH^{2k+*}(-;\lQ_p).
    \]
    Therefore, we may restate \cref{intro-gen-alldeg} as
    \begin{equation}
      \label{intro-gen-ratl}
      \lC \otimes_{\lZ_p} \hat\rK_p^*(\rB G) \iso
      \lC \otimes_{\lQ_p} (\lQ \otimes \hat\rK_p)^*(\cL\rB G).
    \end{equation}
    Hence we may think of character theory as a comparison of the
    rationalized values of K-theory and the values of rationalized
    K-theory. We said above that these things don't agree, in
    particular for non-finite spaces like $\rB G$. Character theory
    says that we can make them agree by applying a loop space functor
    $\cL$ to one side and further extending our coefficients from
    $\lQ$ to $\lC$.

  \item It is not in fact necessary to extend coefficients all the way
    to $\lC$. One may recall that even in the statement of classical
    character theory \cref{intro-classical-thm}, it suffices to extend
    only to a field extension of $\lQ$ containing all of the roots of
    unity. Similarly, in \cref{intro-gen-ratl} we may replace $\lC$
    with the maximal ramified extension $\colim_k \lQ_p(\zeta_{p^k})$
    of $\lQ_p$.

  \item The $p$-adic completion of K-theory is the first member of a
    naturally occurring family of cohomology theories
    $\{E(n)\}_{n \ge 1}$, known as the Morava E-theories. We will show
    that character theory generalizes to these cohomology theories as
    follows. For $E \ce E(n)$, there is a nonzero ring extension
    $C_0^*$ of $\lQ \otimes E^* \iso (\lQ \otimes E)^*$ such that
    \begin{equation}
      \label{intro-gen-ethy}
      C_0^* \otimes_{E^*} E^*(\rB G) \iso
      C_0^* \otimes_{\lQ \otimes E^*}
      (\lQ \otimes E)^*(\cL^n\rB G),
    \end{equation}
    where $\cL^n$ denotes the $n$-fold composition of the $p$-adic
    loop space functor, and these are now tensor products of graded
    rings.

  \item Again in analogy to the situation for abelian groups, we
    should think of rationalization as some kind of localization
    process. In fact, rationalization is the zeroth member of a
    naturally occurring family of localization processes on cohomology
    theories $\{L_{K(t)}\}_{t \ge 0}$. As in the localization of
    abelian groups or modules, for a cohomology theory $E$ there are
    localization maps $E \to L_{K(t)}E$.

    We will prove more generally that for $E \ce E(n)$ and
    $L_t \ce L_{K(t)}E$ with $0 \le t < n$, there is a nonzero
    extension $C_t^*$ of $L_t^*$ (which itself an extension of $E^*$
    via the localization map $E \to L_t$) such that
    \begin{equation}
      \label{intro-gen-ethy-kthy}
      C_t^* \otimes_{E^*} E^*(\rB G) \iso
      C_t^* \otimes_{L_t^*}
      L_t^*(\cL^{n-t}\rB G).
    \end{equation}

  \item In fact, \cref{intro-gen-ethy-kthy} is just one instance of a
    natural isomorphism of $G$-equivariant cohomology theories,
    arising from the Borel equivariant cohomology theories associated
    to $E$ and $L_t$. On finite $G$-CW complexes $X$, this isomorphism
    looks like:
    \begin{equation}
      \label{intro-gen-gequiv}
      C_t^* \otimes_{E_t^*} E^*(\rE G \times_G X) \iso
      C_t^* \otimes_{L_t^*}
      L_t^*(\rE G \times_G \cL^{n-t}X),
    \end{equation}
    where $\rE G \times_G -$ denotes the quotient $(\rE G \times -)/G$
    we discussed earlier.

  \item Observe that \cref{intro-gen-gequiv} is a statement about
    $G$-equivariant cohomology theories which doesn't seem biased in
    any way toward the group $G$ we're working with. In fact, we can
    think of the two sides of this isomorphism as cohomology theories
    which are in some sense ``equivariant with respect to all finite
    groups $G$''. Such things are called \emph{global equivariant
      cohomology theories}. Our proof of these statements will
    critically use this ``global'' perspective, and setting up the
    framework to study these global equivariant cohomology theories is
    one of our primary goals in this work.
  \end{enumerate}
\end{nothing}

Now, finally, we may state (informally) the main result we will be
working towards.

\begin{theorem}[Informal]
  \label{intro-main-informal}
  There is an isomorphism \cref{intro-gen-gequiv}, where each side may
  be viewed as a global equivariant cohomology theory over all finite
  groups $G$.
\end{theorem}

\begin{remark}
  \label{intro-credit}
  We should explicitly note here that this thesis is almost entirely
  expository. However, our strong commitment to the global perspective
  in this exposition is in some sense original, and seems quite
  interesting and useful. In any case, thanks to Hopkins-Kuhn-Ravenel
  and Stapleton for producing some incredibly interesting mathematics.
\end{remark}

%%%%%%%%%%%%%%%%%%%%%%%%%%%%%%%%%%%%%%%%%%%%%%%%%%%%%%%%%%%%%%%%%%%%%%

\subsection{Overview}
\label{intro-overview}

The remainder of our work is organized as follows. In \cref{chrom} we
review some of the main ideas from the field of \emph{chromatic
  homotopy theory}, which is where the Morava E-theories arise. In
\cref{pdiv} we discuss how the theory of \emph{$p$-divisible groups}
in algebraic geometry appears in chromatic homotopy theory, which
appearance is central to understanding character theory. In
\cref{global} we set up our framework for studying global equivariant
cohomology theories and the like. In \cref{abdesc} we study a
particular property of global equivariant cohomology theories which is
key to our proof of character theory. Finally in \cref{char} we
actually give a formal statement and proof of our generalization of
character theory. The following graph depicts the dependency of these
sections on one another, in case the reader is interested in just a
portion of this thesis.

\usetikzlibrary{positioning}
\begin{center}
  \begin{tikzpicture}[%
    ->,>=stealth',shorten >=1pt,auto,node distance=4cm, thick,
    main node/.style={%
      rectangle,draw,font=\normalfont,minimum width=6em,
      execute at begin node={\begin{varwidth}{6em}\centering},
      execute at end node={\end{varwidth}}}]

    \node[main node] (chrom)
      {\cref{chrom}, Chromatic homotopy theory};
    \node[main node] (global) [right=2cm of chrom]
      {\cref{global}, Global equivariant homotopy theory};
    \node[main node] (pdiv) [below=1cm of chrom]
      {\cref{pdiv}, $p$-divisible groups};
    \node[main node] (abdesc) [right=2cm of pdiv]
      {\cref{abdesc}, Abelian descent};
    \node[main node] (char) [below right=1cm and 0cm of pdiv]
      {\cref{char}, Character theory};

    \path
      (chrom) edge (pdiv)
              edge (abdesc)
      (global) edge (abdesc)
      (pdiv) edge (char)
      (abdesc) edge (char);
  \end{tikzpicture}
\end{center}

%%%%%%%%%%%%%%%%%%%%%%%%%%%%%%%%%%%%%%%%%%%%%%%%%%%%%%%%%%%%%%%%%%%%%%

\subsection{Conventions and prerequisites}
\label{intro-conventions}

I have tried throughout the text to give introductions to the many
background ideas relevant in this thesis, so that even readers
unfamiliar with some of these ideas might get something out of reading
(parts of) it. However, it was of course necessary to assume some
foundational language, so I should say a word about some of the
higher-level prerequisites here.

I will freely use higher category theory, in particular the theory of
$\infty$-categories (by which I will always mean
$(\infty,1)$-categories). However, the reader is strongly urged not to
worry about this, as long as they are familiar with ordinary category
theory. The theory will be treated completely as a black-box, and used
formally, analagously to ordinary category theory. I will explicitly
state when we are viewing something as a higher category, but when
working with it I will not distinguish the associated
higher-categorical notions notationally or terminologically from
ordinary categorical notions. For example, if we are dealing with an
$\infty$-category, then all limits and colimits refer to the correct
$\infty$-categorical notions, i.e. homotopy limits and colimits, but
will still just be denoted $\lim$ and $\colim$. The same goes for all
other categorical notions: subcategories, functors, adjunctions, Kan
extensions, and so on.

I will freely use the basic language of stable homotopy theory, most
notably the notions of spectra, ring spectra, and $\rE_\infty$-ring
spectra. The reader unfamiliar with these terms may just replace the
first two terms with cohomology theories and multiplicative cohomology
theories, respectively. The last should be thought of as really nice
multiplicative cohomology theories, which have an associated theory
analagous to commutative rings in algebra. Indeed we will often just
call them $\rE_\infty$-rings. There is a good notion of modules of
$\rE_\infty$-rings and tensor products of these modules, and when one
encounters such notions in the text, one should just imagine them as
the correct counterpart in stable homotopy theory of the usual notions
in algebra.

And finally we state a couple of conventions in notation and
terminology that will be employed throughout:
\begin{itemize}
\item All rings and algebras will be commutative or
  graded-commutative, whichever the context makes more natural.
\item The $\infty$-category of spaces (i.e. topological spaces or Kan
  complexes up to weak equivalence) is denoted $\Space$. The
  $\infty$-category of spectra is denoted $\Spect$.
\item The smash product of spectra will be denoted by $\otimes$ (note
  for $\rE_\infty$-rings it really should be thought of as a tensor
  product). The wedge sum of spectra, which is both the categorical
  product and coproduct, is denoted by $\oplus$.
\end{itemize}

%%%%%%%%%%%%%%%%%%%%%%%%%%%%%%%%%%%%%%%%%%%%%%%%%%%%%%%%%%%%%%%%%%%%%%