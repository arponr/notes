\section{Introduction}
\label{intro}

\newcommand{\Rep}{\operatorname{Rep}}
\newcommand{\Cl}{\operatorname{Cl}}
\newcommand{\tr}{\operatorname{tr}}

We set out from someplace quite classical: the representation theory
of finite groups, in particular the theory of characters
therein. Let's briefly recall how this story goes. Let $G$ be a finite
group. The goal of character theory is to understand a representation
$\rho \c G \to \Aut(V)$ of $G$ on a complex vector space $V$ via its
associated \emph{character}: the function
\begin{equation}
  \label{classical-character}
  \chi_\rho \ce {\tr} \circ \rho \c G \to \lC,
\end{equation}
where $\tr$ denotes the trace map. The beauty of character theory is
that it accomplishes this goal quite thoroughly. This fact can be
articulated succinctly as follows.

\begin{definitions}
  \label{classical-character-dfns}
  \begin{enumerate}[leftmargin=*]
  \item Consider the collection of isomorphism classes of complex
    representations of $G$, which forms a commutative semiring under
    the operations of direct sum and tensor product; the Grothendieck
    ring of this semiring, i.e. the commutative ring obtained by
    formally adjoining additive inverses, is denoted $\Rep(G)$ and
    referred to as the \emph{representation ring} of $G$.
  \item A function on $G$ is called a \emph{class function} if it is
    conjugation-invariant. Let $\Cl(G;\lC)$ denote the $\lC$-algebra
    of class functions $G \to \lC$.
  \end{enumerate}
\end{definitions}

\begin{theorem}[Classical character theory]
  \label{classical-character-thm}
  The assignment $\rho \mapsto \chi_\rho$ defined in
  \cref{classical-character} determines a ring homomorphism
  $\Rep(G) \to \Cl(G;\lC)$, which furthermore induces an isomorphism
  \[
  \lC \otimes_\lZ \Rep(G) \isoto \Cl(G;\lC)
  \]
  of $\lC$-algebras.
\end{theorem}

The goal of the this thesis is to explain a generalization of this
classical character theory, which begins to suggest itself when we
view \cref{classical-character-thm} as a statement in homotopy theory.

\usetikzlibrary{positioning}
\begin{center}
  \begin{tikzpicture}[%
    ->,>=stealth',shorten >=1pt,auto,node distance=4cm, thick,
    main node/.style={%
      rectangle,draw,font=\normalfont,minimum width=6em,
      execute at begin node={\begin{varwidth}{6em}\centering},
      execute at end node={\end{varwidth}}}]

    \node[main node] (chrom)
      {\cref{chrom}, Chromatic homotopy theory};
    \node[main node] (global) [right=2cm of chrom]
      {\cref{global}, Global equivariant homotopy theory};
    \node[main node] (pdiv) [below=1cm of chrom]
      {\cref{pdiv}, $p$-divisible groups};
    \node[main node] (abdesc) [right=2cm of pdiv]
      {\cref{abdesc}, Abelian descent};
    \node[main node] (char) [below right=1cm and 0cm of pdiv]
      {\cref{char}, Character theory};

    \path
      (chrom) edge (pdiv)
              edge (abdesc)
      (global) edge (abdesc)
      (pdiv) edge (char)
      (abdesc) edge (char);
  \end{tikzpicture}
\end{center}
