\documentclass[10pt]{amsart}

\usepackage[tmargin=1.3in,bmargin=1.3in,lmargin=1.7in,rmargin=1.7in]
           {geometry}
\usepackage{fancyhdr}
\usepackage[hyphens]{url}

\usepackage{amssymb}
\usepackage{eucal}
\usepackage{mathtools}
\usepackage{tikz-cd}

\frenchspacing

\pagestyle{fancy}
\renewcommand{\headrulewidth}{0pt}
\renewcommand{\footrulewidth}{0pt}
\fancyhead{}
\fancyfoot[C]{\ \\\small \thepage}

%%%%%%%%%%%%%%%%%%%%%%%%%%%%%%%%%%%%%%%%%%%%%%%%%%%%%%%%%%%%%%%%%%%%%%

\newcommand{\A}{A}
\newcommand{\B}{B}
\newcommand{\C}{\mathbb{C}}
\newcommand{\D}{D}
\newcommand{\E}{E}
\newcommand{\F}{F}
\newcommand{\G}{G}
\renewcommand{\H}{H}
\newcommand{\I}{I}
\newcommand{\J}{J}
\newcommand{\K}{K}
\renewcommand{\L}{L}
\newcommand{\M}{M}
\newcommand{\N}{\mathbb{N}}
\renewcommand{\O}{O}
\renewcommand{\P}{\mathbb{P}}
\newcommand{\Q}{\mathbb{Q}}
\newcommand{\R}{\mathbb{R}}
\renewcommand{\S}{S}
\newcommand{\T}{T}
\newcommand{\U}{U}
\newcommand{\V}{V}
\newcommand{\W}{W}
\newcommand{\X}{X}
\newcommand{\Y}{Y}
\newcommand{\Z}{\mathbb{Z}}

\newcommand{\inj}{\hookrightarrow}
\newcommand{\surj}{\twoheadrightarrow}
\newcommand{\oline}{\overline}
\newcommand{\uline}{\underline}
\newcommand{\del}{\partial}

%%%%%%%%%%%%%%%%%%%%%%%%%%%%%%%%%%%%%%%%%%%%%%%%%%%%%%%%%%%%%%%%%%%%%%

\swapnumbers

\theoremstyle{plain}
\newtheorem{theorem}[subsection]{Theorem}
\newtheorem{proposition}[subsection]{Proposition}
\newtheorem{lemma}[subsection]{Lemma}
\newtheorem{corollary}[subsection]{Corollary}
\newtheorem{conjecture}[subsection]{Conjecture}
\newtheorem{exercise}[subsection]{Exercise}

\theoremstyle{definition}
\newtheorem{definition}[subsection]{Definition}
\newtheorem{definitions}[subsection]{Definitions}
\newtheorem{notation}[subsection]{Notation}
\newtheorem{convention}[subsection]{Convention}
\newtheorem{situation}[subsection]{Situation}
\newtheorem{remark}[subsection]{Remark}
\newtheorem{remarks}[subsection]{Remarks}
\newtheorem{example}[subsection]{Example}
\newtheorem{examples}[subsection]{Examples}

\makeatletter
\renewenvironment{proof}[1][\proofname]{
  \par\pushQED{\qed}
  \normalfont
  \topsep6\p@\@plus6\p@\relax
  \trivlist\item[\hskip\labelsep\bfseries#1\@addpunct{.}]\ignorespaces
}{\popQED\endtrivlist\@endpefalse}
\makeatother

\renewcommand{\labelitemi}{---}
\renewcommand{\labelitemii}{--}
