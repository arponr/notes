\documentclass[10pt,reqno]{amsart}

\usepackage[tmargin=1.3in,bmargin=1.3in,
            lmargin=1.9in,rmargin=1.9in]{geometry}
\usepackage{marginnote}
\usepackage{fancyhdr}
\usepackage{microtype}
\usepackage[hyphens]{url}
\usepackage{enumitem}
\usepackage{array}
\usepackage{appendix}
\usepackage{hyperref}

\usepackage{amssymb}
\usepackage{stmaryrd}
\usepackage{tikz}
\usepackage{tikz-cd}
\usepackage{eucal}

\frenchspacing

\renewcommand*{\marginfont}{\color{gray}\small}

\pagestyle{fancy}
\renewcommand{\headrulewidth}{0pt}
\renewcommand{\footrulewidth}{0pt}
\fancyhf{}
\fancyfoot[C]{\vspace{10pt}\small\thepage}

\renewcommand{\title}[1]{\newcommand{\thetitle}{#1}}
\renewcommand{\author}[1]{\newcommand{\theauthor}{#1}}
\renewcommand{\date}[1]{\newcommand{\thedate}{#1}}

\renewcommand{\maketitle}{%
  \begin{center}
    \textbf{\MakeUppercase{\thetitle}}\\[5pt]
    \MakeUppercase{\footnotesize \theauthor}\\[10pt]
    { \small
      \setlength{\tabcolsep}{3pt}
      \begin{tabular}{rl}
        original: & \thedate \\
        updated: & \today
      \end{tabular}
    }
  \end{center}
  \thispagestyle{empty}

  \vspace{20pt}
}

%%%%%%%%%%%%%%%%%%%%%%%%%%%%%%%%%%%%%%%%%%%%%%%%%%%%%%%%%%%%%%%%%%%%%%

\newcommand{\A}{\mathrm{A}}
\newcommand{\B}{\mathrm{B}}
\newcommand{\C}{\mathbb{C}}
\newcommand{\D}{\mathrm{D}}
\newcommand{\E}{\mathrm{E}}
\newcommand{\F}{\mathbb{F}}
\newcommand{\G}{\mathrm{G}}
\renewcommand{\H}{\mathrm{H}}
\newcommand{\I}{\mathrm{I}}
\newcommand{\J}{\mathrm{J}}
\newcommand{\K}{\mathrm{K}}
\renewcommand{\L}{\mathrm{L}}
\newcommand{\M}{\mathrm{M}}
\newcommand{\N}{\mathbb{N}}
\renewcommand{\O}{\mathrm{O}}
\renewcommand{\P}{\mathbb{P}}
\newcommand{\Q}{\mathbb{Q}}
\newcommand{\R}{\mathbb{R}}
\renewcommand{\S}{\mathrm{S}}
\newcommand{\T}{\mathrm{T}}
\newcommand{\U}{\mathrm{U}}
\newcommand{\V}{\mathrm{V}}
\newcommand{\W}{\mathrm{W}}
\newcommand{\X}{\mathrm{X}}
\newcommand{\Y}{\mathrm{Y}}
\newcommand{\Z}{\mathbb{Z}}

\usetikzlibrary{arrows}
\tikzcdset{
  arrow style=tikz,
  diagrams={>={stealth}},
}

\renewcommand{\to}{\mathrel{\tikz[baseline]%
    \draw[>=stealth,->](0,0.5ex)--(0.45,0.5ex);}}
\newcommand{\lblto}[1]{\mathrel{\tikz[baseline]%
    \draw[>=stealth,->,font=\scriptsize,inner sep=1.5pt]%
    (0,0.5ex)--node[above]{$\!#1$}(0.45,0.5ex);}}
\renewcommand{\mapsto}{\mathrel{\tikz[baseline]%
    \draw[>=stealth,|->](0,0.5ex)--(0.45,0.5ex);}}
\newcommand{\inj}{\mathrel{\tikz[baseline]%
    \draw[>=stealth,right hook->](0,0.5ex)--(0.45,0.5ex);}}
\newcommand{\surj}{\mathrel{\tikz[baseline]%
    \draw[>=stealth,->>](0,0.5ex)--(0.45,0.5ex);}}
\newcommand{\isoto}{\mathrel{\tikz[baseline]%
    \draw[>=stealth,->,font=\scriptsize,inner sep=1pt]%
    (0,0.5ex)--node[above]{$\!\sim$}(0.45,0.5ex);}}
\newcommand{\doubto}{%
  \hspace{3pt}%
  \begin{tikzpicture}[baseline]%
    \draw[>=stealth,->](0,0.15ex)--(0.45,0.15ex);%
    \draw[>=stealth,->](0,0.85ex)--(0.45,0.85ex);%
  \end{tikzpicture}%
  \hspace{3pt}}

\renewcommand{\l}{\left}
\renewcommand{\r}{\right}
\newcommand{\f}{\frac}
\renewcommand{\o}{\overline}
\renewcommand{\u}{\underline}
\newcommand{\til}{\widetilde}
\newcommand{\del}{\partial}
\newcommand{\dash}{\text{-}}
\renewcommand{\c}{\colon}
\newcommand{\ce}{\mathrel{:=}}
\newcommand{\ec}{\mathrel{=:}}
\newcommand{\iso}{\simeq}

%%%%%%%%%%%%%%%%%%%%%%%%%%%%%%%%%%%%%%%%%%%%%%%%%%%%%%%%%%%%%%%%%%%%%%

\makeatletter
\def\swappedhead#1#2#3{%
  \thmnumber{\@upn{\the\thm@headfont#2\@ifnotempty{#1}{.\hspace{3pt}}}}%
  \thmname{#1}%
  \thmnote{ {\the\thm@notefont(#3)}}}
\makeatother

\swapnumbers

\theoremstyle{plain}
\newtheorem{theorem}[equation]{Theorem}
\newtheorem{proposition}[equation]{Proposition}
\newtheorem{lemma}[equation]{Lemma}
\newtheorem{corollary}[equation]{Corollary}
\newtheorem{conjecture}[equation]{Conjecture}
\newtheorem{question}[equation]{Question}
\newtheorem{exercise}[equation]{Exercise}

\theoremstyle{definition}
\newtheorem{nothing}[equation]{}
\newtheorem{definition}[equation]{Definition}
\newtheorem{definitions}[equation]{Definitions}
\newtheorem{notation}[equation]{Notation}
\newtheorem*{notation*}{Notation}
\newtheorem{convention}[equation]{Convention}
\newtheorem*{convention*}{Convention}
\newtheorem{situation}[equation]{Situation}
\newtheorem{remark}[equation]{Remark}
\newtheorem{remarks}[equation]{Remarks}
\newtheorem{example}[equation]{Example}
\newtheorem{examples}[equation]{Examples}
\newtheorem{algorithm}[equation]{Algorithm}
\newtheorem{construction}[equation]{Construction}

\numberwithin{equation}{section}

\setlist{leftmargin=20pt, itemsep=2pt, topsep=2pt}
\setlist[enumerate, 1]{label=\textup{\arabic*}., ref=\theequation.\arabic*}
\setlist[enumerate, 2]{label=\textit{\alph*}., ref=\theequation.\alph*}
\renewcommand{\labelitemi}{$\vcenter{\hbox{\footnotesize$\bullet$}}$}
\renewcommand{\labelitemii}{--}

\newcommand{\pref}[1]{(\ref{#1})}
