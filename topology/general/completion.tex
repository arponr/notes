%%%%%%%%%%%%%%%%%%%%%%%%%%%%%%%%%%%%%%%%%%%%%%%%%%%%%%%%%%%%%%%%%%%%%%

\newcommand{\ob}{\oper{ob}}
\renewcommand{\hom}{\oper{hom}}
\newcommand{\id}{\oper{id}}
\newcommand{\im}{\oper{im}}
\newcommand{\op}{\oper{op}}

\newcommand{\Top}{\oper{Top}}
\newcommand{\Set}{\oper{Set}}
\newcommand{\Ab}{\oper{Ab}}
\newcommand{\Grp}{\oper{Grp}}
\newcommand{\Mod}{\oper{Mod}}
\newcommand{\Simplex}{\Delta}
\newcommand{\s}{\oper{s}}
\newcommand{\Ch}{\oper{Ch}}

\newcommand{\Sing}{\oper{Sing}}
\renewcommand{\H}{\mathrm{H}}

%%%%%%%%%%%%%%%%%%%%%%%%%%%%%%%%%%%%%%%%%%%%%%%%%%%%%%%%%%%%%%%%%%%%%%

%%%%%%%%%%%%%%%%%%%%%%%%%%%%%%%%%%%%%%%%%%%%%%%%%%%%%%%%%%%%%%%%%%%%%%

\newcommand{\ob}{\oper{ob}}
\renewcommand{\hom}{\oper{hom}}
\newcommand{\id}{\oper{id}}
\newcommand{\im}{\oper{im}}
\newcommand{\op}{\oper{op}}

\newcommand{\Top}{\oper{Top}}
\newcommand{\Set}{\oper{Set}}
\newcommand{\Ab}{\oper{Ab}}
\newcommand{\Grp}{\oper{Grp}}
\newcommand{\Mod}{\oper{Mod}}
\newcommand{\Simplex}{\Delta}
\newcommand{\s}{\oper{s}}
\newcommand{\Ch}{\oper{Ch}}

\newcommand{\Sing}{\oper{Sing}}
\renewcommand{\H}{\mathrm{H}}

%%%%%%%%%%%%%%%%%%%%%%%%%%%%%%%%%%%%%%%%%%%%%%%%%%%%%%%%%%%%%%%%%%%%%%


\title{Math 131 Section, IV:\\Completion of metric spaces}
\author{Arpon Raksit}
\date{\today}

\begin{document}
\maketitle
\thispagestyle{fancy}

%%%%%%%%%%%%%%%%%%%%%%%%%%%%%%%%%%%%%%%%%%%%%%%%%%%%%%%%%%%%%%%%%%%%%%

\begin{notation}
  Throughout we let $(X,d_X)$ be a metric space.
\end{notation}

\section{Preliminaries}

Let's first remember what convergent and Cauchy sequences and
completeness are in metric spaces.

\begin{definition}
  A sequence $(x_k) \in X^\N$ is called \textit{convergent} if there
  exists $x \in X$ such that for each $\epsilon > 0$ there exists some
  $n \in \N$ such that $d_X(x,x_k) < \epsilon$ for all $k \ge n$. In
  this case, we say $(x_k)$ \textit{converges to} $x$ and write
  $\lim_{k \to \infty} x_k = x$, or equivalently $x_k \to x$ as $k \to
  \infty$.
\end{definition}

\begin{definition}
  A sequence $(x_k) \in X^\N$ is called \textit{Cauchy} if for each
  $\epsilon > 0$ there exists some $n \in \N$ such that $d_X(x_j,x_k)
  < \epsilon$ for all $j,k \ge n$.
\end{definition}

You might have seen Cauchy sequences defined in the past for the case
$X = \R$. And you might have learned that a sequence being Cauchy is
\textit{equivalent} to the sequence being convergent. But this
equivalence is something special to $\R$---it says that $\R$ is a
complete metric space, which we will define in a second. But first
check that we always have one implication.

\begin{exercise}
  \label{convergentcauchy}
  All convergent sequences are Cauchy.
\end{exercise}

\begin{definition}
  We say $X$ is \textit{complete} if all Cauchy sequences in $X$ are
  convergent in $X$.
\end{definition}

\begin{examples}
  As stated above $\R$ is complete. This is sort of a key fact when
  we're learning real analysis. So maybe it's counterintuitive that
  there are spaces which are \textit{not} complete. But we don't have
  to stray too far for some examples!
  \begin{itemize}
  \item $(0,1)$ is not complete: it's easy to check that the sequence
    given by $x_k \coloneqq 1/k$ for $k \in \N$ is Cauchy, but for any
    $x \in (0,1)$ there exists $n \in \N$ such that $1/k < x$ for all
    $k \ge n$, whence $(x_k)$ can't converge to $x$.
  \item $\Q$ is not complete: I won't say this is as rigourously as
    the previous example, but rational approximations to irrational
    numbers (e.g., $1.4, 1.41, 1.414, \ldots$ tending to $\sqrt 2$)
    are Cauchy but cannot be convergent in $\Q$.
  \end{itemize}
  For another example of a complete metric space, look at the fourth
  problem set: we can take $X$ the space of bounded sequences in $\R$,
  equipped with the $\sup$ metric.
\end{examples}

\begin{remark}
  Perhaps you've noticed something a little subtle about the
  difference between convergent and Cauchy sequences. Namely, the
  notion of convergence depends on the space $X$ in a way that notion
  of Cauchyness does not. What I mean is: we said that $(1/k)_{k \in
    \N}$ doesn't converge in $(0,1)$ even though it's Cauchy---but
  certainly it converges in $[0, 1]$ or $\R$! On the other hand,
  passing to a larger ambient space doesn't change anything about what
  it means to be a Cauchy sequence.

  This is essentially because our definition of convergence requires
  us to actually produce a limit $x \in X$ (so of course it depends on
  $X$ in the way described above), whereas our definition of Cauchy is
  completely intrinsic, in that it only refers to the elements of our
  given sequence.
\end{remark}

Finally let's recall a couple of properties that maps between metric
spaces can have.

\begin{definition}
  Let $(Y, d_Y)$ another metric space. A map $\phi : X \to Y$ is:
  \begin{enumerate}
  \item \textit{uniformly continuous} if for each $\epsilon > 0$ there
    exists a $\delta > 0$ such that
    \[
    d_X(x,y) < \delta \implies d_Y(\phi(x),\phi(y)) < \epsilon
    \]
    for all $x,y \in X$;
  \item an \textit{isometry} (or ``distance-preserving'') if
    \[
    d_X(x,y) = d_Y(\phi(x),\phi(y))
    \]
    for all $x,y \in X$;
  \item an \textit{isomorphism (of metric spaces)} if $\phi$ is a
    bijective isometry. (Note the inverse of an isomorphism is clearly
    automatically an isometry.)
  \end{enumerate}
\end{definition}

\begin{exercise}
  \label{maps}
  Let $\phi : X \to Y$ a map of metric spaces.
  \begin{enumerate}
  \item If $\phi$ is uniformly continuous and $(x_k) \in X^\N$ is a
    Cauchy sequence, then $(f(x_k)) \in Y^\N$ is also Cauchy.
  \item If $\phi$ is an isometry then $\phi$ is uniformly continuous
    and an embedding of topological spaces.
  \end{enumerate}
\end{exercise}

%%%%%%%%%%%%%%%%%%%%%%%%%%%%%%%%%%%%%%%%%%%%%%%%%%%%%%%%%%%%%%%%%%%%%%

\section{Metric completion}

The goal for the rest of these notes is to prove the following.

\begin{proposition}
  \label{completion}
  There exist a metric space $X_c$ and an isometry $\phi : X \to X_c$
  such that:
  \begin{enumerate}
  \item $X_c$ is complete;
  \item $\phi(X)$ is dense in $X_c$;
  \item for any complete metric space $Y$ and uniformly continuous map
    $\psi : X \to Y$, there exists a unique uniformly continuous map
    $\psi_c : X_c \to Y$ such that $\psi_c \circ \phi = \psi$;
  \item if $Y_c$ is a complete metric space and $\psi : X \to Y_c$ an
    isometry such that $\psi(X)$ is dense in $Y_c$, then there exists
    a unique isomorphism $\eta : X_c \to Y_c$ such that $\eta \circ
    \phi = \psi$.
  \end{enumerate}
\end{proposition}

\begin{definition}
  Since $X_c$ is unique up to unique isomorphism, we are justified in
  calling $X_c$ \textit{the completion} of $X$.
\end{definition}

There are a number of steps in proving the proposition, so I won't
even try to put them all inside one proof environment! (Actually,
as---well, if---you read along, you should try to prove the claims
before reading on. This is a decent exercise in knowing how to prove
fundamental things about metric spaces.) Here we go.

%%%%%%%%%%%%%%%%%%%%%%%%%%%%%%%%%%%%%%%%%%%%%%%%%%%%%%%%%%%%%%%%%%%%%%

\subsection{Construction}

First define $C \coloneqq \{(x_k) \in X^\N \mid (x_k)\text{ is
  Cauchy}\}$ the set of all Cauchy sequences in $X$. We then define an
equivalence relation on $C$ by saying
\[
(x_k) \sim (y_k) \iff \lim_{k \to \infty} d_X(x_k,y_k) = 0.
\]
Reflexifity and symmetry of this relation are obvious. For
transitivity we just use nonnegativity of the metric and the triangle
inequality: if $(x_k) \sim (y_k)$ and $(y_k) \sim (z_k)$ then
\begin{align*}
0 \le \lim_{k \to \infty} d_X(x_k,z_k) &\le \lim_{k \to \infty}
d_X(x_k,y_k) + d_X(y_k,z_k) \\ &\le \lim_{k \to \infty} d_X(x_k,y_k) +
\lim_{k \to \infty} d_X(y_k,z_k) = 0,
\end{align*}
which implies $(x_k) \sim (z_k)$.

We then define $X_c \coloneqq C/{\sim}$ to be the set of equivalence
classes of $C$ under this relation, and define a metric $d_{X_c}$ on
$X_c$ as follows. For $\sigma, \tau \in X_c$, with equivalence class
representatives $(x_k), (y_k) \in C$, respectively, we set
\[
d_{X_c}(\sigma, \tau) \coloneqq \lim_{k \to \infty} d_X(x_k, y_k).
\]
Before we define $\phi$ or prove properties (1) and (2) from the
proposition, we should $d_{X_c}$ is well-defined and satifies the
axioms of a metric.

%%%%%%%%%%%%%%%%%%%%%%%%%%%%%%%%%%%%%%%%%%%%%%%%%%%%%%%%%%%%%%%%%%%%%%

\subsection{The metric is well-defined}

Let $\sigma, \tau \in X_c$ with representatives $(x_n),(y_n) \in C$,
repspectively. Since $(x_n),(y_n)$ are Cauchy, there exists $n \in \N$
such that
\[
d_X(x_j,x_k) < \epsilon \quad\text{and}\quad d_E(y_j,y_k) < \epsilon
\quad\text{if } j,k \ge n.
\]
By the triangle inequality we have
\[
  d_X(x_j,y_j) \le d_X(x_j,x_k) + d_X(x_k,y_k) + d_X(y_j,y_k) \implies
  d_X(x_j,y_j) - d_X(x_k,y_k) < 2\epsilon,
\]
and similarly $d_X(x_k,y_k) - d_X(x_j,y_j) < 2\epsilon$, for $j,k \ge
n$.  It follows that the sequence $(d_E(x_n,y_n))$ in $\R$ is
Cauchy. Since $\R$ is complete, $d_{X_c}(\sigma, \tau) = \lim_{n \to
  \infty} d_E(x_n,y_n)$ exists.

Now suppose $(x_k) \sim (x'_k)$ and $(y_k) \sim (y'_k)$. Then by the
triangle inequality
\begin{align*}
\lim_{k \to \infty} d_X(x'_k, y'_k) &\le \lim_{k \to \infty} d_X(x'_k,
x_k) + \lim_{k \to \infty} d_X(x_k, y_k) + \lim_{k \to \infty}
d_X(y_k, y'_k) \\ &= \lim_{k \to \infty} d_X(x_k, y_k).
\end{align*}
By symmetry the reverse inequality holds as well, so $d_{X_c}$ is
well-defined.

%%%%%%%%%%%%%%%%%%%%%%%%%%%%%%%%%%%%%%%%%%%%%%%%%%%%%%%%%%%%%%%%%%%%%%

\subsection{The metric is a metric}

Next we show $d_{X_c}$ is in fact a metric on $X_c$. Let $\sigma,\tau
\in X_c$ with representatives $(x_n),(y_n) \in C$,
repspectively. Nonnegativity symmetry, and the triangle inequality for
$d_{X_c}$ follow immediately from these holding for $d_X$. And that
$d_{X_c}(\sigma, \tau) = 0$ if and only if $\sigma = \tau$ follows by
definition of $d_{X_c}$ and the relation $\sim$.

%%%%%%%%%%%%%%%%%%%%%%%%%%%%%%%%%%%%%%%%%%%%%%%%%%%%%%%%%%%%%%%%%%%%%%

\subsection{The isometric embedding}

Define $\phi : X \to X_c$ by letting $\phi(x)$ be the equivalence
class of the constant sequence given by $x_k \coloneqq x$ for $k \in
\N$. Then for $x,y \in X$ we have
\[
d_X(\phi(x),\phi(y)) = \lim_{k \to \infty} d_X(x,y) = d_X(x,y),
\]
so $\phi$ is an isometry.

Next we show that $\phi(X)$ is dense in $X_c$. Let $\sigma \in X_c$
with representative $(x_n) \in C$. Let $\epsilon > 0$. Since $(x_n)$
is Cauchy there exists $n \in \N$ such that $d_X(x_k,x_n) < \epsilon$
for all $k \ge N$. It follows that
\[
d_{X_c}(\sigma, \phi(x_n)) = \lim_{k \to \infty} d_E(x_k, x_n) <
\epsilon.
\]
I.e., $B_{d_{X_c}}(\sigma, \epsilon)$ intersects $\phi(X)$ for any
$\sigma \in X_c$ and $\epsilon > 0$. So indeed $\phi(X)$ is dense in
$X_c$.

%%%%%%%%%%%%%%%%%%%%%%%%%%%%%%%%%%%%%%%%%%%%%%%%%%%%%%%%%%%%%%%%%%%%%%

\subsection{The completion is complete}

Finally we show that $X_c$ is complete. Let $(\sigma_k) \in X_c^{\N}$
a Cauchy sequence in $X_c$. Then for each $r \in \N$ there exists $n_r
\in \N$ such that $d_{X_c} (\sigma_k, \sigma_{n_r}) < 1/r$ for $k \ge
n_r$. Since $\phi(X)$ is dense in $X_c$, there exists $x_r \in E$ such
that $d_{X_c}(\sigma_{n_r}, \phi(x_r)) < 1/r$ for each $r \in
\N$.\footnote{Observe (for intuition) that we start with a sequence of
  sequences, which can visualise as a grid, and are producing a new
  sequence by taking some type of rapidly converging diagonal from
  this grid.} We claim $(x_r) \in X^\N$ is Cauchy. Let $\epsilon > 0$,
and choose $n \in \N$ such that $3/n < \epsilon$. Then for $k \ge j
\ge n$ we have by the triangle ineqality and the fact that $\phi$ is
an isometry that
\begin{align*}
  d_X(x_j,x_k) &= d_{X_c}(\phi(x_j),\phi(x_k)) \\ &\le
  d_{X_c}(\phi(x_j), \sigma_{n_j}) + d_{X_c}(\sigma_{n_j},
  \sigma_{n_k}) + d_{X_c}(\sigma_{n_k}, \phi(x_k)) \\ &\le 1/j + 1/j +
  1/k \le 3/n < \epsilon,
\end{align*}
so indeed $(x_r)$ is Cauchy.

Then let $\sigma \in X_c$ be the equivalence class of $(x_r)$. We will
show $\lim_{k \to \infty} \sigma_k = \sigma$, and since $(\sigma_k)$
was arbitrary this will show that $X_c$ is indeed complete. So let
$\epsilon > 0$. Choose $r \in \N$ such that $4/r < \epsilon$. We have
by the triangle inequality that
\[
d_{X_c}(\sigma_{n_r}, \sigma) \le d_{X_c}(\sigma_{n_r}, \phi(x_r)) +
d_{X_c}(\phi(x_r), \sigma).
\]
We know $d_{X_c}(\sigma_{n_r}, \phi(x_r)) < 1/r$, and we showed above
that $d_X(x_k,x_r) < 3/r$ for $k \ge r$, which implies
\[
d_{X_c}(\phi(x_r), \sigma) = \lim_{k \to \infty} d_X(x_r,x_k) < 3/r.
\]
It follows that $d_{X_c}(\sigma_{n_r}, \sigma) < 4/r < \epsilon$. Then
since $(\sigma_{n_r})$ is Cauchy, it is easy to see we must then have
$\lim_{k \to \infty} \sigma_k = \sigma$. So we have proven (1) and (2)
of Proposition \ref{completion}.

%%%%%%%%%%%%%%%%%%%%%%%%%%%%%%%%%%%%%%%%%%%%%%%%%%%%%%%%%%%%%%%%%%%%%%

\subsection{The universal property}

We now prove (3) and (4). These will follow from the following more
general facts.\footnote{The presentation here is essentially taken
  from Pete Clark's notes,
  \url{math.uga.edu/~pete/8410Chapter2v2.pdf}.}

\begin{lemma}
  Let $A,B$ be topological spaces, and $A_0 \subseteq A$ a dense
  subspace. Let $f : A_0 \to B$ a continuous map.
  \begin{enumerate}
  \item If $B$ is Hausdorff, there exists at most one extension of $f$
    to $A$, that is, there exists at most one continuous map $g : A
    \to B$ such that $g|_{A_0} = f$.
  \item If $A$ and $B$ are metric spaces, $B$ is complete, and $f$ is
    uniformly continuous, then there exists a unique uniformly
    continuous extension $g : A \to B$ of $f$. If $f$ is an isometry,
    then so is $g$.
  \end{enumerate}
\end{lemma}

\begin{proof}
  (1) Let $g_1,g_2 : A \to B$ two extensions of $f$. Suppose $g_1(a)
  \ne g_2(a)$ for some $a \in A$. Then we can choose $V_1$ and $V_2$
  disjoint neighbourhoods of $g_1(a)$ and $g_2(a)$, respectively, by
  the Hausdorff hypothesis. Let $U_i \coloneqq g_i^{-1}(V_i)$, open by
  continuity of $g_i$, for $i \in \{1,2\}$. Then $a \in U_1 \in U_2$,
  so $U_1 \cap U_2$ is a nonempty open set in $X$, and hence there
  exists $a_0 \in U_1 \cap U_2 \cap A_0$ by density of $A_0$. Now
  observe that since $g_1,g_2$ extend $f$ we must have
  \[
  V_1 \ni g_1(a_0) = f(a_0) = g_2(a_0) \in V_2,
  \]
  contradicting the disjointness of $V_1$ and $V_2$.

  (2) Uniqueness is immediate from (1), so it suffices to show
  existence. Define $g : A \to B$ as follows. Let $a \in A$. By
  density of $A_0$ there is a sequence $(a_k) \in A_0^\N$ such that
  $a_k \to a$ as $k \to \infty$. By Exercise \ref{convergentcauchy}
  $(a_k)$ is Cauchy, whence $(f(a_k)) \in B^\N$ is Cauchy by Exercise
  \ref{maps}. Then $f(a_k) \to b$ as $k \to \infty$ for some $b \in B$
  by completeness. We define $g(a) \coloneqq b$. We claim this is
  independent of the choice of sequence $(a_k)$. Indeed let $(a_k')
  \in A_0^\N$ another sequence converging to $a$. Then for any $\delta
  > 0$ there exists $n \in \N$ such that, by the triangle inequality,
  \[
  d(a,a_k) < \delta/2\ \ \text{and}\ \ d(a,a_k') < \delta/2 \implies
  d(a_k,a_k') < \delta
  \]
  for $k \ge n$. Therefore by uniform continuity there exists $n \in
  \N$ such that
  \[
  d(f(a_k),f(a_k')) < \epsilon \quad\text{for all }k \ge n.
  \]
  It follows that $\lim_{k \to \infty} f(a_k) = \lim_{k \to \infty}
  f(a_k')$, proving our claim. In particular, if $a \in A_0$ then we
  can choose the constant sequence $a_k \coloneqq a_0$ for $k \in \N$,
  from which we see $g(a) = \lim_{k \to \infty} f(a_0) = f(a_0)$, so
  $g$ in fact extends $f$.

  We now show $g$ is uniformly continuous. Let $\epsilon > 0$. Since
  $f$ is uniformly continuous there exists $\delta > 0$ such that
  $d_B(f(a),f(a')) < \epsilon/2$ whenever $a_0,a_0' \in A_0$ are such
  that $d_A(a_0,a_0') < \delta$. Suppose $a,a' \in A$ are such that
  $d(a,a') < \delta/3$. Choose sequences $(a_k)$ and $(a_k')$ in $A_0$
  converging to $a$ and $a'$, respectively. Let $n \in \N$ such that,
  by the triangle inequality,
  \begin{align*}
  d_X(a_k,a) < \delta/3\ \ \text{and}\ \ d_X(a_k',a') < \delta/3
  &\implies d_X(a_k,a_k') < \delta \\ &\implies d_Y(f(a_k),f(a_k')) <
  \epsilon/2
  \end{align*}
  for all $k \ge n$. Then by continuity\footnote{One could also argue
    more directly, but see, e.g., my solutions to problem set 4 for a
    proof of the continuity of the metric.} of $d_Y$ and definition of
  $g$ we have
  \begin{equation}
    \tag{$\dagger$} d_Y(g(a),g(a')) = \lim_{k \to \infty}
    d_Y(f(a_k),f(a_k')) \le \epsilon/2 < \epsilon.
  \end{equation}
  Thus $g$ is uniformly continuous. If $f$ is moreover an isometry,
  the first equality in $(\dagger)$ also clearly implies that $g$ is
  an isometry.
\end{proof}

We now apply the lemma to finish the proof of the proposition. Suppose
$Y$ is a complete metric space and $\psi : X \to Y$ uniformly
continuous. Since $\phi : X \to X_c$ is an isometry with dense image,
the lemma gives that there is a unique uniformly continuous map
$\psi_c : X_c \to Y$ such that $\psi_c \circ \phi = \phi$, proving
(3).

Finally, (4) is proven by the general argument giving uniqueness of
objects satisfying a universal property like (3). Suppose $Y_c$ is a
complete metric space and $\psi : X \to Y_c$ an isometry with dense
image. Then by (3) there is a unique uniformly continuous map $\psi_c
: X_c \to Y_c$ such that $\psi_c \circ \phi = \psi$, and the lemma
tells us that $\psi_c$ is in fact an isometry. But our proof of (3)
applies to $Y_c$ equipped with $\psi$ as well, so we symmetrically
have a unique isometry $\phi_c : Y_c \to X_c$ such that $\phi_c \circ
\psi = \phi$. I.e., we have the diagram
\[
\begin{tikzcd}[column sep = large, row sep = large]
  X \rar{\phi} \dar{\psi} & X_c \ar[shift left = .7ex]{dl}{\psi_c}
  \\ Y_c \ar[shift left = .7ex]{ur}{\phi_c}.
\end{tikzcd}
\]
Observe then that
\[
\phi_c \circ \psi_c \circ \phi = \phi = \id_{X_c} \circ\, \phi
\quad\text{and}\quad \psi_c \circ \phi_c \circ \psi = \psi = \id_{Y_c}
\circ\, \psi.
\]
Then the \textit{uniqueness} statement in (3) implies that we must
have $\phi_c \circ \psi_c = \id_{X_c}$ and $\psi_c \circ \phi_c =
\id_{Y_c}$.\footnote{This argument takes a couple of reads to absorb
  and understand, I think (it certainly did for me!). But it's
  extremely general and very useful so try to do so!} I.e.,
$\phi_c,\psi_c$ are inverse isomorphisms of metric spaces. And thus we
are done!


\end{document}
