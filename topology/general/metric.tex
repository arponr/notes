%%%%%%%%%%%%%%%%%%%%%%%%%%%%%%%%%%%%%%%%%%%%%%%%%%%%%%%%%%%%%%%%%%%%%%

\newcommand{\ob}{\oper{ob}}
\renewcommand{\hom}{\oper{hom}}
\newcommand{\id}{\oper{id}}
\newcommand{\im}{\oper{im}}
\newcommand{\op}{\oper{op}}

\newcommand{\Top}{\oper{Top}}
\newcommand{\Set}{\oper{Set}}
\newcommand{\Ab}{\oper{Ab}}
\newcommand{\Grp}{\oper{Grp}}
\newcommand{\Mod}{\oper{Mod}}
\newcommand{\Simplex}{\Delta}
\newcommand{\s}{\oper{s}}
\newcommand{\Ch}{\oper{Ch}}

\newcommand{\Sing}{\oper{Sing}}
\renewcommand{\H}{\mathrm{H}}

%%%%%%%%%%%%%%%%%%%%%%%%%%%%%%%%%%%%%%%%%%%%%%%%%%%%%%%%%%%%%%%%%%%%%%

%%%%%%%%%%%%%%%%%%%%%%%%%%%%%%%%%%%%%%%%%%%%%%%%%%%%%%%%%%%%%%%%%%%%%%

\newcommand{\ob}{\oper{ob}}
\renewcommand{\hom}{\oper{hom}}
\newcommand{\id}{\oper{id}}
\newcommand{\im}{\oper{im}}
\newcommand{\op}{\oper{op}}

\newcommand{\Top}{\oper{Top}}
\newcommand{\Set}{\oper{Set}}
\newcommand{\Ab}{\oper{Ab}}
\newcommand{\Grp}{\oper{Grp}}
\newcommand{\Mod}{\oper{Mod}}
\newcommand{\Simplex}{\Delta}
\newcommand{\s}{\oper{s}}
\newcommand{\Ch}{\oper{Ch}}

\newcommand{\Sing}{\oper{Sing}}
\renewcommand{\H}{\mathrm{H}}

%%%%%%%%%%%%%%%%%%%%%%%%%%%%%%%%%%%%%%%%%%%%%%%%%%%%%%%%%%%%%%%%%%%%%%


\title{Math 131 Section, I:\\Metric topology}
\author{Arpon Raksit}
\date{September 10, 2013 (original); \today\ (last edit).}

\begin{document}
\maketitle
\thispagestyle{fancy}

%%%%%%%%%%%%%%%%%%%%%%%%%%%%%%%%%%%%%%%%%%%%%%%%%%%%%%%%%%%%%%%%%%%%%%

\section{Some more on countability}

Professor McMullen didn't get quite as far as he had planned to last
Thursday, so let's talk a little more about countability. (The
material for this part of section is all on pp. 8--11 of Professor
McMullen's notes on the course website.)

\begin{exercise}
  Let $S$ be an enemy submarine travelling indefinitely along a line
  $\R$ (underwater!). Suppose our intelligence is intelligent enough
  to know that, at time $t=0$, $S$ is at some $i \in \Z \subset \R$,
  and that $S$ moves with some constant velocity $v \in \Z$; but not
  intelligent to know $i$ or $v$. You need to shoot down $S$ to win
  the war; luckily your set of torpedos is in bijection with
  $\N$. Give a strategy to guarantee victory in finite time.
\end{exercise}

%%%%%%%%%%%%%%%%%%%%%%%%%%%%%%%%%%%%%%%%%%%%%%%%%%%%%%%%%%%%%%%%%%%%%%

\section{Quick review of metric spaces}

I bet you'll find yourself thinking at some point that point-set
topology is just absurdly, unreasonably, unmotivatedly, unnecessarily
general.\footnote{cf.
  \url{http://en.wikipedia.org/wiki/Pointless_topology}} When this
happens, do two things:
\begin{enumerate}
\item Stop thinking that. Bask in the generality. Like Professor
  McMullen said in the introductory lecture, it turns out all this
  generality is unbelievably useful all over mathematics. Otherwise
  this course probably wouldn't be taught.
\item But seriously, if you find yourself miserable, amidst a downpour
  of statements about open sets, seek shelter in the land of metric
  spaces! Metric spaces can give you a nice picture to keep in your
  head when thinking about weird topological definitions. Moreover, if
  you're stuck in a proof, maybe try proving the statement with a
  metric, and see if you can't translate things back into topological
  terms.
\end{enumerate}
In light of point (2), I thought I'd review the basic definitions of
metric topology and how they relate to general topology.

\begin{definition}
  A \textit{metric space} is the following data:
  \begin{itemize}
  \item a set $X$,
  \item a function $d : X \times X \to \R$,
  \end{itemize}
  satisfying the following conditions:
  \begin{itemize}
  \item for $x,y \in X$, $d(x,y) = 0$ if and only if $x = y$,
  \item for $x,y \in X$, $d(x,y) = d(y,x)$,
  \item for $x,y,z \in X$, $d(x,y) \le d(x,z) + d(z,y)$.
  \end{itemize}
  We refer to a metric space by the pair $(X, d)$, or, by the usual
  abuse of notation when no confusion might arise, simply by $X$. We
  refer to $d$ as the \textit{metric} on $X$.
\end{definition}

\begin{example}
  The first example one thinks of is $X \coloneqq \R^n$ and $d(x,y)
  \coloneqq |x-y|$. Indeed you're probably picturing $\R^2$ in your
  head whenever you're thinking about a topological space (but be wary
  of this as we tread into weirder corners of topology!). Of course
  there are other interesting examples, but I don't want to spend time
  developing the theory of metric spaces\footnote{Professor Elkies has
    some nice notes that do this, see
    \url{http://www.math.harvard.edu/~elkies/M55b.10/index.html}. You
    can find some cool problems there also.} right now, so I'll be a
  bad person and leave them out.
\end{example}

Of course metric spaces are usually presented as a more general place
where we can talk about limits and continuous functions in terms of
our trusty $\epsilon$ and $\delta$.

\begin{definition}
  Let $X$ a metric space. A sequence $(x_k)_{k \in \N}$ of points in
  $X$ is said to \textit{converge} to $x \in X$ if for each $\epsilon
  > 0$ there exists $n \in \N$ such that $d(x_k,x) < \epsilon$ for all
  $k \ge n$. We write $\lim_{k \to \infty} x_k = x$, or $x_k \to x$ as
  $k \to \infty$.
\end{definition}

\begin{exercise}
  A sequence converges to a unique point.
\end{exercise}

\begin{definition}
  Let $(X, d_X), (Y, d_Y)$ metric spaces. A function $\phi : X \to Y$
  is
  \begin{itemize}
  \item \textit{continuous at} $x \in X$ if for each $\epsilon > 0$
    there exists a $\delta > 0$ such that
    \[
    d_X(x,y) < \delta \implies d_Y(\phi(x),\phi(y)) < \epsilon
    \quad\text{for } y \in X.
    \]
  \item a \textit{continuous map} (or just \textit{map} if we're
    feeling lazy) if $\phi$ is continuous at each $x \in X$.
  \end{itemize}
\end{definition}

Ok, that's all well and good, and we can do things like differential
calculus with these definitions. But in this class we are topologists,
not analysts. We don't want to push these loathsome epsilons around!
Even the smallest of epsilons is a burden. So let's rephrase things.

\begin{definition}
  Let $X$ a metric space, $x \in X$, and $r > 0$. The \textit{open
    ball of radius $r$ centred at $x$} is defined as $B(x,r) \coloneqq
  \{y \in X \mid d(x,y) < r\}$.
\end{definition}

\begin{definition}
  Let $X$ a metric space. We say a subset $U \subseteq X$ is
  \emph{open} if for each $x \in U$ there exists an $r > 0$ such that
  $B(x,r) \subseteq U$.
\end{definition}

\begin{exercise}
  $B(x,r) \subseteq X$ is open for any $x \in X, r > 0$.
\end{exercise}

\begin{proposition}
  Let $X$ a metric space. Then $\mathcal{T} \coloneqq \{U \subseteq X
  \mid U\ \text{open}\}$ is a topology on $X$.
\end{proposition}

\begin{proof}
  Let $\{U_\alpha\}_{\alpha \in A} \subseteq \mathcal{T}$, set $U
  \coloneqq \bigcup_{\alpha \in A} U_\alpha$, and let $x \in U$. Pick
  $\alpha \in A$ such that $x \in U_\alpha$. By definition there
  exists $r > 0$ such that $B(x,r) \subseteq U_\alpha \subseteq U$.

  \medskip
  Next let $U_1,\ldots,U_n \in \mathcal{T}$, set $U \coloneqq
  \bigcap_{i=1}^n U_i$, and let $x \in U$. By definition there exists
  $r_i > 0$ such that $B(x,r_i) \subseteq U_i$ for $1 \le i \le
  n$. Then if we set $r \coloneqq \min \{r_i \mid 1 \le i \le n\}$
  (this is where we use that the intersection is finite!) we have
  $B(x,r) \subseteq U_i$ for $1 \le i \le n$, whence $B(x,r) \subseteq
  U$.

  \medskip
  Since obviously $\emptyset, X \in \mathcal{T}$, $\mathcal{T}$ is
  indeed a topology on $X$.
\end{proof}

\begin{exercise}
  \label{unionofballs}
  A subset $U \subseteq X$ is open if and only if it's a union
  $\bigcup_{\alpha \in A} B(x_\alpha, r_\alpha)$ of open balls. (Soon
  you'll learn that the open balls form a \textup{basis} for this
  topology on $X$.)
\end{exercise}

\begin{proposition}
  Let $X$ a metric space and $A \subseteq X$. The following are
  equivalent:
  \begin{enumerate}
  \item $X - A$ is open.
  \item For any sequence $(x_k)_{k \in \N}$ in $A$ converging to $x
    \in X$, we have $x \in A$.
  \end{enumerate}
\end{proposition}

\begin{proof}
  Assume (1). Let $(x_k)_{k \in \N}$ a sequence in $A$ converging to
  $x \in X$. If $x \in X - A$ then by hypothesis there exists
  $\epsilon > 0$ such that $B(x, \epsilon) \subseteq X - A$. But one
  easily sees that this contradicts that $x_k \to x$ as $k \to
  \infty$.

  \medskip
  Assume (2). Let $x \in X$. Assume $B(x,\epsilon) \cap A \ne
  \emptyset$ for each $\epsilon > 0$. In particular, for $k \in \N$ we
  have $x_k \in A$ such that $d(x_k,x) < 1/k$. Here one easily checks
  that $x_k \to x$ as $k \to \infty$. So $x \in A$ by hypothesis.
\end{proof}

So being closed under the operation of taking limits of sequences is
the same thing as being closed in the topological sense.

\begin{proposition}
  Let $X, Y$ metric spaces and $\phi : X \to Y$ a function. The
  following are equivalent
  \begin{enumerate}
  \item $\phi^{-1}(U) \subseteq X$ is open whenever $U \subseteq Y$ is
    open.
  \item $\phi$ is a continuous map.
  \end{enumerate}
\end{proposition}

\begin{proof}
  Assume (1). Let $x \in X$ and $\epsilon > 0$. Since
  $B(\phi(x),\epsilon)$ is open in $Y$, by hypothesis
  $\phi^{-1}(B(\phi(x), \epsilon))$ is open in $X$. Thus there exists
  $\delta > 0$ such that $\phi(B(x, \delta)) \subseteq B(\phi(x),
  \epsilon)$, which says exactly that $\phi$ is continuous at $x$.

  \medskip
  Assume (2). Let $U \subseteq Y$ an open set, $x \in \phi^{-1}(U)$,
  and $\epsilon > 0$ such that $B(\phi(x),\epsilon) \subseteq Y$. By
  continuity we can choose $\delta > 0$ such that
  \[
  \phi(B(x,\delta)) \subseteq B(y,\epsilon) \implies B(x,\delta)
  \subseteq \phi^{-1}(U).
  \]
  It follows that $\phi^{-1}(U)$ is open.
\end{proof}

So a map of metric spaces is continuous in the $\epsilon$-$\delta$
sense if and only if it's continuous in the topological sense. Ok,
hopefully this provides some motivation for why things are defined the
way they are for general topological spaces. (Of course, it's easy for
me to say this is all motivated---the hard thing was for the actual
mathematicians to come up with the correct general formulation for
these notions!)


%%%%%%%%%%%%%%%%%%%%%%%%%%%%%%%%%%%%%%%%%%%%%%%%%%%%%%%%%%%%%%%%%%%%%%

\end{document}
