%%%%%%%%%%%%%%%%%%%%%%%%%%%%%%%%%%%%%%%%%%%%%%%%%%%%%%%%%%%%%%%%%%%%%%

\newcommand{\ob}{\oper{ob}}
\renewcommand{\hom}{\oper{hom}}
\newcommand{\id}{\oper{id}}
\newcommand{\im}{\oper{im}}
\newcommand{\op}{\oper{op}}

\newcommand{\Top}{\oper{Top}}
\newcommand{\Set}{\oper{Set}}
\newcommand{\Ab}{\oper{Ab}}
\newcommand{\Grp}{\oper{Grp}}
\newcommand{\Mod}{\oper{Mod}}
\newcommand{\Simplex}{\Delta}
\newcommand{\s}{\oper{s}}
\newcommand{\Ch}{\oper{Ch}}

\newcommand{\Sing}{\oper{Sing}}
\renewcommand{\H}{\mathrm{H}}

%%%%%%%%%%%%%%%%%%%%%%%%%%%%%%%%%%%%%%%%%%%%%%%%%%%%%%%%%%%%%%%%%%%%%%

%%%%%%%%%%%%%%%%%%%%%%%%%%%%%%%%%%%%%%%%%%%%%%%%%%%%%%%%%%%%%%%%%%%%%%

\newcommand{\ob}{\oper{ob}}
\renewcommand{\hom}{\oper{hom}}
\newcommand{\id}{\oper{id}}
\newcommand{\im}{\oper{im}}
\newcommand{\op}{\oper{op}}

\newcommand{\Top}{\oper{Top}}
\newcommand{\Set}{\oper{Set}}
\newcommand{\Ab}{\oper{Ab}}
\newcommand{\Grp}{\oper{Grp}}
\newcommand{\Mod}{\oper{Mod}}
\newcommand{\Simplex}{\Delta}
\newcommand{\s}{\oper{s}}
\newcommand{\Ch}{\oper{Ch}}

\newcommand{\Sing}{\oper{Sing}}
\renewcommand{\H}{\mathrm{H}}

%%%%%%%%%%%%%%%%%%%%%%%%%%%%%%%%%%%%%%%%%%%%%%%%%%%%%%%%%%%%%%%%%%%%%%


\title{Math 131 Section, VIII:\\Fibrations and cofibrations}
\author{Arpon Raksit}
\date{\today}

\begin{document}
\maketitle
\thispagestyle{fancy}

%%%%%%%%%%%%%%%%%%%%%%%%%%%%%%%%%%%%%%%%%%%%%%%%%%%%%%%%%%%%%%%%%%%%%%

\section{Introduction}

We've now seen in lecture what covering spaces are. In particular, if
$E \to B$ is a covering, we've talked about the ability to lift paths
$[0,1] \to B$ to paths $[0,1] \to E$, and we used this path-lifting to
compute $\pi_1(S^1) \simeq \Z$. More generally, Prof. McMullen has
described a \textit{lifting problem} as:
\begin{quote}
  Given maps $p : (E,e) \to (B,b)$ and $f : (X,x) \to (B,b)$ of based
  spaces, find a map $\tilde{f} : (X,x) \to (B,b)$ such that $p \circ
  \tilde{f} = f$.
\end{quote}
We can rewrite this lifting problem with the following diagram:
\[
\begin{tikzcd}[column sep = large, row sep = large]
  * \dar{x} \rar{e} & E \dar{p} \\ X \ar[dashed]{ur}{\tilde{f}}
  \rar{f} & B,
\end{tikzcd}
\]
where $*$ is the one-point space, and $e : * \to E$ and $x : * \to X$
are the inclusions of $*$ as the points $e \in E$ and $x \in X$,
respectively. Then the problem becomes:
\begin{quote}
  Given the commutative diagram of solid arrows, find a dashed arrow
  which keeps the diagram commutative.
\end{quote}
The notion of a fibration arises when we generalise this phrasing of
the lifting problem, and the notion of a cofibration arises by
``dualising'', as we will see.

%%%%%%%%%%%%%%%%%%%%%%%%%%%%%%%%%%%%%%%%%%%%%%%%%%%%%%%%%%%%%%%%%%%%%%

\section{Preliminaries}

We're going to need some topological constructions. Now, I'm going to
brush some point-set details under the rug, because they're a pain and
not really interesting. Basically some of the facts I'm about to state
aren't actually true for general topological spaces (you've witnessed
some of the crazy pathologies that come up---we don't want to really
deal with this anymore in algebraic topology). So maybe just imagine
all spaces in sight are locally compact Hausdorff or
something.\footnote{If you're really interested in what's underneath
  the rug, you can glance at
  \url{http://neil-strickland.staff.shef.ac.uk/courses/homotopy/cgwh.pdf}.}

\subsection{Pushouts}

%%%%%%%%%%%%%%%%%%%%%%%%%%%%%%%%%%%%%%%%%%%%%%%%%%%%%%%%%%%%%%%%%%%%%%

\section{Fibrations}

\begin{definition}
  A map $p : E \to B$ is called a \textit{fibration} if whenever we
  are given a diagram
  \[
  \begin{tikzcd}[column sep = large, row sep = large]
    X \ar[hookrightarrow]{d} \rar{f} & E \dar{p} \\ X \times I
    \ar[dashed]{ur}{\tilde{F}} \rar{F} & B,
  \end{tikzcd}
  \]
  for which the solid arrows commute, there exists a dashed arrow
  which makes the entire diagram commute. Here the inclusion $X \inj X
  \times I$ is given by $x \mapsto (x,0)$. I.e., given a homotopy $X
  \times I \to B$ between two maps $X \to B$ and a lift $X \to E$ of
  one of these maps, there is a lift of the other map and of the
  homotopy.
\end{definition}

\begin{remark}
  Note that if we set $X = *$ in the above then we get exactly the
  (based) path-lifting problem. We will now see that, in some sense,
  path lifting is all one really needs to be a fibration.
\end{remark}

I'm not sure if I can say too much about why fibrations are so
important right now---you should trust me.

\end{document}
