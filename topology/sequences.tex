%%%%%%%%%%%%%%%%%%%%%%%%%%%%%%%%%%%%%%%%%%%%%%%%%%%%%%%%%%%%%%%%%%%%%%

\newcommand{\ob}{\oper{ob}}
\renewcommand{\hom}{\oper{hom}}
\newcommand{\id}{\oper{id}}
\newcommand{\im}{\oper{im}}
\newcommand{\op}{\oper{op}}

\newcommand{\Top}{\oper{Top}}
\newcommand{\Set}{\oper{Set}}
\newcommand{\Ab}{\oper{Ab}}
\newcommand{\Grp}{\oper{Grp}}
\newcommand{\Mod}{\oper{Mod}}
\newcommand{\Simplex}{\Delta}
\newcommand{\s}{\oper{s}}
\newcommand{\Ch}{\oper{Ch}}

\newcommand{\Sing}{\oper{Sing}}
\renewcommand{\H}{\mathrm{H}}

%%%%%%%%%%%%%%%%%%%%%%%%%%%%%%%%%%%%%%%%%%%%%%%%%%%%%%%%%%%%%%%%%%%%%%

%%%%%%%%%%%%%%%%%%%%%%%%%%%%%%%%%%%%%%%%%%%%%%%%%%%%%%%%%%%%%%%%%%%%%%

\newcommand{\ob}{\oper{ob}}
\renewcommand{\hom}{\oper{hom}}
\newcommand{\id}{\oper{id}}
\newcommand{\im}{\oper{im}}
\newcommand{\op}{\oper{op}}

\newcommand{\Top}{\oper{Top}}
\newcommand{\Set}{\oper{Set}}
\newcommand{\Ab}{\oper{Ab}}
\newcommand{\Grp}{\oper{Grp}}
\newcommand{\Mod}{\oper{Mod}}
\newcommand{\Simplex}{\Delta}
\newcommand{\s}{\oper{s}}
\newcommand{\Ch}{\oper{Ch}}

\newcommand{\Sing}{\oper{Sing}}
\renewcommand{\H}{\mathrm{H}}

%%%%%%%%%%%%%%%%%%%%%%%%%%%%%%%%%%%%%%%%%%%%%%%%%%%%%%%%%%%%%%%%%%%%%%


\title{Math 131 Section, II:\\Inadequacy of sequences}
\author{Arpon Raksit}
\date{September 17, 2013 (original); \today\ (last edit).}

\begin{document}
\maketitle
\thispagestyle{fancy}

%%%%%%%%%%%%%%%%%%%%%%%%%%%%%%%%%%%%%%%%%%%%%%%%%%%%%%%%%%%%%%%%%%%%%%

\section{Apologies}

Professor McMullen has mentioned a few times that one of the reasons
we talk about everything in terms of open and closed sets in general
topology is that the notion of convergent sequences doesn't capture
all phenomena in this generality. In this section I wanted to talk
about a way we can still talk about everything in terms of
convergence---but convergence of something more general than
sequences, called \textit{filters}.

\medskip
Unfortunately, I don't really know too much about filters right now,
and I didn't get myself to learn them well enough before
section. Sorry! So I'll (optimistically) leave that for another
section, and just give an example\footnote{Professor McMullen also
  gave this example in lecture, but I did it the day before, so ha!}
to motivate them. I.e., let me prove to you that sequences aren't good
enough for us.

\renewcommand{\T}{\mathcal{T}}

\section{The cocountable topology}

\begin{definition}
  For any set $X$, the \textit{cocountable topology} on $X$ is
\[
\T \coloneqq \{U \subseteq X \mid X - U \text{ is countable}\} \cup
\{\emptyset\}.
\]
I.e., we let the closed sets of $X$ be the countable subsets along
with $X$ (where we consider finite, and in particular empty, sets to
be countable). Note that $\T$ is a topology because arbitrary
intersections and finite unions of countable sets are countable.
\end{definition}

Ok, now fix an uncountable set $X$ and put it in the cocountable
topology. (You can just take $X \coloneqq \R$ for concreteness; I'm just
wrapping some notation around it since the topology we've put on $X$
is far, far away from the intuitive topologies you know on $\R$.)  And
suppose we have $A \subsetneq X$ a proper, uncountable subset. (Again,
take $A \coloneqq [0, 1] \subset \R$ for concreteness.)

\medskip
I claim that $A$ is dense in $X$, but that any convergent sequence in
$A$ necessarily converges (uniquely) to a point in $A$. What is this
saying? Recall that $A$ being dense means that it's closure is all of
$X$, that is $\oline A = X$. On the other hand, there is no sequence
is $A$ converging to any point outside of $A$. (Keep in mind, $A \ne
X$ so that this is nontrivial!) So basically, in this topology, being
closed in the topological sense is rather far from being closed under
the operation of taking limits of convergent sequences. Contrast this
with the familiar setting of metric spaces, where these two concepts
are the same!\footnote{See my notes from section I for a proof of
  this.} Ok, let's prove it.

\begin{exercise}
  \label{dense}
  Show that $\oline A = X$ if and only if $U \cap A \ne \emptyset$ for
  every $\emptyset \ne U \subseteq X$ open.
\end{exercise}

\begin{lemma}
  $\oline A = X$.
\end{lemma}

\begin{proof}
  We use the characterisation of denseness given in (\ref{dense}). Let
  $U \subseteq X$ nonempty and open. By definition of the cocountable
  topology $X - U$ is countable. Since $A$ is uncountable, we cannot
  then have $A \subseteq X - U$, that is, $U \cap A \ne \emptyset$.
\end{proof}

Before we examine convergent sequences in $A$, let's just quickly
recall what it means for sequences to converge in topological spaces.

\begin{definition}
  We say a sequence $(x_k) \in X^\N$ \textit{converges} to $x \in X$
  if for each open set $U \subseteq X$ containing $x$ there exists $n
  \in \N$ such that $x_k \in U$ for $k \ge n$.
\end{definition}

\begin{lemma}
  Let $(x_k) \in A^\N$ a convergent sequence in $A$. Then there exists
  $x \in A$ and $n \in \N$ such that $x_k = x$ for $k \ge n$. That is,
  $(x_k)$ is \textup{eventually constant}.
\end{lemma}

\begin{proof}
  Since $(x_k)$ is convergent we can pick $x \in X$ such that $x_n \to
  x$ as $n \to \infty$. Observe that $F \coloneqq \{x_k \mid k \in \N, x_k
  \ne x\}$ is countable, and hence closed, and by definition does not
  contain $x$. I.e., $X - F$ is an open set containing $x$. So by
  definition of convergence, there exists $n \in \N$ such that $x_k
  \in X - F$ for $k \ge n$. But then by definition of $F$ we must have
  $x_k = x$ for $k \ge n$. Then of course $x \in A$ since $x_k \in A$
  for all $k \in \N$.
\end{proof}

Note that eventually constant sequences obviously converge to a unique
point.



%%%%%%%%%%%%%%%%%%%%%%%%%%%%%%%%%%%%%%%%%%%%%%%%%%%%%%%%%%%%%%%%%%%%%%

\end{document}
