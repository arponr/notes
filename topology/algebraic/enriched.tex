%%%%%%%%%%%%%%%%%%%%%%%%%%%%%%%%%%%%%%%%%%%%%%%%%%%%%%%%%%%%%%%%%%%%%%

\newcommand{\ob}{\oper{ob}}
\renewcommand{\hom}{\oper{hom}}
\newcommand{\id}{\oper{id}}
\newcommand{\im}{\oper{im}}
\newcommand{\op}{\oper{op}}

\newcommand{\Top}{\oper{Top}}
\newcommand{\Set}{\oper{Set}}
\newcommand{\Ab}{\oper{Ab}}
\newcommand{\Grp}{\oper{Grp}}
\newcommand{\Mod}{\oper{Mod}}
\newcommand{\Simplex}{\Delta}
\newcommand{\s}{\oper{s}}
\newcommand{\Ch}{\oper{Ch}}

\newcommand{\Sing}{\oper{Sing}}
\renewcommand{\H}{\mathrm{H}}

%%%%%%%%%%%%%%%%%%%%%%%%%%%%%%%%%%%%%%%%%%%%%%%%%%%%%%%%%%%%%%%%%%%%%%

%%%%%%%%%%%%%%%%%%%%%%%%%%%%%%%%%%%%%%%%%%%%%%%%%%%%%%%%%%%%%%%%%%%%%%

\newcommand{\ob}{\oper{ob}}
\renewcommand{\hom}{\oper{hom}}
\newcommand{\id}{\oper{id}}
\newcommand{\im}{\oper{im}}
\newcommand{\op}{\oper{op}}

\newcommand{\Top}{\oper{Top}}
\newcommand{\Set}{\oper{Set}}
\newcommand{\Ab}{\oper{Ab}}
\newcommand{\Grp}{\oper{Grp}}
\newcommand{\Mod}{\oper{Mod}}
\newcommand{\Simplex}{\Delta}
\newcommand{\s}{\oper{s}}
\newcommand{\Ch}{\oper{Ch}}

\newcommand{\Sing}{\oper{Sing}}
\renewcommand{\H}{\mathrm{H}}

%%%%%%%%%%%%%%%%%%%%%%%%%%%%%%%%%%%%%%%%%%%%%%%%%%%%%%%%%%%%%%%%%%%%%%


\title{Enriched Categories}
\author{Arpon Raksit}
\thanks{These notes are me digesting material presented by Hiro Lee
  Tanaka and Sitan Chen on September 18 and 25 in a reading group on
  infinity categories held at Harvard in Fall 2013.}
\date{\today}

\begin{document}
\maketitle
\thispagestyle{fancy}

%%%%%%%%%%%%%%%%%%%%%%%%%%%%%%%%%%%%%%%%%%%%%%%%%%%%%%%%%%%%%%%%%%%%%%

\renewcommand{\C}{\mathcal{C}}

\section{Motivation}

Recall that in a category $\C$ we have a \textit{hom-set}
$\Hom_\C(X,Y)$ for any two objects $X,Y \in \Ob \C$. But sometimes we
really want $\Hom_\C(X,Y)$ to be more than just a set, i.e. have the
structure of an object in some category. Our primary motivation is the
idea that some categories should have \textit{hom-spaces}.

\begin{example}
  Consider $\C = \Top$ the category of spaces. (Actually we'll need to
  restrict our class of spaces, but let's leave the details for a
  couple of pages.) We'll see that $\Hom_\C(X,Y)$ has a natural
  topology, which gives us a mapping \textit{space} $\Map(X,Y)$. An
  important property this satisfies---an incarnation of the
  ``hom-tensor adjunction''---is that there is a natural isomorphism
  \begin{equation}
    \label{homtensor}\tag{$\ast$}
    \Map(X \times Y, Z) \simeq \Map(X, \Map(Y, Z)).
  \end{equation}
  And one particularly important example of this important property is
  when $X = [0,1]$. Then (\ref{homtensor}) tells us that talking about
  \textit{homotopies} of maps $Y \to Z$ is equivalent to talking about
  \textit{paths} in $\Map(Y,Z)$.
\end{example}

%%%%%%%%%%%%%%%%%%%%%%%%%%%%%%%%%%%%%%%%%%%%%%%%%%%%%%%%%%%%%%%%%%%%%%

\renewcommand{\T}{\mathcal{T}}

\section{Monoidal and enriched categories}

So we want to somehow talk about categories $\C$ in which we have
hom-objects in some other category $\T$. It turns out the structure we
want on $\T$ is some sort of ``tensor product''. In fact it's easy to
see why we need some way of taking any two objects in $\T$ and
producing a third: we need a way to write down a composition law
$\Hom_\C(X,Y) \times \Hom_\C(Y,Z) \to \Hom_\C(X,\Z)$ as a map in $\T$!
But we'll use the notation $\otimes$ instead of $\times$.

\begin{definition}
  A \textit{monoidal category} $\T$ is a category equipped with
  \begin{itemize}
  \item a functor $- \otimes - : \T \times \T \to \T$, the
    \textit{tensor product}
  \item an object $1 \in \T$, the \textit{unit}
  \item a natural isomorphism $\alpha : (- \otimes -) \otimes - \to -
    \otimes (- \otimes -)$, the \textit{associator}
  \item a natural isomorphism $\lambda : 1 \otimes - \to \id_\T$, the
    \textit{left unitor}
  \item a natural isomorphism $\rho : - \otimes 1 \to \id_\T$, the
    \textit{right unitor}
  \end{itemize}
  satisfying coherence conditions given by two commutative diagrams:
  the \textit{triangle identity},
    \[
    \begin{tikzcd}[column sep = tiny]
      (- \otimes 1) \otimes - \ar{rr}{\alpha} \ar{dr}[below]{\rho
        \otimes \id_\T\hspace{30pt}} & \ & - \otimes (1 \otimes -)
      \ar{dl}{\id_\T \otimes \lambda} \\ \ & - \otimes -, & \
    \end{tikzcd}
    \]
    and the \textit{pentagon identity},
    \[
    \begin{tikzcd}[column sep = tiny]
      \ & (- \otimes -) \otimes (- \otimes -) \ar{dr}{\alpha} \ \\ ((-
      \otimes -) \otimes -) \otimes - \ar{ur}{\alpha} \ar{d}{\alpha
        \otimes \id_\T} & \ & - \otimes (- \otimes (- \otimes -))
      \\ (- \otimes (- \otimes -)) \otimes - \ar{rr}{\alpha} & \ & -
      \otimes ((- \otimes -) \otimes -) \ar{u}{\id_\T \otimes \alpha}.
    \end{tikzcd}
    \]
\end{definition}

\begin{remark}
  The coherence conditions look sort of intimidating. Basically the
  point is that the tensor product should behave as one might
  expect. Perhaps even more concretely, as the name suggests, a
  symmetric monoidal category is a categorification of the notion of a
  commutative monoid. So just think that the tensor product gives a
  nice, commutative, associative multiplication on $\T$.
\end{remark}

\renewcommand{\Z}{\mathbb{Z}}

\begin{examples}
  \label{examples-monoidal}
  One thing to note immediately is that we can put different monoidal
  structures on the same category. We have the following familiar
  examples.
  \begin{enumerate}
  \item $\T \coloneqq \Set$ the category of sets; $\otimes \coloneqq
    \times$ the direct product; $1 \coloneqq \{*\}$ the one-point set.
  \item $\T \coloneqq \Set$; $\otimes \coloneqq \amalg$ the disjoint
    union; $1 \coloneqq \emptyset$ the empty set.
  \item $\T \coloneqq \Ab$ the category of abelian groups; $\otimes
    \coloneqq \otimes_\Z$ the tensor product over $\Z$; $1 \coloneqq
    \Z$.
  \item $\T \coloneqq \Ab$; $\otimes \coloneqq \oplus$ the direct sum;
    $1 \coloneqq 0$ the trivial group.
  \item If you're feeling fancy, $\T \coloneqq \Ch$ the category of
    chain complexes; $\otimes \coloneqq \otimes_\Z$ the tensor product
    of complexes over $\Z$, $1 \coloneqq \Z$ the complex with $\Z$
    concentrated in degree $0$.
  \item Most relevant to our immediate goals, $\T \coloneqq \Top$ the
    category of topological spaces; $\otimes \coloneqq \times$ the
    direct product; $1 \coloneqq \{*\}$ the one-point space.
  \end{enumerate}
\end{examples}

Now we want to define precisely what it means for a category to have
hom-objects in a monoidal category.

\begin{definition}
  Let $\T$ be a monoidal category. Then a \textit{$\T$-enriched
    category} (or a \textit{category enriched over $\T$}) $\C$ is the
  data of
  \begin{itemize}
  \item a collection $\Ob\C$ of \textit{objects}
  \item a \textit{hom-object} $\Hom_\C(X,Y) \in \Ob\T$ for all pairs
    $X,Y \in \Ob\C$
  \item a \textit{composition} map (in the category $\T$!)
    \[
    \circ : \Hom_\C(Y,Z) \otimes \Hom_\C(X,Y) \to \Hom_\C(X,Z)
    \]
    for all triples of objects $X,Y,Z \in \Ob\C$
  \item an \textit{identity} $\id_X : 1 \to \Hom_\C(X,X)$ for all
    objects $X \in \C$
  \end{itemize}
  satisfying
  \begin{itemize}
  \item the commutative diagram for associativity for $W,X,Y,Z \in \Ob
    \C$,
    \[
    \begin{tikzcd}
      (\Hom_\C(Z,Y) \otimes \Hom_\C(X,Y)) \otimes \Hom_\C(W,X)
      \rar{\circ \otimes \id} \ar{dd}{\alpha} & \Hom_\C(X,Z) \otimes
      \Hom_\C(W,X) \dar{\circ} \\ \ & \Hom_\C(W,Z) \\ \Hom_\C(Z,Y)
      \otimes (\Hom_\C(X,Y) \otimes \Hom_\C(W,X)) \rar{\id \otimes
        \circ} & \Hom_\C(Z,Y) \otimes \Hom_\C(W,Y),
      \ar{u}[right]{\circ}
    \end{tikzcd}
    \]
  \item the commutative diagrams for the identity maps for $X,Y \in
    \Ob\C$,
    \[
    \begin{tikzcd}[column sep = tiny]
      1 \otimes \Hom_\C(X,Y) \ar{rr}{\id_Y \otimes \id}
      \ar{dr}[below]{\lambda\hspace{10pt}} & \ & \Hom_\C(Y,Y) \otimes
      \Hom_\C(X,Y) \ar{dl}{\circ} \\ \ & \Hom_\C(X,Y), & \
    \end{tikzcd}
    \]
    and
    \[
    \begin{tikzcd}[column sep = tiny]
      \Hom_\C(X,Y) \otimes 1 \ar{rr}{\id \otimes \id_X}
      \ar{dr}[below]{\rho\hspace{10pt}} & \ & \Hom_\C(X,Y) \otimes
      \Hom_\C(X,X) \ar{dl}{\circ} \\ \ & \Hom_\C(X,Y). & \
    \end{tikzcd}
    \]
  \end{itemize}
\end{definition}

\begin{examples}
  \ \begin{enumerate}
  \item Any ordinary category is a $\Set$-enriched category when we
    put a monoidal structure on $\Set$ as in Example
    \ref{examples-monoidal}.1.
  \item $\Set$, $\Ab$, and $\Ch$ are all enriched over themselves.
  \item When we say a category has \textit{hom-spaces} we mean mean
    the category is be enriched over $\Top$, given the monoidal
    structure in Example \ref{examples-monoidal}.6.
  \end{enumerate}
\end{examples}

%%%%%%%%%%%%%%%%%%%%%%%%%%%%%%%%%%%%%%%%%%%%%%%%%%%%%%%%%%%%%%%%%%%%%%

\newcommand{\CG}{\mathrm{CG}}
\newcommand{\WH}{\mathrm{WH}}

\section{Replacing $\Top$ with $\CG$}

Ok I just lied in that last example. As mentioned earlier, it turns
out that the full category of topological spaces $\Top$ has some
pathologies that we really don't want to deal with. So we're going to
make a slight modification and replace $\Top$ with a certain
subcategory $\CG$ which does not have these pathologies.\footnote{A
  much more detailed and comprehensive account of $\CG$ can be found
  at \url{neil-strickland.staff.shef.ac.uk/courses/homotopy/cgwh.pdf},
  and the presentation here is mostly taken from there, in addition to
  May's book, \textit{A Concise Course in Algebraic Topology}.}

\begin{convention}
  We will follow the convention of calling a space \textit{compact} if
  it is quasicompact and Hausdorff, and the same for \textit{locally
    compact}.
\end{convention}

Before I begin defining things, a disclaimer: I honestly don't know
any great motivation for these definitions besides that they turn out
to give us a more convenient category to work with.

\begin{definitions}
  Let $X$ a space, and $A \subseteq X$ a subspace.
  \begin{enumerate}
  \item We say $A$ is \textit{$k$-closed} if $f^{-1}(A)$ is closed in
    $K$ for all compact spaces $K$ and continuous maps $f : K \to X$.
  \item It is easy to check that the collection of $k$-closed subsets
    form the closed sets of a topology on $X$. We call $X$ with this
    topology the \textit{$k$-ification} of $X$, and denote it by $kX$.
  \item We say $X$ is a \textit{$k$-space} if $X = kX$, that is, if
    $A$ is $k$-closed if and only if $A$ is closed.
  \item We say $X$ is \textit{weakly Hausdorff} if $f(K)$ is closed in
    $X$ for all compact spaces $K$ and continuous maps $f : K \to X$.
  \item We say $X$ is \textit{compactly generated} if it is a weakly
    Hausdorff $k$-space.
  \end{enumerate}
\end{definitions}

\begin{remarks}
  \label{CGWH-rems}
  Let us note a couple of immediate facts regarding the above
  definitions.
  \begin{enumerate}
  \item If $A \subseteq X$ is closed then certainly $A$ is $k$-closed,
    i.e., $\id : kX \to X$ is continuous. Thus $X$ is a $k$-space if
    and only if all $k$-closed sets are closed.
  \item If $A$ is $k$-closed, then in particular for $K \subseteq X$
    any compact subspace, with inclusion $i : K \to X$, we have $A
    \cap K = i^{-1}(A)$ is closed.
  \item If $K$ is compact then a map $f : K \to X$ is continuous if
    and only if it is continuous as a map $K \to kX$. Hence $k(kX) =
    kX$, so $kX$ is a $k$-space.
  \item If $X$ is weakly Hausdorff, then so is $kX$, and hence $kX$ is
    compactly generated by the above.
  \item Certainly if $X$ is Hausdorff then $X$ is weakly
    Hausdorff. And certainly if $X$ is weakly Hausdorff then $X$ is
    $T_1$ (points are closed).
  \end{enumerate}
\end{remarks}

\begin{lemma}
  \label{WH-props}
  Let $X$ be a weakly Hausdorff space.
  \begin{enumerate}
  \item If $K$ is compact and $f : K \to X$ continuous then $f(K)$ is
    compact (in the induced topology).
  \item A subspace $A \subseteq X$ is compact in $X$ if and only if it
    is compact in $kX$.
  \item A subspace $A \subseteq X$ is $k$-closed if and only if $A
    \cap K$ is closed in $K$ for every compact subspace $K \subseteq
    X$.
  \end{enumerate}
\end{lemma}

\begin{proof}
  Todo.
\end{proof}

The following shows that many of the spaces familiar to us are
compactly generated.

\begin{lemma}
  A locally compact space $X$ is compactly generated and weakly
  Hausdorff.
\end{lemma}

\begin{proof}
  Since $X$ is Hausdorff, $X$ is weakly Hausdorff. So it suffices to
  show a $k$-closed set $A \subseteq X$ must be closed. Let $x \in
  \oline A$; choose a compact neighbourhood $K$ of $x$. For any open
  neighbourhood $U$ of $x$, certainly $U \cap K$ is also a
  neighbourhood of $x$, and hence $U \cap K \cap A \ne \emptyset$. But
  this says that $x \in \oline{K \cap A}$, and since $A$ is $k$-closed
  we know $K \cap A$ is closed, so in fact $x \in K \cap A$. Thus
  $\oline A = A$ and we're done.
\end{proof}

\begin{lemma}
  \label{C-CG-map}
  Let $f : X \to Y$ a map, $X$ compactly generated. The following are
  equivalent:
  \begin{enumerate}
  \item $f$ is continuous as a map $X \to Y$;
  \item the restriction $f|_K : X \to Y$ is continuous for all compact
    subspaces $K \subseteq X$;
  \item $f$ is continuous as a map $X \to kY$.
  \end{enumerate}
\end{lemma}

\begin{proof}
  (1) $\Rightarrow$ (2) is tautological, and (3) $\Rightarrow$ (1)
  follows from $\id : kY \to Y$ being continuous (Remark
  \ref{CGWH-rems}.1). Assume (2). By Remark \ref{CGWH-rems}.3, this is
  equivalent to assuming $f|_K$ is continuous as a map $K \to kY$ for
  all compact $K \subseteq X$. Thus if $A \subseteq kY$ is any closed
  set and $K \subseteq X$ any compact set, then $f|_K^{-1}(A) =
  f^{-1}(A) \cap K$ is closed, whence by Lemma \ref{WH-props} this
  implies $f^{-1}(A)$ is closed. This gives (3).
\end{proof}

\begin{definition}
  Let $\WH$ the full subcategory of $\Top$ consisting of weakly
  Hausdorff spaces. Let $\CG$ the full subcategory of $\WH$ consisting
  of compactly generated spaces.
\end{definition}

\begin{proposition}
  We have functors $j : \CG \to \WH$, the forgetful inclusion, and $k
  : \WH \to \CG$, given by $k$-ification. Moreover, there is a natural
  isomorphism
  \[
  \Hom_\WH(jX, Y) \simeq \Hom_\CG(X, kY),
  \]
 i.e., $k$ is right-adjoint to $j$.
\end{proposition}

\begin{proof}
  Obviously $j$ is a functor. That $k$ is a functor, and that the
  adjunction given simply by the identity (on sets) is well-defined,
  are immediate from Lemma \ref{C-CG-map}.
\end{proof}

\begin{definition}
  Let $X,Y$ spaces. The \textit{compact-open topology} on
  $\Hom_\Top(X,Y)$ is defined as having subbasis elements of the form
\end{definition}

Today we're interested in $\T = \Top$, $\otimes = \times$, and $1 =
\{*\}$. And we're going to enrich $\CG$ compactly generatedHausdorff
spaces with $\T$. We're interested in these spaces because we needs
some conditions to get $\Map(X,Y) \times \Map(Y,Z) \to \Map(X,Z)$ to
be continuous (i.e., a map in the category $\Top$). There's an article
on nLab about ``Convenient categories of topological spaces'' that
deals with this.

But we jumped the gun. We need to define a topology on
$\Map(X,Y)$. We'll define a subbasis for the compact-open topology:
sets $(K,U)$ of maps $X \to Y$ which send $K \to Y$, where $K
\subseteq X$ compact and $U \subseteq Y$ open.

We claim that the following identity holds:
\[
\Map(A \times B, C) \simeq \Map(A, \Map(B,C)).
\]
Where $f \mapsto g$ where $g(a)(b) = f(a,b)$.

For continuity in forward direction, suffices to show $(K,(L,U))
\subseteq \Map(A, \Map(B,C))$ has open preimage (this is a lemma). But
preimage is just $(K \times L, U)$ so clear. The other direciton is
harder, and you probably need to use the compactly generated Hasudorff
condition. Break $K$ into compact boxes $K_i \times L_j$ and then do
shit. ``We're in the thick jungle of point-set topology.''

Important example $A = [0,1]$. This tells us homotopies are paths in
the mapping space.

\end{document}
