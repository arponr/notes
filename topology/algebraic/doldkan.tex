%%%%%%%%%%%%%%%%%%%%%%%%%%%%%%%%%%%%%%%%%%%%%%%%%%%%%%%%%%%%%%%%%%%%%%

\renewcommand{\A}{\mathbb{A}}
\renewcommand{\O}{\mathcal{O}}

\renewcommand{\a}{\mathfrak{a}}
\newcommand{\p}{\mathfrak{p}}
\newcommand{\q}{\mathfrak{q}}

\newcommand{\height}{\operatorname{ht}}

%%%%%%%%%%%%%%%%%%%%%%%%%%%%%%%%%%%%%%%%%%%%%%%%%%%%%%%%%%%%%%%%%%%%%%

%%%%%%%%%%%%%%%%%%%%%%%%%%%%%%%%%%%%%%%%%%%%%%%%%%%%%%%%%%%%%%%%%%%%%%

\renewcommand{\A}{\mathbb{A}}
\renewcommand{\O}{\mathcal{O}}

\renewcommand{\a}{\mathfrak{a}}
\newcommand{\p}{\mathfrak{p}}
\newcommand{\q}{\mathfrak{q}}

\newcommand{\height}{\operatorname{ht}}

%%%%%%%%%%%%%%%%%%%%%%%%%%%%%%%%%%%%%%%%%%%%%%%%%%%%%%%%%%%%%%%%%%%%%%


\title{The Dold-Kan correspondence}
\author{Arpon Raksit}
\date{\today}

\begin{document}
\maketitle
\thispagestyle{fancy}

%%%%%%%%%%%%%%%%%%%%%%%%%%%%%%%%%%%%%%%%%%%%%%%%%%%%%%%%%%%%%%%%%%%%%%

\renewcommand{\C}{\mathcal{C}}

\section{Introduction}

\begin{definition}
  A \textit{simplicial object} in a category $\C$ is a contravariant
  functor $\Simplex \to \C$. We denote the category
  $\Fun(\Simplex^\op, \C)$ of simplicial objects in $\C$ by
  $\s\C$. E.g., $\s\Set$ is the category of \textit{simplicial sets}
  and $\s\Ab$ is the category of \textit{simplicial abelian groups}.
\end{definition}

Recall we have a functor $\Sing : \Top \to \s\Set$, sending $X \mapsto
\Hom_\Top(|\Delta^\bullet|, X)$. Lately we've been talking about
$\Sing$ for two reasons:
\begin{enumerate}
\item It's a right adjoint to geometric realisation $|-| : \s\Set \to
  \Top$.
\item $\Sing(X)$ is a Kan complex for all $X \in \Top$. In this sense,
  ``Kan complexes  are like spaces''.
\end{enumerate}
But this isn't the first place one sees $\Sing$, probably. Indeed, the
singular homology functors are essentially defined by a composition
\[
\begin{tikzcd}[column sep = large]
  \H_n(-;\Z) \coloneqq \Top \rar{\Sing} & \s\Set \rar{\Z} & \s\Ab
  \rar{\sum (-1)^i d_i} & \Ch_{\ge 0} \rar{\H_n} & \Ab.
\end{tikzcd}
\]
Here $\Z$ denotes the functor which takes free abelian groups
level-wise, from which we get the \textit{singular chain complex} by
letting the boundary map be given by $\del \coloneqq \sum (-1)^i d_i$.

This was just to remind us that we've seen a natural functor $\s\Ab
\to \Ch$ relating simplicial abelian groups and chain complexes. We'll
look at it a bit more carefully in a second, and develop this
relationship much further.

%%%%%%%%%%%%%%%%%%%%%%%%%%%%%%%%%%%%%%%%%%%%%%%%%%%%%%%%%%%%%%%%%%%%%%

\renewcommand{\A}{\mathcal{A}}

\section{Stating the correspondence}

We fix $\A$ any abelian category---but we'll probably be imagining $\A
= \Ab$ or $\A = R\text{-}\Mod$ (for some commutative ring $R$).

\begin{notation}
  We denote the category of nonnegatively graded chain complexes in
  $\A$ (and chain maps) by $\Ch_{\ge 0}(\A)$.
\end{notation}

Let's now make precise the $\del \coloneqq \sum (-1)^i d_i$ business
with which we started this discussion.

\begin{definition}
  Let $A_\bullet \in \s\A$ a simplicial object in $\A$ (e.g., a
  simplicial abelian group). We define the \textit{associated chain
    complex} $C_\bullet(A) \in \Ch_{\ge 0}(\A)$ by
  \[
  \textstyle{C_n(A) \coloneqq A_n \quad\text{and}\quad \del_n
    \coloneqq \sum_{i=0}^n (-1)^i d_i : C_n(A) \to C_{n-1}(A)}
  \]
  for $n \ge 0$. Note that the simplicial identities clearly imply
  $\del^2 = 0$, so $C_\bullet(A)$ is indeed a chain complex. Moreover,
  this evidently defines a functor $C : \s\A \to \Ch_{\ge 0}(\A)$.
\end{definition}

This is perhaps the most natural---or familiar, at least---functor
$\s\A \to \Ch_{\ge 0}(\A)$, but it turns out not to be the cleanest to
use when discussing the relationship between the two categories. In
fact, we will want to use the following alternative.

\begin{definition}
  Again let $A_\bullet \in \s\A$ a simplicial object in $\A$. We
  define the \textit{normalised chain complex} $N_\bullet(A) \in
  \Ch_{\ge 0}(\A)$ by
  \[
  \textstyle{N_n(A) \coloneqq \bigcap_{i=0}^{n-1} \ker(d_i) \subseteq
    A_n \quad\text{and}\quad \del_n \coloneqq (-1)^n d_n : N_n(A) \to
    N_{n-1}(A)}
  \]
  for $n \ge 0$. Again the simplicial identities imply $\del^2 = 0$,
  and we have a functor $N : \s\A \to \Ch_{\ge 0}(\A)$.
\end{definition}

What is this unmotivated nonsense? Have no fear, for $C$ and $N$ are
intimately related! For instance we can note immediately from the
definitions that the natural inclusion $N_\bullet(A) \to C_\bullet(A)$
is in fact a chain map for $A \in \s\A$. But there's more!

\begin{definition}
  Let $A \in \s\A$. We define the \textit{degenerate subcomplex}
  $D_\bullet(A)$ of $C_\bullet(A)$ by
  \[
  \textstyle{D_0(A) \coloneqq 0 \quad\text{and}\quad D_n(A) \coloneqq
    \sum_{i=0}^{n-1} \im(s_i)}
  \]
  for $n \ge 1$. That is, $D_\bullet(A)$ is generated by the images of
  the degeneracy maps. Note that by the simplicial identities
  \[
  \textstyle{\del s_j = \sum_{i=0}^n (-1)^i d_i s_j = \sum_{i=0}^{j-1}
    (-1)^i s_{j-1} d_i + \sum_{i=j+2}^n (-1)^i s_j d_i,}
  \]
  so $D_\bullet(A)$ is indeed a subcomplex.
\end{definition}

\begin{proposition}
  Let $A \in \s\A$. For $n \ge 0$ the natural map
  \[
  \phi : N_n(A) \oplus D_n(A) \to C_n(A)
  \]
  induced by the inclusions is an isomorphism. Therefore we have a
  natural isomorphism
  \[
  N_\bullet(A) \simeq C_\bullet(A)/D_\bullet(A).
  \]
\end{proposition}

\begin{proof}
  Fix $n \ge 0$. For $0 \le i \le n - 1$, the simplicial identity
  $d_is_i = \id$ implies that we have a canonical splitting
  \[
  A_n \simeq \ker(d_i) \oplus \im(s_i).
  \]
  It follows easily that $N_n(A) \cap D_n(A) \simeq 0$, so we're just
  left to show that $\phi$ is surjective. We prove by downward
  induction on $0 \le j \le n-1$ that
  \[
  \textstyle{\im(\phi) \supseteq N_j \coloneqq\bigcap_{i=0}^j
    \ker(d_i).}
  \]
  The base case $j = n-1$ is tautological and the final case $j = 0$
  will finish the proof. Now consider the map
  \[
  \psi \coloneqq \id - s_jd_j : C_n \to C_n.
  \]
  Observe by the simplicial identities that
  \[
  d_j\psi = d_j - d_js_jd_j = d_j - d_j = 0 \quad\text{and}\quad
  d_i\psi = d_i - d_is_jd_j = d_i - s_{j-1}d_{j-1}d_i
  \]
  for $i < j$, implying that $\psi(N_{j+1}) \subseteq N_j$. So by
  induction $\im(\psi \circ \phi) \supseteq N_j$. But it's easy to see
  that $\im(\psi \circ \phi) \subseteq \im(\phi)$, since $\im(s_jd_j)
  \subseteq D_n(A)$ by definition.
\end{proof}

So there's the relationship between $C$ and $N$. With these
definitions in hand, we can now state our main goal, the
\textit{Dold-Kan correspondence}.

\begin{theorem}[Dold-Kan]
  The functor $N : \s\A \to \Ch_{\ge 0}(\A)$ is an equivalence of
  categories, and inclusion $N_\bullet(A) \to C_\bullet(A)$ for $A \in
  \s\A$ gives a natural (chain) homotopy equivalence.
\end{theorem}

%%%%%%%%%%%%%%%%%%%%%%%%%%%%%%%%%%%%%%%%%%%%%%%%%%%%%%%%%%%%%%%%%%%%%%

\section{Proving the correspondence}

%%%%%%%%%%%%%%%%%%%%%%%%%%%%%%%%%%%%%%%%%%%%%%%%%%%%%%%%%%%%%%%%%%%%%%

\nocite{goerssjardine, riehl-ssets, mathew-doldkan, weibel}
\bibliographystyle{amsplain}
\bibliography{refs}

\end{document}
