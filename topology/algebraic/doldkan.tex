%%%%%%%%%%%%%%%%%%%%%%%%%%%%%%%%%%%%%%%%%%%%%%%%%%%%%%%%%%%%%%%%%%%%%%

\renewcommand{\A}{\mathbb{A}}
\renewcommand{\O}{\mathcal{O}}

\renewcommand{\a}{\mathfrak{a}}
\newcommand{\p}{\mathfrak{p}}
\newcommand{\q}{\mathfrak{q}}

\newcommand{\height}{\operatorname{ht}}

%%%%%%%%%%%%%%%%%%%%%%%%%%%%%%%%%%%%%%%%%%%%%%%%%%%%%%%%%%%%%%%%%%%%%%

%%%%%%%%%%%%%%%%%%%%%%%%%%%%%%%%%%%%%%%%%%%%%%%%%%%%%%%%%%%%%%%%%%%%%%

\renewcommand{\A}{\mathbb{A}}
\renewcommand{\O}{\mathcal{O}}

\renewcommand{\a}{\mathfrak{a}}
\newcommand{\p}{\mathfrak{p}}
\newcommand{\q}{\mathfrak{q}}

\newcommand{\height}{\operatorname{ht}}

%%%%%%%%%%%%%%%%%%%%%%%%%%%%%%%%%%%%%%%%%%%%%%%%%%%%%%%%%%%%%%%%%%%%%%


\title{The Dold-Kan correspondence}
\author{Arpon Raksit}
\date{\today}

\begin{document}
\maketitle
\thispagestyle{fancy}

%%%%%%%%%%%%%%%%%%%%%%%%%%%%%%%%%%%%%%%%%%%%%%%%%%%%%%%%%%%%%%%%%%%%%%

\renewcommand{\C}{\mathcal{C}}

\section{Introduction}

\begin{definition}
  A \textit{simplicial object} in a category $\C$ is a contravariant
  functor $\Simplex \to \C$. We denote the category
  $\Fun(\Simplex^\op, \C)$ of simplicial objects in $\C$ by
  $\s\C$. E.g., $\s\Set$ is the category of \textit{simplicial sets}
  and $\s\Ab$ is the category of \textit{simplicial abelian groups}.
\end{definition}

Recall we have a functor $\Sing : \Top \to \sSet$, sending $X \mapsto
\Hom_\Top(|\Delta^\bullet|, X)$. Lately we've been talking about
$\Sing$ for two reasons:
\begin{enumerate}
\item It's a right adjoint to geometric realisation $|-| : \sSet \to
  \Top$.
\item $\Sing(X)$ is a Kan complex for all $X \in \Top$. In this sense,
  ``Kan complexes  are like spaces''.
\end{enumerate}
But this isn't the first place one sees $\Sing$, probably. Indeed, the
singular homology functors are essentially defined by a composition
\[
\begin{tikzcd}[column sep = large]
  \H_n(-;\Z) \coloneqq \Top \rar{\Sing} & \sSet \rar{\Z} & \sAb
  \rar{\sum (-1)^i d_i} & \Ch \rar{\H_n} & \Ab.
\end{tikzcd}
\]
Here $\Z$ denotes the functor which takes free abelian groups
level-wise, from which we get the \textit{singular chain complex} by
letting the boundary map be given by $\del \coloneqq \sum (-1)^i d_i$.

This was just to remind us that we've seen a natural functor $\sAb \to
\Ch$ relating simplicial abelian groups and chain complexes. We'll
look at it a bit more carefully in a second, and develop this
relationship much further.

\nocite{goerssjardine, riehl-ssets, mathew-doldkan}
\bibliographystyle{amsplain}
\bibliography{refs}

\end{document}
