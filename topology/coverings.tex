%%%%%%%%%%%%%%%%%%%%%%%%%%%%%%%%%%%%%%%%%%%%%%%%%%%%%%%%%%%%%%%%%%%%%%

\newcommand{\ob}{\oper{ob}}
\renewcommand{\hom}{\oper{hom}}
\newcommand{\id}{\oper{id}}
\newcommand{\im}{\oper{im}}
\newcommand{\op}{\oper{op}}

\newcommand{\Top}{\oper{Top}}
\newcommand{\Set}{\oper{Set}}
\newcommand{\Ab}{\oper{Ab}}
\newcommand{\Grp}{\oper{Grp}}
\newcommand{\Mod}{\oper{Mod}}
\newcommand{\Simplex}{\Delta}
\newcommand{\s}{\oper{s}}
\newcommand{\Ch}{\oper{Ch}}

\newcommand{\Sing}{\oper{Sing}}
\renewcommand{\H}{\mathrm{H}}

%%%%%%%%%%%%%%%%%%%%%%%%%%%%%%%%%%%%%%%%%%%%%%%%%%%%%%%%%%%%%%%%%%%%%%

%%%%%%%%%%%%%%%%%%%%%%%%%%%%%%%%%%%%%%%%%%%%%%%%%%%%%%%%%%%%%%%%%%%%%%

\newcommand{\ob}{\oper{ob}}
\renewcommand{\hom}{\oper{hom}}
\newcommand{\id}{\oper{id}}
\newcommand{\im}{\oper{im}}
\newcommand{\op}{\oper{op}}

\newcommand{\Top}{\oper{Top}}
\newcommand{\Set}{\oper{Set}}
\newcommand{\Ab}{\oper{Ab}}
\newcommand{\Grp}{\oper{Grp}}
\newcommand{\Mod}{\oper{Mod}}
\newcommand{\Simplex}{\Delta}
\newcommand{\s}{\oper{s}}
\newcommand{\Ch}{\oper{Ch}}

\newcommand{\Sing}{\oper{Sing}}
\renewcommand{\H}{\mathrm{H}}

%%%%%%%%%%%%%%%%%%%%%%%%%%%%%%%%%%%%%%%%%%%%%%%%%%%%%%%%%%%%%%%%%%%%%%


\title{Covering spaces}
\author{Arpon Raksit}
\date{November 22, 2013 (original); \today\ (last edit).}

\begin{document}
\maketitle
\thispagestyle{fancy}

%%%%%%%%%%%%%%%%%%%%%%%%%%%%%%%%%%%%%%%%%%%%%%%%%%%%%%%%%%%%%%%%%%%%%%

\section{Preliminaries}

\begin{notation}
  \begin{enumerate}[leftmargin=*]
  \item Denote the interval $[0,1]$ by $I$.
  \item Denote the fundamental group of a pointed space $(X,x)$ by
    $\pi_1(X,x)$.
  \item If we write an element $g \in \pi_1(X,x)$ as $[\gamma]$, we
    mean that $g$ is the homotopy class of the loop $\gamma : I \to X$
    based at $x$.
  \item Denote the group homomorphism $\pi_1(X,x) \to \pi_1(X',x')$
    induced by a map of pointed spaces $f : (X,x) \to (X',x')$ by
    $f_*$.
  \end{enumerate}
\end{notation}

We quickly remind ourselves of some basic general topology. This is
not actually important, and really just to make sure we understand
some restrictions we will be putting on the topological spaces we
consider here.

\begin{proposition}
  \label{locpathconn}
  Let $X$ be a locally path connected space. Then the connected
  components and path components of $X$ agree and are open. In
  particular, $X$ is homeomorphic to a disjoint union of spaces which
  are path connected and locally path connected.
\end{proposition}

\begin{proof}
  Omitted. Do it as an exericse, or see, e.g., \cite{munkres}.
\end{proof}

\begin{definition}
  A space $X$ is \emph{semilocally simply connected} if each $x \in X$
  has an open neighbourhood $U$ such that $i_* : \pi_1(U,x) \to
  \pi_1(X,x)$ is the trivial (constant) group homomorphism, where $i :
  U \to X$ is the inclusion.
\end{definition}

%%%%%%%%%%%%%%%%%%%%%%%%%%%%%%%%%%%%%%%%%%%%%%%%%%%%%%%%%%%%%%%%%%%%%%

\section{Setting up}

\begin{nothing}
  \label{nice-space}
  For the remainder we fix a nice pointed space $(B,b)$. Sometimes we
  won't use the basepoint, and sometimes we will. Here \emph{nice}
  means that $B$ is connected, locally path connected, and semilocally
  simply connected.
\end{nothing}

\begin{definition}
  \label{covering}
  Let $p : E \to B$ be a map of spaces. We say a subset $U \subseteq
  B$ is \emph{evenly covered} if $p^{-1}(U) = \coprod_{j \in J} V_j$,
  where $J \ne \emptyset$ and for each $j \in J$:
  \begin{enumerate}
  \item $V_j \subseteq E$ is open,
  \item the restriction $p : V_j \to U$ is a homeomorphism.
  \end{enumerate}
  We say $p$ is a \emph{covering} if each $x \in B$ has an open
  neighbourhood $U$ which is evenly covered. In this case we say $E$
  is a \emph{covering space} over $B$. As usual we often abusively
  denote a covering $p : E \to B$ just by the space $E$.
\end{definition}

\begin{remarks}
  \label{covering-rmk}
  The definition (\ref{covering}) is just how one formalises the idea
  that $E$ locally looks like a bunch---more precisely, a discrete
  set---of copies of its projection to $B$. In particular, note
  immediately from the definition that coverings are local
  homeomorphisms. Hence, given the assumptions in (\ref{nice-space})
  any covering space $E$ over $B$ is also locally path connected.
\end{remarks}

\newcommand{\Cov}{\operatorname{Cov}}
\newcommand{\Gal}{\operatorname{Gal}}

\begin{definition}
  \label{cover-category}
  We define $\Cov(B)$, the \emph{category of coverings of $B$}, as
  follows.
  \begin{enumerate}
  \item An object of $\Cov(B)$ is a covering $p : E \to B$.
  \item \label{cover-morphism} A \emph{morphism of coverings} from $p
    : E \to B$ to $p' : E' \to B$ is given by a map $f : E \to E'$
    such that $p'f = p$, i.e., a commutative diagram
    \[
    \begin{tikzcd}
      E \ar{rr}{f} \drar[swap]{p} & & E'. \dlar{p'} \\ & B &
    \end{tikzcd}
    \]
  \end{enumerate}
\end{definition}

\begin{nothing}
  As one might expect, we call a pointed map $p : (E,e) \to (B,b)$ a
  covering if the underlying unpointed map $p : E \to B$ is a
  covering. Similarly if $p : (E,e) \to (B,b)$ and $p' : (E',e') \to
  (B,b)$ are coverings, we will call $f : (E,e) \to (E',e')$ a
  morphism of coverings simply if the underlying $f : E \to E'$ is a
  morphism of coverings as defined above.
\end{nothing}

\begin{definition}
  \label{fibre}
  If $p : E \to B$ is a covering, for $x \in B$ we denote the
  \emph{fibre} $p^{-1}(x)$ by $F_x(E)$. Note that if $p' : E' \to B$
  is another covering, the condition
  (\ref{cover-category}.\ref{cover-morphism}) defining a morphism $f :
  E \to E'$ of coverings can be rephrased as the condition that
  \[
  f(F_x(E)) \subseteq F_x(E') \quad\text{for all } x \in B,
  \]
  i.e., that $f$ restricts to give maps on the fibres $F_x(E) \to
  F_x(E')$.
\end{definition}

%%%%%%%%%%%%%%%%%%%%%%%%%%%%%%%%%%%%%%%%%%%%%%%%%%%%%%%%%%%%%%%%%%%%%%

\section{Lifting things}

\begin{nothing}
  In this section we investigate the ``lifting properties'' of
  coverings, and their consequences. Suppose we have a covering $p :
  (E,e) \to (B,b)$ and a map of pointed spaces $f : (X,x) \to (B,b)$,
  giving us the data of the solid arrows in the diagram
  \begin{equation}
    \label{lifting-problem}
    \begin{tikzcd}
      \ & (E,e) \dar{p} \\ (X,x) \rar{f} \ar[dashed]{ur}{F} & (B,b).
    \end{tikzcd}
  \end{equation}
  The finding of the dashed arrow $F$ is called a \emph{lifting
    problem}. Indeed, such a map $F : (X,x) \to (E,e)$ is called a
  \emph{lift} of $f$ (along $p$), or alternatively, a \emph{solution}
  to the lifting problem. The ultimate result is the following.
\end{nothing}

\begin{proposition}
  \label{general-lifting}
  Let $p : (E,e) \to (B,b)$ be a covering and $f : (X,x) \to (B,b)$ a
  map of pointed spaces with $X$ locally path connected.
  \begin{enumerate}
  \item There is a lift $F : (X,x) \to (E,e)$ of $f$ if and only if
    $\im(f_*) \subseteq \im(p_*)$.
  \item If $X$ is connected and the lift $F$ does exist, then it is
    unique.
  \end{enumerate}
\end{proposition}

But in order to prove this we need some lemmas. In particular, we
first prove two important special cases of this general result.

\begin{lemma}
  \label{path-lifting}
  The statements of (\ref{general-lifting}) hold in the case $(X,x) =
  (I,0)$. In particular, if $f$ is the constant path at $b$, then the
  unique lift $F$ is the constant path at $e$.
\end{lemma}

\renewcommand{\U}{\mathfrak{U}}

\begin{proof}
  By definition of a covering there is a cover $\U$ of $B$ consisting
  of evenly covered open sets. Then $\{f^{-1}(U) \mid U \in \U\}$ is
  an open cover of $I$. Since $I$ is a compact metric space, we can
  then find by the Lebesgue number lemma $0 = t_0 < \cdots < t_{n+1} =
  1$ such that for $0 \le i \le n$,
  \[
  [t_i,t_{i+1}] \subseteq f^{-1}(U_i) \implies f([t_i,t_{i+1}])
  \subseteq U_i
  \]
  for some $U_i \in \U$. We define the lift $F : (I,0) \to (E,e)$
  inductively on the intervals $[t_i,t_{i+1}]$. Assume we have defined
  $F(t_i)$. Since $U_i$ is evenly covered, there is an open
  neighbourhood $V_i \subseteq E$ of $F(t_i)$ such that $p$ restricts
  to a homeomorphism $V_i \to U_i$, with inverse $q_i : U_i \to
  V_i$. Thus, given initial value $F(t_i)$, $f$ lifts uniquely on
  $[t_i,t_{i+1}]$ by $F \coloneqq q_i \circ f$. Since we are required
  to have $F(t_0) = F(0) = e$, this uniquely defines $F$.

  The particular statement about constant paths follows from
  uniqueness.
\end{proof}

\begin{lemma}
  \label{homotopy-lifting}
  The statements of (\ref{general-lifting}) hold in the case $(X,x) =
  (I \times I, (0,0))$. Moreover, if
  \begin{enumerate}
  \item $f$ fixes endpoints, in the sense that $f(0,-), f(1,-) : (I,0)
    \to (B,b)$ are constant, or
  \item $f$ is a null-homotopy, in the sense that $f(-,1) : (I,0) \to
    (B,b)$ is constant,
  \end{enumerate}
  then $F$ also has this property.
\end{lemma}

\begin{proof}
  The proof of (\ref{path-lifting}) goes through if, instead of
  partitioning $I$ into subintervals $[t_i,t_{i+1}]$, we partition $I
  \times I$ into subsquares $[s_i,s_{i+1}] \times [t_j,t_{j+1}]$, and
  proceed row by row.

  The preservation of the properties (1) and (2) follows from the
  statement about constant paths in (\ref{path-lifting}).
\end{proof}

\begin{remark}
  Let's summarise what (\ref{path-lifting}) and
  (\ref{homotopy-lifting}) are saying. Given a path in $B$ beginning
  at $b$, there is a unique lift of this path to $E$ beginning at any
  $e \in F_b(E)$. Moreover, a homotopy of such paths in $B$ lifts to a
  homotopy of the lifts in $E$, and if the original homotopy keeps the
  endpoints fixed in $B$, the lift does so in $E$ as well.
\end{remark}

We've ended up talking about paths and homotopies of paths. Of course
you know where we're headed then! We now discuss what these lifting
properties can tell us about the relationship between fundamental
groups of coverings.

\begin{notation}
  For convenience we denote $\pi_1(B,b)$ by $G$, and if $p : (E,e) \to
  (B,b)$ is a covering we denote $\im(p_*) \subseteq G$ by $H(E,e)$.
\end{notation}

\begin{proposition}
  \label{cover-inj}
  Let $p : (E,e) \to (B,b)$ be a covering. Then $p_*$ is injective,
  and hence gives an isomorphism $\pi_1(E,e) \simeq H(E,e)$.
\end{proposition}

\begin{proof}
  Let $g = [\gamma] \in \pi_1(E,e)$. To say $g \in \ker(p_*)$ is to
  say there's a null-homotopy through loops (fixing endpoints) $h : I
  \times I \to X$ of $p \circ \gamma$. Suppose this is the case. By
  (\ref{homotopy-lifting}), this lifts to a null-homotopy through
  loops $H : I \times I \to Y$ of $\gamma$. Hence $g =
  \id_{\pi_1(E,e)}$. This implies $\ker(p_*)$ is trivial, as desired.
\end{proof}

\begin{nothing}
  \label{pi1-action}
  Let $p : E \to B$ be a covering. There is a canonical $G$-action on
  the fibre $F_b(E)$, defined as follows. Let $e \in F_b(E)$ and $g =
  [\gamma] \in G$. By (\ref{path-lifting}), $\gamma$ lifts uniquely to
  a path $\tilde\gamma : I \to E$ with $\tilde\gamma(0) = e$. Then
  $p(\tilde\gamma(1)) = \gamma(1) = b$, implying $\tilde\gamma(1) \in
  F_b(E)$. We define the action by $ge \coloneqq
  \tilde\gamma(1)$. Since a different representative loop for $g$
  differs from $\gamma$ by a homotopy through loops, in particular
  fixing endpoints, by (\ref{homotopy-lifting}) the definition of $ge$
  does not depend on the choice of representative $\gamma$.

  In particular, if $g = \id_G$ then we can take $\gamma$ to be the
  constant loop at $b$, which clearly lifts (uniquely) to
  $\tilde\gamma$ the constant loop at $e$. Thus $\id_G e =
  \tilde\gamma(1) = e$.

  Finally, consider another loop $h = [\eta] \in G$. Then $\eta$ lifts
  to a unique $\tilde\eta : I \to E$ satisfying $\tilde\eta(0) =
  ge$. By definition, $h(ge) = \tilde\eta(1)$. But note that the
  composition of loops $\tilde\eta \cdot \tilde\gamma$ is a lift of
  the composition $\eta \cdot \gamma$. Since $hg = [\eta \cdot
    \gamma]$, it follows that also $(hg)e = \tilde\eta(1)$. So indeed
  we have a well-defined $G$-action.
\end{nothing}

\begin{nothing}
  \label{fibre-map}
  Let $p : E \to B$ and $p' : E' \to B$ be coverings, and $f : E \to
  E'$ a morphism of coverings. Recall from (\ref{fibre}) that $f$
  restricts to give a map of sets $F_b(E) \to F_b(E')$. We claim this
  map is in fact $G$-equivariant. Let $e \in F_b(E)$ and $e' \coloneqq
  f(e) \in F_b(E')$. Let $g,\gamma,\tilde\gamma$ as in
  (\ref{pi1-action}) so that $ge = \tilde\gamma(1)$. Observe that $p'
  \circ f \circ \tilde\gamma = p \circ \tilde \gamma = \gamma$ since
  $f$ is a morphism of coverings, so that $f \circ \tilde\gamma$ is a
  lift of $\gamma$ at $e'$. Thus, by definition,
  \[
  gf(e) = ge' = f(\tilde\gamma(1)) = f(ge),
  \]
  proving equivariance.
\end{nothing}

\begin{definition}
  (\ref{pi1-action}) and (\ref{fibre-map}) define for us the
  \emph{fibre functor}
  \[
  F_b : \Cov(B) \to G\dash\Set.
  \]
\end{definition}

\begin{proposition}
  \label{pi1-action-props}
  Let $p : E \to B$ be a covering. The $G$-action on the fibre
  $F_b(E)$ satisfies the following:
  \begin{enumerate}
  \item \label{orbit} Two points $e,e' \in F_b(E)$ are in the same
    orbit if and only if they lie in the same connected component of
    $E$.
  \item \label{stab} The stabiliser subgroup of $e \in F_b(E)$ is
    $H(E,e)$.
  \end{enumerate}
\end{proposition}

\begin{proof}
  (1) By (\ref{covering-rmk}), $E$ is locally path connected, so by
  (\ref{locpathconn}) it suffices to show that $e,e' \in F_b(E)$ are
  in the same orbit if and only if there is a path between them. The
  ``if'' direction is immediate from the definition of the
  $G$-action. Conversely, suppose $\tilde\gamma : I \to E$ is path
  from $e$ to $e'$. Then $\gamma \coloneqq p \circ \tilde \gamma : I
  \to E$ is a loop at $b$ which lifts uniquely at $e$ to
  $\tilde\gamma$. Thus if we set $g \coloneqq [\gamma] \in G$ we have
  $ge = e'$, so $e,e'$ are in the same orbit.

  (2) Consider any $g = [\gamma] \in G$. By definition, we have $ge =
  e$ if and only if $\gamma$ lifts to a \emph{loop} $\tilde \gamma : I
  \to E$ at $e$, which is of course equivalent to $g$ lying in the
  image $H(E,e)$ of $p_*$.
\end{proof}

To end this section, we will finally prove the general lifting result
stated at the beginning of the section.

\begin{proof}[Proof of (\ref{general-lifting})]
  We first observe that the condition is certainly necessary: if we
  can find a lift $F$ then
  \[
  f = p \circ F \implies f_* = p_* \circ F_* \implies \im(f_*)
  \subseteq \im(p_*).
  \]
  So the content is in showing that the condition is sufficient. By
  (\ref{locpathconn}), $X$ is the disjoint union (topologically, not
  just set-theoretically) of its connected components, so it suffices
  to treat the case that $X$ is (path) connected.

  Consider any point $x' \in X$, and let $\gamma : I \to X$ be a path
  from the basepoint $x$ to $x'$. By (\ref{path-lifting}), there is a
  unique lift $\tilde\gamma : I \to E$ of $f \circ \gamma$ with
  $\tilde\gamma(0) = e$. If there is a lift $F : (X,x) \to (E,e)$ of
  $f$, then $F \circ \gamma$ is a lift of $f \circ \gamma$ at $e$,
  whence by uniqueness we have
  \begin{equation}
    \label{path-defn}
    F(x') = F(\gamma(1)) = \tilde\gamma(1).
  \end{equation}
  This determines $F$ uniquely, proving claim (2). We just need to
  show that this in fact gives a well-defined, continuous lift $F$.

  We first verify well-definedness. Suppose $\eta : I \to X$ is
  another choice of path from $x$ to $x'$. Write $\gamma^{-1}$ for the
  reverse path of $\gamma$ from $x'$ to $x$, so the composition $\rho
  \coloneqq \gamma^{-1} \cdot \eta$ is a loop at $x$. By our
  assumption we have
  \[
  g \coloneqq [f \circ \rho] = f_*[\rho] \in \im(f_*) \subseteq
  \im(p_*) = H(E,e).
  \]
  By (\ref{pi1-action-props}.\ref{stab}), this implies $ge = e$, i.e.,
  $f \circ \rho$ lifts to a loop $\tilde\rho$ at $e$. Let $\tilde\eta$
  and $\tilde\eta'$ denote the unique lifts of $f \circ \eta$ and $f
  \circ (\gamma \cdot \rho)$, respectively. Since $\eta$ and $\gamma
  \cdot \rho$ differ by a homotopy fixing endpoints, by
  (\ref{homotopy-lifting}) we have $\tilde\eta(1) =
  \tilde\eta'(1)$. Moreover since $\tilde\rho$ is a loop we have
  \[
  \tilde\eta' = \tilde\gamma \cdot \tilde\rho \implies \tilde\eta(1)=
  \tilde\eta'(1) = \tilde\gamma(1),
  \]
  so the definition (\ref{path-defn}) is independent of the choice of
  $\gamma$, as desired.

  To finish, we prove continuity. Let $V \subseteq E$ an open set and
  $x' \in F^{-1}(V)$. It suffices to find an open neighbourhood $W
  \subseteq X$ of $x'$ such that $F(W) \subseteq V$. We can replace
  $V$ with a smaller open set such that $U \coloneqq p(V)$ is open and
  evenly covered, so in particular $p : V \to U$ is a
  homeomorphism. Note $f(x') \in U$, so since $X$ is locally path
  connected, we can choose a path connected neighbourhood $W$ of $x'$
  such that $W \subseteq f^{-1}(U)$. Let $x'' \in W$, and choose a
  path $\eta : I \to X$ from $x'$ to $x''$. Let $\gamma : I \to X$ be
  a path from the basepoint $x$ to $x'$. Then the composition $\eta
  \cdot \gamma$ is a path from $x$ to $x''$. Let $\tilde\gamma,
  \tilde\eta : I \to E$ be the unique lifts of $f \circ \gamma$ and $f
  \circ \eta$ beginning at $e$ and $F(x')$, respectively. Since $f(W)
  \subseteq U$ and $p : V \to U$ is a homeomorphism, $\im(\tilde\eta)
  \subseteq V$. Now, $\tilde\eta \cdot \tilde\gamma$ is a lift of
  $\eta \cdot \gamma$ at $e$, so by the well-definedness just proved,
  we have
  \[
  F(x'') = (\tilde\eta \cdot \tilde\gamma)(1) = \tilde\eta(1) \in V,
  \]
  so $F(W) \subseteq V$, as desired.
\end{proof}

%%%%%%%%%%%%%%%%%%%%%%%%%%%%%%%%%%%%%%%%%%%%%%%%%%%%%%%%%%%%%%%%%%%%%%

\section{The universal cover}

%%%%%%%%%%%%%%%%%%%%%%%%%%%%%%%%%%%%%%%%%%%%%%%%%%%%%%%%%%%%%%%%%%%%%%

\section{Classifying coverings}

Now we're going to sit down and classify all of the coverings of
$X$. This stuff is beautiful, I hope I can convey that.

\begin{lemma}
  Let $p : (E,e) \to (X,x)$ and $p' : (E',e') \to (X,x)$
  coverings.
  \begin{enumerate}
  \item There is a morphism of coverings $f : (E,e) \to (E',e')$
    if and only if $H(E,e) \subseteq H(E',e')$.
  \item If $E$ is connected and the morphism $f$ does exist, then it
    is unique.
  \end{enumerate}
\end{lemma}

\begin{proof}
  This is immediate from (\ref{general-lifting}).
\end{proof}

%%%%%%%%%%%%%%%%%%%%%%%%%%%%%%%%%%%%%%%%%%%%%%%%%%%%%%%%%%%%%%%%%%%%%%

\nocite{may-concise, munkres}
\bibliographystyle{amsalpha}
\bibliography{refs}

\end{document}

%%%%%%%%%%%%%%%%%%%%%%%%%%%%%%%%%%%%%%%%%%%%%%%%%%%%%%%%%%%%%%%%%%%%%%

Outline

1. Motivation
2. Definitions
3. Universal cover
4. Lifting
5. Galois
   when do we have a morphism of connected coverings?
   equivalence of categories (construct inverse by quotienting
   disjoint union of universals)
   the connected case
Appendix: G-sets as products of quotients

