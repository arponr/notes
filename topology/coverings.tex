%%%%%%%%%%%%%%%%%%%%%%%%%%%%%%%%%%%%%%%%%%%%%%%%%%%%%%%%%%%%%%%%%%%%%%

\newcommand{\ob}{\oper{ob}}
\renewcommand{\hom}{\oper{hom}}
\newcommand{\id}{\oper{id}}
\newcommand{\im}{\oper{im}}
\newcommand{\op}{\oper{op}}

\newcommand{\Top}{\oper{Top}}
\newcommand{\Set}{\oper{Set}}
\newcommand{\Ab}{\oper{Ab}}
\newcommand{\Grp}{\oper{Grp}}
\newcommand{\Mod}{\oper{Mod}}
\newcommand{\Simplex}{\Delta}
\newcommand{\s}{\oper{s}}
\newcommand{\Ch}{\oper{Ch}}

\newcommand{\Sing}{\oper{Sing}}
\renewcommand{\H}{\mathrm{H}}

%%%%%%%%%%%%%%%%%%%%%%%%%%%%%%%%%%%%%%%%%%%%%%%%%%%%%%%%%%%%%%%%%%%%%%

%%%%%%%%%%%%%%%%%%%%%%%%%%%%%%%%%%%%%%%%%%%%%%%%%%%%%%%%%%%%%%%%%%%%%%

\newcommand{\ob}{\oper{ob}}
\renewcommand{\hom}{\oper{hom}}
\newcommand{\id}{\oper{id}}
\newcommand{\im}{\oper{im}}
\newcommand{\op}{\oper{op}}

\newcommand{\Top}{\oper{Top}}
\newcommand{\Set}{\oper{Set}}
\newcommand{\Ab}{\oper{Ab}}
\newcommand{\Grp}{\oper{Grp}}
\newcommand{\Mod}{\oper{Mod}}
\newcommand{\Simplex}{\Delta}
\newcommand{\s}{\oper{s}}
\newcommand{\Ch}{\oper{Ch}}

\newcommand{\Sing}{\oper{Sing}}
\renewcommand{\H}{\mathrm{H}}

%%%%%%%%%%%%%%%%%%%%%%%%%%%%%%%%%%%%%%%%%%%%%%%%%%%%%%%%%%%%%%%%%%%%%%


\title{Covering spaces}
\author{Arpon Raksit}
\date{November 22, 2013 (original); \today\ (last edit).}

\begin{document}
\maketitle
\thispagestyle{fancy}

%%%%%%%%%%%%%%%%%%%%%%%%%%%%%%%%%%%%%%%%%%%%%%%%%%%%%%%%%%%%%%%%%%%%%%

\section{Setting up}

\begin{notation}
  For the remainder we fix a nice pointed space $(X,x)$. Sometimes we
  won't use the basepoint, and sometimes we will. Here \emph{nice}
  means that $X$ is connected, locally path connected, and semilocally
  simply connected.
\end{notation}

\begin{definition}
  \label{covering}
  A map $p : E \to X$ of spaces (or $p : (E,e) \to (X,x)$ of pointed
  spaces) is a \emph{covering} if each $x \in X$ has an open
  neighbourhood $U$ such that $p^{-1}(U) = \coprod_{j \in J} V_j$,
  where $J \ne \emptyset$ and for each $j \in J$:
  \begin{enumerate}
  \item $V_j \subseteq E$ is open,
  \item the restriction $p : V_j \to U$ is a homeomorphism.
  \end{enumerate}
\end{definition}

\begin{remarks}
  (\ref{covering}) is just how one formalises the idea that $E$
  locally looks like a bunch of copies of $X$. Note we'll often be
  abusive and refer to $E$ as a covering, leaving the map $p$
  implicit.
\end{remarks}

\newcommand{\Cov}{\operatorname{Cov}}
\newcommand{\Gal}{\operatorname{Gal}}

\begin{definition}
  We define $\Cov(X)$, the \emph{category of coverings of $X$}, as
  follows.
  \begin{enumerate}
  \item An object of $\Cov(X)$ is a covering $p : E \to X$.
  \item A \emph{morphism of coverings} from $p : E \to X$ to $p' : E'
    \to X$ is given by a map $f : E \to E'$ such that $p'f = p$, i.e.,
    a commutative diagram
    \[
    \begin{tikzcd}
      E \ar{rr}{f} \drar[swap]{p} & & E'. \dlar{p'} \\ & X &
    \end{tikzcd}
    \]
  \end{enumerate}
  Note, if we have pointed coverings $p : (E,e) \to (X,x)$ and $p' :
  (E',e') \to (X,x)$, we will call $f : (E,e) \to (E',e')$ a morphism
  of coverings simply if the underlying $f : E \to E'$ is a morphism
  of coverings as defined above.
\end{definition}

\begin{notation}
  We fix the following for the remainder.
  \begin{enumerate}
  \item Denote the interval $[0,1]$ by $I$.
  \item Denote by $\pi_1(Y,y)$ the fundamental group of a pointed
    space $(Y,y)$.
  \item If we write an element $g \in \pi_1(Y,y)$ as $[\gamma]$, we
    mean that $g$ is the homotopy class of the loop $\gamma : I \to Y$
    based at $y$.
  \item Denote by $f_* : \pi_1(Y,y) \to \pi_1(Y',y')$ the group
    homomorphism induced by a map of pointed spaces $f : (Y,y) \to
    (Y',y')$.
  \item Denote $\pi_1(X,x)$ by $G$.
  \end{enumerate}
\end{notation}

%%%%%%%%%%%%%%%%%%%%%%%%%%%%%%%%%%%%%%%%%%%%%%%%%%%%%%%%%%%%%%%%%%%%%%

\section{Lifting things}

\begin{proposition}
  \label{general-lifting}
  Let $p : (E,e) \to (X,x)$ a covering. Let $f : (Y,y) \to (X,x)$ a
  map of pointed spaces.
  \begin{enumerate}
  \item There is a lift $f; : (Y,y) \to (E,e)$,
    making the diagram
    \[
    \begin{tikzcd}
      \ & (E,e) \dar{p} \\ (Y,y) \rar{f} \ar[dashed]{ur}{f'} & (X,x)
    \end{tikzcd}
    \]
    commute, if and only if $f_*\pi_1(Y,y) \subseteq p_*\pi_1(E,e)$.
  \item If $Y$ is connected and the lift $f'$ does exist, then it is
    unique.
  \end{enumerate}
\end{proposition}

%%%%%%%%%%%%%%%%%%%%%%%%%%%%%%%%%%%%%%%%%%%%%%%%%%%%%%%%%%%%%%%%%%%%%%

\section{The fibre functor}

One of the mottos in this theory, which will surely become clear, is
that everything is determined by the fibres.

\begin{definition}
  Let $p : E \to X$ a covering. We denote the \emph{fibre} $p^{-1}(x)$
  by $F_x(E)$.
\end{definition}

\begin{lemma}
  Let $p : E \to X$ a covering. There is a canonical $G$-action on
  $F_x(E)$.
\end{lemma}

\begin{proof}
  Let $e \in F_x(E)$ and $g = [\gamma] \in G$. Then by a lifting
  lemma, $\gamma$ lifts uniquely to a path $\tilde\gamma : I \to E$
  with $\gamma(0) = e$. Then $p\tilde\gamma(1) = \gamma(1) = x
  \implies \tilde\gamma(1) \in F_x(E)$. We define the action by $g
  \cdot e \coloneqq \tilde\gamma(1)$.

  \medskip
  \textbf{Check that this is independent of the loop
    representative. This should probably be a lemma in the Lifting
    things section}.

  \medskip
  Now if $g = \id_G$ then we can take $\gamma$ to be the constant loop
  at $x$, which clearly lifts to $\tilde\gamma$ the constant loop at
  $e$. Thus $\id_G \cdot e = \tilde\gamma(1) = e$.

  \medskip
  Finally let $h = [\delta] \in G$. Then $\delta$ lifts uniquely to
  $\tilde\delta : I \to E$ with $\tilde\delta(0) = g \cdot e$. By
  definition, $h \cdot (g \cdot e) = \tilde\delta(1)$. But note that
  the composition of $\tilde\delta \cdot \tilde\gamma$ is a lift of
  the composition $\delta \cdot \gamma$. Since $hg = [\delta \cdot
    \gamma]$, it follows that also $hg \cdot e =
  \tilde\delta(1)$. Thus we indeed have a $G$-action.
\end{proof}

Now let $p : E' \to X$ another covering, and $f : E \to E'$ a
morphism of coverings. By definition, $f$ restricts to a set map
$F_x(f) : F_x(E) \to F_x(E')$. Let $e,g,\gamma,\gamma'$ as
above. Then $f\gamma' : I \to E'$ is a lift of $\gamma$ at
$f(e)$. So by uniqueness of lifting we must have $gf(e) =
f(ge)$. I.e., $F_x(f)$ is a morphism of $G$-sets.

\medskip
This gives us the \emph{fibre functor} $F_x : \Cov(X) \to
G\text{-}\Set$.

%%%%%%%%%%%%%%%%%%%%%%%%%%%%%%%%%%%%%%%%%%%%%%%%%%%%%%%%%%%%%%%%%%%%%%

\section{The universal cover}

%%%%%%%%%%%%%%%%%%%%%%%%%%%%%%%%%%%%%%%%%%%%%%%%%%%%%%%%%%%%%%%%%%%%%%

\section{Classifying coverings}

Now we're going to sit down and classify all of the coverings of
$X$. This stuff is beautiful, I hope I can convey that.

\begin{lemma}
  \label{cover-injection}
  Let $p : (E,e) \to (X,x)$ a covering. Then $p_* : \pi_1(E,e) \to
  \pi_1(X,x)$ is injective.
\end{lemma}

\begin{proof}
  Let $[\gamma] \in \pi_1(E,e)$ the homotopy class of a loop $\gamma :
  I \to Y$ at $y$. By definition, $p_*[\gamma] = [p\gamma]$. Suppose
  $[\gamma] \in \ker(p_*)$. Then there's a null-homotopy (through
  loops) $h : I \times I \to X$ of $p\gamma$. But by a lifting lemma,
  this lifts to $h' : I \times I \to Y$ with $h'(0,0) = y$ and, for $t
  \in I$,
  \begin{align*}
    ph'(0,t) = h(0,t) = x, \quad &ph'(1,t) = h(1,t) = x, \\ ph'(t,0) =
    h(t,0) = p\gamma(t), \quad &ph'(t,1) = h(t,1) = x.
  \end{align*}
  By uniqueness of lifting this implies $h'(0,t) = h'(1,t) = h'(t,1) =
  y$ and $h'(t,0) = \gamma(t)$ for $t \in I$. I.e., $h'$ is a
  null-homotopy of $\gamma$, and hence $[\gamma] =
  \id_{\pi_1(Y,y)}$. Thus $\ker(p_*)$ is trivial, implying the claim.
\end{proof}

\begin{notation}
  Let $p : (E,e) \to (X,x)$ is a covering. We will use the following
  notation for the remainder:
  \begin{enumerate}
  \item
  \item $H(E,e) \coloneqq p_*\pi_1(E,e) \subseteq G$.
  \end{enumerate}
  Note that (\ref{cover-injection}) says that $H(E,e) \simeq
  \pi_1(E,e)$.
\end{notation}

\begin{lemma}
  Let $p : (E,e) \to (X,x)$ and $p' : (E',e') \to (X,x)$
  coverings.
  \begin{enumerate}
  \item There is a morphism of coverings $f : (E,e) \to (E',e')$
    if and only if $H(E,e) \subseteq H(E',e')$.
  \item If $E$ is connected and the morphism $f$ does exist, then it
    is unique.
  \end{enumerate}
\end{lemma}

\begin{proof}
  This is immediate from (\ref{general-lifting}).
\end{proof}



%%%%%%%%%%%%%%%%%%%%%%%%%%%%%%%%%%%%%%%%%%%%%%%%%%%%%%%%%%%%%%%%%%%%%%

\bibliographystyle{amsplain}
\bibliography{refs}

\end{document}

%%%%%%%%%%%%%%%%%%%%%%%%%%%%%%%%%%%%%%%%%%%%%%%%%%%%%%%%%%%%%%%%%%%%%%

Outline

1. Motivation
2. Definitions
3. Universal cover
4. Lifting
5. Galois
   when do we have a morphism of connected coverings?
   equivalence of categories (construct inverse by quotienting
   disjoint union of universals)
   the connected case
Appendix: G-sets as products of quotients

