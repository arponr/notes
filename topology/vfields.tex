%%%%%%%%%%%%%%%%%%%%%%%%%%%%%%%%%%%%%%%%%%%%%%%%%%%%%%%%%%%%%%%%%%%%%%

\newcommand{\ob}{\oper{ob}}
\renewcommand{\hom}{\oper{hom}}
\newcommand{\id}{\oper{id}}
\newcommand{\im}{\oper{im}}
\newcommand{\op}{\oper{op}}

\newcommand{\Top}{\oper{Top}}
\newcommand{\Set}{\oper{Set}}
\newcommand{\Ab}{\oper{Ab}}
\newcommand{\Grp}{\oper{Grp}}
\newcommand{\Mod}{\oper{Mod}}
\newcommand{\Simplex}{\Delta}
\newcommand{\s}{\oper{s}}
\newcommand{\Ch}{\oper{Ch}}

\newcommand{\Sing}{\oper{Sing}}
\renewcommand{\H}{\mathrm{H}}

%%%%%%%%%%%%%%%%%%%%%%%%%%%%%%%%%%%%%%%%%%%%%%%%%%%%%%%%%%%%%%%%%%%%%%

%%%%%%%%%%%%%%%%%%%%%%%%%%%%%%%%%%%%%%%%%%%%%%%%%%%%%%%%%%%%%%%%%%%%%%

\newcommand{\ob}{\oper{ob}}
\renewcommand{\hom}{\oper{hom}}
\newcommand{\id}{\oper{id}}
\newcommand{\im}{\oper{im}}
\newcommand{\op}{\oper{op}}

\newcommand{\Top}{\oper{Top}}
\newcommand{\Set}{\oper{Set}}
\newcommand{\Ab}{\oper{Ab}}
\newcommand{\Grp}{\oper{Grp}}
\newcommand{\Mod}{\oper{Mod}}
\newcommand{\Simplex}{\Delta}
\newcommand{\s}{\oper{s}}
\newcommand{\Ch}{\oper{Ch}}

\newcommand{\Sing}{\oper{Sing}}
\renewcommand{\H}{\mathrm{H}}

%%%%%%%%%%%%%%%%%%%%%%%%%%%%%%%%%%%%%%%%%%%%%%%%%%%%%%%%%%%%%%%%%%%%%%


%%%%%%%%%%%%%%%%%%%%%%%%%%%%%%%%%%%%%%%%%%%%%%%%%%%%%%%%%%%%%%%%%%%%%%

\title{Vector fields and the J-homomorphism}
\author{Arpon Raksit}
\date{original: May 7, 2014; last edit: \today}

\begin{document}
\maketitle
\thispagestyle{fancy}

%%%%%%%%%%%%%%%%%%%%%%%%%%%%%%%%%%%%%%%%%%%%%%%%%%%%%%%%%%%%%%%%%%%%%%

\section{The problem}

For the sake of completeness, we state the basic definitions.

\begin{notation}
  Throughout we have $k,n \in \N$ with $n \ge 2$.
\end{notation}

\begin{definitions}
  \label{vfield-dfn}
  Let $M$ be a differentiable manifold. Let $\pi \c T \to M$ be the
  tangent bundle on $M$.
  \begin{enumerate}
  \item A \emph{vector field} on $M$ is a continuous section $v \c M
    \to T$ of $\pi$.
  \item A set of vector fields $\{v_1,\ldots,v_k\}$ on $M$ is
    \emph{linearly independent} if for each $p \in M$ the vectors
    $v_1(p),\ldots,v_k(p)$ are linearly independent in the tangent
    space $T_p$. In particular a single vector field $v$ forms a
    linearly independent set if and only if it is nowhere vanishing.
  \end{enumerate}
\end{definitions}

Now the actual story: we are taught to love spheres from our very
first days in the land of topology. But perhaps it is the following
result---or perhaps really its title---which first truly beguiles us.

\begin{theorem}[Hairy ball]
  \label{hairy-ball}
  The sphere $\S^{n-1}$ admits a nowhere vanishing vector field if and
  only if $n$ is even.
\end{theorem}

Of course, so enticed, we cannot just leave it there. We must ask the
following.

\begin{question}
  \label{vfield-prob}
  Then how many vector fields does $\S^{n-1}$ admit? Or more
  precisely, what is the maximum size of a set of linearly independent
  vector fields on $\S^{n-1}$?
\end{question}

This is one of those problems that occupied people for a while. To get
a lower bound on the problem is not so hard (especially with
hindsight). The theory of Clifford algebras gives the following result
(see, e.g., \cite{hopkins-256y, miller-vfields}).

\begin{definition}
  \label{radon-hurwitz}
  Write $n = 2^ab$ with $b$ odd, and write $a = 4c + d$ with $0 \le d
  \le 3$. Then we define the \emph{Radon-Hurwitz number} $\rho(n) \ce
  2^d + 8c$. Note in particular that $\rho(n) = 1$ if $n$ is odd.
\end{definition}

\begin{theorem}
  \label{vfield-lower-bound}
  There exists a set of $\rho(n)-1$ linearly independent vector fields
  on $\S^{n-1}$.
\end{theorem}

Getting an upper bound is where the real difficulty is. Well we know
one upper bound: $\S^{n-1}$ certainly can't admit $n$ linearly
independent vector fields since $\dim \S^{n-1} = n - 1$. And to say
$\S^{n-1}$ admits $n-1$ linearly independent vector fields is to say
$\S^{n-1}$ is parallelisable, which famously is true if and only if $n
\in \{2,4,8\}$. An optimal upper bound would at least tell us this
parallelisability result. So let's think about it for a second and
reduce the problem to one more attackable by algebra.

\begin{nothing}
  \label{gram-schmidt}
  First of all, we have our very nice embedding $\S^{n-1} \inj \R^n$,
  which gives the tangent spaces of $\S^{n-1}$ a very concrete
  description. In particular, a vector field on $\S^{n-1}$ is just a
  map $v \c \S^{n-1} \to \R^n$ such that $v(x) \perp x$ (in $\R^n$)
  for all $x \in \S^{n-1}$. Now, by Gram-Schmidt, giving $k$ linearly
  independent vector fields $v_1,\ldots,v_k \c \S^{n-1} \to \R^n$ is
  equivalent to giving $k$ (pointwise) orthonormal vector fields. This
  leads us to recall the following definition.
\end{nothing}

\begin{definition}
  \label{stiefel}
  Let $1 \le l \le n$. The \emph{Stiefel manifold} $\V_{l,n}$ is the
  space
  \[
  \{(v_1,\ldots,v_l) \mid v_i \in \S^{n-1}, \langle v_i, v_j \rangle =
  \delta_{i,j}\}
  \]
  of orthonormal $l$-frames on $\R^n$.
\end{definition}

\begin{lemma}
  Let $\pi_k \c \V_{k+1,n} \to \S^{n-1}$ be the projection
  $(v_1,\ldots,v_{k+1}) \mapsto v_{k+1}$. Then $\S^{n-1}$ admits a set
  of $k$ linearly independent vector fields if and only if there is a
  section $\S^{n-1} \to \V_{k+1,n}$ of $\pi_k$.
\end{lemma}

\begin{proof}
  This is immediate from the discussion in (\ref{gram-schmidt}).
\end{proof}

So we've reduced our problem to the existence of some map. Already one
can imagine using algebra to get at the problem now. E.g., for what
$k$ can this map exist in singular homology or cohomology?  This was
the strategy of Steenrod and Whitehead \cite{steenrod-vfields}, who
achieve an upperbound of $2^a$, in the notation of
(\ref{radon-hurwitz}). Of course this result doesn't tell us that
$\S^{15}$ is not parallelisable, and leaves a large gap from the lower
bound (\ref{vfield-lower-bound}). This gap was closed by Adams, who
employed $\K$-theory instead to show the lower bound
(\ref{vfield-lower-bound}) is in fact optimal.

\begin{theorem}[\cite{adams-vfields}]
  \label{vfield-upper-bound}
  There does not exist a set of $\rho(n)$ linearly independent vector
  fields on $\S^{n-1}$.
\end{theorem}

Our goal for the remainder is to prove this theorem, and hence solve
our problem. Following \cite{miller-vfields}, the argument we give is
not in the original form presented in \cite{adams-vfields}, but rather
a consequence of Adams's later work on bounding the image of the
$J$-homomorphism \cite{adams-J-II}.

%%%%%%%%%%%%%%%%%%%%%%%%%%%%%%%%%%%%%%%%%%%%%%%%%%%%%%%%%%%%%%%%%%%%%%

\bibliographystyle{amsalpha}
\bibliography{refs}

\end{document}
