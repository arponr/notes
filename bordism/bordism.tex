%%%%%%%%%%%%%%%%%%%%%%%%%%%%%%%%%%%%%%%%%%%%%%%%%%%%%%%%%%%%%%%%%%%%%%

\newcommand{\ob}{\oper{ob}}
\renewcommand{\hom}{\oper{hom}}
\newcommand{\id}{\oper{id}}
\newcommand{\im}{\oper{im}}
\newcommand{\op}{\oper{op}}

\newcommand{\Top}{\oper{Top}}
\newcommand{\Set}{\oper{Set}}
\newcommand{\Ab}{\oper{Ab}}
\newcommand{\Grp}{\oper{Grp}}
\newcommand{\Mod}{\oper{Mod}}
\newcommand{\Simplex}{\Delta}
\newcommand{\s}{\oper{s}}
\newcommand{\Ch}{\oper{Ch}}

\newcommand{\Sing}{\oper{Sing}}
\renewcommand{\H}{\mathrm{H}}

%%%%%%%%%%%%%%%%%%%%%%%%%%%%%%%%%%%%%%%%%%%%%%%%%%%%%%%%%%%%%%%%%%%%%%

%%%%%%%%%%%%%%%%%%%%%%%%%%%%%%%%%%%%%%%%%%%%%%%%%%%%%%%%%%%%%%%%%%%%%%

\newcommand{\ob}{\oper{ob}}
\renewcommand{\hom}{\oper{hom}}
\newcommand{\id}{\oper{id}}
\newcommand{\im}{\oper{im}}
\newcommand{\op}{\oper{op}}

\newcommand{\Top}{\oper{Top}}
\newcommand{\Set}{\oper{Set}}
\newcommand{\Ab}{\oper{Ab}}
\newcommand{\Grp}{\oper{Grp}}
\newcommand{\Mod}{\oper{Mod}}
\newcommand{\Simplex}{\Delta}
\newcommand{\s}{\oper{s}}
\newcommand{\Ch}{\oper{Ch}}

\newcommand{\Sing}{\oper{Sing}}
\renewcommand{\H}{\mathrm{H}}

%%%%%%%%%%%%%%%%%%%%%%%%%%%%%%%%%%%%%%%%%%%%%%%%%%%%%%%%%%%%%%%%%%%%%%


%--------------------------------------------------------------------%

\title{Bordism}
\author{Arpon Raksit}
\date{June 12, 2014}

\begin{document}
\maketitle
\thispagestyle{fancy}

%--------------------------------------------------------------------%

\renewcommand{\d}{\partial}
\newcommand{\e}{\emptyset}
\newcommand{\Mfd}{\mathrm{Mfd}}
\renewcommand{\C}{\mathcal{C}}

%--------------------------------------------------------------------%

\section{Generalities}

\begin{definition}
  A \emph{bordism category} is a category $\C$ equipped with a functor
  $\d \c \C \to \C$ and natural transformation $\iota \c \d \to
  \id_\C$ such that:
  \begin{enumerate}
  \item $\C$ has finite coproducts $M \amalg N$, including an initial
    object $\e$;
  \item $\C$ is essentially small, i.e., has a small subcategory
    $\C_0$ such that every $M \in \C$ is isomorphic to some $M_0 \in
    \C_0$;
  \item $\d$ and $\iota$ preserve coproducts;
  \item $\d\e \simeq \e$ and $\d\d M \simeq \e$ for all $M \in \C$.
  \end{enumerate}
\end{definition}

\begin{example} \label{main-example}
  Hopefully the notation is suggestive enough so that this motivating
  example is predictable. We take $\C$ to be the category $\Mfd$ of
  compact smooth manifolds (with boundary) and smooth maps, $\d$ to be
  the boundary, and $\iota$ to be the inclusion of the
  boundary. Finite coproducts in $\Mfd$ are given by disjoint union,
  and $\e$ is the empty manifold. Note, $\Mfd$ is essentially small
  because every object can be embedded as a subset of $\R^\infty$.
\end{example}

In the remainder of this section we state some elementary facts about
a general bordism category $\C$. Then in the next section we will
address this main example (although in slightly more generality, by
letting our manifolds have extra structure and working over a base
space). We will in particular discuss at the end why this definition
of bordism is equivalent to the more familiar one for manifolds.

\begin{definitions}
  \begin{itemize}[leftmargin=*]
  \item We say $M,N \in \C$ are \emph{bordant}, and write $M \equiv
    N$, if there exist $U,V \in \C$ such that $M \amalg \d U \simeq N
    \amalg \d V$.
  \item The relation $\equiv$ is called \emph{bordism}.
  \item We say $M \in \C$ is \emph{closed} if $\d M \simeq \e$ and a
    \emph{boundary} if $M \equiv \e$.
  \end{itemize}
\end{definitions}

\begin{proposition}
  \label{bordism-props}
  Let $K,L,M,N \in \C$.
  \begin{enumerate}
  \item Bordism is an equivalence relation, whose equivalence classes
    form a set.
  \item $M \equiv N \implies \d M \simeq \d N$.
  \item $\d M \equiv \e$.
  \item \label{bordism-sum} $K \equiv M, L \equiv N \implies K \amalg
    L \equiv M \amalg N$.
  \item If $M \equiv N$ then $M$ is closed (resp. a boundary) if and
    only if $N$ is closed (resp. a boundary).
  \item \label{closed-sum} If $M,N$ are closed (resp. boundaries) then
    $M \amalg N$ is closed (resp. a boundary).
  \item If $M$ is a boundary then $M$ is closed.
  \end{enumerate}
\end{proposition}

\begin{proof}
  Easy. (Note that we required $\C$ to be essentially small precisely
  so that the bordism classes form a set.)
\end{proof}

\begin{definition}
  Let $\Omega(\C)$ denote the set of bordism classes of closed objects
  of $\C$. By \pref{bordism-sum} and \pref{closed-sum}, coproduct in
  $\C$ induces a commutative monoid structure on $\Omega(\C)$, with
  identity given by $\e$.
\end{definition}

%--------------------------------------------------------------------%

\section{Bordism of (structured) manifolds}

\begin{definition}
  \label{path-homotopy}
  Suppose we have a space $U$, two points $u,v \in U$, and two paths
  $p,q \c u \doubto v$. Recall that a \emph{homotopy of paths} $h \c p
  \to q$ is a map $h \c I \times I \to U$ such that
  \[
  h(0,t) = p(t) \qquad
  h(1,t) = q(t) \qquad
  h(t,0) = u \qquad
  h(t,1) = v
  \]
  for all $t \in I$. Applying this to a mapping space $U = \map(X,Y)$
  gives a notion of \emph{homotopy of homotopies}.
\end{definition}

\begin{definitions}
  \label{map-f-struct}
  Let $f \c Y \to X$ and $g \c Z \to X$ be two maps. A \emph{homotopy
    lift} of $g$ along $f$ is a pair $(\til g, h)$ where $\til g$ is a
  map $Z \to Y$ and $h$ is a homotopy $f\til g \to g$. In other words,
  this is the data of a homotopy-commutative diagram
  \[
  \begin{tikzcd}
    & Y \ar[d, "f"] \\ Z \ar[ur, "\til g"] \ar[r, "g"] & X.
  \end{tikzcd}
  \]
  We say two homotopy lifts $(\til g_1, h_1)$ and $(\til g_2, h_2)$
  are \emph{equivalent} if there is a homotopy $\til h \c \til g_1 \to
  \til g_2$ such that the diagram of homotopies
  \begin{equation}
    \label{lift-equiv}
    \begin{tikzcd}
      f\til g_1 \ar[rr, "f\til h"] \ar[dr, swap, "h_1"] & & f\til g_2
      \\ & g \ar[ur, swap, "h_2^{-1}"] &
    \end{tikzcd}
  \end{equation}
  is homotopy-commutative, i.e. there is a homotopy of homotopies
  $h_2^{-1}h_1 \to f\til h$, as defined in \pref{path-homotopy}. An
  equivalence class of homotopy lifts of $g$ along $f$ is called an
  \emph{$f$-structure} on $g$.
\end{definitions}

\begin{remarks}
  Suppose in \pref{map-f-struct} that $f$ is a fibration. We may then
  make two additional observations:
  \begin{enumerate}
  \item
    \label{strict-homotopy}
    Suppose $k \c h_2^{-1}h_1 \to f\til h$ is a homotopy
    exhibiting the equivalence of homotopy lifts $(\til g_1, h_1)$ and
    $(\til g_2, h_2)$. Then $k$ can be lifted to a homotopy $\til k \c
    \til h' \to \til h$ of homotopies $\til g_1 \doubto \til g_2$,
    with $f\til h' = h_2^{-1}h_1$. By replacing $\til h$ with $\til
    h'$, we may always make the diagram \pref{lift-equiv} commute on
    the nose.
  \item
    \label{strict-lift}
    Suppose given a homotopy lift $(\til g_1, h_1)$ of $g$ along
    $f$. Since $f$ is a fibration, the homotopy $h_1 \c f\til g_1 \to
    g$ can be lifted to a homotopy $\til h \c \til g_1 \to \til
    g_2$. If we set $h_2 \c f\til g_2 = g \to g$ to be a the constant
    homotopy, then by definition $(\til g_1, h_1)$ and $(\til g_2,
    h_2)$ will be equivalent. Thus, any equivalence class of homotopy
    lifts contains an element $(\til g, h)$ where $\til g$ lifts $g$
    strictly and $h$ is the constant homotopy.

    Now suppose $(\til g_1, h_1)$ and $(\til g_2, h_2)$ are two such
    strict lifts (and constant homotopies). By the first observation,
    they are equivalent if and only if there is a homotopy $\til h \c
    \til g_1 \to \til g_2$ such that $f\til h$ is a constant homotopy,
    that is, $\til h$ is a fibrewise homotopy. So finally, when $f$ is
    a fibration, we may redefine an $f$-structure on $g$ to be a
    fibrewise homotopy class of strict lifts $\til g$ of $g$ along
    $f$.
  \end{enumerate}
\end{remarks}

\newcommand{\st}{\mathrm{st}}
\begin{notation}
  For $M$ a compact smooth manifold of dimension $d$:
  \begin{itemize}
  \item Let $\T M$ denote the tangent bundle of $M$.
  \item Let $\tau_M \c M \to \B\O(d)$ be a fixed map classifying $\T
    M$.
  \item Let $\tau_M^\st$ denote the composition $M \lblto{\tau_M}
    \B\O(d) \to \B\O$, classifying $\T M$ stably.
  \end{itemize}

\end{notation}

\begin{definition}
  \label{mfd-f-struc}
  Let $f \c B \to \B\O$ be any map. An \emph{$f$-structure} on a
  compact smooth manifold $M$ is an $f$-structure on its stable
  tangent bundle $\tau^\st_M \c M \to \B\O$, as defined in
  \pref{map-f-struct}.
\end{definition}

% \begin{definition} \label{normal-bundle} Let $i \c M \to N$ be an
%   immersion of smooth manifolds. The \emph{normal bundle} $\nu_i$ of
%   $i$ (or of $M$ in $N$) is the vector bundle on $M$ defined by the
%   short exact sequence $0 \to TM \to i^*TN \to \nu_i \to 0$.
% \end{definition}

% \begin{proposition} \label{normal-composition}
%   Let $i \c L \to M$ and $j \c M \to N$ be immersions of smooth
%   manifolds. Let $k \ce j \circ i$. Then there is a short exact
%   sequence $0 \to \nu_i \to \nu_k \to i^*\nu_j \to 0$ of bundles on
%   $L$.
% \end{proposition}

% \begin{proof}
%   Since pullback of vector bundles is exact we have
%   \[
%   i^*\nu_j \simeq i^*\l(\f{j^*TN}{TM}\r) \simeq \f{k^*TN}{i^*TM} \simeq
%   \f{k^*TN/TL}{i^*TM/TL} \simeq \f{\nu_k}{\nu_i}. \qedhere
%   \]
% \end{proof}

% \begin{definition} \label{stable-normal}
%   Let $M$ be a compact smooth manifold. let $\xi \in [M, BO]$ be the
%   homotopy class classifying the tangent bundle $TM \to M$
%   (stably). The \emph{stable normal bundle} $\nu_M$ of $M$ is defined
%   as the inverse of $\xi$ in the group $[M, BO] \simeq \til{KO}(M)$.

%   Note if $i \c M \to \R^n$ is an immersion (which always exists, as
%   remarked in \pref{main-example}) then by definition we have a short
%   exact sequence
%   \[
%   0 \to TM \to i^*T\R^n \to \nu_i \to 0.
%   \]
%   Since $T\R^n$ is trivial it follows that $\nu_i$ is inverse to $TM$
%   in $\til{KO}(M)$. Thus the stable classifying map $M \to BO$ of
%   $\nu_i$ determines the homotopy class of the stable normal bundle
%   $\nu_M$ (hence the terminology). But note that, by definition,
%   $\nu_M$ does not depend on the choice of dimension $n$ and immersion
%   $i$.
% \end{definition}

% \begin{definitions} \label{f-structure} Let $f \c B \to BO$ be a
%   fibration. Let $\xi$ be a vector bundle on a compact smooth manifold
%   $M$. Let $g \c M \to BO$ be a map classifying $\xi$. An
%   \emph{$f$-structure} on $\xi$ is the fibrewise homotopy\footnote{A
%     \emph{fibrewise homotopy} of lifts $\til g, \til g'$ is a homotopy
%     $h_t \c M \to B$ such that $f \circ h_t = g$ for all $t \in
%     [0,1]$.}  class $[\til g]$ of a lift in the diagram
%   \begin{equation} \label{f-structure-lift}
%     \begin{tikzcd}
%       \ & B \ar{d}{f} \\ M \ar[dashed]{ur}{\til g} \ar{r}{g} & BO.
%     \end{tikzcd}
%   \end{equation}
%   The $f$-structure only depends on the homotopy class $[g]$, or
%   equivalently on the stable bundle determined by $\xi$. This will be
%   explained below in \pref{f-structure-explanation} after we state some
%   examples to keep in mind.

%   An \emph{$f$-structure} on a compact smooth manifold $M$ is an
%   $f$-structure on its stable normal bundle $\nu_M$.
% \end{definitions}

\begin{examples}
  \label{f-examples}
  Why should we care about lifting against maps $B \to \B\O$? Well,
  maybe the primary class of examples will answer that. Whenever we
  have a morphism of topological groups $\phi \c G \to \O$ we get a
  fibration $f \ce \B\phi \c \B G \to \B\O$. In particular if we have
  a sequence of morphisms
  \[
  \begin{tikzcd}
    G(d_0) \ar[r] \ar[d] & G(d_1) \ar[r] \ar[d] & G(d_2) \ar[r] \ar[d]
    & \cdots \ar[d] \\ \O(d_0) \ar[r] & \O(d_1) \ar[r] & \O(d_2)
    \ar[r] & \cdots
  \end{tikzcd}
  \]
  for some increasing cofinal sequence $\{d_n\} \subseteq \Z_{\ge 0}$,
  we get a morphism
  \[
  \colim_n G(d_n) \ec G \to \O \ce \colim_n \O(n).
  \]
  
  In this situation, we think of a map $g \c Z \to \B\O$ as
  classifying a stable bundle on $Z$ (assuming $Z$ is, say,
  paracompact). Then a homotopy lift of $g$ along $f = \B\phi$ is
  precisely a reduction of structure group from $\O$ to $G$, and
  equivalence of homotopy lifts is precisely equivalence of
  reductions. Accordingly, in this situation we will abusively refer
  to $f$-structures as \emph{$G$-structures}. Here are some important
  subexamples:

  \newcommand{\SO}{\mathrm{SO}}
  \begin{enumerate}
  \item The most basic case comes from the identity $\id \c \O \to
    \O$, inducing $f = \id \c \B\O \to \B\O$. It follows from
    \pref{strict-lift} that any map $Z \to \B\O$, in particular
    $\tau^\st_M \c M \to \B\O$, has a unique $\O$-structure.
  \item Less trivially we have the inclusion $\SO \to \O$ coming from
    the inclusions $\SO(d) \to \O(d)$. This induces the two-to-one
    cover $f \c \B\SO \to \B\O$ (which arises from a choice of
    orientation on linear subspaces when we construct $\B\SO$ and
    $\B\O$ via the grassmanians).

    If $Z$ is compact, any $g \c Z \to \B\O$ factors as $Z \lblto{g_d}
    \B\O(d) \to \B\O$ for some $d \in \Z_{\ge 0}$. Then a strict lift
    of $g$ along $f$ is equivalent to a strict lift of $g_d$ along
    $f_d \c \B\SO(d) \to \B\O(d)$, since $f_d \c \B\SO(d) \to \B\O(d)$
    is just the pullback of $f$ along the inclusion $\B\O(d) \to
    \B\O$. Similarly, a fibrewise homotopy of strict lifts of $g$ will
    factor through some finite stage $f_{d+k} \c \B\SO(d+k) \to
    \B\O(d+k)$. It follows that an $\SO$-structure on $g$ is
    equivalent to an equivalence class of reductions of structure
    group to $\SO(d)$ of the $d$-dimensional bundle $\xi$ classified
    by $g_d$. This in turn is well known to be equivalent to orienting
    $\xi$.

    Since orienting a compact smooth manifold $M$ is equivalent to
    orienting its tangent bundle $\T M$, applying the above discussion
    to $\tau^\st_M$ shows that giving an $\SO$-structure on $M$ is
    equivalent to giving an orientation on $M$.
  \item The inclusions $\U(d) \to \O(2d)$ determine a morphism $\U \to
    \O$ and hence a fibration $\B\U \to \B\O$. A $\U$-structure on a
    manifold is called a \emph{stable almost-complex structure}.
  \end{enumerate}
\end{examples}

I think it's worth discussing for a bit why \pref{f-structure} should
be stated in the way it is. It certainly confused me for a while. We
will explain this by considering the case $B = BSO$, and actually
proving what we just claimed: giving an $SO$-structure on a compact
smooth manifold $M$ is equivalent to giving an orientation on $M$.

\begin{lemma} \label{normal-orientation}
  Let $i \c M \to \R^m$ be an immersion. Then:
  \begin{enumerate}
  \item Orienting the normal bundle $\nu_i$ is equivalent to orienting
    $M$.
  \item If $j \c M \to \R^n$ is a another immersion, there is a
    bijection
  \end{enumerate}
\end{lemma}

\begin{remarks} \label{f-structure-explanation}
  As claimed in \pref{f-examples}(ii), Given an immersion $i \c M \to
  \R^n$, orienting $\nu_i$ is equivalent to orienting $TM$ (by
  \pref{normal-bundle} of $\nu_i$ and the fact that $\R^n$ has a
  canonical orientation) which in turn is equivalent to orienting
  $M$. This is the essential point, but there are still some details
  to work through since we are working with the \emph{stable} normal
  bundle.

  Recall that:
  \begin{itemize}
  \item we can always find such an immersion $i$;
  \item an orientation on a $k$-dimensional vector bundle is
    equivalent to a reduction of its structure group to $SO(k)$,
    i.e. a lift from the classifying map to $BO(k)$ to a map to
    $BSO(k)$.
  \end{itemize}
  Since $\nu_i$ determines $\nu_M$ this implies than an orientation on
  $M$ indeed determines a lift in \pref{f-structure-lift}, and hence a
  $BSO$-structure on $M$. For the other direction, note that since $M$
  is compact, a lift $\til g \c M \to BSO$ factors through some
  $BSO(k)$, and thus a lift determines (for large enough embedding
  dimension $n$) a reduction of the structure group of $\nu_i$ to
  $SO(k)$, and hence an orientation on $M$.

  However, these correspondences aren't really well-defined, since we
  have made choices. I won't discuss why these are independent of the
  choice of immersion $i$
\end{remarks}

\begin{proposition} \label{trivial-normal}
  Let $i \c M \to N$ be an immersion of compact smooth manifolds whose
  normal bundle $\nu_i$ is trivial. Then $\nu_M = i^*\nu_N$, and for
  any fibration $f \c B \to BO$, an $f$-structure on $N$ induces a
  canonical $f$-structure on $M$.
\end{proposition}

\begin{proof}
  Choose an immersion $j \c N \to \R^n$ and let $k \ce j \circ i$. By
  \pref{normal-composition} we have an exact sequene $0 \to \nu_i \to
  \nu_k \to i^*\nu_j \to 0$. Since $\nu_i$ is trivial, $\nu_k$ and
  $i^*\nu_j$ are equal in $\til{KO}(M) \simeq [M,BO]$. By the discussion
  in \pref{stable-normal} we know then that if we take a map $N \to BO$
  classifying the stable normal bundle $\nu_N$, then its restriction
  to $M$ will classify $\nu_M$, as desired. It follows immediately
  that any lift defining an $f$-structure on $N$ canonically restricts
  to a lift defining an $f$-structure on $M$.
\end{proof}

Finally we may define the bordism categories of manifolds, whose
bordism classes we will analyse in future posts.

\begin{definition}
  Let $X$ be a space and $f \c B \to BO$ a fibration. The category
  $\Mfd^f_X$ of \emph{compact smooth manifolds with $f$-structure
    over $X$*} is defined as follows:
  \begin{itemize}
  \item objects are continuous maps $M \to X$ where $M$ is a compact
    smooth manifold equipped with an $f$-structure (note we will often
    abuse as usual and refer to an object of simply by the underlying
    manifold $M$);
  \item morphisms are commutative diagrams
    \[
    % \begin{xy}
    %   \xymatrix {
    %   M \ar[rr]^i \ar[dr] & & N \ar[dl] \\
    %   & X &
    % }
    % \end{xy}
    \]
    where $i$ is a smooth immersion whose normal bundle $\nu_i$ is
    trivial, such that the $f$-structure induced by $i$ on $M$ by
    \pref{trivial-normal} agrees with its equipped $f$-structure.
  \end{itemize}

  We claim $\Mfd^f_X$ is in fact a bordism category. Generalising
  \pref{main-example}, we still take $\d$ to be the boundary and
  $\iota$ to be the inclusion of the boundary, noting that the
  inclusion $\d M \to M$ always has trivial normal
  bundle\footnote{That $\d M \to M$ has trivial normal bundle can be
    seen by the existence of a collar: an open neighbourhood $U
    \subseteq M$ of $\d M$ isomorphic to $\d M \times [0, 1)$ with $\d
    M$ identified with $\d M \times \{0\}$. For $\d M \times \{0\} \to
    \d M \times [0,1)$ certainly has trivial normal bundle, and the
    open embedding $U \to M$ obviously has zero-dimensional normal
    bundle, so then it's immediate from \pref{normal-composition}. See,
    e.g., [Kosinski][kosinski] for a proof of the existence of a
    collar.} so that $M$ induces a canonical $f$-structure on $\d M$
  by \pref{trivial-normal}. Moreover:
  \begin{enumerate}
  \item Obviously the unique map from the empty manifold $\e \to X$
    gives the initial object. We claim disjoint union still gives
    finite coproducts. Given maps $M \to X$ and $N \to X$ there is of
    course a canonical map $M \amalg N \to X$. Moreover $[M \amalg N,
    B] \simeq [M, B] \times [N, B]$ so $f$-structures on $M$ and $N$
    induce a canonical $f$-structure on $M \amalg N$. The maps $M \to
    M \amalg N$ and $N \to M \amalg N$ are open embeddings, so are
    clearly immersions with trivial (zero-dimensional) normal
    bundles. It's not difficult to see then that $M \amalg N \to X$
    really does define a coproduct in $\Mfd^f_X$.
  \item Essential smallness follows again from the embedding theorem.
  \item Clearly $\d$ and $\iota$ preserve coproducts.
  \item It's also evident that $\d\e \simeq \e$ and $\d\d M \simeq \e$.
  \end{enumerate}
  So indeed we have a bordism category.
\end{definition}

\begin{remark}
  One might be concerned that we haven't properly generalised
  \pref{main-example} since we have restricted our set of
  morphisms. But since we have the same coproduct and boundary, and we
  only need to care about \emph{isomorphisms} to define the set of bordism
  classes $\Omega(\Mfd^f_X)$, indeed setting $X$ to be a point and $f
  \ce \id \c BO \to BO$ recovers the same bordism classes
  $\Omega(\Mfd)$ of compact smooth manifolds.
\end{remark}

We finish by assuring ourselves that we aren't working with some
ridiculous notion of bordism, as promised above. Fix a space $X$ and
fibration $f \c B \to BO$ for the remainder.

\begin{lemma} \label{cross-interval}
  Suppose $M \in \Mfd^f_X$ is closed. Then $M \times I$ has the
  structure (not necessarily unique) of an object of $\Mfd^f_X$ such
  that the inclusion $i \c M \simeq M \times \{0\} \inj M \times I$
  determines a morphism in $\Mfd^f_X$.
\end{lemma}

\begin{proof}
  Since $M$ is closed, $M \times I$ is a compact smooth manifold and
  composition with projection $M \times I \to M \to X$ gives us a map
  to $X$. Now, $\nu_i$ is clearly trivial, so by \pref{trivial-normal}
  we have $\nu_M = i^*\nu_{M \times I}$. I.e., we have a commutative
  diagram
  \[
  % \begin{xy}
  %   \xymatrix{
  %     M \ar[r]^{\til g} \ar[d]_i \ar[dr]^g & B \ar[d]^f \\
  %     M \times I \ar[r]^h & BO
  %   }
  % \end{xy}
  \]
  where $g$ classifies $\nu_M$, $h$ classifies $\nu_{M \times I}$, and
  $\til g$ gives the $f$-structure on $M$. Since $f$ is a fibration
  there exists a lift $\til h \c M \times I \to B$ in the diagram. It's
  then clear that the $f$-structure $[\tilde h]$ on $M \times I$ is as
  desired.
\end{proof}

\begin{definition} \label{inverse}
  Let $M \in \Mfd^f_X$ be closed. Let $M \times I \in
  \Mfd^f_X$ be as given by \pref{cross-interval}. Observe that
  \[
  \d(M \times I) = M \times \{0\} \amalg M \times \{1\} \simeq M \amalg M'
  \]
  where $M'$ has the same underlying manifold and map to $X$ as $M$,
  but possibly a different $f$-structure. We call $M'$ an *inverse* of
  $M$; this is not unique since the $f$-structure on $M \times I$ is
  not unique, but we abusively denote any inverse by $-M$. (This
  terminology and notation will be earned below.)
\end{definition}

\begin{remark} \label{inverse-props}
  One may check easily that the following properties hold for $M,N \in
  \Mfd^f_X$:
  \begin{enumerate}
  \item If we have an isomorpshism $M \to N$ then the canonical
    morphisms $M \times I \to N \times I$ and $-M \to -N$ are
    isomorphisms.
  \item Inverse respects coproducts: $-(M \amalg N) \simeq -M \amalg
    -N$.
  \item Inverse is an involution: $-(-M) \simeq M$.
  \end{enumerate}
\end{remark}

\begin{proposition}
  Let $M,N \in \Mfd^f_X$ be closed. Then $M \equiv N$ if and
  only if there exists $W \in \Mfd^f_X$ such that $M \amalg -N
  \simeq \d W$ for some inverse $-N$ of $N$.
\end{proposition}

\begin{proof}
  ($\Rightarrow$) Assume $M \amalg -N \simeq \d W$. By \pref{inverse} we
  have $N \times I \in \Mfd^f_X$ such that $\d(N \times I) \simeq
  -N \amalg N$. Then
  \[
  M \amalg \d(N \times I) \simeq M \amalg -N \amalg N \simeq \d W
  \amalg N
  \]
  so $M \equiv N$.

  ($\Leftarrow$) Assume $M \equiv N$, so $M \amalg \d U \simeq N
  \amalg d V$.
\end{proof}

%--------------------------------------------------------------------%

\bibliographystyle{amsalpha}
\bibliography{refs}

\end{document}
