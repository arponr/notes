\section{Solving $\mA\vx = \vb$ (row reduced form)}

Now we're going to completely solve the question of solving
\[ \mA \vx = \vb, \]
how to tell when there is a solution, and if there what the solutions look like, etc.

\bex
We're going to do this with an example, in fact using the system of equations described by the matrix we used in the previous example:
\bea
x_1+2x_2+2x_3+2x_4=b_1\\
2x_1+4x_2+6x_3+6x_4=b_2\\
3x_1+6x_2+8x_3+10x_4=b_3\\
\eea
We can quite easily see that the third row is the sum of the first two, so we immediately have the condition that $b_3=b_2+b_1$. Let's see how elimination discovers this fact. 

So we do elimination on the (augmented) matrix for this equation:
\[ \mat{1&2&2&2&|&b_1\\2&4&6&8&|&b_2\\3&6&8&10&|&b_3} \rightarrow \mat{1&2&2&2&|&b_1\\0&0&2&4&|&b_2-2b_1\\0&0&2&4&|&b3-3b_1} \rightarrow \mat{1&2&2&2&|&b_1\\0&0&2&4&|&b_2-2b_1\\0&0&0&0&|&b3-b_2-b_1}. \]
That last row in the final step tells us $b_3=b_2+b_1$, exactly as we expected, and this is our condition for solvability on $\vb$. I.e., if $\vb$ satisfies this condition, then $\vb \in C(\mA)$. We can also notice through this example that if some combination of the rows of $\mA$ give the zero row, then the same combination of the entries of $\vb$ must give 0. And both these statements are actually equivalent in terms of solvability, which isn't immediate. 

Now how do we actually find the solutions? We start by finding one particular solution, $\vx_p$. We can do this by setting all free variables to zero, and then solving for the pivot variables. In this case this means we set $x_2=x_4=0$. It follows that
\[ x_3 = \frac{b_2-2b_1}2, \]
and
\[ x_1 = 3b_1-b_2. \]
Now since our solvability criterion only limits our values of $b_3$, we can choose any $b_1$ and $b_2$, say, for example, 1 and 5. So that gives us that $x_1 = -2$ and $x_3= 3/2$. So one solution would be
\[ \vx_p = \mat{-2\\0\\\frac 32\\0}. \]
And one can see by plugging into the original system that this solution vector does in fact work out. 

Now how do we get from this one solution to all the solutions? Well what we do this find any solution from the nullspace of $\mA$, $\vx_n$ and then add this to our particular solution, and this describes the whole solution set. This is because $\mA\vx_p = \vb$ and $\mA\vx_n = 0$, so
\[ \mA(\vx_p + \vx_n) = \vb. \]
So if we remember from the previous section, all $\vx_n$ could be expressed as
\[ \vx_n = c\mat{-2\\1\\0\\0} + d\mat{2\\0\\-2\\1}. \]
Then it follows that all solutions to $\mA\vx = \vb$ (for our choice of $\mA$ and $\vb$) can be expressed as 
\[ \mat{-2\\0\\\frac 32\\0} + c\mat{-2\\1\\0\\0} + d\mat{2\\0\\-2\\1}. \]
This solution set would be a plane intersecting the particular solution and in the same directions as the special solutions of the null space. Notice that this cannot be a vector space because it doesn't go through the origin. 
\eex

So that's the algorithm for solving $\mA\vx = \vb$.
\brm
All solutions can be expressed in as a sum of the particular solution and a nullspace vector because, given any two solutions $\vx_1$ and $\vx_2$, we have that
\[ \mA(\vx_1-\vx_2) = \mA\vx_1 - \mA\vx_2 = \vb - \vb = 0, \]
so the difference between any two solutions is a vector in the nullspace.
\erm

\subsection{In more generality}

So in general, let's thing of an $m \times n$ matrix $\mA$ of rank $r$. We know that
\[ r \le m, \]
since there can't be more than $m$ pivots. The same can be said about $n$:
\[ r \le n. \]

Let's examine the case of \textbf{full column rank}, meaning that $r = n$. It's clear that this implies that there are no free variables and thus the nullspace only has the zero vector, or
\[ N(\mA) = \{\vct{0}\}. \]
And this implies that the particular solution to $\mA\vx = \vb$ (if it exists) is unique, i.e.
\[ \vx = \vx_p. \]
So there are either 0 or 1 solutions. 

Now let's look at \textbf{full row rank}, so $r = m$. What does this say about solvability? Well since there are pivots in every row, there are no zero rows in elimination, which means there are no requirements on $\vb$. I.e., there exists a solution for every $\vb$. Also observe that we have $n-r=n-m$ free variables. 

It follows that if we have a square matrix with $m=n$ (then we have completely full rank), then there will be no free variables, and thus exactly one solution, and moreover, $\mA$ is invertible. And the reduced row echelon of these matrices will be the identiy ($\mR = \mI$) and the nullspace will only be the zero vector.  

So the rank really tells us everything about what the solutions look like. If $r = m = n$ , then $\mR = \mI$ and there is one solution to $\mA\vx = \vb$. If $r = m < n$, then $\mR = \mat{\mI\ \mF}$ and there are infinite solutions. If $r = n < m$, then $\mR = \mat{\mI \\ 0}$ and there are either 0 or 1 solutions. And if $r < n; r < m$, then $\mR = \mat{\mI&\mF\\0&0}$ and there are either 0 or infinite solutions.