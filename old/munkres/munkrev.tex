%%%%%%%%%%%%%%%%%%%%%%%%%%%%%%%%%%%%%%%%%%%%%%%%%%%%%%%%%%%%%%%%%%%%%%

\renewcommand{\A}{\mathbb{A}}
\renewcommand{\O}{\mathcal{O}}

\renewcommand{\a}{\mathfrak{a}}
\newcommand{\p}{\mathfrak{p}}
\newcommand{\q}{\mathfrak{q}}

\newcommand{\height}{\operatorname{ht}}

%%%%%%%%%%%%%%%%%%%%%%%%%%%%%%%%%%%%%%%%%%%%%%%%%%%%%%%%%%%%%%%%%%%%%%


\title{Review of basic concepts in topology}
\author{Arpon Raksit}
\date{}

\begin{document}

\begin{abstract}
  These are (at least partial) solutions to a review exercise given in the section ``*Supplementary Exercises: Review of the Basics'' at the end of Chapter 4 of \emph{Topology}, by Munkres. Any references in the solutions are references within this text.
\end{abstract}
\maketitle
\thispagestyle{fancy}

\subsection*{a}
  $S_\Omega$ (in the order topology) is
  \begin{enumerate}
  \item not connected, not locally connected (hence not path connected, not locally path connected),
  \item not Lindel\"of (hence not compact),
  \item limit point compact,
  \item locally compact,
  \item normal (hence completely regular, regular, Hausdorff),
  \item first countable,
  \item not separable (hence not second countable),
  \item not metrisable, and
  \item locally metrisable.
  \end{enumerate}

\begin{proof}
  Let $a_0$ denote the minimum element of $S_\Omega$ throughout.
  \begin{enumerate}[leftmargin=*]
  \item Let $b \in S_{\Omega}$ an element without an immediate predecessor. (Note, there are uncountably many such elements. Suppose not: then the set of such elements is bounded above by some $d_1 \in S_{\Omega}$. Then define $d_n$ the immediate successor of $d_{n-1}$ for $n \in \N - \{1\}$. Since $\{d_n\}$ is countable it is bounded above and hence has a least upper bound $d \in S_{\Omega}$. But then clearly $d$ cannot have an immediate predecesor, a contradiction since $d_1 < d$.) Any neighbourhood $U$ of $b$ must contain some basis element $(a,c) \ni b$. Since $b$ has no immediate predecessor, there exists $x \in (a,b)$, and moreover if $x'$ is the immediate successor of $x$ then we must have $x' \in (a,b)$. Then we have the separation $U = A \cup B$ where $A := \{y \in U \mid y < x'\}$ and $B := \{y \in U \mid y > x\}$. Thus no neighbourhood of $b$ is connected, so $S_\Omega$ cannot be locally connected or connected.
  \item For $a \in S_{\Omega}$ let $S_a := \{x \in S_\Omega \mid x < a\}$. Since $S_\Omega$ has no maximum element, $\{S_a\}$ is an open cover of $S_{\Omega}$. However since $S_a$ is countable for each $a \in S_\Omega$, any countable subcover can cover only countably many elements of $S_\Omega$.
  \item Let $A$ an infinite subset of $S_\Omega$. Let $B$ a countable subset of $A$. $B$ must have an upper bound $b$, whence $B \subset [a_0,b]$. Since $S_\Omega$ is well ordered and hence has the least upper bound property, $[a_0,b]$ is a compact, hence limit point compact, subset of $S_\Omega$. Then $B \subset A$ has a limit point in $[a_0,b] \subset S_\Omega$. So $A$ is limit point compact.
  \item Let $a \in S_\Omega$. Since $S_\Omega$ has no maximum element, there exists $b \in S_\Omega$ such that $a < b$. Then $[a_0,b]$ is compact in $S_\Omega$ and $a \in [a_0,b) \subset [a_0,b]$.
  \item Every well ordered set is normal in the order topology.
  \item Let $a_1$ the immediate successor of $a_0$. Then $\{a_0\} = [a_0,a_1)$ is open, and hence $S_\Omega$ trivially has a countable basis at $a_0$. Now take any $b \in S_\Omega$ different from $a_0$. Let $b'$ its immediate sucessor. Take the countable collection $\mathcal B_b := \{(a, b') \mid a < b\}$ of neighbourhoods of $b$. Any neigbourhood $U$ of $a$ must contain some basis element $(a, c) \ni b$. Then since $\mathcal B_b \ni (a,b') \subset (a,c)$, $\mathcal B_b$ is a countable basis for $S_\Omega$ at $b$.
  \item Let $A$ any countable subset of $S_\Omega$. $A$ has an upper bound $b$. Then the open set $(b, \Omega)$ is nonempty and disjoint from $A$, so we cannot have $\bar A = S_\Omega$.
  \item If $S_\Omega$ were metrisable then limit point compactness would imply compactness.
  \item We have $\{a_0\}$ a trivially metrisable neighbourhood of $a_0$. For $a \in S_\Omega$ different from $a_0$, we have the neighbourhood $[a_0,a]$ of $a$ which is compact and countable. Since $S_\Omega$ is Hausdorff and first countable, it follows that $[a_0,a]$ is normal and second countable, hence metrisable by the Urysohn metrisation theorem. \qedhere
  \end{enumerate}
\end{proof}

%%%%%%%%%%%%%%%%%%%%%%%%%%%%%%%%%%%%%%%%%%%%%%%%%%%%%%%%%%%%%%%%%%%%%%%%%%%%%%%%%

\subsection*{b}
  $\bar S_\Omega$ (in the order topology) is
  \begin{enumerate}
  \item not connected, not locally connected (hence not path connected, not locally path connected),
  \item compact (hence Lindel\"of, limit point compact, locally compact),
  \item normal (hence completely regular, regular, Hausdorff),
  \item not first countable (hence not second countable),
  \item not separable, and
  \item not metrisable (hence not locally metrisable).
  \end{enumerate}

\begin{proof}
  \begin{enumerate}[leftmargin=*]
  \item See the argument for $S_\Omega$ above.
  \item $\bar S_\Omega$ is equal to the closed interval $[a_0, \Omega]$ ($a_0$ as defined above) and is hence compact.
  \item Every well ordered set is normal in the order topology.
  \item Clearly $\Omega$ is a limit point of $S_\Omega \subset \bar S_\Omega$. But any sequence in $\bar S_\Omega$ forms a countable set and is thus bounded above in $S_\Omega$, so no sequence in $\bar S_\Omega$ converges to $\Omega$. So $\bar S_\Omega$ cannot be first countable.
  \item See the argument for $S_\Omega$ above.
  \item If $\bar S_\Omega$ were metrisable then $S_\Omega \subset \bar S_\Omega$ would be metrisable, which is not the case (as proved above). Since a locally metrisable regular Lindel\"of space is metrisable, $\bar S_\Omega$ thus cannot be locally metrisable either. \qedhere
  \end{enumerate}
\end{proof}

%%%%%%%%%%%%%%%%%%%%%%%%%%%%%%%%%%%%%%%%%%%%%%%%%%%%%%%%%%%%%%%%%%%%%%%%%%%%%%%%%

\subsection*{c}
  $S_\Omega \times \bar S_\Omega$ (in the product topology) is
  \begin{enumerate}
  \item not connected, not locally connected (hence not path connected, not locally path connected),
  \item not Lindel\"of (hence not compact),
  \item limit point compact,
  \item locally compact,
  \item not normal,
  \item completely regular (hence regular, Hausdorff),
  \item not first countable (hence not second countable),
  \item not separable, and
  \item not locally metrisable (hence not metrisable).
  \end{enumerate}

\begin{proof}
  Let $\pi_1 : S_\Omega \times \bar S_\Omega \to S_\Omega$ the projection onto the first coordinate and $\pi_2 : S_\Omega \times \bar S_\Omega \to \bar S_\Omega$ the projection onto the second coordinate. Recall that these are both continuous open maps.
  \begin{enumerate}[leftmargin=*]
  \item Since $\pi_1$ is continuous and open, if $S_\Omega \times \bar S_\Omega$ were locally connected (resp. connected) then $\im \pi_1 = S_\Omega$ would be locally connected (resp. connected), which is not the case (as proved above).
  \item Since $\pi_1$ is continuous, if $S_\Omega \times \bar S_\Omega$ were Lindel\"of then $\im \pi_1 = S_\Omega$ would be Lindel\"of, which is not the case (as proved above).
  \item Let $A$ an infinite subset of $S_\Omega \times \bar S_\Omega$. Let $B$ any countable subset of $A$. Then $\pi_1(B)$ is countable and hence bounded above by some $b \in S_\Omega$. Then $B \subset [a_0,b] \times \bar S_\Omega$. Since both $[a_0,b]$ and $\bar S_\Omega$ are compact, so is $[a_0,b] \times \bar S_\Omega$. It follows that $[a_0,b] \times \bar S_\Omega$ is limit point compact, and thus $B$ has a limit point in $[a_0,b] \times \bar S_\Omega \subset S_\Omega \times \bar S_\Omega$. So $A$ has a limit point, whence $S_\Omega \times \bar S_\Omega$ is limit point compact.
  \item Let $(x,y) \in S_\Omega \times \bar S_\Omega$. Take $a > x$; then $[a_0,a] \times \bar S_\Omega \supset [a_0,a) \times \bar S_\Omega \ni (x,y)$ is compact.
  \item See example 2 in \S 32.
  \item Since $S_\Omega, \bar S_\Omega$ are completely regular, so is $S_\Omega \times \bar S_\Omega$.
  \item Since $\pi_2$ is continuous and open, if $S_\Omega \times \bar S_\Omega$ were first countable then so would $\im \pi_2 = \bar S_\Omega$, which is not the case (as proved above).
  \item Since $\pi_2$ is continuous, if $S_\Omega \times \bar S_\Omega$ were separable then so would $\im \pi_2 = \bar S_\Omega$, which is not the case (as proved above).
  \item Assume $S_\Omega \times \bar S_\Omega$ is locally metrisable. Let $(x,y) \in S_\Omega \times \bar S_\Omega$ and $U \ni (x,y)$ a metrisable neighbourhood. Then $U$ is first countable. Since $\pi_2$ is continuous and open, $\pi_2(U) \ni y$ is first countable, implying that $\bar S_\Omega$ has a countable basis at $y$. Since $y$ was arbitrary, this implies $\bar S_\Omega$ is first countable, a contradiction since this is shown not to be the case above.  \qedhere
  \end{enumerate}
\end{proof}

%%%%%%%%%%%%%%%%%%%%%%%%%%%%%%%%%%%%%%%%%%%%%%%%%%%%%%%%%%%%%%%%%%%%%%%%%%%%%%%%%

\subsection*{d}
  $I_o^2$ ($[0,1]^2$ in the dictionary order topology) is
  \begin{enumerate}
  \item connected,
  \item locally connected,
  \item not path connected,
  \item not locally path connected,
  \item compact (hence Lindel\"of, limit point compact, locally compact),
  \item normal (hence completely regular, regular, Hausdorff),
  \item first countable,
  \item not separable (hence not second countable), and
  \item not metrisable (hence not locally metrisable).
  \end{enumerate}

\begin{proof}
  \begin{enumerate}[leftmargin=*]
  \item $I_o^2$ is a linear continuum in the order topology and is hence connected.
  \item Let $x \in I_o^2$. Any neighbourhood of $x$ in $I_o^2$ must contain an interval containing $x$, which is connected.
  \item Let $0 \le a < c \le 1$ and $0 \le b, d \le 1$. We claim there is no path in $I_o^2$ from $(a,b)$ to $(c,d)$. Suppose there were such a path $\gamma$; then for each $x \in (a,c)$, let $q_x$ a rational in $\gamma^{-1}(\{x\} \times (0,1))$ (which is open by continuity of $\gamma$ and hence intersects the rationals, which are dense in the reals). Then the mapping $x \mapsto q_x$ is an injection since each $\gamma^{-1}(\{x\} \times (0,1))$ is disjoint from the others. This contradicts the uncountability of $(a,c)$.
  \item Any neighbourhood of $(0,1) \in I_o^2$ must contain an interval containing $(0,1)$ and hence must contain a point $(0,b)$ as well as a point $(c,d)$, where $0 < b,c,d < 1$. By the argument in (3), it follows that no neighbourhood of $(0,1)$ is path connected, so $I_o^2$ is not locally path connected.
  \item Since $I_o^2$ is a closed interval in its order topology, it is compact.
  \item We clearly have that $I_o^2$ is Hausdorff. Then we know a compact Hausdorff space is normal.
  \item Clear (don't want to write all of the cases right now).
  \item Let $\pi_1 : I_o^2 \to [0,1]$ the projection onto the first coordinate. Let $A$ any countable subset of $I_o^2$. Then $\pi_1(A)$ is countable, so $[0,1] - \pi_1(A)$ is uncountable, and in particular nonempty. Let $x \in [0,1] - \pi_1(A)$. Then the open interval $((x,1/3), (x, 2/3))$ is nonempty and disjoint from $A$, so $A$ is not dense in $I_o^2$.
  \item A compact metrisable space is second countable, so since $I_o^2$ is compact but not second countable, it is not metrisable. A locally metrisable compact Hausdorff space is metrisable, so $I_o^2$ cannot be locally metrisable either. \qedhere
  \end{enumerate}
\end{proof}

%%%%%%%%%%%%%%%%%%%%%%%%%%%%%%%%%%%%%%%%%%%%%%%%%%%%%%%%%%%%%%%%%%%%%%%%%%%%%%%%%

\subsection*{e}
  $\R_\ell$ (the reals in the lower limit topology) is
  \begin{enumerate}
  \item totally disconnected (hence not connected, not path connected, not locally connected, not locally path connected),
  \item not limit point compact (hence not compact),
  \item Lindel\"of,
  \item not locally compact,
  \item normal (hence completely regular, regular, Hausdorff),
  \item first countable,
  \item separable,
  \item not second countable
  \item not metrisable (hence not locally metrisable).
  \end{enumerate}

\begin{proof}
  \begin{enumerate}[leftmargin=*]
  \item Let $A$ any subset of $\R_\ell$ with more than one point, let $a \in A$ such that there exists $b \in A$ with $b < a$. Then $(-\infty, a) \cap A, [a, \infty) \cap A$ form a separation of $A$.
  \item The infinite set $\Z$ of integers clearly has no limit points in $\R_\ell$.
  \item See example 3 in \S 30.
  \item Suppose $\R_\ell$ is locally compact at, say, $0$. Then we have a compact set $K \subset \R_\ell$ containing a neighbourhood $[0, a)$ of $0$. Since $[0, a)$ is closed in $\R_\ell$, it is closed in $K$, and hence must be compact. But there is clearly no finite subcover of the open cover $\{[0, a-1/n) \mid n \in \N, a > 1/n\}$, so contradiction.
  \item Clearly $\R_\ell$ is Hausdorff. Now let $A,B$ disjoint closed sets in $\R_\ell$. For each $a \in A$, since $a \notin B$ and $B$ is closed, $a$ is not a limit point of $B$, so there exists $x_a \in \R_\ell$ such that $[a,x_a)$ is disjoint from $B$. Similarly for each $b \in B$ there exists $y_b \in R_\ell$ such that $[b,y_b)$ is disjoint from $A$. Then $\bigcup_{a \in A} [a, x_a)$ and $\bigcup_{b \in B} [b, y_b)$ are disjoint opens containing $A$ and $B$, respectively.
  \item For any $x \in \R_\ell$ we have the countable basis $\{[x, x + 1/n) \mid n \in \N\}$ at $x$.
  \item The countable set $\Q$ of rationals are clearly dense in $\R_\ell$.
  \item Let $\mathcal B$ a basis for $\R_\ell$. For each $x \in \R_\ell$ let $U_x \in \mathcal B$ such that $x \in U_x \subset [x, x+1)$. Then the mapping $x \mapsto U_x$ is an injection since $x = \inf U_x$ for all $x \in \R_\ell$. Thus $\{U_x\} \subset \mathcal B$ must be uncountable, so no basis for $\R_\ell$ is countable.
  \item A Lindel\"of metrisable space is second countable, so since $\R_\ell$ is Lindel\"of but not second countable, it is not metrisable. A locally metrisable regular Lindel\"of space is metrisable, so $\R_\ell$ cannot be locally metrisable either. \qedhere
  \end{enumerate}
\end{proof}

%%%%%%%%%%%%%%%%%%%%%%%%%%%%%%%%%%%%%%%%%%%%%%%%%%%%%%%%%%%%%%%%%%%%%%%%%%%%%%%%%

\subsection*{f}
  $\R_\ell^2$ (in the product topology) is
  \begin{enumerate}
  \item totally disconnected (hence not connected, not path connected, not locally connected, not locally path connected),
  \item not limit point compact (hence not compact),
  \item not Lindel\"of (hence not second countable),
  \item not locally compact,
  \item not normal (hence not metrisable),
  \item completely regular (hence regular, Hausdorff),
  \item first countable, and
  \item separable.
  \end{enumerate}

\begin{proof}
   Let $\pi_1,\pi_2 : \R_\ell^2 \to \R_\ell$ the projections onto the first and second coordinates, respectively.
  \begin{enumerate}[leftmargin=*]
  \item Let $A$ any subset of $\R_\ell^2$ with more than one point. Then either $\pi_1(A)$ or $\pi_2(A)$ must contain more than one point, implying (as proved above) that either $\pi_1(A)$ or $\pi_2(A)$ is not connected. Since $\pi_1,\pi_2$ are both continuous, this implies $A$ is not connected.
  \item The infinite set $\Z \times \Z$ of lattice points clearly has no limit points in $\R_\ell^2$.
  \item See example 4 in \S 30.
  \item Suppose there were a compact set $K \subset \R_\ell^2$ containing a neighbourhood $[0, a) \times [0, b)$ of $0$. Then $\pi_1(K)$ would be a compact set containing a neighbourhood $[0, a)$ of $0 \in \R_\ell$, a contradiction to what we showed above.
  \item See example 3 in \S 31.
  \item This follows from the fact that $\R_\ell$ is completely regular.
  \item This follows from the fact that $\R_\ell$ is first countable.
  \item This follows from the fact that $\R_\ell$ is separable. \qedhere
  \end{enumerate}
\end{proof}

%%%%%%%%%%%%%%%%%%%%%%%%%%%%%%%%%%%%%%%%%%%%%%%%%%%%%%%%%%%%%%%%%%%%%%%%%%%%%%%%%

\subsection*{g}
  $\R^\omega$ (in the product topology) is
  \begin{enumerate}
  \item path connected (hence connected), 
  \item locally path connected (hence locally connected),
  \item metrisable (hence normal (hence completely regular, regular, Hausdorff), first countable, locally metrisable),
  \item not locally compact (hence not compact, not limit point compact),
  \item separable (hence second countable, Lindel\"of).
  \end{enumerate}

\begin{proof}
  \begin{enumerate}[leftmargin=*]
  \item This follows from the fact that $\R$ is path connected.
  \item This follows from the fact that $\R$ is locally path connected.
  \item It is shown in \S 20 that the metric defined for $\b x = (x_1,x_2,\ldots), \b y = (y_1,y_2,\ldots) \in \R^\omega$ as
\[ d(\b x, \b y) := \sup_{n \in \N} \frac{\min(|x_n - y_n|, 1)}n \]
gives the product topology on $\R^\omega$.
  \item For a collection $\{X_\alpha\}$ of topological spaces, we have (proved in an exercise) that $\prod_\alpha X_\alpha$ is locally compact if and only if $X_\alpha$ is locally compact for all $\alpha$ and compact for all but finitely many $\alpha$. Since $\R$ is not compact, $\R^\omega$ is not locally compact. (And note, for metrisable spaces, compactness and limit point compactness are equivalent.)
  \item This follows from the fact that $\R$ is separable (a countable product of separable spaces is separable). And separability, the Lindel\"of property, and second countability are equivalent for metrisable spaces. \qedhere
  \end{enumerate}
\end{proof}

%%%%%%%%%%%%%%%%%%%%%%%%%%%%%%%%%%%%%%%%%%%%%%%%%%%%%%%%%%%%%%%%%%%%%%%%%%%%%%%%%

\subsection*{h}
  $\R^\omega$ (in the uniform topology) is
  \begin{enumerate}
  \item not connected (hence not path connected),
  \item locally path connected (hence locally connected),
  \item metrisable (hence normal (hence completely regular, regular, Hausdorff), first countable, locally metrisable),
  \item not locally compact (hence not compact, not limit point compact),
  \item not second countable (hence not separable, not Lindel\"of). 
  \end{enumerate}

\begin{proof}
  Let $\rho$ be the metric which defines the uniform topology on $\R^\omega$, which is defined for $\b x = (x_1,x_2,\ldots), \b y = (y_1,y_2,\ldots) \in \R^\omega$ as
\[ \rho(\b x, \b y) := \sup_{n \in \N} \min(|x_n - y_n|, 1). \]
  \begin{enumerate}[leftmargin=*]
  \item View elements of $\R^\omega$ as $\R$-valued sequences. Then let $A \subset \R^\omega$ the set of unbounded sequences and $B = \R^\omega - A$ the set of bounded sequences. We claim $A,B$ are both open in, and hence constitute a separation of, $\R^\omega$ in the uniform topology. Suppose $\b x = (x_1,x_2,\ldots) \in B$, such that $|x_n| \le M$ for all $n \in \N$. Let $\b y = (y_1, y_2, \ldots) \in B_\rho(\b x, 1)$. Since $\rho(\b x, \b y) < 1$ we must have $|x_n - y_n| < 1$ for all $n \in \N$. Then it follows that $|y_n| \le M+1$ for all $n \in \N$, so that $\b y$ is bounded and hence an element of $B$. So indeed $B$ is open, and in fact this by symmetry shows that $A$ is open as well.
  \item For $\b x = (x_1, x_2, \ldots), \b y = (y_1,y_2,\ldots) \in \R^\omega$, we have (proved in an exercise) that $\b x$ and $\b y$ are in the same component in the uniform topology if and only if $\b x - \b y$ is bounded. So take any neighbourhood $U$ of $\b x$. A basis element $B_\rho(\b x, \epsilon)$ is contained in $U$. Without loss of generality $\epsilon < 1/2$. It is clear then that for all $\b y, \b z \in B_\rho(\b x, \epsilon)$ that $\rho(\b y, \b z) < 1$ and hence $\b y - \b z$ is bounded. Thus $B_\rho(\b x, \epsilon)$ is a path connected neighbourhood of $\b x$ contained in $U$.
  \item By definition.
  \item Let $U$ any neighbourhood of $\b 0 = (0, 0, \ldots) \in \R^\omega$. There exists $\epsilon > 0$ such that $B_\rho(\b 0, \epsilon) \subset U$. Consider the $\R^\omega$-valued sequence $\{\b x_n\}$ defined for $n \in \N$ by
\[ \b x_n := (\overbrace{0, 0, \ldots, 0}^{n-1\ \text{zeros}}, \epsilon/2, 0, 0, \ldots) \in  B_\rho(\b 0, \epsilon) \subset U. \]
Then for any $n, m \in \N$ distinct we have $\rho(\b x_n, \b x_m) = \epsilon/2$, so no subsequence of $\{\b x_n\}$ can converge in $\R^\omega$. It follows that any set containing $U$ is not sequentially compact, and by the equivalence of sequential compactness and compactness for metrisable spaces, there is no compact set in $\R^\omega$ containing a neighbourhood of $\b 0$. 
  \item If $\R^\omega$ were second countable then (as proved in an exercise) every uncountable set would contain uncountable many limit points. However the uncountable set $\{0,1\}^\omega$ clearly has no limit points. \qedhere
  \end{enumerate}
\end{proof}

%%%%%%%%%%%%%%%%%%%%%%%%%%%%%%%%%%%%%%%%%%%%%%%%%%%%%%%%%%%%%%%%%%%%%%%%%%%%

\subsection*{i}
  $\R^\omega$ (in the box topology) is
  \begin{enumerate}
  \item not connected (hence not path connected),
  \item not locally connected (hence not locally path connected),
  \item completely regular (hence regular, Hausdorff),
  \item not locally compact (hence not compact),
  \item not limit point compact,
  \item not first countable (hence not locally metrisable, not metrisable, not second countable),
  \item not separable, and not Lindel\"of.
  \end{enumerate}
Note, it is not known whether or not $\R^\omega$ is normal in the box topology (although it has been shown that the continuum hypothesis implies that indeed it is normal).

\begin{proof}
  \begin{enumerate}[leftmargin=*]
  \item The box topology is finer than the uniform topology and $\R^\omega$ isn't connected in the uniform topology, as shown above. 
  \item It was shown in an exercise that $\b x = (x_1, x_2, \ldots), \b y = (y_1, y_2, \ldots) \in \R^\omega$ are in the same component in the box topology if and only if $\b x - \b y$ is eventually zero, i.e. if $x_n = y_n$ for all but finitely many $n \in \N$. But, e.g., any neighbourhood of $\b 0 = (0,0,\ldots) \in \R^\omega$ contains a point $\b x = (x_1, x_2, \ldots)$ such that $x_n \ne 0$ for infinitely many $n \in \N$. Hence no neighbourhood of $\b 0$ is connected. 
  \item A product of completely regular spaces is completely regular in the box topology (proof omitted).
  \item Suppose there exists a compact set $K \subset \R^\omega$ which contains a neighbourhood $\prod_{n=1}^\infty (a_n,b_n)$ of some $\b x = (x_1,x_2, \ldots) \in \R^\omega$. Then the closure $\prod_{n=1}^\infty [a_n,b_n]$ would be a closed subset of $K$ and hence compact. For $n \in \N$ let 
    \[ U_n := \left[a_n, a_n + \frac{2(b_n-a_n)}3\right)\quad\text{and}\quad V_n := \left(a_n + \frac{b_n-a_n}3, b_n\right]. \]
    Then $\{U_n, V_n \mid n \in \N\}$ is an open cover of $\prod_{n=1}^\infty [a_n,b_n]$ which clearly has no finite subcover, contradiction.
  \item $\Z^\omega$ is infinite but has no limit points in $\R^\omega$ with the box topology.
  \item Suppose $\{U_n\}$ is a countable basis at $\b x \in \R^\omega$. For each $n \in \N$ we can choose $a_{n,i},b_{n,i} \in \R$ for $i \in \N$ such that $U_n \supset \prod_{i=1}^\infty (a_{n,i}, b_{n,i})$. It follows that  $\{\prod_{i=1}^\infty (a_{n,i}, b_{n,i})\}$ is also a countable basis at $\b x$. But clearly $\prod_{n=1}^\infty (a_{n,n}/2, b_{n,n}/2)$ does not contain $\prod_{i=1}^\infty (a_{n,i}, b_{n,i})$ for any $n \in \N$, so contradiction. 
  \item Since the box topology is finer than the uniform topology, if $\R^\omega$ were separable (resp. Lindel\"of) in the box topology, it would be separable (resp. Lindel\"of) in the uniform topology, which is proved not to be the case above. \qedhere
  \end{enumerate}
\end{proof}

%%%%%%%%%%%%%%%%%%%%%%%%%%%%%%%%%%%%%%%%%%%%%%%%%%%%%%%%%%%%%%%%%%%%%%%%%%%%

\subsection*{j}
  $\R^I$ where $I := [0, 1]$ (in the product topology) is
  \begin{enumerate}
  \item path connected (hence connected),
  \item locally path connected (hence locally connected),
  \item not normal,
  \item completely regular (hence regular, Hausdorff),
  \item not locally compact,
  \item not limit point compact (hence not compact),
  \item not first countable (hence not locally metrisable, not metrisable, not second countable)
  \item separable, and 
  \item not Lindel\"of.
  \end{enumerate}

\begin{proof}
  \begin{enumerate}[leftmargin=*]
  \item This follows from $\R$ being path connected.
  \item This follows from $\R$ being locally path connected.
  \item Proof omitted (outlined in exercises).
  \item This follows from $\R$ being completely regular.
  \item Same argument as $\R^\omega$ in the product topology.
  \item The infinite set $\Z^I$ has no limit points in $\R^I$.
  \item Suppose $\{B_n\}$ is a countable basis at $\b x = (x_i)_{i \in I} \in \R^I$. For each $n \in \N$, $B_n$ must contain some open set $U_n := \prod_{i \in I} U_{i,n}$, where $U_{i,n}$ is open in $\R$ for all $i \in I$ and $U_{i,n} = \R$ for all $i \in I - J_n$, $J_n$ a finite subset of $I$. Then $\{U_n\}$ is also a countable basis at $\b x$. Since $J_n$ is finite for all $n \in \N$, we have $J := \bigcup_{n=1}^\infty J_n \subset I$ is countable, and hence $I -  J$ is uncountable. So let $\alpha \in I -  J$, and let $V := \prod_{i \in I} V_i$, where $V_i := \R$ if $i \ne \alpha$ and $V_i := (x_\alpha - 1,x _\alpha + 1)$ if $i = \alpha$. Clearly then $V$ is a neighbourhood of $\b x$ which does not contain $U_n$ for any $n \in \N$, contradiction.
  \item Let $Q := \Q \cap I$. For $n \in \N$ let 
    \begin{align*}
      A_n := \{&(i_0, i_1,\ldots,i_{n+1},x_0,x_1,\ldots,x_n) \in Q^{n+2} \times \Q^{n+1} \\ &\mid 0 = i_0, i_1 < i_2 < \cdots < i_n < i_{n+1} = 1\}. \end{align*}
Now for each $n \in \N$ define $f_n : A_n \to \R^I$ by 
\[ f (i_0, i_1,i_2,\ldots,i_n,i_{n+1},x_0,x_1,\ldots,x_n) := \b x, \]
where for $0 \le k \le n$, $\b x(i) := x_k$ for $i \in [i_k,i_{k+1}]$. Since $A_n$ is clearly countable for $n \in \N$, so is $f_n(A_n)$, and hence $A := \bigcup_{n=0}^\infty f_n(A_n)$ is a countable subset of $\R^I$. We claim $A$ is dense in $\R^I$. Let $U := \prod_{i \in I} U_i$ a basis element in $\R^I$, so each $U_i$ is open in $\R$ and $U_i = \R$ for all $i \in I - J$, $J$ a finite set $\{j_1,j_2,\ldots j_n\} \subset I$. For $1 \le k \le n - 1$, since $\Q$ is dense in $\R$ we can choose $i_k \in \Q \cap [j_k, j_{k+1})$ and $x_{k-1} \in \Q \cap U_{j_k}$; similarly we can choose $i_n \in \Q \cap [j_n, 1]$, $x_{n-1} \in \Q \cap U_{j_{n-1}}$, and $x_n \in \Q \cap U_{j_n}$. Then we clearly have
\[ f_n(0, i_1,i_2,\ldots,i_n,1,x_0,x_1,\ldots,x_n) \in A \cap U, \]
so indeed $A$ is dense in $\R^I$.
  \item Let 
\[ B := \{(x_i)_{i \in I} \in \R^I \mid (\exists \alpha \in I)\ (x_\alpha = 1\ \text{and}\ (\forall i \in I - \{\alpha\})\,x_i = 0)\}. \]
Clearly $B$ is closed in $\R^I$. Hence if $\R^I$ were Lindel\"of then so would $B$ (in the subspace topology). But $B$ is uncountable and discrete, so contradiction.  
\qedhere
  \end{enumerate}
\end{proof}

%%%%%%%%%%%%%%%%%%%%%%%%%%%%%%%%%%%%%%%%%%%%%%%%%%%%%%%%%%%%%%%%%%%%%%%%%%%%

\subsection*{k}
  $\R_K$  (the reals in the $K$-topology) is
  \begin{enumerate}
  \item connected,
  \item not path connected,
  \item not locally connected (hence not locally path connected),
  \item Hausdorff,
  \item not regular (hence not completely regular, not normal, not metrisable, not locally metrisable),
  \item not locally compact,
  \item not limit point compact (hence not compact), and
  \item second countable (hence first countable, separable, Lindel\"of)
  \end{enumerate}

\begin{proof}
  \begin{enumerate}[leftmargin=*]
  \item Let $A := (-\infty, 0) \subset \R_K, B := (0, \infty) \subset \R_K$ in the subspace topologies. The topology on $A$ is definitionally equivalent to the standard subspace topology on $(-\infty, 0)$ and hence $A$ is connected. We also claim the topology on $B$ is equivalent to the standard subspace topology on $(0, \infty)$. This is because a basis element of the form $(a, b) - K$ (with $a \ge 0$) in the $K$-topology is equal to a union of open intervals (hence open) in the standard topology. Thus $B$ is connected as well. Then $\bar A := (\infty, 0]$ and $\bar B := [0, \infty)$ are connected, and since $\bar A \cap \bar B \ne \emptyset$ we then have $\bar A \cup \bar B = \R_K$ is connected. 
  \item Suppose there exists a path $f : [a, b] \to \R_K$ from $0$ to $1$. We know $\im f$ is compact. By the intermediate value theorem we have $[0,1] \subset \im f$; since $[0,1]$ is closed, this implies $[0,1]$ is compact in the $K$-topology. But consider the open cover $\{[0,1] - K\} \cup \{(1/n, 1] \mid n \in \N\}$. There is clearly no finite subcover, contradiction. 
  \item We claim $\R_K$ is not locally connected at $0$. Consider the neighbourhood $U := (-1, 1) - K$ of $0$. Let $V$ any neighbourhood of $0$ contained in $U$. There exists $a,b \in \R$ such that $0 \in (a, b) - K \subset V$. Let $n \in \N$ such that $1/n < b$. Then since $1/n \notin U \implies 1/n \notin V$, we have $V = (V \cap (-\infty, 1/n)) \cup (V \cap (1/n, \infty))$ is a separation of $V$. Hence $V$ is not connected. 
  \item This follows from the fact that $\R$ is Hausdorff in the standard topology, which is coarser than the $K$-topology. 
  \item We know the set $K := \{1/n \mid n \in \N\}$ is closed in $\R_K$, but it is easy to check that any neighbourhood of $0$ intersects any open set containing $K$. This argument also gives that there is no metrisable neighbourhood of $0$, so $\R_K$ is not locally metrisable.
  \item Any neighbourhood of $0$ contains infinitely many points of $K$ which have no limit point. Hence no set containing a neighbourhood of $0$ can be limit point compact.
  \item The infinite set $K$ has no limit point in $\R_K$. 
  \item We have the countable basis $\{(a,b), (a,b) - K \mid a,b \in \Q, a < b\}$. \qedhere
  \end{enumerate}
\end{proof}

\end{document}