\section{Functions of more than one variable}

Functions of one variable, $f(x)$, depend on one parameter and can be represented on on a plane as a collection of points $(x,f(x))$. Functions of two variables, $f(x,y)$, depend on two parameters and can be represented as surfaces in space formed by the points $(x,y,f(x,y))$. Just like functions of one variable have certain domains, functions of multiple variables can also have certain domains. 
\bex
The function
\[ f(x,y) = \frac{x}{x+y} \]
has the domain $x+y \ne 0$. 
\eex

Many real-world functions depend on many variables; e.g. temperature can depend on the position in space, $(x,y,z)$, and time, $t$. 

Functions depending on many parameters are hard to represent visually, so we focus mostly on functions depending on two and sometimes three variables. 

\bex
Take the function
\[ f(x,y)=-y. \]
Visually, this will be a plane, looking like a line with slope -1 in the $yz$-plane and constant in the $xz$-plane. 
\eex

\bex
Take the function 
\[ f(x,y) = 1 - x^2 - y^2. \]
In the $yz$-plane (take $x=0$) we have a downwards parabola starting at 1. In the $xz$-plane (take $y=0$) we also have a downwards parabola starting at 1. In the $xy$-plane (take $z=0$) we have a circle of radius 1 centred at the origin. It thus looks like an upside down cup (sort of).
\eex

One can also represent the graph of a function of two variables using a contour plot. On the $xy$-plane one draws a sequence of ``level curves'', each showing all the points where $f(x,y)$ takes on a certain constant value. Imagine slicing the graph of the function by horizontal planes, and then placing all of these slices onto the $xy$-plane. (This is the same as the contour plots of a topographical map.) Based on the contour plot one can tell qualitatively what happens to $f$ as $x$ increases and $y$ remains constant, or vice-versa. To do this quantitatively, we turn to partial derivatives. 

\section{Partial derivatives}

So, for a function of one variable, $f(x)$, then
\[ f'(x) = \drv{f}{x} = \lim_{\Delta x \to 0} \frac{f(x+\Delta x)-f(x)}{\Delta x}. \]
Now how do we do the same for functions of two variables, $f(x,y)$. The difficulty is that we can change $x$ or change $y$ or change both. Thus, we have several notions of the derivative for a function of multiple variables. We define the notation for a partial derivative, where we consider the change of $f$ with respect to only one variable at a time. E.g., 
\[ \pd{f}{x}(x_0,y_0)= \lim_{\Delta x \to 0} \frac{f(x_0 + \Delta x,y_0)-f(x_0,y_0)}{\Delta x} \]
\[ \pd{f}{y}(x_0,y_0)= \lim_{\Delta y \to 0} \frac{f(x_0,y_0+\Delta y)-f(x_0,y_0)}{\Delta y} \]

We can turn this into a more general definition for functions of any number of variables.
\bdf
For a function of $n\ge 1$ variables, $f(x_1,x_2,\ldots,x_n)$, the partial derivative of $f$ with respect to one of its variables, $x_i$, at a point $(a_1,\ldots,a_n)$ is defined (and noted) as
\[ \pd{f}{x_i}(a_1,\ldots,a_n) = f_{x_i}(a_1,\ldots,a_n) = \lim_{h\to 0} \frac{f(a_1,\ldots,a_i+h,\ldots,a_n) - f (a_1,\ldots,a_n)}{h}. \]
\edf

Geometrically, the partial derivative means that we take a slice of the surface in the relevant plane ($xz$ plane for respect to $x$, $yz$ plane for respect to $y$, etc.) and look at the slope of the curve in that plane (or slice). 

To compute the partial derivative, we just imagine the variables not being considered as constant. For example, when computing 
\[ \pd{f}{x} = f_x\]
for a function $f(x,y)$, we treat $y$ as constant. 

\bex
Take the function
\[ f(x,y) = x^3y + y^2. \]
Then 
\[\pd{f}{x} =3x^2 \]
and 
\[\pd{f}{y}=x^3 + 2y. \]
\eex