\section{Basic properties of numbers}

\subsection{Notes}
This chapter is essentially a more rigourous examination of, as the chapter title suggests, the basic properties of numbers. Included are the associative, commutative, identity and inverse properties of addition and multiplication, the distributive property combining addition and multiplication and three properties of the collection of positive numbers, $P$---that for any number $a$, $a = 0$, $a$ is in $P$ or $-a$ is in $P$, and that $P$ is closed under addition and multiplication. Also examined are properties of the absolute value, including its definitions both as a piecewise function for positive and negative numbers and as the positive square root of the square of a number, as well as the property that $|a + b| \le |a| + |b|$ .

\subsection{Exercises}
\begin{problem}[1-1] \ 
\begin{enumerate}
\item[(i)]
$ax = a$ \\
$a^{-1}(ax) = a^{-1}a$ \\
$(a^{-1}a)x = 1$ \\
$x = 1$ 

\item[(ii)]
$x^2 - y^2 =$ \\
$x^2 - xy + xy - y^2 =$ \\
$x(x - y) + y(x - y) =$ \\
$(x + y)(x - y)$ 

\item[(iii)]
$x^2 = y^2$ \\
$x^2 - y^2 = 0$ \\
$(x - y)(x + y) = 0$ \\
$x - y = 0$ or $x + y = 0$ \\
$x = y$ or $x = -y$

\item[(iv)]
$x^3 - y^3 =$ \\
$x^3 + x^2y + xy^2 - x^2y - xy^2 - y^3 =$ \\
$x(x^2 + xy + y^2) -y(x^2 + xy + y^2) = $ \\
$(x - y)(x^2 + xy + y^2)$

\item[(v)]
$x^n - y^n =$ \\
$x^n + \sum_{i=1}^{n-1} x^{n-i}y^i - \sum_{i=1}^{n-1} x^{n-i}y^i - y^n =$ \\
$x^n + x\sum_{i=1}^{n-1} x^{n-i-1}y^i - y\sum_{i=1}^{n-1} x^{n-i}y^{i-1} - y^n =$ \\
$x\sum_{i=0}^{n-1} x^{n-i-1}y^i -y\sum_{i=0}^{n-1} x^{n-i-1}y^i =$\\
$(x - y)\sum_{i=0}^{n-1} x^{n-i-1}y^i$

\item[(vi)]
$x^3 + y^3 =$ \\
$x^3 - (-y^3) =$ \\
$x^3 - (-y)^3 =$ \\
$(x - (-y))(x^2 + x(-y) + (-y)^2) = $\\
$(x + y)(x^2 - xy + y^2)$
\end{enumerate}
\end{problem}

\begin{problem}[1-7]
We consider the terms of the inequality in pairs. Let us first consider the statement that $a < \sqrt{ab}$. If we square both sides of the inequality, we have that $a^2 < ab$. Dividing by $a$, or multiplying by $a^{-1}$, gives us that $a < b$, which is given as true, so the first inequality must hold.

We next consider the statement that $\sqrt{ab} < \frac{a+b}{2}$. Squaring both sides and then multiplying by 4 gives us that $4ab < a^2 + 2ab + b^2$. Subtracting $4ab$ from both sides of the inequality gives us that $a^2 - 2ab + b^2 > 0$. It follows that $(a-b)^2 > 0$. Since $a \ne b$ this must be true, and thus our original statement is true.

Finally, we consider $\frac{a+b}{2} < b$. Multiplying by 2 gives us that $a + b < 2b$, and subtracting $b$ from each side gives us that $a < b$. This is given as true, and therefore the original inequality must hold true. 

Since these three inequalities hold, the given inequality is proven to be true.
\end{problem}

\begin{problem}[1-20]
We know that $|a + b| \le |a| + |b|$. So,
\[ |(x + y) - (x_0 + y_0)| = |(x - x_0) + (y - y_0)| \le |x - x_0| + |y - y_0| < \epsilon \]
\end{problem}

\begin{problem}[1-21]
As a preliminary, we show that $|a| - |b| \le |a - b|$. We can see this by using the definition of absolute value, $|k| = \sqrt{k^2}$. So we have that $\sqrt{a^2} - \sqrt{b^2} \le \sqrt{(a-b)^2}$. Squaring both sides gives us that $a^2 - 2|ab| + b^2 \le a^2 - 2ab + b^2$, or that $ab \le |ab|$. Since this last statement is true, our original claim is true. 

So, since $|x - x_0| < 1$, we know that $|x| - |x_0| < 1$, or that $|x| < |x_0| + 1$. Next we see that $|xy - x_0y_0| = |x(y - y_0) + y_0(x - x_0)| \le |x||y - y_0| + |y_0||x - x_0|$. Using the statement at the beginning of this paragraph and the inequalities given in the problem, it is clear that we can finish the proof:
\begin{eqnarray*}
|xy - x_0y_0| &&< (1 + |x_0|)\frac{\epsilon}{2(|x_0| + 1)} \,+\, (1 + |y_0|)\frac{\epsilon}{2(|y_0| + 1)} \\
              &&< \epsilon
\end{eqnarray*}
\end{problem}

\begin{problem}[1-22]
We know that $|y_0| - |y| < |y_0 - y| = |y - y_0| < \frac{|y_0|}{2}$. It follows that $|y| > \frac{|y_0|}{2}$, and thus $y \ne 0$. 
For the second part of the problem, we say that 
\[ \left|\frac{1}{y} - \frac{1}{y_0}\right| = \frac{|y - y_0|}{|y||y_0|} < \frac{\epsilon|y_0|^2}{2 |y||y_0|} = \epsilon \frac{|y_0|}{2} \frac{1}{y} < \epsilon \frac{y}{y} = \epsilon. \]
\end{problem}
