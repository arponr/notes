\section{Continuous functions}

\subsection{Notes}
\begin{definition}
The function $f$ is continuous at $a$ if
\[ \lim_{x \to a} f(x) = f(a). \]
More precisely, this means that for every $\epsilon > 0$, there is some $\delta$ such that if $|x - a| < \delta$, then $|f(x) - f(a)| < \epsilon$.

\begin{remark}
Firstly, it is implied in this definition that both the limit and the function exist at $x = a$. Also note that in our precise definition, we no longer need that $|x - a| > 0$, simply because it is trivially true that $|f(a) - f(a)| < \epsilon$.
\end{remark}
\end{definition}

\begin{theorem}
If $f$ and $g$ are continuous at $a$, then
\begin{enumerate}
\item
$f + g$ is continuous at $a$,
\item
$fg$ is continuous at $a$.
\end{enumerate}
Moreover, if $g(a) \ne 0$, then
\begin{enumerate}
\item[(3)]
$\frac{1}{g}$ is continuous at $a$.
\end{enumerate}

\begin{proof}
Given that $f$ and $g$ are continuous at $a$ allows us to say that
\[ lim_{x \to a} f(x) = f(a) \textrm{ and } \lim_{x \to a} g(x) = g(a). \]
By part (1) of Theorem \ref{limthm}, we can say
\[ \lim_{x \to a} (f + g)(x) = f(a) + g(a) = (f + g)(a), \]
which proves (1).

By part (2) of Theorem \ref{limthm}, we can say
\[ \lim_{x \to a} (fg)(x) = f(a)g(a) = (fg)(a), \]
which proves (2).

By part (3) of Theorem \ref{limthm}, we can say
\[ \lim_{x \to a} \left(\frac{1}{g}\right)(x) = \frac{1}{g(a)} = \left(\frac{1}{g}\right)(a), \]
which proves (3).
\end{proof}
\end{theorem}

\begin{theorem} \label {contcomp}
If $g$ is continuous at $a$, and $f$ is continuous at $g(a)$, then $f \circ g$ is continuous at $a$.

\begin{proof}
By the continuity of $f$ at $g(a)$, we can say that for all $\epsilon > 0$, there is some $\delta ' > 0$ such that when
\[ |g(x) - g(a)| < \delta ' \textrm{, } |f(g(x)) - f(g(a))| < \epsilon. \]
Similarly, by the continuity of $g$ at $a$, we can say that for all $\delta '$ there is some $\delta$ such that when
\[ |x - a| < \delta ' \textrm{, } |g(x) - g(a)| < \delta '. \]
Combining this two proves that for every $\epsilon > 0$ there is some $\delta > 0$ such that when $|x - a| < \delta$, $|(f \circ g)(x) - (f \circ g)(a)| < \epsilon$, which is exactly the statement of continuity for $f \circ g$.
\end{proof}
\end{theorem}

The definition of continuity at a point can be generalised to continuity on intervals.

\begin{definition}
A function $f$ is continuous on the interval $(a, b)$ if $f$ is continuous at every $x$ in $(a, b)$. A function $f$ is continuous on the interval $[a, b]$ if
\begin{enumerate}
\item
$f$ is continuous on (a, b),
\item
$\lim_{x \to a^+} f(x) = f(a)$ and $\lim_{x \to b^-} f(x) = f(b)$.
\end{enumerate}

\begin{remark}
Functions that are coninuous on an interval gives us a starting point for rigourously stating what exactly we like to see in a ``resonable'' or ``well-behaved'' function.
\end{remark}
\end{definition}

The following theorem connects the results of continuity at a point, a few of which we have shown already to be easy, and continuity on an interval, which are apparently more difficult.

\begin{theorem} \label{contposneg}
Suppose $f$ is continuous at $a$, and $f(a) > 0$. Then there is a number $\delta > 0$ such that $f(x) > 0$ for all $x$ satisfying $|x - a| < \delta$. Similarly, if $f(a) < 0$, then there is a number $\delta > 0$ such that $f(x) < 0$ for all $x$ satisfying $|x - a| < \delta$.

\begin{proof}
We first consider the case when $f(a) > 0$. By the continuity of $f$ at $a$, we can say that for all $\epsilon > 0$, there is some $\delta > 0$ such that if
\[ |x - a| < \delta \textrm{, then } |f(x) - f(a)| < \epsilon. \]
Since $f(a) > 0$, we can use $f(a)$ as one such $\epsilon$, giving us that if
\[ |x - a| < \delta \textrm{, then } |f(x) - f(a)| < f(a). \]
The fact that $|f(x) - f(a)| < f(a)$ implies that $f(x) > 0$.

An identical argument can be made for $f(a) < 0$. Instead, we can say that
\[ |x - a| < \delta \textrm{, then } |f(x) - f(a)| < -f(a). \]
The fact that $|f(x) - f(a)| < -f(a)$ implies that $f(x) < 0$.
\end{proof}
\end{theorem}

\subsection{Exercises}
\begin{problem}[6-7]
Firstly, we can say that $f(0) = 0$, since $f(0 + 0)$ must equal $f(0) + f(0) = 2f(0)$, which can only occur if $f(0) = 0$. The definition of continuity at 0, then, gives us that for every $\epsilon > 0$, there is some $\delta > 0$ such that
\[ \textrm{if } |x| < \delta \textrm{, then } |f(x)| < \epsilon. \]

Now, consider the continuity at any point $a$. This can be expressed by saying fore every $\epsilon > 0$, there is some $\delta > 0$ such that
\[ \textrm{if } |h| < \delta \textrm{, then } |f(a + h) - f(a)| < \epsilon. \]
It follows, since $f(a + h) = f(a) + f(h)$, that
\[ \textrm{if } |h| < \delta \textrm{, then } |f(h)| < \epsilon. \]
This has already been shown to be true, using the continuity of $f$ at 0, and so we have shown that $f$ must be continuous as all $a$.
\end{problem}

\begin{problem}[6-12] \ 
\begin{enumerate}
\item[(a)]
As defined in the hint, it is clear that the function $G$ is continuous at $a$, with
\[ \lim_{x \to a} G(x) = \lim_{x \to a} g(x) = l = G(a). \]
And, since $f$ is continuous at $l$, or equivalently at $G(a)$, by Theorem \ref{contcomp}, $(f \circ G)$ is continuous at $a$. This means that 
\[ \lim_{x \to a} f(G(x)) = f(G(a)) = f(l). \]

\item[(b)]
With $\lim_{x \to a} g(x) = l$, we can rewrite the statement as
\[ \lim_{x \to l} f(x) = f(l). \]
Now, this is the definition of continuity of $f$ at $l$; if $f$ is not continuous, then this statement is not necessarily true, and thus the claim in the problem is not necessarily true.
\end{enumerate}
\end{problem}
