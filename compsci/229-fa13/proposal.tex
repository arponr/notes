%%%%%%%%%%%%%%%%%%%%%%%%%%%%%%%%%%%%%%%%%%%%%%%%%%%%%%%%%%%%%%%%%%%%%%

\newcommand{\ob}{\oper{ob}}
\renewcommand{\hom}{\oper{hom}}
\newcommand{\id}{\oper{id}}
\newcommand{\im}{\oper{im}}
\newcommand{\op}{\oper{op}}

\newcommand{\Top}{\oper{Top}}
\newcommand{\Set}{\oper{Set}}
\newcommand{\Ab}{\oper{Ab}}
\newcommand{\Grp}{\oper{Grp}}
\newcommand{\Mod}{\oper{Mod}}
\newcommand{\Simplex}{\Delta}
\newcommand{\s}{\oper{s}}
\newcommand{\Ch}{\oper{Ch}}

\newcommand{\Sing}{\oper{Sing}}
\renewcommand{\H}{\mathrm{H}}

%%%%%%%%%%%%%%%%%%%%%%%%%%%%%%%%%%%%%%%%%%%%%%%%%%%%%%%%%%%%%%%%%%%%%%

%%%%%%%%%%%%%%%%%%%%%%%%%%%%%%%%%%%%%%%%%%%%%%%%%%%%%%%%%%%%%%%%%%%%%%

\newcommand{\ob}{\oper{ob}}
\renewcommand{\hom}{\oper{hom}}
\newcommand{\id}{\oper{id}}
\newcommand{\im}{\oper{im}}
\newcommand{\op}{\oper{op}}

\newcommand{\Top}{\oper{Top}}
\newcommand{\Set}{\oper{Set}}
\newcommand{\Ab}{\oper{Ab}}
\newcommand{\Grp}{\oper{Grp}}
\newcommand{\Mod}{\oper{Mod}}
\newcommand{\Simplex}{\Delta}
\newcommand{\s}{\oper{s}}
\newcommand{\Ch}{\oper{Ch}}

\newcommand{\Sing}{\oper{Sing}}
\renewcommand{\H}{\mathrm{H}}

%%%%%%%%%%%%%%%%%%%%%%%%%%%%%%%%%%%%%%%%%%%%%%%%%%%%%%%%%%%%%%%%%%%%%%


\title{CS 229, Project Proposal}
\author{Aleksandar Makelov \and Arpon Raksit}
\date{\today}

\begin{document}
\maketitle
\thispagestyle{fancy}

%%%%%%%%%%%%%%%%%%%%%%%%%%%%%%%%%%%%%%%%%%%%%%%%%%%%%%%%%%%%%%%%%%%%%%

There are two problems we're interested in thinking about. We hope
that's okay.

\section{The log-rank conjecture}

Earlier in the semester we talked about the \textit{deterministic
  communication complexity} $\D(M)$ of a boolean function $M : X
\times Y \to \{0,1\}$. Regarding this, there is the following
interesting \textit{log-rank conjecture}.

\begin{conjecture}[log-rank, \cite{lovasz-1988}]
  Let $M \in \{0,1\}^{X \times Y}$. Then
  \[
  \D(M) \le \poly(\log(\rk(M))),
  \]
  where $\rk(M)$ denote the rank of $M$ viewed as a matrix over $\R$.
\end{conjecture}

\subsection{Previous work}

The conjecture has evaded just about any progress until very
recently. It's current state is given by the following bounds:
\[
\Omega(\log^{\log_3(6)}(\rk(M)) \le \D(M) \le
O(\sqrt{\rk(M)}\log(\rk(M))),
\]
the lower bound given quite a while ago in \cite{nisan-1995} and the
upper bound given a few months ago in \cite{lovett-2013}. The same
upper bound is given in \cite{tsang-2013} in the case that $X = Y =
\{0,1\}^n$ and $M(x,y) = f(x \oplus y)$ for some $f : \{0,1\}^n \to
\{0,1\}$, however using seemingly fairly different techniques. There
was a conditional upper bound
\[
\D(M) \le O(\rk(M) / \log(\rk(M)))
\]
given previously in \cite{ben-sasson-2012}, again using rather
different techniques, in particular relying on a conjecture in
additive combinatorics.

Finally, several equivalent formulations of the conjecture are given
in \cite{gavinsky-2013}. Interestingly, it is shown in particular that
it is equivalent to consider other notions of communication
complexity---\textit{randomised communication complexity},
\textit{zero communication complexity}, and \textit{information
  complexity}---to resolve the conjecture.

\subsection{Proposal}

Obviously we will resolve the conjecture. But here are some things we
might think about along the way.

Firstly, it is curious that the three upper bounds mentioned above
were proved with rather different tools. A connection between
communication complexity and the \textit{discrepanacy} of a boolean
function is used in \cite{lovett-2013}; fourier analysis of boolean
functions is used in \cite{tsang-2013}; and additive combinatorics is
used in \cite{ben-sasson-2012}. It would be interesting, as noted in
\cite{lovett-2013}, to see if there are significant connections
between these techniques, and if so whether this gives some new
insight into the problem.

Secondly, in these papers it seems that only the deterministic
communication complexity has been analysed. In light of the equivalent
formulations above, we are interested to see if one can bound in some
new way, say, randomised communication complexity in terms of rank.

%%%%%%%%%%%%%%%%%%%%%%%%%%%%%%%%%%%%%%%%%%%%%%%%%%%%%%%%%%%%%%%%%%%%%%

\section{Optimal derandomization of JL}

\subsection{The problem}

Following one of the project tips on the course website, we'd like to
look at ways of derandomizing the Johnson-Lindenstrauss transform. The
current state of the art is in \cite{kane-2011}, where the
authors use $k$-wise independence derandomization techniques
iteratively to achieve explicit generators for a projection
$\Pi:\R^d\to\R^s$ with target dimension $s=O(\log(1/\delta)/\eps^2)$
that guarantees $|\|\Pi x\|^2-1|\leq\eps$ with probability at least
$1-\delta$ over all $\|x\|=1$. A purely random seed of length
\[
O \l(\log d + \log(1/\delta)\log \l(\frac{\log(1/\delta)}{\eps}\r)\r)
\]
is used, whereas from the probabilistic method one can show existence
of a seed of length $O \l(\log d + \log(1/\delta)\r)$.

%%%%%%%%%%%%%%%%%%%%%%%%%%%%%%%%%%%%%%%%%%%%%%%%%%%%%%%%%%%%%%%%%%%%%%

\subsection{Previous work}

Apart from \cite{kane-2011}, there have been other results on
obliviously derandomizing JL: \cite{karnin-2011} achieves (using
Euclidean sections, something we need to learn about more) a better
seed length when we want to cluster asymptotically less than
$\poly(d)$ points (more specifically, when
$\delta,\eps=e^{-o(\sqrt{\log d})}$), but does worse than
\cite{kane-2011} in the other case. Apparently, the non-oblivious
derandomizations of JL \cite{engebretsen-2002,sivakumar-2002} are used
for the latter result.

%%%%%%%%%%%%%%%%%%%%%%%%%%%%%%%%%%%%%%%%%%%%%%%%%%%%%%%%%%%%%%%%%%%%%%

\subsection{Proposal}

We have no concrete plan of attack right now, only a boundless
enthusiasm for exploiting the limits of randomness and using expander
graphs. We plan to search for ways of using (random walks on) expander
graphs to tackle the problem, as they're known to achieve
almost-optimal derandomization, and to make the task easier we might
assume we're explicitly given some expander that current theory has
not yet exhibited. We also plan to look at other derandomization
techniques---we're familiar with the discussions on derandomization in
\cite{motwani-2010}, and will consult another useful source, Salil
Vadhan's book on pseudorandomness,
\url{people.seas.harvard.edu/~salil/pseudorandomness}.

%%%%%%%%%%%%%%%%%%%%%%%%%%%%%%%%%%%%%%%%%%%%%%%%%%%%%%%%%%%%%%%%%%%%%%

\bibliographystyle{amsplain}
\bibliography{refs}

\end{document}
