%%%%%%%%%%%%%%%%%%%%%%%%%%%%%%%%%%%%%%%%%%%%%%%%%%%%%%%%%%%%%%%%%%%%%%

\newcommand{\ob}{\oper{ob}}
\renewcommand{\hom}{\oper{hom}}
\newcommand{\id}{\oper{id}}
\newcommand{\im}{\oper{im}}
\newcommand{\op}{\oper{op}}

\newcommand{\Top}{\oper{Top}}
\newcommand{\Set}{\oper{Set}}
\newcommand{\Ab}{\oper{Ab}}
\newcommand{\Grp}{\oper{Grp}}
\newcommand{\Mod}{\oper{Mod}}
\newcommand{\Simplex}{\Delta}
\newcommand{\s}{\oper{s}}
\newcommand{\Ch}{\oper{Ch}}

\newcommand{\Sing}{\oper{Sing}}
\renewcommand{\H}{\mathrm{H}}

%%%%%%%%%%%%%%%%%%%%%%%%%%%%%%%%%%%%%%%%%%%%%%%%%%%%%%%%%%%%%%%%%%%%%%

%%%%%%%%%%%%%%%%%%%%%%%%%%%%%%%%%%%%%%%%%%%%%%%%%%%%%%%%%%%%%%%%%%%%%%

\newcommand{\ob}{\oper{ob}}
\renewcommand{\hom}{\oper{hom}}
\newcommand{\id}{\oper{id}}
\newcommand{\im}{\oper{im}}
\newcommand{\op}{\oper{op}}

\newcommand{\Top}{\oper{Top}}
\newcommand{\Set}{\oper{Set}}
\newcommand{\Ab}{\oper{Ab}}
\newcommand{\Grp}{\oper{Grp}}
\newcommand{\Mod}{\oper{Mod}}
\newcommand{\Simplex}{\Delta}
\newcommand{\s}{\oper{s}}
\newcommand{\Ch}{\oper{Ch}}

\newcommand{\Sing}{\oper{Sing}}
\renewcommand{\H}{\mathrm{H}}

%%%%%%%%%%%%%%%%%%%%%%%%%%%%%%%%%%%%%%%%%%%%%%%%%%%%%%%%%%%%%%%%%%%%%%


\usetikzlibrary{decorations.markings}
\usetikzlibrary{arrows,positioning}

\tikzset{
  ->-/.style={decoration={markings,
      mark=at position .5 with {\arrow{>}}},postaction={decorate}},
  nd/.style = {circle,fill=black,inner sep=1.5pt,font=\tiny},
  redarr/.style={->-, red, fill=none,>=stealth},
  bluarr/.style={->-,blue,fill=none,>=stealth}
}

\title{Cayley graphs as covering spaces}
\author{Arpon Raksit}
\date{December 18, 2013 (original); \today\ (last edit).}

\begin{document}
\maketitle
\thispagestyle{fancy}

%%%%%%%%%%%%%%%%%%%%%%%%%%%%%%%%%%%%%%%%%%%%%%%%%%%%%%%%%%%%%%%%%%%%%%

\section{The definition}

\begin{definition}
  Let $G$ be a group and $S \subseteq G$ a generating set. The
  \emph{cayley graph} $\Gamma(G,S)$ is defined to have:
  \begin{itemize}
  \item vertex set $G$,
  \item (directed) edge set $\{(g,gs) \mid g \in G, s \in S\}$.
  \end{itemize}
\end{definition}

\begin{remark}
  We generally think of $\Gamma(G,S)$ as a \emph{decorated graph}, in
  the sense that we label an edge $(g,gs)$ by $s$ (or, to make things
  more fun, decorate it with some colour assigned to $s$).
\end{remark}

\begin{examples}
  If $G = \Z$ then the generating sets $S = \{1\}$ and $S = \{1,-1\}$
  produce the following cayley graphs, respectively.
  \[
  \begin{tikzpicture}
    \node[nd] (o)  at (0,0) [label=below:$0$] {};
    \node[nd] (l)  [left=of o, label=below:$-1$] {};
    \node[nd] (ll) [left=of l, label=below:$-2$] {};
    \node[nd] (r)  [right=of o, label=below:$1$] {};
    \node[nd] (rr) [right=of r, label=below:$2$] {};

    \node (lll) [left=of ll] {$\cdots$};
    \node (rrr) [right=of rr] {$\cdots$};

    \draw[bluarr] (lll) to (ll);
    \draw[bluarr] (ll) to (l);
    \draw[bluarr] (l) to (o);
    \draw[bluarr] (o) to (r);
    \draw[bluarr] (r) to (rr);
    \draw[bluarr] (rr) to (rrr);
  \end{tikzpicture}
  \]
  \[
  \begin{tikzpicture}
    \node[nd] (o)  at (0,0) [label=below:$0$] {};
    \node[nd] (l)  [left=of o, label=below:$-1$] {};
    \node[nd] (ll) [left=of l, label=below:$-2$] {};
    \node[nd] (r)  [right=of o, label=below:$1$] {};
    \node[nd] (rr) [right=of r, label=below:$2$] {};

    \node (lll) [left=of ll] {$\cdots$};
    \node (rrr) [right=of rr] {$\cdots$};

    \draw[bluarr] (lll) to [bend left = 15] (ll);
    \draw[bluarr] (ll) to [bend left = 15] (l);
    \draw[bluarr] (l) to [bend left = 15] (o);
    \draw[bluarr] (o) to [bend left = 15] (r);
    \draw[bluarr] (r) to [bend left = 15] (rr);
    \draw[bluarr] (rr) to [bend left = 15] (rrr);

    \draw[redarr] (rrr) to [bend left = 15] (rr);
    \draw[redarr] (rr) to [bend left = 15] (r);
    \draw[redarr] (r) to [bend left = 15] (o);
    \draw[redarr] (o) to [bend left = 15] (l);
    \draw[redarr] (l) to [bend left = 15] (ll);
    \draw[redarr] (ll) to [bend left = 15] (lll);
  \end{tikzpicture}
  \]
  These simple examples serve to illustrate (literally!) that
  $\Gamma(G,S)$ certainly depends not only on $G$, but also on $S$.

  If $G = Q_8 = \langle -1,i,j,k \mid (-1)^2 = 1,\ i^2=j^2=k^2=ijk=-1
  \rangle$ is the quaternion group then the generating set $S =
  \{i,j\}$ produces the following cayley graph.
  \[
  \begin{tikzpicture}
    \node[nd] (otl) at (0,0) [label=left:$k$] {};
    \node[nd] (itl) [below right=of otl, label=left:$1$] {};
    \node[nd] (itr) [right=of itl, label=right:$i$] {};
    \node[nd] (otr) [above right=of itr, label=right:$-j$] {};
    \node[nd] (ibl) [below=of itl, label=left:$-i$] {};
    \node[nd] (obl) [below left=of ibl, label=left:$j$] {};
    \node[nd] (ibr) [right=of ibl, label=right:$\!\!-1$] {};
    \node[nd] (obr) [below right=of ibr, label=right:$-k$] {};

    \draw[bluarr] (otr) to (otl);
    \draw[bluarr] (otl) to (obl);
    \draw[bluarr] (obl) to (obr);
    \draw[bluarr] (obr) to (otr);
    \draw[bluarr] (itl) to (itr);
    \draw[bluarr] (itr) to (ibr);
    \draw[bluarr] (ibr) to (ibl);
    \draw[bluarr] (ibl) to (itl);

    \draw[redarr] (itl) to (obl);
    \draw[redarr] (itr) to (otl);
    \draw[redarr] (ibl) to (obr);
    \draw[redarr] (ibr) to (otr);
    \draw[redarr] (otl) to (ibl);
    \draw[redarr] (otr) to (itl);
    \draw[redarr] (obl) to (ibr);
    \draw[redarr] (obr) to (itr);
  \end{tikzpicture}
  \]
  This non-abelian example serves to emphasise that edges are given by
  \emph{right} multiplication by generators.
\end{examples}

\begin{nothing}
  Observe that rather trivially each vertex in $\Gamma(G,S)$ has
  both in-degree and out-degree equal to $|S|$ for any $G,S$.
\end{nothing}

%%%%%%%%%%%%%%%%%%%%%%%%%%%%%%%%%%%%%%%%%%%%%%%%%%%%%%%%%%%%%%%%%%%%%%

\section{Coverings}

Here we discuss one way in which cayley graphs appear in the theory of
covering spaces. This is really nice in that it gives some extremely
geometric examples in which we can think about the (beautifully)
abstract Galois theory of coverings more concretely.

\begin{nothing}
  Let $(X,*) \coloneqq \bigvee_{i=1}^n (S^1,*)$ be the bouquet of $n
  \in \N$ circles. Recall that van Kampen tell us that $\pi_1(X,*)
  \simeq F_n$, where $F_n$ is the free group on $n$ generators. Thus
  the general theory implies that giving a connected Galois covering
  $p : E \to X$ with automorphism group $G$ is equivalent to giving a
  surjective homomorphism $F_n \to G$.\footnote{I was a bit loose with
    my language here. I should really say that giving such a covering
    \emph{up to isomorphism} is equivalent to giving such a
    homomorphism \emph{up to automorphism of $G$}. This is a
    restatement of the bijection between isomorphism classes of
    connected Galois covers with normal subgroups of the fundamental
    group of the base space.} But of course, by definition of $F_n$,
  this is furthermore equivalent to specifying $n$ generators of
  $G$. Now cayley graphs enter our thoughts, 
\end{nothing}

\begin{proposition}
  Let $G$ be a group and $S = \{s_1,\ldots,s_n\} \subseteq G$ a
  generating set.
\end{proposition}

\end{document}
