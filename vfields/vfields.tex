%%%%%%%%%%%%%%%%%%%%%%%%%%%%%%%%%%%%%%%%%%%%%%%%%%%%%%%%%%%%%%%%%%%%%%

\newcommand{\ob}{\oper{ob}}
\renewcommand{\hom}{\oper{hom}}
\newcommand{\id}{\oper{id}}
\newcommand{\im}{\oper{im}}
\newcommand{\op}{\oper{op}}

\newcommand{\Top}{\oper{Top}}
\newcommand{\Set}{\oper{Set}}
\newcommand{\Ab}{\oper{Ab}}
\newcommand{\Grp}{\oper{Grp}}
\newcommand{\Mod}{\oper{Mod}}
\newcommand{\Simplex}{\Delta}
\newcommand{\s}{\oper{s}}
\newcommand{\Ch}{\oper{Ch}}

\newcommand{\Sing}{\oper{Sing}}
\renewcommand{\H}{\mathrm{H}}

%%%%%%%%%%%%%%%%%%%%%%%%%%%%%%%%%%%%%%%%%%%%%%%%%%%%%%%%%%%%%%%%%%%%%%


\title{Vector fields and the J-homomorphism}
\author{Arpon Raksit}
\date{May 9, 2014}

\begin{document}
\maketitle
\thispagestyle{fancy}

%%%%%%%%%%%%%%%%%%%%%%%%%%%%%%%%%%%%%%%%%%%%%%%%%%%%%%%%%%%%%%%%%%%%%%

\newcommand{\RP}{\mathbb{RP}}
\newcommand{\CP}{\mathbb{CP}}
\newcommand{\Gr}{\operatorname{Gr}}
\newcommand{\Th}{\operatorname{Th}}
\newcommand{\Spin}{\mathrm{Spin}}
\newcommand{\Sq}{\operatorname{Sq}}

%%%%%%%%%%%%%%%%%%%%%%%%%%%%%%%%%%%%%%%%%%%%%%%%%%%%%%%%%%%%%%%%%%%%%%

\section{The problem}

To start at the very beginning, we state the basic definitions.

\begin{notation}
  \label{ints}
  Throughout we have integers $n \ge 2$ and $0 \le k \le n$.
\end{notation}

\begin{definitions}
  \label{vfield-dfn}
  Let $M$ be a differentiable manifold. Let $\pi \c T \to M$ be the
  tangent bundle on $M$.
  \begin{enumerate}
  \item A \emph{vector field} on $M$ is a continuous section $v \c M
    \to T$ of $\pi$.
  \item A set of vector fields $\{v_1,\ldots,v_k\}$ on $M$ is
    \emph{linearly independent} if for each $p \in M$ the vectors
    $v_1(p),\ldots,v_k(p)$ are linearly independent in the tangent
    space $T_p$. In particular a single vector field $v$ forms a
    linearly independent set if and only if it is nowhere vanishing.
  \end{enumerate}
\end{definitions}

Now the actual story: we are taught to love spheres from our very
first days in the land of topology. But perhaps it is the following
result---or perhaps really its title---which first truly beguiles us.

\begin{theorem}[Hairy ball]
  \label{hairy-ball}
  The sphere $S^{n-1}$ admits a nowhere vanishing vector field if and
  only if $n$ is even.
\end{theorem}

Of course, so enticed, we cannot just leave it there. We must ask the
following.

\begin{question}
  \label{vfield-prob}
  Then how many vector fields does $S^{n-1}$ admit? Or more
  precisely, what is the maximum size of a set of linearly independent
  vector fields on $S^{n-1}$?
\end{question}

This is one of those questions that occupied people for a while. Its
answer was one of the first applications of generalised cohomology
theory, and involves some really nice ideas, some of which are
hopefully conveyed in this exposition.

%%%%%%%%%%%%%%%%%%%%%%%%%%%%%%%%%%%%%%%%%%%%%%%%%%%%%%%%%%%%%%%%%%%%%%

\section{Lower bound}

We first briefly review the positive side of the problem, that is, how
to actually construct some vector fields and achieve a lower bound. We
do this for completeness, and because it clarifies the sort of
numerology we see in the upper bound. One can see, e.g.,
\cite{hopkins-256y, miller-vfields} for more details.

\begin{nothing}
  \label{gram-schmidt}
  First of all, we have our very nice embedding $S^{n-1} \inj \R^n$,
  which gives the tangent spaces of $S^{n-1}$ a very concrete
  description. In particular, a vector field on $S^{n-1}$ is just a
  map $v \c S^{n-1} \to \R^n$ such that $v(x) \perp x$ (in $\R^n$)
  for all $x \in S^{n-1}$. Note that by Gram-Schmidt, giving $k$
  linearly independent vector fields $v_1,\ldots,v_k \c S^{n-1} \to
  \R^n$ is equivalent to giving $k$ pointwise orthonormal maps
  $v_1,\ldots,v_k \c S^{n-1} \to S^{n-1}$.
\end{nothing}

\begin{notation}
  \label{radon-hurwitz}
  \begin{itemize}[leftmargin=*]
  \item Write $n = 2^ab$ with $b$ odd, and write $a = 4c + d$ with $0
    \le d \le 3$. We define $\rho(n) \ce 2^d + 8c$.\footnote{The
      $\rho(n)$ are called the \emph{Radon-Hurwitz numbers}.} Note in
    particular $\rho(n) = 1$ if $n$ is odd.
  \item Define $e_k \ce |\{0 < j \le k : j \equiv 0,1,2,4 \pmod
    8\}|$. It's easy to see that
    \begin{equation}
      \label{radon-max}
      \rho(n) - 1 = \max \{l \ge 0 : 2^{e_l} \mid n\} = \max \{l
      \ge 0 : e_l \le a\}.
  \end{equation}
  \end{itemize}
\end{notation}

\newcommand{\Cl}{\mathrm{Cl}}
\newcommand{\HH}{\mathbb{H}}
\begin{nothing}
  \label{clifford}
  The \emph{Clifford algebras} $\Cl_l$ for $l \ge 0$ are the free
  associative $\R$-algebras with generators $q_1,\ldots,q_l$ subject
  to the relations $q_i^2 = -1$ and $q_iq_j = -q_jq_i$ for $i \ne
  j$. For small values of $l$ these are as follows\footnote{One should
    note however that the descriptions in this table are given by
    completely non-canonical isomorphisms.}, where $A(d)$
  denotes the algebra of $d$-by-$d$ matrices in $A$.
  \[
    \begin{tabular}{|c||c|c|c|c|c|c|c|c|c|}
      \hline
      $l$ & 0 & 1 & 2 & 3 & 4 & 5 & 6 & 7 & 8 \\
      \hline
      $\Cl_l$ & $\R$ & $\C$ & $\HH$ & $\HH^{\oplus 2}$ & $\HH(2)$ &
      $\C(4)$ & $\R(8)$ & $\R(8)^{\oplus 2}$ & $\R(16)$ \\
      \hline
    \end{tabular}
  \]
  For larger values of $l$ we have a periodicity $\Cl_{l+8} \simeq
  \Cl_l \otimes_\R \Cl_8$.

  What does this have to do with vector fields? Well, suppose $V$ is
  an $\R$-vector space of dimension $n$ with a representation $\Cl_l
  \otimes_\R V \to V$. Let $G_l \subset \Cl_l$ be the multiplicative
  group genereated by $\{\pm q_i\}$; it's easy to see $|G_l| =
  2^{l+1}$. We can construct a $G_l$-invariant inner product on $V$
  (e.g., by averaging any inner product over $G_l$), and then we see
  that (under this inner product) we get $l$ orthonormal vector fields
  on the unit sphere $S(V) \simeq S^{n-1}$ via $x \mapsto q_ix$ for
  $1 \le i \le l$.

  Thus we are interested in knowing the minimal dimension of a
  representation of $\Cl_l$. By the periodicity above, one can show
  without much difficulty that this dimension is precisely $2^{e_l}$,
  where $e_l$ is as defined in \eqref{radon-hurwitz}. Thus there is a
  representation of $\Cl_l$ on $\R^n$ whenever $2^{e_l} \mid n$, by
  writing $\R^n \simeq \R^{2^{e_l}} \times \cdots \times \R^{2^{e_l}}$
  and acting diagonally via the minimal representation. This gives the
  following lower bound on our question \eqref{vfield-prob}.
\end{nothing}

\begin{theorem}
  \label{vfield-lower-bound}
  There is a set of $\rho(n)-1$ linearly independent vector fields on
  $S^{n-1}$.
\end{theorem}

\begin{proof}
  Let $l \ce \rho(n)-1$. By \eqref{radon-max}, $2^{e_l} \mid n$, so the
  discussion in \eqref{clifford} gives the claim.
\end{proof}

%%%%%%%%%%%%%%%%%%%%%%%%%%%%%%%%%%%%%%%%%%%%%%%%%%%%%%%%%%%%%%%%%%%%%%

\section{Upper bound: a reduction}

Getting an upper bound is where the real difficulty lies. Well, we
know one upper bound: $S^{n-1}$ certainly can't admit $n$ linearly
independent vector fields, since $\dim S^{n-1} = n - 1$. And to say
$S^{n-1}$ admits $n-1$ linearly independent vector fields is to say
$S^{n-1}$ is parallelisable, which famously is true if and only if $n
\in \{2,4,8\}$. An optimal upper bound would at least tell us this
parallelisability result. So let's think about it for a second and
reduce the question to one more attackable by algebra.

\begin{definition}
  \label{stiefel}
  For $l \in \N$, the \emph{Stiefel manifold} $V_{l,n}$ is the space
  \[
  \{(v_1,\ldots,v_l) : v_i \in S^{n-1}, \langle v_i, v_j \rangle =
  \delta_{i,j}\}
  \]
  of orthonormal $l$-frames on $\R^n$.
\end{definition}

\begin{lemma}
  \label{section-reduction}
  Let $\pi_k \c V_{k+1,n} \to S^{n-1}$ be the projection
  $(v_1,\ldots,v_{k+1}) \mapsto v_1$. Then $S^{n-1}$ admits a set of
  $k$ linearly independent vector fields if and only if there is a
  section $S^{n-1} \to V_{k+1,n}$ of $\pi_k$.
\end{lemma}

\begin{proof}
  This is immediate from the discussion in \eqref{gram-schmidt}.
\end{proof}

So we've reduced our problem to the existence of some map. Already one
can imagine using algebra to get at the problem now. E.g., we can ask
for what $k$ this map can exist in singular homology or
cohomology. This was the strategy of Steenrod and Whitehead
\cite{steenrod-vfields}, who achieve an upper bound $k \le 2^a$, in
the notation of \eqref{radon-hurwitz}. Of course this result doesn't
tell us that $S^{15}$ is not parallelisable, and leaves a large gap
from the lower bound \eqref{vfield-lower-bound}. This gap was finally
closed by Adams, who ingeniously employed K-theory instead to show the
lower bound \eqref{vfield-lower-bound} is in fact optimal.

\begin{theorem}[\cite{adams-vfields}]
  \label{vfield-upper-bound}
  There does not exist a set of $\rho(n)$ linearly independent vector
  fields on $S^{n-1}$.
\end{theorem}

Our goal for the remainder is to explain the ideas behind a proof of
this theorem, which of course completely answers our question
\eqref{vfield-prob}. Following \cite{miller-vfields}, the argument we
give is not in the original form presented in \cite{adams-vfields},
but rather one utilising Adams's later work on bounding the image of
the J-homomorphism \cite{adams-J-II}.

However, we will only work in complex K-theory. As a result we will
achieve a very slightly worse upper bound than promised by
\eqref{vfield-upper-bound}, but this way we get to avoid a few
subtleties that arise in studying real K-theory. At least to the
author, it seems the argument in complex K-theory retains the main
ideas and yet is much easier to absorb. At the end we will explain
why, after sorting out the subtleties, translating the argument into
real K-theory gives the full result.

%%%%%%%%%%%%%%%%%%%%%%%%%%%%%%%%%%%%%%%%%%%%%%%%%%%%%%%%%%%%%%%%%%%%%%

\section{Connection to the J-homomorphism}

We first review our basic notation and the general setup of Adams's
study of the J-homomorphism.

\begin{notation}
  \begin{enumerate}[leftmargin=*]
  \item $X$ will always denote a connected finite CW-complex.
  \item Denote\footnote{We use Adams's notation for K-theory rather
      than the what is the standard notation these days, since it
      seems more convenient for our purposes.} real and complex
    K-theory by $K_\Lambda(X)$ with $\Lambda = \R$ or $\Lambda = \C$,
    respectively, and reduced K-theory by $\til K_\Lambda(X)$.
  \item We have maps:
    \begin{itemize}
    \item $\iota \c K_\C(X) \to K_\R(X)$ by forgetting complex
      structure;
    \item $\kappa \c K_\R(X) \to K_\C(X)$ by complexification, i.e.,
      tensoring with $X \times \C$.
    \end{itemize}
    Note that since $\C \simeq \R \oplus \R$ the map $\iota \circ
    \kappa$ is just multiplication by $2$.
  \item If $\xi \to X$ is a real or complex vector bundle, we
    abusively denote its class in $K_\Lambda(X)$ by $\xi$ as well.
  \item Let $\epsilon_\Lambda$ for $\Lambda = \R$ or $\Lambda = \C$
    denote the trivial real or complex line bundle (over a space
    understood from context), respectively.
  \end{enumerate}
\end{notation}

\begin{remark}
  \label{dimension-welldefined}
  Since we always work over a connected base, any vector bundle $\xi
  \to X$ has a well-defined dimension over a specified $\Lambda$, and
  this extends to a ring morphism $\dim \c K_\Lambda(X) \to \Z$. Recall
  $\til K_\Lambda(X) = \ker(\dim)$.
\end{remark}

\begin{definitions}
  \label{J-dfns}
  \begin{enumerate}[leftmargin=*]
  \item A \emph{spherical fibration} over $X$ is a fibre bundle $S \to
    X$ whose fibre has the homotopy type of a sphere.
  \item Let $S \to X$ and $S' \to X$ be two spaces over $X$. We say
    $S$ and $S'$ are \emph{fibre homotopy equivalent} if there are
    maps $f \c S \to S'$ and $g \c S' \to S$ over $X$ and homotopies
    $gf \simeq \id_S$ and $fg \simeq \id_{S'}$ over $X$.\footnote{As
      usual, ``over $X$'' just means preserving fibres.}
  \item Denote by $SF(X)$ the Grothendieck group of the monoid of
    fibre homotopy equivalence classes of spherical fibrations over
    $X$ with fibre-wise smash product.
  \item The \emph{(real) J-homomorphism} is the map $J_\R \c K_\R(X)
    \to SF(X)$ induced by $\xi \mapsto \xi-0$ for bundles $\xi \to
    X$, i.e., removing the zero section. Or if we equip $\xi$ with a
    metric, then the \emph{unit sphere bundle} $S(\xi) \ce \{v \in \xi
    : |v| = 1\}$ is evidently fibre homotopy equivalent to $\xi-0$, so
    we have $J(\xi) = S(\xi)$.
  \item We get a \emph{complex J-homomorphism} $J_\C \ce J_\R \circ
    \iota \c K_\C(X) \to K_\R(X) \to SF(X)$.
  \item The \emph{image of the J-homomorphism} is denoted
    $J_\Lambda(X)$ with $\Lambda = \R$ or $\Lambda = \C$.
  \end{enumerate}
\end{definitions}

We now translate our problem into a question about the J-homomorphism.

\begin{lemma}[Dold-Lashof]
  \label{dold-fibre}
  Let $Y \to X$ and $Y' \to X$ be fibre bundles. Assume $Y$ and $Y'$
  have the homotopy type of CW-complexes and $X$ is connected. Then a
  map $Y \to Y'$ over $X$ inducing a homotopy equivalence on all
  fibres is in fact a fibre homotopy equivalence.
\end{lemma}

\begin{proof}
  Omitted.
\end{proof}

\begin{notation}
  \label{bundle-notation}
  Let $\gamma$ denote the tautological real line bundle over $\RP^k$.
\end{notation}

\begin{lemma}
  \label{RP-reduction}
  Assume $\pi_k \c V_{k+1,n} \to S^{n-1}$ has a section. Then there
  is a fibre homotopy equivalence $S(\epsilon_\R^{\oplus n}) \to
  S(\gamma^{\oplus n})$ over $\RP^k$.
\end{lemma}

\begin{proof}
  First let $Y \ce (S^k \times S^{n-1})/(\Z/2)$, with $\Z/2$ acting
  diagonally via the antipodal maps, and observe there is a
  homeomorphism $Y \simeq S(\gamma^{\oplus n})$ over $\RP^k$, where the
  map $Y \to \RP^k$ is induced by projection
  \[
  S^k \times S^{n-1} \to S^k \to \RP^k, \quad (v,x) \mapsto \ell_v.
  \]
  Indeed, viewing $S(\gamma^{\oplus n}) \subset \gamma^{\oplus n}
  \subset \RP^k \times (\R^{k+1})^n$, the homeomorphism is induced by
  \[
  S^k \times S^{n-1} \to \RP^k \times (\R^{k+1})^n, \quad (v,x)
  \mapsto (\ell_v, x_1v, \ldots, x_nv),
  \]
  where $(x_1,\ldots,x_n) \ce x \in S^{n-1} \subset \R^n$. Then since
  $S(\epsilon_\R^{\oplus n}) \simeq \RP^k \times S^{n-1}$, we are
  left to give a fibre homotopy equivalence $\RP^k \times S^{n-1} \to
  Y$.

  If $s \c S^{n-1} \to V_{k+1,n}$ is a section of $\pi_k$ then we
  get a map $f \c \RP^k \times S^{n-1} \to Y$ over $\RP^k$ induced by
  the map
  \[
  S^k \times S^{n-1} \to S^k \times S^{n-1}, \quad (v,x) \mapsto
  (v, s(x)v),
  \]
  where $s(x) \in V_{k+1,n}$ acts on $S^k$ by identifying an
  orthonormal frame with an $n$-by-$(k+1)$ matrix. Note that since $s$
  is a section, if $e_1 \in S^k$ is the first canonical basis vector,
  then $s(x)e_1 = x$.

  Since $f$ preserves fibres, over each $\ell \in \RP^k$ it induces a
  map $f_\ell \c S^{n-1} \to S^{n-1}$. On any connected open set $U
  \subseteq \RP^k$ over which $\gamma^{\oplus n}$ is trivial, the
  assignment $\ell \mapsto f_\ell$ gives a continuous map $U \to
  \map(S^{n-1},S^{n-1})$, whose image lies in a single connected
  component. Since $\RP^k$ is connected and can be covered by such
  $U$, it follows that $\deg(f_\ell)$ is constant. But above we noted
  that $f$ induces the identity over $\ell_{e_1}$, so in fact $f_\ell$
  must be a homotopy equivalence for all $\ell \in \RP^k$. We are then
  done by \eqref{dold-fibre}.
\end{proof}

\begin{notation}
  \label{RP-notation}
  Define $\lambda \ce \gamma - 1 \in \til K_\R(\RP^k)$ and $\nu \ce
  \kappa(\lambda) = \kappa(\gamma) - 1 \in \til K_\C(\RP^k)$.
\end{notation}

\begin{lemma}
  \label{j-reduction}
  Suppose $n = 2m$. If $S^{n-1}$ admits a set of $k$ linearly
  independent vector fields, then $J_\C(m\nu) = J_\R(n\lambda) = 0$.
\end{lemma}

\begin{proof}
  Given the hypothesis, it is immediate from \eqref{section-reduction}
  and \eqref{RP-reduction} that $J_\R(n\lambda) = 0$. And by
  definition we have $J_\C(m\nu) = J_\R(\iota(\kappa(m\lambda))) =
  J_\R(2m\lambda) = J_\R(n\lambda)$.
\end{proof}

So solving our problem now reduces to understanding
$K_\Lambda(\RP^k)$ and $J_\Lambda(\RP^k)$, and as indicated earlier
we will work with $\Lambda = \C$. We first address the former.

%%%%%%%%%%%%%%%%%%%%%%%%%%%%%%%%%%%%%%%%%%%%%%%%%%%%%%%%%%%%%%%%%%%%%%

\section{K-theory of projective space}

\begin{notation}
  \label{kthy-order}
  Let $f_k \ce \lfloor k/2 \rfloor$.
\end{notation}

\begin{theorem}
  \label{RP-kthy}
  $\til K_\C(\RP^k)$ is generated by $\nu$, which satisfies the
  relations $\nu^2 = -2\nu$ and $\nu^{f_k+1} = 0$. This determines a
  group isomorphism $\til K_\C(\RP^{k-1}) \simeq \Z/2^{f_k}$.  Finally,
  the Adams operations are given by
  \[
  \psi^l(\nu) =
  \begin{cases}
    0 & \text{if }l\text{ even} \\
    \nu & \text{if }l\text{ odd}.
  \end{cases}
  \]
\end{theorem}

\begin{proof}
  The case $k=1$ is trivial, so assume $k > 1$. Since real line
  bundles have structure group $\O(1) \simeq \{\pm 1\}$, we
  automatically have $\gamma^2 = 1 \in K_\R(\RP^k)$. It follows that
  $\kappa(\gamma)^2 = 1 \in K_\C(\RP^k)$, and hence that $\nu^2 =
  -2\nu$.

  Next we prove $\nu^{f_k+1} = 0$. Since the tautological line bundle on
  $\RP^{k+1}$ pulls back to the tautological line bundle on $\RP^k$
  via the inclusion $\RP^k \to \RP^{k+1}$, by naturality we may assume
  $k = 2f_k+1$ is odd. Let $\pi \c \RP^k \to \CP^{f_k}$ be the canonical
  projection, and let $\xi \to \CP^{f_k}$ be the tautological (complex)
  line bundle. It is easy to see directly that $\pi^*\xi \simeq
  \kappa(\gamma)$. Thus that $\nu^{f_k+1} = 0$ follows from the
  fact\footnote{One can see \cite{atiyah-kthy} or \cite{adams-vfields}
    for (two different) proofs.}  that $\til K_\C(\CP^{f_k}) \simeq
  \Z[t]/t^{f_k+1}$ where $t \ce \xi - 1$.

  We now show $\til K_\C(\RP^k) \simeq \Z/2^{f_k}$, using the
  Atiyah-Hirzebruch spectral sequence
  \[
  E_2^{p,q} \ce H^p(\RP^k,K_\C^q(*)) \implies K_\C^{p+q}(\RP^k).
  \]
  Recall Bott periodicity and the cohomology of projective space:
  \[
  K_\C^q(*) \simeq
  \begin{cases}
    \Z & \text{if }q\text{ even} \\
    0 & \text{if }q\text{ odd},
  \end{cases}
  \hspace{30pt}
  H^p(\RP^k;\Z) \simeq
  \begin{cases}
    \Z & \text{if }p=0 \\
    \Z/2 & \text{if }0 < p \le k,\ p\text{ even} \\
    \Z & \text{if }p=k,\ k\text{ odd} \\
    0 & \text{otherwise}.
  \end{cases}
  \]  
  Thus the $E_2$ page looks as follows, where $A$ depends on the parity
  of $k$ as indicated above.

  \tikzset{
  lb/.style = {font=\tiny},
  nd/.style = {font=\small},
}

\begin{center}
  \begin{tikzpicture}[scale=0.7]
    \pgfmathtruncatemacro{\A}{-1}
    \pgfmathtruncatemacro{\B}{7}
    \pgfmathtruncatemacro{\C}{-4}
    \pgfmathtruncatemacro{\D}{4}

    \draw[gray,very thin] (\A,\C) grid (\B+1,\D+1);
    \draw[thin,<->] (0,\C) -- (0,\D+1);
    \draw[thin,<->] (\A,0) -- (\B+1,0);

    \pgfmathtruncatemacro{\E}{\B-3}
    \foreach \x in {\A,...,\E} {
      \node[lb] at (\x+0.5,-0.2) {\x};
    }
    \node[lb] at (\B-1.5,-0.2) {$\cdots$};
    \node[lb] at (\B-0.5,-0.2) {$k$};
    \node[lb] at (\B+0.5,-0.23) {$k$+1};

    \foreach \y in {\C,...,\D} {
      \node[lb] at (-0.2,\y+0.5) {\y};
    }

    \pgfmathtruncatemacro{\F}{\C+2}
    \foreach \y in {\C,\F,...,\D} {
      \node[nd] at (0.5,\y+0.5) {$\Z$};
      \node[lb] at (\B-1.5,\y+0.5) {$\cdots$};
      \node[nd] at (\B-0.5,\y+0.5) {$A$};
      \foreach \x in {2,4,...,\E} {
        \node[nd] at (\x+0.5,\y+0.45) {$\Z/2$};
      }
    }
  \end{tikzpicture}
\end{center}


  We claim the spectral sequence is trivial, i.e., all the
  differentials on every page vanish. By naturality of the spectral
  sequence in a point inclusion $* \to \RP^k$ we know all the
  differentials on the column $p=0$ vanish. All differentials change
  the parity of the total degree $p+q$, so a differential from a
  $\Z/2$ can only possibly map to $0$ or $A \simeq\Z$ (in the case $k$
  odd) and hence must vanish. And obviously the differentials on the
  column $p=k$ vanish. So we conclude
  \[
  \Gr_pK_\C(\RP^k) \simeq E_\infty^{p,-p} \simeq E_2^{p,-p} \simeq
  \begin{cases}
    \Z & \text{if }p = 0\\
    \Z/2 & \text{if }1 \le p \le f_k\\
    0 & \text{otherwise}.
  \end{cases}
  \]

  We next claim that $\Gr_pK_\C(\RP^k) \simeq \Z/2$ is generated by
  the class of $\nu^p$ for $1 \le p \le f_k$. It suffices to show this
  for $p=1$, because the spectral sequence is multiplicative, and has
  multiplication induced by the cup product in singular cohomology on
  $E_2$, and if $x \in H^2(\RP^k;\Z)$ is a generator then we know
  $x^p \ne 0$ for $1 \le p \le f_k$. And for $p=1$ it suffices to
  treat the case $k=2$, since the spectral sequence is natural and the
  inclusion $\RP^2 \to \RP^k$ both:
  \begin{itemize}
  \item pulls back the tautological bundle on $\RP^k$ to the
    tautological bundle on $\RP^2$;
  \item induces an isomorphism $H^2(\RP^k;\Z) \simeq H^2(\RP^2;\Z)
    \simeq \Z/2$.
  \end{itemize}
  But in the case $k=2$, we have $ \Gr_1K_\C(\RP^2) \simeq
  \til K_\C(\RP^2)$, so to say that $\nu$ is a generator is just to say
  $\nu \ne 0$, or equivalently $\kappa(\gamma) \ne 1$. One can prove
  this with the Stiefel-Whitney class: it suffices to show $1 \ne
  \iota(\kappa(\gamma)) = 2\gamma \in K_\R(\RP^2)$, which is
  witnessed by $w(\gamma\oplus\gamma) = w(\gamma)^2 = 1+x^2 \ne 1$,
  where $x \in H^1(\RP^2;\Z/2)$ is a generator.

  Now we can determine the extensions needed to compute $K_\C(\RP^k)$
  from the associated graded $\Gr_*K_\C(\RP^k)$. Let $F_p$ be the
  $p$-th filtered piece of $K_\C(\RP^k)$, so that $F_p/F_{p+1} \simeq
  \Gr_pK_\C(\RP^k) \simeq \Z/2$ for $1 \le p \le f_k$ and $F_1 \simeq
  \til K_\C(\RP^k)$. We inductively show that $F_p \simeq
  \Z/2^{f_k-p+1}$ with generator $\nu^{f_k}$; the base case $p=f_k$ is
  done already. To induct, we have the extension problem
  \[
  0 \to \Z/2^{f_k-p} \to F_p \to \Z/2 \to 0.
  \]
  We just need to show $\nu^p$ has order $2^{f_k-p+1}$. But we have
  the identity $\nu^{p+1} = -2\nu^p$, and we inductively know $\nu^p$
  has order $2^{f_k-p}$. Since we know $F_p$ is a $2$-group this
  implies the claim.

  To finish the proof we just need to verify the Adams operations, but
  this is evident from the identity $\kappa(\gamma)^2 = 1$ shown
  above, since $\psi^l(\nu) = \psi^l(\kappa(\gamma)) - 1 =
  \kappa(\gamma)^l - 1$.
\end{proof}

\begin{remark}
  \label{RP-ringhom}
  Given the information from \eqref{RP-kthy}, the multiplicative
  structure on $\til K_\C(\RP^k)$ can be clarified by observing that we
  can define an injective morphism of (non-unital) rings $\alpha \c
  \til K_\C(\RP^{k-1}) \to \Z/2^{f_k+1}$ via $\nu \mapsto -2$.
\end{remark}

%%%%%%%%%%%%%%%%%%%%%%%%%%%%%%%%%%%%%%%%%%%%%%%%%%%%%%%%%%%%%%%%%%%%%%

\section{The Thom isomorphism}

Before moving on to studying $J_\C(\RP^{k-1})$ we briefly review the
basic theory of the Thom isomorphism.

\begin{definition}
  \label{thom-space}
  Let $\xi \to X$ be a vector bundle of real dimension $r$. The
  \emph{Thom space} $\Th(\xi)$ of $\xi$ is obtained by fibre-wise
  one-point compactifying $\xi$ and then identifying all the points at
  infinity. More precisely we have $\Th(\xi) \simeq \P(\xi \oplus
  \epsilon_\R)/\P(\xi)$, or if we equip $\xi$ with a metric then we
  have $\Th(\xi) \simeq D(\xi)/S(\xi)$ the unit disk bundle quotiented
  by the unit sphere bundle.
\end{definition}

\begin{remark}
  \label{thom-fibre}
  For any $p \in X$ the fibre $\R^r \simeq \xi_p \inj \xi$ determines
  a ``fibre'' $S^r \simeq \Th(\xi_p) \inj \Th(\xi)$.
\end{remark}

\begin{nothing}
  \label{thom-iso}
  Let $\xi \to X$ be an oriented vector bundle of real dimension
  $r$. Let $E$ be a multiplicative cohomology theory. Then we have
  \[
  \til E^*(\Th(\xi)) \simeq E^*(D(\xi),S(\xi)) \simeq E^*(\xi, \xi-0),
  \]
  and in this way $\til E^*(\Th(\xi))$ is a module over $E^*(X) \simeq
  E^*(\xi)$.
  
  A \emph{Thom class} in $E$ of $\xi$ is an element $u \in \til
  E^r(\Th(\xi))$ such that for any fibre $i \c S^r \simeq \Th(\xi_p)
  \to \Th(\xi)$, where the identification $S^r \simeq \Th(\xi_p)$ is
  determined by the orientation of $\xi$, the restriction
  \[
  \begin{tikzcd}
    \til E^r(\Th(\xi)) \rar{i^*} & \til E^r(S^r) \rar{\sim} & E^0(*)
  \end{tikzcd}
  \]
  sends $u$ to the canonical unit $1 \in E^0(*)$.
  
  The \emph{Thom isomorphism theorem} states that if $\xi$ has a Thom
  class $u$ then the map
  \[
  \phi \c E^*(X) \to \til E^{*+r}(\Th(\xi)), \quad x \mapsto x \cdot u
  \]
  is an isomorphism of $E^*(X)$-modules.
\end{nothing}

\begin{proposition}
  \label{thom-K}
  There exist Thom classes $u_\xi$ in $K_\C$ for all complex vector
  bundles $\xi \to X$. These can be chosen to satisfy the following
  pleasant properties.
  \begin{enumerate}
  \item Naturality: for any pullback square
    \[
    \begin{tikzcd}
      f^*\xi \rar{g} \dar & \xi \dar \\ Y \rar{f} & X
    \end{tikzcd}
    \]
    of complex vector bundles, $u_{f^*\xi} = g^*(u_\xi)$.
  \item Multiplicativity: let $\xi \to X$ and $\eta \to Y$ be complex
    vector bundles. Consider the product bundle $\xi \times \eta \to X
    \times Y$, and let $\pi_\xi \c \xi \times \eta \to \xi$ and
    $\pi_\eta \c \xi \times \eta \to \eta$ be the projections. Then
    \[
    u_{\xi \times \eta} = \pi_\xi^*(u_\xi) \cdot \pi_\eta^*(u_\eta).
    \]
    By naturality this implies the same multiplicativity when $X=Y$
    and we replace $\xi \times \eta$ with $\xi \oplus \eta$.
  \end{enumerate}
\end{proposition}

\begin{proof}
  This is achieved by the ``difference bundle'' construction of
  \cite{abs-clifford}, but this topic is omitted here, unfortunately.
\end{proof}

\begin{example}
  \label{taut-thom}
  Let $\xi \to \CP^\infty$ be the tautological line bundle. We claim
  that in fact $\Th(\xi) \simeq \CP^\infty$. Indeed since $S(\xi)
  \simeq S^\infty$ is contractible, we have $\Th(\xi) \simeq
  D(\xi)/S(\xi) \simeq D(\xi) \simeq \CP^\infty$.\footnote{I learned
    this argument from \cite{may-mu1}.} It is easy to see then that
  the $K_\C^*(\CP^{\infty})$-module structure on $\K^*_\C(\Th(\xi))$
  is just given by multiplication in $K_\C^*(\CP^{\infty})$, and
  hence the Thom class $u_\xi \in \til K_\C^2(\Th(\xi)) \simeq
  \til K_\C(\CP^\infty)$ must be a generator $\pm(\xi-1)$.
\end{example}

%%%%%%%%%%%%%%%%%%%%%%%%%%%%%%%%%%%%%%%%%%%%%%%%%%%%%%%%%%%%%%%%%%%%%%

\section{Characteristic classes}

We now begin our quest to understand $J_\C(\RP^k)$. This will fall
out of Adams's more general results on the J-homomorphism, namely, the
construction of a certain quotient $J_\Lambda(X) \surj
J'_\Lambda(X)$, a ``lower bound'' on $J_\Lambda(X)$. We begin with a
general discussion of characteristic classes, which are used to define
the group $J'_\Lambda(X)$.

\renewcommand{\V}{\mathcal{V}}
\newcommand{\cl}{\mathrm{cl}}
\begin{nothing}
  \label{chclass-general}
  Let $E$ and $F$ be multiplicative cohomology theories. Consider the
  following data.
  \begin{enumerate}
  \item Let $\V$ be some class of vector bundles $\xi \to X$ equipped
    with natural\footnote{Here ``natural'' is used in the same sense
      as \eqref{thom-K}.} Thom classes $u_\xi$ in $E$ and $t_\xi$ in
    $F$.
  \item Let $T \c E \to F$ be a natural transformation of cohomology
    theories.
  \end{enumerate}
  Then for any $\xi \to X$ in $\V$ we can form a \emph{characteristic
    class}
  \[
  \cl(T,\xi) \ce \psi_\xi^{-1}T\phi_\xi(1) = T(u_\xi)/t_\xi \in
  F^*(X),
  \]
  where $\phi_\xi,\psi_\xi$ denote the Thom isomorphisms in $E,F$
  respectively. Note by naturality of our Thom classes, $\cl(T,\xi)$
  is natural in bundle maps $f^*\xi \to \xi$.
\end{nothing}

\begin{example}
  \label{stiefel-whitney}
  If we take $E = F = H^*(-;\Z/2)$ in \eqref{chclass-general} then we
  have natural orientations on all vector bundles. It turns out that
  if we set $T$ to be the Steenrod square $\Sq^i$ then the resulting
  characteristic class $\cl(\Sq^i,-)$ is just the Stiefel-Whitney
  class $w_i$.
\end{example}

\begin{definition}
  Take $E = F = K_\C^*$ and $T = \psi^l$ the Adams operation for $l
  \in \N$ in \eqref{chclass-general}. The resulting classes $\rho^l \ce
  \cl(\psi^l,-)$ are called the \emph{cannibalistic
    classes}.\footnote{Because they live in K-theory and also eat
    things in K-theory, and because Adams was awesome.}
\end{definition}

\begin{remark}
  By \eqref{thom-K} the cannibalistic classes $\rho^l$ are defined on
  all complex vector bundles, and moreover satisfy the exponential
  property
  \begin{equation}
    \label{exponential}
    \rho^l(\xi \oplus \eta) = \rho^l(\xi)\rho^l(\eta).
  \end{equation}
\end{remark}

\begin{convention}
  \label{complex-convention}
  For the remainder, all bundles are complex vector bundles, and
  dimension always refers to complex dimension.
\end{convention}

\begin{lemma}
  \label{cannibal-line}
  Let $\xi \to X$ be a line bundle. Then $\rho^l(\xi) = 1 + \xi +
  \cdots + \xi^{l-1}$ for $l \in \N$.
\end{lemma}

\begin{proof}
  By naturality it suffices to prove this for the universal line
  bundle $\xi \to \CP^{\infty}$. By \eqref{taut-thom}, $\Th(\xi) \simeq
  \CP^\infty$ with Thom class $\pm(\xi - 1) \in
  \til K_\C(\CP^\infty)$. Then by definition of $\rho^l$ and $\psi^l$
  we have
  \[
  \rho^l(\xi) = \frac{\psi^l(\pm(\xi-1))}{\pm(\xi-1)} = \frac{\xi^l -
    1}{\xi - 1} = 1 + \xi + \cdots + \xi^{l-1}. \qedhere
  \]
\end{proof}

\begin{remark}
  \label{cannibal-dim}
  By the splitting principle, \eqref{exponential} and
  \eqref{cannibal-line} imply that $\dim \rho^l(\xi) = l^{\dim \xi}$
  for any bundle $\xi \to X$.
\end{remark}

\begin{nothing}
  \label{cannibal-extend}
  Combining \eqref{exponential} and \eqref{cannibal-line} gives
  $\rho^l(\epsilon_\C^{\oplus r}) = l^r$. This tells us that, after
  inverting $l$, we can extend the definition of $\rho^l$ from bundles
  to all of $K_\C(X)$. Namely, since any element of $K_\C(X)$ can be
  written in the form $\xi - r$ for some bundle $\xi$, we can define
  \[
  \rho^l \c K_\C(X) \to K_\C(X)_l, \quad \xi - r \mapsto
  \rho^l(\xi)/l^r,
  \]
  where $K_\C(X)_l$ denotes the localisation at $l$. Since $\xi - r =
  \eta - s \iff \xi + s = \eta + r$, this is obviously well
  defined. And note we have preserved the exponential property:
  $\rho^l(x+y) = \rho^l(x)\rho^l(y)$ for $x,y \in K_\C(X)$.
\end{nothing}

We next show how the cannibalistic classes $\rho^l$ allows us to bound
$J_\C(X)$.

\begin{lemma}
  \label{jprime-condition}
  Suppose $J_\C(\xi) = J_\C(\eta)$ for two bundles $\xi \to X$ and
  $\eta \to X$. Then there exists $y \in \til K_\C(X)$ such that $1+y
  \in K_\C(X)$ is invertible and
  \[
  \rho^l(\eta) = \rho^l(\xi) \cdot \frac{\psi^l(1+y)}{1+y} \quad
  \text{for all } l \in \N.
  \]
\end{lemma}

\begin{proof}
  The hypothesis is that there is a fibre homotopy equivalence
  $f \c S(\xi) \to S(\eta)$ over $X$. This then extends to a homotopy
  equivalence of Thom complexes $g \c \Th(\xi) \to \Th(\eta)$. On each
  fibre, $g$ induces a map $g_p \c \Th(\xi_p) \to \Th(\eta_p)$ which
  is just (homotopic to) the suspension of
  $f_p \c S(\xi_p) \to S(\eta_p)$; since $f_p$ is a homotopy
  equivalence so is $g_p$.

  Let $v \ce \phi_\xi^{-1}g^*\phi_\eta(1) = g^*(u_\eta)/u_\xi \in
  K_\C(X)$. Since $g_p$ is a homotopy equivalence,
  \[
  g_p^*(u_{\eta_p}) = \pm u_{\xi_p} \quad \text{for }p \in X.
  \]
  It follows that $\dim v = \pm 1$. Let $\epsilon \ce \dim v \in
  K_\C(X)$, so that $\dim \epsilon v = 1$, and hence $\epsilon v = 1
  + y$ for some $y \in \til K_\C(X)$.

  Let $h \c \Th(\eta) \to \Th(\xi)$ be a homotopy inverse to $g$;
  define $w \ce \phi_\eta^{-1}h^*\phi_\xi(1) \in K_\C(X)$. Since
  $\phi_\xi,\phi_\eta$ are isomorphisms of $K_\C(X)$-modules and
  pullback respects multiplication, we have
  \[
  vw = \phi_\xi^{-1}g^*\phi_\eta(1) \cdot
  \phi_\eta^{-1}h^*\phi_\xi(1) =
  \phi_\eta^{-1}h^*\phi_\xi(\phi_\xi^{-1}g^*\phi_\eta(1)) = 1,
  \]
  and symmetrically $wv = 1$. So $w$ is inverse to $v$, and hence
  $\epsilon w$ is inverse to $\epsilon v$, implying $1+y$ is
  invertible.

  Finally let $l \in \N$. By naturality of the Adams operation
  $\psi^l$ we have
  \[
  ((\phi_\xi^{-1}g^*\phi_\eta)(\phi_\eta^{-1}\psi^l\phi_\eta))(1) =
  ((\phi_\xi^{-1}\psi^l\phi_\xi)(\phi_\xi^{-1}g^*\phi_\eta))(1).
  \]
  Then by definition of $\rho^l$, multplicativity of $\psi_l$, and the
  fact that $\phi_\xi$ is an isomorphism of $K_\C(X)$-modules, we get
  \[
  v \cdot \rho^l(\eta) = \phi_\xi^{-1}(\psi^l(v \cdot u_\xi)) =
  \phi_\xi^{-1}(\psi^l(u_\xi)) \cdot \psi^l(v) = \rho^l(\xi) \cdot
  \psi^l(v).
  \]
  Now, multiplying this equation by $\epsilon = \psi^l(\epsilon)$ we
  get
  \[
  (1+y) \cdot \rho^l(\eta) = \rho^l(\xi) \cdot \psi^l(1+y),
  \]
  and since $1+y$ is invertible we are done.
\end{proof}

\begin{definitions}
  \label{jprime}
  \begin{enumerate}[leftmargin=*]
  \item Let $V_\C(X) \subseteq K_\C(X)$ be the subgroup of elements
    $x$ for which there exists $y \in \til K_\C(X)$ such that $1+y$ is
    invertible and
    \begin{equation}
      \label{jprime-y}
      \rho^l(x) = \frac{\psi^l(1+y)}{1+y} \in K_\C(X)_l \quad \text{for
      } l \in \N.
    \end{equation}
    That $V_\C(X)$ is in fact a subgroup
    follows from the fact that each $\rho^l$ is exponential and each
    $\psi^l$ is multiplicative.
  \item Define $J_\C'(X) \ce K_\C(X)/V_\C(X)$.
  \end{enumerate}
\end{definitions}

\begin{lemma}
  \label{jprime-dim}
  $V_\C(X) \subseteq \til K_\C(X)$.
\end{lemma}

\begin{proof}
  Suppose $x \in V_\C(X)$, and let $y$ is as in \eqref{jprime}. Writing
  $x = \xi - \eta$ for bundles $\xi$ and $\eta$, we must have
  \[
  \rho^l(x) = \frac{\psi^l(1+y)}{1+y} \in K_\C(X)_l \implies
  l^r(1+y)\rho^l(\xi) = l^r\psi^l(1+y)\rho^l(\eta) \in K_\C(X).
  \]
  The Adams operations preserve dimension (by the splitting principle
  and their definition), so $\dim \psi^l(1+y) = \dim {(1+y)}$. It
  follows that $\dim \rho^l(\xi) = \dim \rho^l(\eta)$, whence $\dim
  \xi = \dim \eta$ by \eqref{cannibal-dim}. Therefore $x \in
  \til K_\C(X)$.
\end{proof}

\begin{proposition}
  \label{jprime-bound}
  The quotient map $K_\C(X) \to J_\C'(X)$ factors through
  $J_\C$. That is, $\ker(J_\C) \subseteq V_\C(X)$, so $J_\C'(X)$
  is a quotient of $J_\C(X)$.
\end{proposition}

\begin{proof}
  Immediate from \eqref{jprime-condition}.
\end{proof}

%%%%%%%%%%%%%%%%%%%%%%%%%%%%%%%%%%%%%%%%%%%%%%%%%%%%%%%%%%%%%%%%%%%%%%

\section{Finishing}

We now specialise to the case $X = \RP^k$. Computing $J_\C'(\RP^k)$
will give us enough information about $J_\C(\RP^k)$ to finally give
an upper bound to our question \eqref{vfield-prob} on vector fields.

\begin{lemma}
  \label{RP-cannibal}
  Let $\nu$ be the generator of $\til K_\C(\RP^k)$ as in
  \eqref{RP-kthy}. For $l \in \N$ odd we have
  \[
  \rho^l(t\nu) = 1 + \frac{l^t-1}{2l^t}\nu \quad\text{for }0 \le t <
  2^{f_k}.
  \]
\end{lemma}

\begin{proof}
  Let $\xi \ce \kappa(\gamma)$. By \eqref{cannibal-line} and the
  identity $\xi^2 = 1$ we have
  \[
  \rho^l(\xi) = 1 + \xi + \cdots + \xi^{l-1} = \frac{l+1}2 +
  \frac{l-1}2\xi = l + \frac{l-1}{2}\nu.
  \]
  The desired identity for $t=1$ then follows from $\nu = \xi - 1
  \implies \rho^l(\nu) = \rho^l(\xi)/l$. Then $t > 1$ follows by
  induction using the relation $\nu^2 = -2\nu$ (and $t=0$ is trivial).
\end{proof}

\begin{lemma}
  \label{RP-jprime}
  $V_\C(\RP^k) \subseteq \{0,2^{f_k-1}\nu\}$.
\end{lemma}

\begin{proof}
  Let $x \in V_\C(\RP^k)$, and let $y \in \til K_\C(\RP^k)$ be such
  that \eqref{jprime-y} holds. By \eqref{RP-kthy} we have $\psi^l(1+y) =
  1+y$ for $l$ odd, so we in fact have $\rho^l(x) = 1$ for $l$ odd.

  Next, $x \in \til K_\C(X)$ by \eqref{jprime-dim}, so by \eqref{RP-kthy}
  we can write $x = t\nu$ for some $0 \le t < 2^{f_k}$. Then from
  \eqref{RP-cannibal} we know that
  \[
  \rho^l(x) = \rho^l(t\nu) = 1 + \frac{l^t-1}{2l^t}\nu.
  \]
  Recall the ring morphism $\alpha \c \til K_\C(\RP^k) \to
  \Z/2^{f_k+1}$ defined in \eqref{RP-ringhom} by $\alpha(\nu) \ce
  -2$. Observe this gives a morphism of multiplicative groups $\beta
  \c 1 + \til K_\C(\RP^k) \to (\Z/2^{f_k+1})^\times$ via $\beta(1+z)
  \ce 1+\alpha(z)$.

  So finally let $l$ be odd. Note $\til K_\C(\RP^k) \simeq \Z/2^{f_k}
  \implies \til K_\C(\RP^k)_l \simeq \til K_\C(\RP^k)$. Thus from the
  above we get
  \[
  1 = \rho^l(x) \implies 1 = \beta\l(1 + \frac{l^t-1}{2l^t}\nu\r) =
  1/l^t.
  \]
  We now use the fact that $(\Z/2^{f_k+1})^\times \simeq \Z/2 \times
  \Z/2^{f_k-1}$ has an element of order $2^{f_k-1}$ to see that
  choosing $l$ appropriately implies $2^{f_k-1} \mid t$, as desired.
\end{proof}

\begin{theorem}
  \label{complex-upper}
  There does not exist a set of $\rho(n) + 4$ linearly independent
  vector fields on $S^{n-1}$.
\end{theorem}

\begin{proof}
  If $n$ is odd this is trivial by the hairy ball theorem
  \eqref{hairy-ball}. So assume $n = 2m$, and suppose $S^{n-1}$ admits
  a set of $k$ vector fields. Combining \eqref{j-reduction},
  \eqref{jprime-bound}, and \eqref{RP-jprime} gives that $2^{f_k-1} \mid
  m \implies 2^{f_k} \mid n$. Then observe that $f_k \ge e_k - 1 \ge
  e_{k-4}$. By \eqref{radon-max} it follows that $k-4 \le \rho(n) - 1
  \implies k \le \rho(n) + 3$.
\end{proof}

So ends our journey.

\begin{remark}
  Our final upper bound in \eqref{complex-upper} is off by $4$ from the
  right answer \eqref{vfield-upper-bound}. If we were computer
  scientists, that would be good enough. As remarked above, that $4$
  goes away if one translates our work from complex K-theory into real
  K-theory. Indeed one can define $\rho^l$ using the Adams operations
  in $K_\R$ rather than $K_\C$, and then define analogous groups
  $V_\R(X)$ and $J'_\R(X)$. But there is a bit of nastiness:
  \begin{itemize}
  \item Thom classes don't exist for all real vector bundles, so the
    analogue of \eqref{thom-K} is more subtle. However, in
    \cite{abs-clifford} natural Thom clases are also constructed for
    certain $\Spin$ bundles, and this is where one must begin.
  \item This means our proof of \eqref{cannibal-line} won't go through
    in the real case, and indeed this implies that extending $\rho^l$
    to $K_\R(X)$ is much more technical than the simple discussion of
    \eqref{cannibal-extend}.
  \end{itemize}
  These subtleties are handled in \cite{adams-J-II}, and the reward is
  the correct bound. The argument is essentially the same: one uses
  $J_\R(n\lambda) = 0$ from \eqref{j-reduction}, and this time computes
  $V_\R(\RP^k) \simeq 0$, so $K_\R(\RP^k) \simeq J_\R(\RP^k) \simeq
  J_\R'(\RP^k)$. The crucial difference is that $\til K_\R(\RP^k)
  \simeq \Z/2^{e_k}$, so now we get $2^{e_k} \mid n$ rather than
  $2^{f_k} \mid n$ when proving the upper bound
  \eqref{complex-upper}. Getting precisely $e_k$ instead of the
  approximation $f_k$ means that no pesky $4$ will show up.
\end{remark}

\begin{remark}
  In addition to computing $J_\R'(\RP^k)$, Adams computes $J_\R'$
  for spheres as well, by using another characteristic class coming
  from the formalism of \eqref{chclass-general}. This gives bounds on
  the image of the J-homomorphism for spheres, where the study of the
  J-homomorphism originated.
\end{remark}

%%%%%%%%%%%%%%%%%%%%%%%%%%%%%%%%%%%%%%%%%%%%%%%%%%%%%%%%%%%%%%%%%%%%%%

\nocite{mathew-J, shah-vfields}
\bibliographystyle{amsalpha}
\bibliography{refs}

\end{document}
