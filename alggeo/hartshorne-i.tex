%%%%%%%%%%%%%%%%%%%%%%%%%%%%%%%%%%%%%%%%%%%%%%%%%%%%%%%%%%%%%%%%%%%%%%

\renewcommand{\A}{\mathbb{A}}
\renewcommand{\O}{\mathcal{O}}

\renewcommand{\a}{\mathfrak{a}}
\newcommand{\p}{\mathfrak{p}}
\newcommand{\q}{\mathfrak{q}}

\newcommand{\height}{\operatorname{ht}}

%%%%%%%%%%%%%%%%%%%%%%%%%%%%%%%%%%%%%%%%%%%%%%%%%%%%%%%%%%%%%%%%%%%%%%

%%%%%%%%%%%%%%%%%%%%%%%%%%%%%%%%%%%%%%%%%%%%%%%%%%%%%%%%%%%%%%%%%%%%%%%

\renewcommand{\A}{\mathbb{A}}
\renewcommand{\O}{\mathcal{O}}

\renewcommand{\a}{\mathfrak{a}}
\newcommand{\p}{\mathfrak{p}}
\newcommand{\q}{\mathfrak{q}}

\newcommand{\height}{\operatorname{ht}}

%%%%%%%%%%%%%%%%%%%%%%%%%%%%%%%%%%%%%%%%%%%%%%%%%%%%%%%%%%%%%%%%%%%%%%


%%%%%%%%%%%%%%%%%%%%%%%%%%%%%%%%%%%%%%%%%%%%%%%%%%%%%%%%%%%%%%%%%%%%%%

\title{Solutions to Hartshorne:\\Chapter I, Varieties}
\author{Arpon Raksit}
\date{December 19, 2013 (original); \today\ (last edit)}

\begin{document}
\maketitle
\thispagestyle{fancy}

%%%%%%%%%%%%%%%%%%%%%%%%%%%%%%%%%%%%%%%%%%%%%%%%%%%%%%%%%%%%%%%%%%%%%%

\section{Affine varieties}

%% 1.1
\begin{nothing}
  (a) We have $A(Y) \simeq k[x,y]/(y-x^2)$. The maps $k[x,y]/(y-x^2)
  \to k[t]$ sending $(x,y) \mapsto (t,t^2)$ and $k[t] \to
  k[x,y]/(y-x^2)$ sending $t \mapsto x$ are inverse isomorphisms.

  \medskip\noindent
  (b) We have $A(Y) \simeq k[x,y]/(xy-1)$. Any map of $k$-algebras
  $k[x,y]/(xy-1) \to k[t]$ must send $x$ to a unit, hence to an
  element of $k$, so cannot be injective.

  \medskip\noindent
  (c) Skipped.
\end{nothing}

%% 1.2
\begin{nothing}
  Let $I \coloneqq (z-x^3,y-x^2) \subseteq k[x,y,z]$. Then clearly
  $V(I) = Y$. Similar to 1.1a, we have an isomorphism $k[x,y,z]/I \to
  k[t]$ sending $(x,y,z) \mapsto (t,t^2,t^3)$. Since $k[t]$ is a
  domain of Krull dimension 1, this implies that $Y$ is an affine
  variety of dimension 1, with $A(Y) \simeq k[t]$.
\end{nothing}

%% 1.3
\begin{nothing}
  Factoring $xz-x = x(z-1)$, we have $(x,y,z) \in V(x^2-yz,xz-x)$ if
  and only if one of the following holds:
  \begin{enumerate}
  \item $x = 0$ and $yz = 0$, i.e., $y = 0$ or $z = 0$;
  \item $z = 1$ and $y = x^2$.
  \end{enumerate}
  Thus if we define $Y_1 \coloneqq V(x,y)$, $Y_2 \coloneqq V(x,z)$,
  and $Y_3 \coloneqq V(z-1,y-x^2)$ we have $Y = Y_1 \cup Y_2 \cup
  Y_3$. Now observe
  \begin{align*}
  A(Y_1) &\simeq k[x,y,z]/(x,y) \simeq k[t], \\
  A(Y_2) &\simeq k[x,y,z]/(x,z) \simeq k[t], \\
  A(Y_3) &\simeq k[x,y,z]/(z-1,y-x^2) \simeq k[x,y]/(y-x^2) \simeq
  k[t],
  \end{align*}
  with the last isomorphism by 1.1a. This implies $Y_1,Y_2,Y_3$ are
  irreducible. Finally, it's evident that none of them contains
  another.
\end{nothing}

%% 1.4
\begin{nothing}
  
\end{nothing}

%%%%%%%%%%%%%%%%%%%%%%%%%%%%%%%%%%%%%%%%%%%%%%%%%%%%%%%%%%%%%%%%%%%%%%

\bibliographystyle{amsalpha}
\bibliography{refs}

\end{document}
