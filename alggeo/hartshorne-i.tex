%%%%%%%%%%%%%%%%%%%%%%%%%%%%%%%%%%%%%%%%%%%%%%%%%%%%%%%%%%%%%%%%%%%%%%

\newcommand{\ob}{\oper{ob}}
\renewcommand{\hom}{\oper{hom}}
\newcommand{\id}{\oper{id}}
\newcommand{\im}{\oper{im}}
\newcommand{\op}{\oper{op}}

\newcommand{\Top}{\oper{Top}}
\newcommand{\Set}{\oper{Set}}
\newcommand{\Ab}{\oper{Ab}}
\newcommand{\Grp}{\oper{Grp}}
\newcommand{\Mod}{\oper{Mod}}
\newcommand{\Simplex}{\Delta}
\newcommand{\s}{\oper{s}}
\newcommand{\Ch}{\oper{Ch}}

\newcommand{\Sing}{\oper{Sing}}
\renewcommand{\H}{\mathrm{H}}

%%%%%%%%%%%%%%%%%%%%%%%%%%%%%%%%%%%%%%%%%%%%%%%%%%%%%%%%%%%%%%%%%%%%%%

%%%%%%%%%%%%%%%%%%%%%%%%%%%%%%%%%%%%%%%%%%%%%%%%%%%%%%%%%%%%%%%%%%%%%%

\newcommand{\ob}{\oper{ob}}
\renewcommand{\hom}{\oper{hom}}
\newcommand{\id}{\oper{id}}
\newcommand{\im}{\oper{im}}
\newcommand{\op}{\oper{op}}

\newcommand{\Top}{\oper{Top}}
\newcommand{\Set}{\oper{Set}}
\newcommand{\Ab}{\oper{Ab}}
\newcommand{\Grp}{\oper{Grp}}
\newcommand{\Mod}{\oper{Mod}}
\newcommand{\Simplex}{\Delta}
\newcommand{\s}{\oper{s}}
\newcommand{\Ch}{\oper{Ch}}

\newcommand{\Sing}{\oper{Sing}}
\renewcommand{\H}{\mathrm{H}}

%%%%%%%%%%%%%%%%%%%%%%%%%%%%%%%%%%%%%%%%%%%%%%%%%%%%%%%%%%%%%%%%%%%%%%


%%%%%%%%%%%%%%%%%%%%%%%%%%%%%%%%%%%%%%%%%%%%%%%%%%%%%%%%%%%%%%%%%%%%%%

\title{Solutions to Hartshorne:\\Chapter I, Varieties}
\author{Arpon Raksit}
\date{December 19, 2013 (original); \today\ (last edit)}

\begin{document}
\maketitle
\thispagestyle{fancy}

\section*{Introduction}

As the title suggests, these are some (hopefully correct) solutions to
exercises in the book \cite[Ch. I]{hartshorne}.

\begin{convention}
  References in parthentheses are to \emph{exercises} and references
  in brackets are to \emph{results} in the text. E.g., ``(2.3)''
  refers to Exercise 2.3, but ``[2.3]'' refers to Corollary 2.3.
\end{convention}

%%%%%%%%%%%%%%%%%%%%%%%%%%%%%%%%%%%%%%%%%%%%%%%%%%%%%%%%%%%%%%%%%%%%%%

\section{Affine varieties}

%% 1.1
\begin{nothing}
  (a) We have $A(Y) \simeq k[x,y]/(y-x^2)$. The maps $k[x,y]/(y-x^2)
  \to k[t]$ sending $(x,y) \mapsto (t,t^2)$ and $k[t] \to
  k[x,y]/(y-x^2)$ sending $t \mapsto x$ are inverse isomorphisms.

  \medskip\noindent
  (b) We have $A(Y) \simeq k[x,y]/(xy-1)$. Any map of $k$-algebras
  $k[x,y]/(xy-1) \to k[t]$ must send $x$ to a unit, hence to an
  element of $k$, so cannot be injective.

  \medskip\noindent
  (c) Skipped. (Pretty sure you just apply some coordinate changes to
  reduce the general quadratic equation.)
\end{nothing}

%% 1.2
\begin{nothing}
  Let $I \coloneqq (z-x^3,y-x^2) \subseteq k[x,y,z]$. Then clearly
  $V(I) = Y$. Similar to (1.1a), we have an isomorphism $k[x,y,z]/I
  \to k[t]$ sending $(x,y,z) \mapsto (t,t^2,t^3)$. Since $k[t]$ is a
  domain of Krull dimension 1, this implies that $Y$ is an affine
  variety of dimension 1, with $A(Y) \simeq k[t]$.
\end{nothing}

%% 1.3
\begin{nothing}
  Factoring $xz-x = x(z-1)$, we have $(x,y,z) \in V(x^2-yz,xz-x)$ if
  and only if one of the following holds:
  \begin{enumerate}
  \item $x = 0$ and $yz = 0$, i.e., $y = 0$ or $z = 0$;
  \item $z = 1$ and $y = x^2$.
  \end{enumerate}
  Thus if we define $Y_1 \coloneqq V(x,y)$, $Y_2 \coloneqq V(x,z)$,
  and $Y_3 \coloneqq V(z-1,y-x^2)$ we have $Y = Y_1 \cup Y_2 \cup
  Y_3$. Now observe
  \begin{align*}
  A(Y_1) &\simeq k[x,y,z]/(x,y) \simeq k[t], \\
  A(Y_2) &\simeq k[x,y,z]/(x,z) \simeq k[t], \\
  A(Y_3) &\simeq k[x,y,z]/(z-1,y-x^2) \simeq k[x,y]/(y-x^2) \simeq
  k[t],
  \end{align*}
  with the last isomorphism by (1.1a). This implies $Y_1,Y_2,Y_3$ are
  irreducible. Finally, it's evident that none of them contains
  another.
\end{nothing}

%% 1.4
\begin{nothing}
  Closed sets in the product topology on $\A^1 \times \A^1$ are
  intersections of sets of the form $X \times Y$ where $X,Y \subseteq
  \A^1$ are closed, i.e., either the whole space or finite unions of
  points and lines parallel to the axes. This of course does not
  include all the Zariski closeds of $\A^2$, e.g., the diagonal $y=x$.

  Note though that the Zariski topology is strictly finer than the
  product topology. The intuition here is that the closeds of $\A^2$
  which come from the product topology are those defined by
  polynomials in which ``$x$ and $y$ are independent'', e.g., $y-1$
  or $x(y-1)$ or $(x,y-1)$.
\end{nothing}

%% 1.5
\begin{nothing}
  The ``if'' direction is clear since $I(X)$ is reduced for any
  algebraic set $X \subseteq \A^n$. Conversely if $B$ is a finitely
  generated $k$-algebra without nilpotents then $B \simeq
  k[x_1,\ldots,x_n]/I$ with $I$ reduced. By the Nullstellensatz,
  $I(V(I)) = I$ so then $B \simeq A(V(I))$.
\end{nothing}

%% 1.6
\begin{nothing}
  Let $X$ be a topological space. By taking complements, the condition
  that no two proper closeds in $X$ union to $X$ is equivalent to the
  condition that the condition that no two nonempty opens in $X$ have
  empty intersection. This implies that $X$ is irreducible if and only
  if all nonempty open sets are dense.

  If $X$ is irreducible and $U \subseteq X$ is open, then open subsets
  of $U$ are also open subsets of $X$, so by the above
  characterisation we see that $U$ is also irreducible.

  \medskip\noindent
  Now suppose $Y \subseteq X$. By definition of closure, if $Z$ is a
  proper closed subset of $\oline Y$ then $Z \cap Y$ is a proper
  closed subset of $Y$. It follows that if $\oline Y$ is reducible
  then $Y$ is reducible.
\end{nothing}

\renewcommand{\Y}{\mathcal{Y}}

%% 1.7
\begin{nothing}
  (a, i $\Rightarrow$ ii) Let $\Y$ be a family of closed subsets of
  $X$. If $\Y$ is nonempty then we can choose some $Y_1 \in
  \Y$. Inductively, if $Y_i$ is not minimal then we can choose
  $Y_{i+1} \in \Y$ with $Y_{i+1} \subsetneq Y_i$. If $X$ is
  noetherian, this cannot occur for all $i \ge 1$, whence $\Y$ has a
  minimal element.

  \medskip\noindent
  (a, ii $\Rightarrow$ i) Let $Y_1 \supseteq Y_2 \supseteq \cdots$ be
  a descending chain of closeds in $X$. The family $\{Y_i\}_{i \ge 1}$
  must have a minimal element, whence the chain stabilises. Thus $X$
  is noetherian.

  \medskip\noindent
  (a, i $\Leftrightarrow$ iii and ii $\Leftrightarrow$ iv) Take
  complements.

  \medskip\noindent
  (b) Let $\{U_i\}_{i \in I}$ be an open cover of a noetherian space
  $X$. The family $\{\bigcup_{j \in J} U_j \mid J \subseteq I \text{
    finite}\}$ has a maximal element $U$ by (a, iv). If $U \ne X$ then
  there exists $i \in I$ such that $U \cup U_i \supsetneq U$,
  contradicting maximality. Thus $U = X$, giving a finite subcover.

  \medskip\noindent
  (c) Let $X$ be a noetherian space and $Y \subseteq X$. Let $Z_1
  \supseteq Z_2\supseteq \cdots$ a chain of closeds in $Y$. Taking
  closures in $X$ gives a chain $\oline{Z_1} \supseteq \oline{Z_2}
  \supseteq \cdots$ of closeds in $X$, which must stabilise since $X$
  is noetherian. But then since each $Z_i$ is closed in $Y$, we have
  $Z_i = \oline{Z_i} \cap Y$, so this implies the chain in $Y$ also
  stabilises. Hence $Y$ is noetherian as well.

  \medskip\noindent
  (d) Let $Y$ be a hausdorff space. Suppose $Y$ has two distinct
  points $x,y$. Choose disjoint open neighbourhoods $U \ni x$ and $V
  \ni y$. Then $Y-U$ and $Y-V$ are proper closeds which union to
  $Y$, so $Y$ is reducible.

  Now let $X$ be a noetherian hausdorff space. Let $X =
  \bigcup_{i=1}^n X_1$ be the decomposition into irreducible
  components. Each $X_i$ is hausdorff, so by the above has exactly one
  point. Each of these points must be closed, and hence also open,
  giving the claim.
\end{nothing}

%% 1.8
\begin{nothing}
  Let $A \coloneqq k[x_1,\ldots,x_n]$. Suppose $H = V(f)$ and $Y =
  V(\p)$, where $\p \subseteq A$ is prime and $f \in A - \p$. Then $Y
  \cap H = V(\p + (f))$, so an irreducible component $Z$ of $Y \cap H$
  correponds to a minimal prime in $A$ containing $\p + (f)$, or
  equivalently, to a minimal prime $\q$ in $A/\p$ containing $f$. By
  the Hauptidealsatz, $\height(\q) = 1$, so then we have
  \[
  \height(\q) + \dim((A/\p)/\q) = \dim(A/\p) \implies \dim(Z) =
  \dim(Y) - 1.
  \]
\end{nothing}

%% 1.9
\begin{nothing}
  We proceed inductively. The base case $r=0$ is trivial. Now write
  $\a = (f_1,\ldots,f_r)$, and let $\a' \coloneqq
  (f_2,\ldots,f_r)$. By induction every irreducible component of
  $V(\a')$ has dimension at least $n-r+1$. If $f_1 \in \a'$ then
  $V(\a) = V(\a')$ so we're done. Else, we can apply (1.8) to finish.
\end{nothing}

%% 1.10
\begin{nothing}
  (a) Let $Z$ be a closed in $Y$ and $\oline Z$ its closure in $X$. As
  noted in (1.7c) we have $Z = \oline Z \cap Y$. Any by (1.6), if $Z$
  is irreducible then so is $\oline Z$. It follows that if $Z_0
  \subsetneq Z_1 \subsetneq \cdots \subsetneq Z_n$ is a chain of
  irreducible closeds in $Y$ of length $n$, then $\oline{Z_0}
  \subsetneq \oline{Z_1} \subsetneq \cdots \subsetneq \oline{Z_n}$ is
  a chain of irreducible closeds in $X$ of length $n$. Hence $\dim(X)
  \ge \dim(Y)$.

  \medskip\noindent
  (b) By (a), $\dim(X) \ge \dim(U_i)$ for all $i$ so certainly
  $\dim(X) \ge \sup \dim(U_i)$. Conversely, suppose $Z_0 \subsetneq
  Z_1 \subsetneq \cdots \subsetneq Z_n$ is a chain of closed
  irreducibles in $X$. Fix any $i$. Then for $0 \le j \le n$, define
  $Z_{i,j} \coloneqq U_i$. Then $Z_{i,j}$ is an open subset of $Z_j$
  and hence dense and irreducible by (1.6). It follows that $Z_{i,0}
  \subsetneq Z_{i,1} \subsetneq \cdots \subsetneq Z_{i,n}$ is a chain
  of irreducible closeds in $U_i$. Since each of these must stabilise
  at length $\dim(U_i)$ and $\{U_i\}$ covers $X$, we get that $\dim(X)
  \le \sup \dim(U_i)$.

  \medskip\noindent
  (c) Consider $X \coloneqq \A^1$ and $U \coloneqq \A^1 -
  \{0\}$. Certainly $U$ is a dense open subset of $X$.

  \medskip\noindent
  (d) If $Y \ne X$ then for any chain $Z_0 \subsetneq Z_1 \subsetneq
  \cdots \subsetneq Z_n$ of irreducible closeds in $Y$, since $Y$ is
  closed in $X$ and $X$ is irreducible, $Z_0 \subsetneq Z_1 \subsetneq
  \cdots \subsetneq Z_n \subsetneq X$ is a chain of irreducible
  closeds in $X$. Since $X$ is finite-dimensional, so is $Y$ by (a),
  so the above implies $\dim(X) \ge \dim(Y) + 1$.

  \medskip\noindent
  (e) 
\end{nothing}

%%%%%%%%%%%%%%%%%%%%%%%%%%%%%%%%%%%%%%%%%%%%%%%%%%%%%%%%%%%%%%%%%%%%%%

\bibliographystyle{amsalpha}
\bibliography{refs}

\end{document}
