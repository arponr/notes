
%%%%%%%%%%%%%%%%%%%%%%%%%%%%%%%%%%%%%%%%%%%%%%%%%%%%%%%%%%%%%%%%%%%%%%

\newcommand{\ob}{\oper{ob}}
\renewcommand{\hom}{\oper{hom}}
\newcommand{\id}{\oper{id}}
\newcommand{\im}{\oper{im}}
\newcommand{\op}{\oper{op}}

\newcommand{\Top}{\oper{Top}}
\newcommand{\Set}{\oper{Set}}
\newcommand{\Ab}{\oper{Ab}}
\newcommand{\Grp}{\oper{Grp}}
\newcommand{\Mod}{\oper{Mod}}
\newcommand{\Simplex}{\Delta}
\newcommand{\s}{\oper{s}}
\newcommand{\Ch}{\oper{Ch}}

\newcommand{\Sing}{\oper{Sing}}
\renewcommand{\H}{\mathrm{H}}

%%%%%%%%%%%%%%%%%%%%%%%%%%%%%%%%%%%%%%%%%%%%%%%%%%%%%%%%%%%%%%%%%%%%%%


\title{The Cartier isomorphism and Hodge degeneration}
\author{Arpon Raksit}
\date{2017-04-18}

\numberwithin{block}{section}
% \makeatletter
% \let\c@equation\c@subblock
% \makeatother

\begin{document}
\maketitle

\newcommand{\an}{\mathrm{an}}
\newcommand{\chr}{\operatorname{char}}
\newcommand{\dR}{\mathrm{dR}}

% ---------------------------------------------------------------------

\section{Introduction}
\label{intro}

\begin{nothing}
  \label{intro-dr}
  Let $X$ be a scheme over a field $\kappa$. As Dan discussed last time, we have versions of differential forms and the de Rham complex in this completely algebraic setting. Let me first recall this construction.

  \begin{subdefinition}
    \label{intro-dr-differentials}
    The \emph{sheaf of K\"ahler differentials} $\Omega^1_{X/\kappa}$ on $X$ is a quasicoherent $\sO_X$-module equipped with the initial $\kappa$-linear derivation $\rd \c \sO_X \to \Omega^1_{X/\kappa}$. Concretely, $\Omega^1_{X/\kappa}$ is generated as an $\sO_X$-module by sections $\rd t \in \Gamma(U,\Omega^1_{X/\kappa})$ for $t \in \Gamma(U,\sO_X)$, subject to relations enforcing that the map of sheaves $\rd \c \sO_X \to \Omega^1_{X/\kappa}$ sending $t \mapsto \rd t$ is in fact a $\kappa$-linear derivation\footnote{Not $\sO_X$-linear!}.
  \end{subdefinition}

  \begin{subexample}
    \label{intro-dr-affine}
    Let $A \ce \kappa[t_1,\ldots,t_n]$. If $X = \bA^n_\kappa = \Spec(A)$, then $\Omega^1_{X/\kappa}$ is (the quasicoherent sheaf associated to) the free $A$-module on the symbols $\rd t_1,\ldots, \rd t_n$:
    \[
      \Omega_{X/k} \iso \bigoplus_{i=1}^n A \cdot \rd t_i.
    \]

    More generally, if $X = \Spec(A/I)$ for an ideal $I = (f_1,\ldots,f_m)$, then $\Omega^1_{X/\kappa}$ is the $A/I$-module generated by the symbols $\rd t_1, \ldots \rd t_n$ subject to the relations $\rd f_1 = \cdots = \rd f_m = 0$:
    \[
      \Omega_{X/k} \iso \frac{\bigoplus_{i=1}^n A/I \cdot \rd t_i}{\langle\rd f_1, \cdots, \rd f_m\rangle}.
    \]
  \end{subexample}

  \begin{subdefinition}
    \label{intro-dr-complex}
    Next, for $i \in \bZ_{\ge 0}$ we define $\Omega^i_{X/\kappa} \ce \bigwedge^i \Omega^1_{X/\kappa}$, the $i$-th exterior power over $\sO_X$. The differential extends via the Leibniz rule to give the \emph{de Rham complex}
    \[
      \sO_X = \Omega^0_{X/\kappa} \lblto{\rd} \Omega^1_{X/\kappa} \lblto{\rd} \Omega^2_{X/\kappa} \lblto{\rd} \cdots
    \]
    which we denote $\Omega^\bullet_{X/\kappa}$. The universal property of $\Omega^1_{X/\kappa}$ extends to a universal property of $\Omega^\bullet_{X/\kappa}$: it is the initial commutative differential-graded $\kappa$-algebra $\sA^\bullet$ on $X$ (meaning a complex of sheaves of $\kappa$-modules on $X$ with a commutative algebra structure) equipped with a map of sheaves $\sO_X \to \sA^0$.

    The \emph{de Rham cohomology} $\rH^*_\dR(X/\kappa)$ of $X$ over $\kappa$ is then defined to be the sheaf cohomology of the de Rham complex $\Omega^\bullet_{X/\kappa}$:
    \[
      \rH^*_\dR(X/\kappa) \ce \rR^*\Gamma(X,\Omega^\bullet_{X/\kappa}).
    \]
  \end{subdefinition}

  \begin{subconstruction}
    \label{intro-dr-ss}
    As with any complex, there is a so-called \emph{stupid filtration} on $\Omega^\bullet_{X/\kappa}$, i.e. the decreasing filtration
    \[
      \cdots \inj \Omega^{\bullet \ge i}_{X/\kappa} \inj \cdots \inj \Omega^{\bullet \ge 1}_{X/\kappa} \inj \Omega^{\bullet \ge 0}_{X/\kappa} = \Omega^{\bullet}_{X/\kappa},
    \]
    where $\Omega^{\bullet \ge i}_{X/\kappa}$ denotes the de Rham complex truncated to degrees $\ge i$ (at the chain level):
    \[
      0 \to \cdots \to 0 \to \Omega^i_{X/\kappa} \lblto{\rd} \Omega^{i+1}_{X/\kappa} \lblto{\rd} \cdots.
    \]
    Noting that the graded pieces of this filtration are just the terms of the de Rham complex, we have that this filtration determines a spectral sequence
    \begin{equation}
      \label{intro-dr-ss-formula}
      E_1^{i,j} = \rH^j(X;\Omega^i_{X/\kappa})
      \quad \implies \quad
      \rH_\dR^{i+j}(X/\kappa).
    \end{equation}
    This is known as the \emph{Hodge--de Rham spectral sequence}.
  \end{subconstruction}
\end{nothing}

\begin{nothing}
  \label{intro-complex}
  Now suppose $X$ is in fact a smooth projective variety over $\kappa = \bC$. Let $X^\an$ be its analytification, a compact K\"ahler complex manifold. Let $\Omega^\bullet_{X^\an}$ denote the holomorphic de Rham complex of $X^\an$.

  \begin{subnothing}
    \label{intro-complex-dr}
    As Dan said last time, the holomorphic Poincar\'e lemma implies that the sheaf cohomology of $\Omega^\bullet_{X^\an}$ computes the usual de Rham cohomology $\rH_\dR^*(X^\an)$ of $X^\an$ (with $\bC$ coefficients), and therefore GAGA supplies canonical isomorphisms
    \[
      \rH^*(X;\Omega^i_{X/\bC}) \iso \rH^*(X^\an;\Omega^i_{X^\an}), \quad
      \rH_\dR^*(X/\bC) \iso \rH_\dR^*(X^\an).
    \]
  \end{subnothing}

  \begin{subnothing}
    \label{intro-complex-degeneration}
    Hodge theory over $\bC$ gives isomorphisms of $\bC$-vector spaces
    \[
      \rH_\dR^k(X^\an) \iso \bigoplus_{i+j=k} \rH^j(X^\an;\Omega^i_{X^\an})
    \]
    (for all $k$). From the GAGA isomorphisms stated in \cref{intro-complex-dr} we then deduce isomorphisms
    \[
      \rH_\dR^k(X/\bC) \iso \bigoplus_{i+j=k} \rH^j(X;\Omega^i_{X/\bC}).
    \]
    Since $X$ is projective all these cohomology groups are finite-dimensional, so dimension-counting implies that the Hodge--de Rham spectral sequence \cref{intro-dr-ss-formula} for $X$ degenerates immediately at $E_1$. We'll refer to this phenomenon as \emph{Hodge degeneration}.

    Of course, classical Hodge theory over $\bC$ relies heavily on analytic notions, but we've found this purely algebraic consequence. This raises the following questions: Is there a purely algebraic proof of Hodge degeneration over $\bC$? Does Hodge degeneration occur over other fields $\kappa$ too?
  \end{subnothing}
\end{nothing}

\begin{nothing}
  \label{intro-deligne-illusie}
  It turns out that there is indeed a purely algebraic proof of Hodge degeneration, due to Deligne--Illusie, which applies to any smooth proper scheme over any characteristic zero field $\kappa$. The argument itself is perhaps even more interesting than the result, proceeding by understanding Hodge degeneration over positive characteristic fields (though degeneration only occurs in special circumstances in positive characteristic).

  Before discussing the rigorous ideas of the Deligne-Illusie argument, let us motivate what's to come with a fantasy and then a heuristic.

  \begin{subfantasy}
    \label{intro-deligne-illusie-fantasy}
    We saw in \cref{intro-complex-degeneration} that to prove Hodge degeneration---for $X$ proper over $\kappa$, so that we have finite-dimensional cohomology groups---it would suffice by dimension-counting to find any isomorphism of $\kappa$-vector spaces
    \[
      \rH^k_\dR(X/\kappa) \iso \bigoplus_{i+j=k} \rH^j(X;\Omega^i_{X/\kappa}).
    \]
    The left-hand side is by definition the $k$-th sheaf cohomology group of the de Rham complex $\Omega^\bullet_{X/\kappa}$. We may rewrite the right-hand side as the $k$-th sheaf cohomology group of the complex $\bigoplus_i \Omega^i_{X/\kappa}[i]$, which has the same terms as the de Rham complex but trivial differentials.

    Thus, to prove Hodge degeneration for $X/\kappa$, it would suffice to find an equivalence
    \begin{equation}
      \label{intro-deligne-illusie-fantasy-qiso}
      \Omega^\bullet_{X/\kappa} \iso \bigoplus_i \Omega^i_{X/\kappa}[i]
    \end{equation}
    in the derived category $\rD(X)$. The existence of such an equivalence is the fantasy behind the Deligne-Illusie argument. Note that, by passing to homology sheaves, such an equivalence would in particular beget isomorphisms
    \begin{equation}
      \label{intro-deligne-illusie-fantasy-hiso}
      \rH^i(\Omega^\bullet_{X/\kappa}) \iso \Omega^i_{X/\kappa}.
    \end{equation}
  \end{subfantasy}
  
  \begin{subheuristic}
    \label{intro-deligne-illusie-heuristic}
    I now want to recall a calculation Jesse made in the introductory talk, in order to investigate the plausibility of the fantasy \cref{intro-deligne-illusie-fantasy}. Set $X = \bA^1_\kappa = \Spec(\kappa[t])$ (which isn't proper but this is just a heuristic so don't worry about that). From \cref{intro-dr-affine} we see that the de Rham complex $\Omega^\bullet_{X/\kappa}$ is simply the two-term complex
    \[
      \kappa[t] \lblto{\rd} \kappa[t] \cdot \rd t,
    \]
    where $\rd \c \kappa[t] \to \kappa[t] \cdot \rd t$ is determined by the formula $\rd(t^n) = nt^{n-1} \rd t$.

    The key observation is that this complex has very different behavior in positive characteristic than in characteristic zero:
    \begin{enumerate}
    \item \label{intro-deligne-illusie-heuristic-z} Suppose $\chr(\kappa) = 0$. Then de Rham cohomology is as you would expect from the analytic intuition: the kernel of $\rd$ is given by the constants $\kappa \subset \kappa[t]$ since $\rd(t^n) \ne 0$ for $n > 0$; and $\rd$ is surjective since $t^n = \rd(\frac{1}{n+1}t^{n+1})$ for $n \ge 0$. So we have
      \[
        \rH_\dR^0(X/\kappa) \iso \kappa, \quad
        \rH_\dR^1(X/\kappa) \iso 0.
      \]
      In particular, the fantasized isomorphism \cref{intro-deligne-illusie-fantasy-hiso} fails dramatically here.

    \item \label{intro-deligne-illusie-heuristic-p} Suppose $\chr(\kappa) = p > 0$. Then the fact that $\rd(t^p) = pt^{p-1}\rd t = 0$ significantly alters the de Rham cohomology: the kernel of $\rd$ consists of polynomials of the form $f(t^p)$ for $f \in \kappa[t]$; and the image of $\rd$ consists of polynomials which do not contain any monomials of degree congruent to $-1$ modulo $p$. So we have
      \[
        \rH_\dR^0(X/\kappa) \iso \kappa[t^p], \quad
        \rH_\dR^1(X/\kappa) \iso t^{p-1}\kappa[t^p] \cdot \rd t.
      \]
      These of course are abstractly isomorphic to the original terms of the de Rham complex, $k[t]$ and $k[t] \cdot \rd t$. Thus, modulo the intervention of $p$-th powers, the fantasized isomorphism \cref{intro-deligne-illusie-fantasy-hiso} actually looks reasonable in this setting!
    \end{enumerate}
  \end{subheuristic}
\end{nothing}

The conclusion to draw from \cref{intro-deligne-illusie-heuristic} is that the argument outlined in \cref{intro-deligne-illusie-fantasy} may yield degeneration results in positive characteristic. This is indeed the case, as we will see in \cref{cartier,degenp}. On the other hand, \cref{intro-deligne-illusie-heuristic} also suggests the same argument will not apply in characteristic zero. Instead, one can deduce the characteristic zero degeneration result originally desired from the positive characteristic degeneration results by a separate argument, using the standard ``spreading out'' technique; this will be explained in \cref{degenz}.

% ---------------------------------------------------------------------

\section{The Cartier isomorphism}
\label{cartier}

\begin{notation}
  \label{cartier-char}
  Throughout this section we fix a prime $p$.
\end{notation}

Our goal in this section is to formalize the intuition we gained in \cref{intro-deligne-illusie-heuristic} about the de Rham complex in characteristic $p$. Namely, we will formulate and prove the precise version of the fantasized isomorphism \cref{intro-deligne-illusie-fantasy-hiso} that holds when $\chr(\kappa) = p$, known as the \emph{Cartier isomorphism}. To do so, we must carefully handle the $p$-th powers appearing, so we'll begin by recalling the Frobenius apparatus that comes with characterstic $p$ algebraic geometry.

\begin{definition}
  \label{cartier-frob}
  \begin{enumerate}[leftmargin=*]
  \item Let $S$ be an $\bF_p$-scheme. The \emph{absolute Frobenius morphism} $\rF_S \c S \to S$ is defined to be the identity map on the underlying topological space $|S|$ and the $p$-th power map $t \mapsto t^p$ on the structure sheaf $\sO_S$.

    This construction is natural in the sense that if  $\pi \c X \to S$ is a morphism of $\bF_p$-schemes then the diagram
    \begin{equation}
      \label{cartier-frob-natural}
      \begin{tikzcd}
        X \ar[r, "\rF_X"] \ar[d, "\pi", swap] &
        X \ar[d, "\pi"] \\
        S \ar[r, "\rF_S"] &
        S
      \end{tikzcd}
    \end{equation}
    evidently commutes.

  \item Let $\pi \c X \to S$ be a morphism of $\bF_p$-schemes. We define $\pi' \c X' \to S$ to be the base-change of $\pi$ by $\rF_S$; if there is potential for confusion then we will instead denote this base-change by $\pi'_S \c X'_S \to S$. The resulting morphism $X' \to X$ is denoted $\rG_{X/S}$; i.e. we have a fiber square
    \[
      \begin{tikzcd}
        X' \ar[r, "\rG_{X/S}"] \ar[d, "\pi'", swap] &
        X \ar[d, "\pi"] \\
        S \ar[r, "\rF_S"] &
        S.
      \end{tikzcd}      
    \]
    The diagram \cref{cartier-frob-natural} then defines a morphism $\rF_{X/S} \c X \to X'$ making the following diagram commute:
    \[
      \begin{tikzcd}
        X \ar[rd, "\rF_{X/S}"] \ar[rrd, "\rF_X", bend left] \ar[rdd, "\pi", bend right] &
        &
        \\
        &
        X' \ar[r, "\rG_{X/S}"] \ar[d, "\pi'", swap] &
        X \ar[d, "\pi"] \\
        &
        S \ar[r, "\rF_S"] &
        S.
      \end{tikzcd}      
    \]
  \end{enumerate}

  \begin{subexample}
    \label{cartier-frob-affine}
    The above definitions are clarified by examining the affine situation. Let $A$ be an $\bF_p$-algebra and $S \ce \Spec(A)$. Of course, $\rF_S \c S \to S$ corresponds to the usual Frobenius ring morphism $\rF_A \c A \to A$ sending $a \mapsto a^p$.

    Now let $B$ be an $A$-algebra and $X \ce \Spec(B)$. Letting $A^\rF$ denote $A$ viewed as an $A$-algebra via $\rF_A$, we have $X' \iso \Spec(B')$ for $B' \ce A^\rF \otimes_A B$. Then $\rG_{X/S} \c X' \to X$ is given by the canonical map $B \to B'$, though we should note that this sends $ab \mapsto a^p \otimes b$ for $a \in A, b \in B$. It follows that $\rF_{X/S} \c X \to X$ is given by the map $B' \to B$ sending $a \otimes b \mapsto ab^p$.

    Finally, this is all made especially clear in the example $B = A[t_1,\ldots,t_n]$, i.e. $X = \bA^n_S$. The base change of affine space is still affine space: $B' \iso A^\rF[t_1,\ldots,t_n]$. The map $\rG_{X/S} \c X' \to X$ is the map $A[t_1,\ldots,t_n] \to A^\rF[t_1,\ldots,t_n]$ sending $at_i \mapsto a^pt_i$ for $a \in A$ and $1 \le i \le n$. And the relative Frobenius $\rF_{X/S} \c X \to X'$ is the map $A^\rF[t_1,\ldots,t_n] \to A[t_1,\ldots,t_n]$ sending $at_i \mapsto at_i^p$ for $a \in A$ and $1 \le i \le n$.
  \end{subexample}
\end{definition}

\begin{notation}
  \label{cartier-situation}
  For the remainder of the section we fix a map of $\bF_p$-schemes $\pi \c X \to S$. If it is helpful, one should think of the case $S = \Spec(\kappa)$ for $\kappa$ a field of characteristic $p$.
\end{notation}

\begin{proposition}
  \label{cartier-frob-homeo}
  The maps $\rF_{X/S}$ and $\rG_{X/S}$ determine inverse homeomorphisms between the underlying topological spaces $|X|$ and $|X'|$. Moreover, the morphism $\rF_{X/S}$ is universally injective (i.e. injective after any base-change).

  \begin{proof}
    By construction we have $\rG_{X/S} \circ \rF_{X/S} = \rF_X$. From the analysis of the affine case in \cref{cartier-frob-affine} we see that also $\rF_{X/S} \circ \rG_{X/S} = \rF_{X'}$. Since the absolute Frobenius morphism is by definition the identity on underlying topological spaces, it follows that $\rF_{X/S}$ and $\rG_{X/S}$ are inverse homeomorphisms.

    As $\rG_{X/S}$ is the base-change of $\rF_S$, we deduce that absolute Frobenius morphisms are in fact universal homeomorphisms. Then the factorization $\rG_{X/S} \circ \rF_{X/S} = \rF_X$ implies that $\rF_{X/S}$ must be universally injective.
  \end{proof}

  \begin{subremark}
    \label{cartier-frob-homeo-univ}
    In fact, $\rF_{X/S}$ is a universal homeomorphism. We won't need this stronger claim, so if you don't care about this fact you can skip past this remark. But if you do, we see this as follows. We have already shown above that $\rF_{X/S}$ is universally injective and surjective (as it is a homeomorphism). Note that surjectivity is stable under base-change. So it suffices to show that $\rF_{X/S}$ is universally closed. But the absolute Frobenius $\rF_X$ is obviously integral, implying $\rF_{X/S}$ is integral. As integral maps are closed and stable under base-change, this proves the claim.
  \end{subremark}
\end{proposition}

\begin{nothing}
  \label{cartier-frob-identify}
  By \cref{cartier-frob-homeo}, we may identify topological spaces underlying $X$ and $X'$. In what follows we will indeed make this (perhaps abusive) identification, and hence think of the maps $\rF_{X/S}$ and $\rG_{X/S}$ as being the identity on underlying topological spaces. This allows us to omit the pushforward and topological pullback functors $(\rF_{X/S})_*, \rF_{X/S}^{-1}, (\rG_{X/S})_*, \rG_{X/S}^{-1}$ on sheaves.

  In particular, we will think of $\sO_X$ and $\sO_{X'}$ as sheaves on the same topological space $|X|$. This essentially allows us to work with the maps $\rF_{X/S}$ and $\rG_{X/S}$ on any scheme as we did in the affine setting of \cref{cartier-frob-affine}. We let $\sO_S^\rF$ denote $\sO_S$ viewed as an algebra over itself via the $p$-th power map $\rF_S^\sharp \c \sO_S \to \sO_S$. We then have $(\pi')^{-1}\sO_S \iso \pi^{-1}\sO_S^\rF$ and $\sO_{X'} \iso \pi^{-1}\sO_S^\rF \otimes_{\pi^{-1}\sO_S} \sO_X$, so sections of $\sO_{X'}$ are generated locally by sections of the form $a \otimes t$ for $a$ a section of $\pi^{-1}\sO_S$ and $t$ a section of $\sO_X$. The map $\rG_{X/S}^\sharp \c \sO_X \to \sO_{X'}$ sends $at \mapsto a^p \otimes t$ and the map $\rF_{X/S}^\sharp \c \sO_{X'} \to \sO_X$ sends $a \otimes t \mapsto a \otimes t^p$ (for $a,t$ as in the previous sentence).
\end{nothing}

\begin{lemma}
  \label{cartier-frob-etale}
  Suppose $\pi \c X \to S$ is \'etale. Then the relative Frobenius $\rF_{X/S}$ is an isomorphism.

  \begin{proof}
    Since $\pi$ is \'etale, so is its base-change $\pi' \c X' \to S$. Since $\pi' \circ \rF_{X/S} = \pi$ it follows that $\rF_{X/S}$ is \'etale. By \cref{cartier-frob-homeo} $\rF_{X/S}$ is also universally injective, and it's a fact that these two conditions together imply that $\rF_{X/S}$ is an open embedding. But \cref{cartier-frob-homeo} also tells us that $\rF_{X/S}$ is a homeomorphism, so we're done.
  \end{proof}
\end{lemma}

\begin{proposition}
  \label{cartier-frob-smooth}
  Suppose $\pi \c X \to S$ is smooth of relative dimension $n$. Then $\sO_X$ is locally free of rank $p^n$ over $\sO_{X'}$ (when viewed as an algebra via $\rF_{X/S}^\sharp$).

  \begin{proof}
    The question is local, so we may $\pi$ assume factors as a composition
    \[
      X \lblto{\rho} \bA^n_S \lblto{\sigma} S
    \]
    where $\rho$ is \'etale and $\sigma$ is the canonical map. Setting $Y \ce \bA^n_S$, we have the commutative diagram
    \[
      \begin{tikzcd}
        X \ar[dr, "\rF_{X/S}"] \ar[d, "\rF_{X/Y}", swap] \\
        X'_Y \ar[r, "\rF'_{Y/S}"] \ar[d, "\rho'_Y", swap] &
        X'_S \ar[r, "\rG_{X/S}"] \ar[d, "\rho'_S"] &
        X \ar[d, "\rho"] \\
        Y \ar[r, "\rF_{Y/S}"] \ar[dr, "\sigma", swap] &
        Y' \ar[r, "\rG_{Y/S}"] \ar[d, "\sigma'"] &
        Y \ar[d, "\sigma"] \\
        &
        S \ar[r, "\rF_S"] &
        S
      \end{tikzcd}
    \]
    in which all squares are pullbacks. Since $\rF_{X/Y}$ is an isomorphism by \cref{cartier-frob-etale}, this reduces us to the case $X = Y = \bA^n_S$, for which the claim follows from the explicit analysis of affine space in   \cref{cartier-frob-affine}. Namely, again localizing so that we may assume $S$ is an affine $\Spec(A)$, we saw that the relative Frobenius $\rF_{X/S}$ is given by the map $A[t_1,\ldots,t_n] \to A[t_1,\ldots,t_n]$ restricting to the identity on $A$ and sending $t_i \mapsto t_i^p$. This map exhibits $A[t_1,\ldots,t_n]$ as a free module over itself on the set of $p^n$ generators $\l\{\prod_i t_i^{r_i} : 0 \le r_i \le p-1\r\}$.
  \end{proof}
\end{proposition}

With the Frobenius apparatus in place, we may now discuss the de Rham complex in characteristic $p$.

\begin{nothing}
  \label{cartier-kahler}
  As $\pi' \c X' \to S$ is the base-change of $\pi \c X \to S$, the natural map $\rG_{X/S}^*\Omega^1_{X/S} \to \Omega^1_{X'/S}$ is an isomorphism. In other words, using the notational conventions described in \cref{cartier-frob-identify}, we may write
  \[
    \Omega^1_{X'/S} \iso \sO_{X'} \otimes_{\sO_X} \Omega^1_{X/S} \iso \pi^{-1}\sO_S^\rF \otimes_{\pi^{-1}\sO_S} \Omega^1_{X/S}.
  \]
  Accordingly, we will denote the image of a local section $\rd t$ of $\Omega^1_{X/S}$ in the canonical map $\Omega_{X/S}^1 \to \Omega^1_{X'/S}$ by $1 \otimes \rd t$.
\end{nothing}

\begin{proposition}
  \label{cartier-d-frob}
  \begin{enumerate}[leftmargin=*]
  \item \label{cartier-d-frob-zero}
    The composition
    \[
      \sO_{X'} \lblto{\rF_{X/S}^\sharp} \sO_X \lblto{\rd} \Omega^1_{X/S}
    \]
    is zero. It follows that $\rF_{X/S}^\sharp \c \Omega^1_{X'/S} \to \Omega^1_{X/S}$ is also zero.
  \item \label{cartier-d-frob-linear}
    The differential $\rd$ of the de Rham complex $\Omega^\bullet_{X/S}$ is $\sO_{X'}$-linear (as usual, via the map $\rF_{X/S}^\sharp \c \sO_{X'} \to \sO_X$). Thus $\Omega^\bullet_{X/S}$ is a commutative differential-graded $\sO_{X'}$-algebra, and hence its cohomology sheaves $\bigoplus_i \rH^i(\Omega_{X/S}^\bullet)$ form a commutative graded $\sO_{X'}$-algebra.
  \end{enumerate}

  \begin{proof}
    \begin{enumerate}[leftmargin=*]
    \item Recall from \cref{cartier-frob-identify} that $\rF_{X/S}^\sharp$ can be described as sending $a \otimes t \mapsto at^p$ for $a$ a local section of $\pi^{-1}\sO_S$ and $t$ a local section of $\sO_X$. The claim then follows from $\rd$ being $\pi^{-1}\sO_S$-linear (by definition) and the fact $\rd(t^p) = pt^{p-1} \rd t = 0$.
    \item This follows immediately from \cref{cartier-d-frob-zero} (by the Leibniz rule). \qedhere
    \end{enumerate}
  \end{proof}
\end{proposition}

\begin{theorem}[Cartier isomorphism]
  \label{cartier-iso}
  \begin{enumerate}[leftmargin=*]
  \item \label{cartier-iso-map} There exists a map of commutative graded $\sO_{X'}$-algebras
    \[
      \gamma_{X/S} = \gamma = \bigoplus_i \gamma^i \c \bigoplus_i \Omega^i_{X'/S} \to \bigoplus_i \rH^i(\Omega_{X/S}^\bullet)
    \]
    uniquely characterized by its behavior in degree $1$, where $\gamma^1$ sends the section $1 \otimes \rd t$ of $\Omega^1_{X'/S}$ (notation as in \cref{cartier-kahler}) to the class $[t^{p-1}\rd t]$ in $\rH^1(\Omega_{X/S}^\bullet)$.
  \item \label{cartier-iso-iso} If $\pi \c X \to S$ is smooth, the above map $\gamma$ is an isomorphism.
  \end{enumerate}

  \begin{proof}
    \begin{enumerate}[leftmargin=*]
    \item First note that the right-hand side does have the structure of a commutative graded $\sO_{X'}$-algebra by \cref{cartier-d-frob}. That the map $\gamma$ is uniquely determined by $\gamma^1$ is immediate from the fact that the left-hand side is the exterior algebra on $\Omega^1_{X'/S}$ over $\sO_{X'}$, which is the free commutative graded graded $\sO_{X'}$-algebra on $\Omega_{X'/S}^1$. To define $\gamma^1$ we use the universal property of $\Omega_{X'/S}^1$, as stated in \cref{intro-dr-differentials}: we just need to show that the map $\delta \c \sO_{X'} \to \rH^1(\Omega_{X/S}^\bullet)$ obtained by sending the section $1 \otimes t$ (notation as in \cref{cartier-frob-identify}) to the class $[t^{p-1}\rd t]$ and extending linearly determines a well-defined $(\pi')^{-1}\sO_S$-linear derivation on $\sO_{X'}$. It suffices to check that this formula is additive and satisfies the Leibniz rule:
      \begin{enumerate}
      \item \label{cartier-iso-map-additive} To show additivity, we compute $\delta(1 \otimes (t_1 + t_2)) - \delta(1 \otimes t_1) - \delta(1 \otimes t_2)$ to be the homology class of
        \[
          (t_1+t_2)^{p-1} \rd(t_1 + t_2) - t_1^{p-1}\rd t_1 - t_2^{p-1}\rd t_2 \in \Omega^1_{X/S},
        \]
        so we must show that this is a boundary. The point is that, while the antiderivative of a single term $t^{p-1} \rd t$ does not exist (since we cannot form $\frac{1}{p} t^p$), the antiderivative
        \[
          \frac{1}{p}\l((t_1+t_2)^p - t_1^p - t_2^p\r)
        \]
        of the above difference \emph{does} exist, since $(u+v)^p - u^p - v^p$ is formally divisible by $p$ as a polynomial in $\bZ[u,v]$.

      \item \label{cartier-iso-map-leibniz} For the Leibniz rule, we calculate for $t_1,t_2$ two local sections of $\sO_X$ that
        \begin{align*}
          \delta(1 \otimes t_1t_2)
          &= [(t_1t_2)^{p-1}\rd(t_1t_2)] \\
          &= [t_1^pt_2^{p-1} \rd t_2] + [t_2^pt_1^{p-1}\rd t_1] \\
          &= t_1 \cdot \delta(1 \otimes t_2) + t_2 \cdot \delta(1 \otimes t_1).
        \end{align*}
        noting that $\rH^1(\Omega^\bullet_{X/S})$ is an $\sO_{X'}$-module via $\rF_{X/S}^\sharp$, so that $t_1$ and $t_2$ indeed act by their $p$-th powers.
      \end{enumerate}

    \item The question is local, so we may assume $\pi$ factors as a composite
      \[
        X \lblto{\rho} \bA^n_S \lblto{\sigma} S
      \]
      where $\rho$ is \'{e}tale and $\sigma$ is the canonical map. Set $Y \ce \bA^n_S$. As in \cref{cartier-frob-smooth} we will reduce to the case that $X = \bA^n_S$ by analyzing the commutative diagram
      \[
        \begin{tikzcd}
          X \ar[dr, "\rF_{X/S}"] \ar[d, "\rF_{X/Y}", swap] \\
          X'_Y \ar[r, "\rF'_{Y/S}"] \ar[d, "\rho'_Y", swap] &
          X' \ar[r, "\rG_{X/S}"] \ar[d, "\rho'"] &
          X \ar[d, "\rho"] \\
          Y \ar[r, "\rF_{Y/S}"] \ar[dr, "\sigma", swap] &
          Y' \ar[r, "\rG_{Y/S}"] \ar[d, "\sigma'"] &
          Y \ar[d, "\sigma"] \\
          &
          S \ar[r, "\rF_S"] &
          S,
        \end{tikzcd}
      \]
      where all squares are cartesian. As $\rF_{X/S}$ is an isomorphism by \cref{cartier-frob-etale}, we deduce that the square
      \begin{equation}
        \label{cartier-iso-iso-pullback}
        \begin{tikzcd}
          X \ar[r, "\rF_{X/S}"] \ar[d, "\rho", swap] &
          X' \ar[d, "\rho'"] \\
          Y \ar[r, "\rF_{Y/S}"]  &
          Y'
        \end{tikzcd}
      \end{equation}
      is cartesian. Since $\rho$ is \'etale, the canonical map
      \begin{equation}
        \label{cartier-iso-iso-rhoetale}
        \rho^*\Omega^\bullet_{Y/S} \to \Omega^\bullet_{X/S}
      \end{equation}
      is an isomorphism of complexes. As $\rho'$ is a base-change of $\rho$, it is also \'etale, in particular flat. The relative Frobenii $\rF_{Y/S}$ and $\rF_{X/S}$ are finite by \cref{cartier-frob-smooth}, in particular quasicompact and quasiseparated, so flat base-change applies to the cartesian square \cref{cartier-iso-iso-pullback}.  Putting this together with the isomorphism \cref{cartier-iso-iso-rhoetale}, we get an isomorphism
      \[
        (\rho')^*(\rF_{Y/S})_*\Omega^\bullet_{Y/S} \isoto
        (\rF_{X/S})_*\rho^*\Omega^\bullet_{Y/S} \isoto (\rF_{X/S})_*\Omega^\bullet_{X/S}.
      \]
      Omitting the notation for pushforward by relative Frobenius as we've been doing since \cref{cartier-frob-identify}, we write this as an isomorphism $(\rho')^*\Omega^\bullet_{Y/S} \isoto \Omega^\bullet_{X/S}$. Again using that $\rho'$ is flat, this induces isomorphisms on homology sheaves
      \[
        (\rho')^*\rH^i(\Omega^\bullet_{Y/S}) \isoto \rH^i(\Omega^\bullet_{X/S}).
      \]
      This identifies the target of the Cartier map for $X/S$ with the pullback of the target of the Cartier map for $Y/S$. To do the same for the source, we observe that $\rho'$ being \'etale implies that the canonical maps $(\rho')^*\Omega^i_{Y'/S} \to \Omega^i_{X'/S}$ are isomorphisms. Finally, one checks that the definition of the Cartier map is natural in the sense that $(\rho')^*\gamma_{Y/S} = \gamma_{X/S}$. The above isomorphisms then reduce us to the case $X = Y = \bA^n_S$.

      So now set $X = \bA^n_S$ (with $\pi \c X \to S$ the canonical map). For convenience, write $S_0 \ce \Spec(\bF_p)$ and $X_0 \ce \bA^n_{S_0}$, and view $\Omega^\bullet_{X_0/S_0}$ as a complex of $\bF_p$-modules. There is a canonical map
      \[
        \pi^{-1}\sO_S \otimes_{\bF_p} \Omega^\bullet_{X_0/S_0} \lblto{\alpha} \Omega^\bullet_{X/S},
      \]
      which by inspection (we understand completely explicitly the de Rham complex of affine space) is an isomorphism. As tensoring over $\bF_p$ is exact, we also get an isomorphism on homology sheaves
      \[
        \pi^{-1}\sO_S \otimes_{\bF_p} \rH^i(\Omega^\bullet_{X_0/S_0}) \lblto{\rH^i(\alpha)} \rH^i(\Omega^\bullet_{X/S}).
      \]
      Using the identification $(\pi')^{-1}\sO_S \iso \pi^{-1}\sO_S^\rF$, we similarly see that the canonical map
      \[
        \pi^{-1}\sO_S^\rF \otimes_{\bF_p} \Omega^\bullet_{X_0/S_0} \lblto{\beta} \Omega^\bullet_{X'/S} \iso \pi^{-1}\sO_S^\rF \otimes_{\sO_S} \Omega^\bullet_{X/S}
      \]
      is an isomorphism. One now checks that the diagram
      \[
        \begin{tikzcd}[column sep = large]
          \pi^{-1}\sO_S^\rF \otimes_{\bF_p} \Omega^i_{X_0/S_0} \ar[r, "\id \otimes \gamma_{X_0/S_0}"] \ar[d, "\beta", swap] &
          \pi^{-1}\sO_S \otimes_{\bF_p} \rH^i(\Omega^\bullet_{X_0/S_0}) \ar[d, "\rH^i(\alpha)"] \\
          \Omega^i_{X'/S}\ar[r] &
          \rH^i(\Omega^\bullet_{X/S}).
        \end{tikzcd}
      \]
      commutes, allowing us to replace $X/S$ by $X_0/S_0 = \Spec(\bF_p[t_1,\ldots,t_n])/\Spec(\bF_p)$.

      Finally, using the canonical isomorphism
      \[
        \l(\Omega^\bullet_{\bF_p[t]/\bF_p}\r)^{\otimes n} \isoto \Omega^\bullet_{\bF_p[t_1,\ldots,t_n]/\bF_p}
      \]
      and the fact that $\gamma$ is a map of algebras, the Kunneth formula reduces us further to the case $n=1$, i.e. $X/S = \bA^1_{\bF_p}/\bF_p$. Here the claim is immediate from the motivating calculation \cref{intro-deligne-illusie-heuristic}\cref{intro-deligne-illusie-heuristic-p} of de Rham cohomology for $\bA^1_\kappa/\kappa$, which tells us that
      \begin{equation}
        \label{cartier-iso-iso-line}
        \rH_\dR^0(\bA^1_\kappa/\kappa) \iso \kappa[t^p], \qquad
        \rH_\dR^1(\bA^1_\kappa/\kappa) \iso t^{p-1} \kappa[t^p]\,\rd t. \qedhere
      \end{equation}
    \end{enumerate}
  \end{proof}
\end{theorem}

Before moving on, let's just reflect on the intuition behind the Cartier isomorphism. The Cartier map $\gamma$ is characterized by two properties:
\begin{enumerate}
\item It is a map of $\sO_{X'}$-algebras. I.e., in degree $0$ we have that $\gamma^0 = \rF_{X/S}^\sharp \c \sO_{X'} \to \sO_X$, and $\gamma$ is multiplicative.
\item In degree $1$ we have that $\gamma^1$ sends $1 \otimes \rd t \mapsto [t^{p-1} \rd t]$.
\end{enumerate}
Besides the condition of multiplicativity, which is natural enough, both of these requirements are directly motivated by our original heuristic calculation for the affine line, restated above as \cref{cartier-iso-iso-line}: the appearance of $\kappa[t^p] \subseteq \kappa[t]$ in zeroth de Rham cohomology motivates the appearance of the relative Frobenius as $\gamma^0$, and the generator $t^{p-1} \rd t$ in first de Rham cohomology motivates the definition of $\gamma^1$. Thus, the Cartier map is  entirely inspired by this one calculation; and as we just saw, the proof that it is an isomorphism (in the smooth case) ultimately boils down to this one calculation.

% ---------------------------------------------------------------------

\section{Hodge degeneration in positive characteristic}
\label{degenp}

\begin{notation}
  \label{degenp-setup}
  In this section we continue to work with our fixed prime $p$ as well as our map of $\bF_p$-schemes $\pi \c X \to S$. In addition, we assume throughout this section that $\pi$ is smooth. The notation regarding Frobenii established in the previous section also persists.
\end{notation}

The Cartier isomorphism \cref{cartier-iso} gives us a rigorous understanding of de Rham cohomology in characteristic $p$, formalizing \cref{intro-deligne-illusie-fantasy-hiso} from our original fantasy. Recall, however, that the Hodge degeneration argument outlined in the fantasy actually rested on something stronger, namely a lift of this isomorphism to an equivalence in the derived category \cref{intro-deligne-illusie-fantasy-qiso}.

To see that this is a serious issue, observe that the Cartier map $\gamma^1$, as characterized in \cref{cartier-iso}\cref{cartier-iso-map}, certainly does not lift immediately to the chain level: examining the proof of \cref{cartier-iso} reveals that passing to homology classes was crucial for making the assignment $1 \otimes \rd t \mapsto [t^{p-1} \rd t]$ a well-defined map.

What we should focus on now is that $t^{p-1} \rd t$ is begging to be interpreted as $\frac{1}{p} \rd(t^p)$. If we could simultaneously have $p$ be invertible and $t \mapsto t^p$ be a ring homomorphism, then this interpretation would allow us to easily see that the Cartier map was well-defined at the chain level. But of course these two desiderata are in direct tension with each other---indeed, all of \cref{cartier} was based on the failure of the former and the existence of the latter in characteristic $p$!

The Deligne-Illusie idea is that it is possible to make simultaneous approximate sense of both dividing by $p$ and the Frobenius when we can lift the characteristic $p$ situation to one just a little bit away from characteristic $p$, e.g. over $\bZ/p^2$. We'll now see how this works.

\begin{definition}
  \label{degenp-lift}
  Suppose given a morphism of schemes $S_1 \to S_2$, as well as an $S_1$-scheme $\pi_1 \c X_1 \to S_1$. A \emph{lifting of $X_1$ from $S_1$ to $S_2$} is an $S_2$-scheme $\pi_2 \c X_2 \to S_2$ equipped with an isomorphism $X_1 \iso S_1 \times_{S_2} X_2$, i.e. a pullback diagram
  \[
    \begin{tikzcd}
      X_1 \ar[r] \ar[d, "\pi_1", swap] &
      X_2 \ar[d, "\pi_2"] \\
      S_1 \ar[r] &
      S_2.
    \end{tikzcd}
  \]
  We say $X_2$ is a \emph{flat (resp. smooth) lifting} if $\pi_2$ is flat (resp. smooth), which of course can only occur if $\pi_1$ was already flat (resp. smooth).
\end{definition}

\begin{proposition}
  \label{degenp-cartier-lift-local}
  Suppose given a flat lifting $T$ of $S$ from $\Spec(\bF_p)$ to $\Spec(\bZ/p^2)$, and smooth liftings $Y$ and $Y'$ of $X$ and $X'$ from $S$ to $T$. Suppose also given a map $F \c Y \to Y'$ making the diagram
  \begin{equation}
    \label{degenp-cartier-lift-local-frob}
    \begin{tikzcd}
      X \ar[r] \ar[d, "\rF_{X/S}", swap] &
      Y \ar[d, "F"] \\
      X' \ar[r]  &
      Y'
    \end{tikzcd}
  \end{equation}
  commute. Then we may construct an explicit map of chain complexes $\phi^1 \c \Omega_{X'/S}^1[1] \to \Omega_{X/S}^\bullet$ inducing the Cartier isomorphism $\gamma^1$ in degree $1$ homology.

  \begin{subremark}
    \label{degenp-cartier-lift-local-intuition}
    The map $\phi^1$ is given roughly by the formula ``$\phi^1 = F^\sharp/p$'' (which one thinks of as making approximate sense of the formula ``$\rd t \mapsto \frac{1}{p} \rd(t^p)$''); the precise meaning of this is explained in the proof below.
  \end{subremark}

  \begin{proof}
    Since $\Spec(\bF_p) \inj \Spec(\bZ/p^2)$ is a square-zero closed immersion, so are all of its base-changes; in particular we may and shall identify the topological spaces underlying $X$ and $Y$, as well as $X'$ and $Y$. Since we are moreover identifying the topological spaces underlying $X$ and $X'$ via $\rF_{X/S}$, we will in fact identify all four topological spaces $|X|,|Y|,|X'|,|Y'|$, and omit certain notation accordingly.

    Let's study these square-zero extensions more precisely. We begin with the exact sequence
    \[
      0 \to \bF_p \lblto{[p]} \bZ/p^2 \to \bF_p \to 0,
    \]
    where $[p]$ denotes multiplication by $p$. Since all of our liftings are flat, we deduce exact sequences
    \begin{equation}
      \label{degenp-cartier-lift-local-functions-multp}
      0 \to \sO_X \lblto{[p]} \sO_Y \to \sO_X \to 0, \qquad
      0 \to \sO_{X'} \lblto{[p]} \sO_{Y'} \to \sO_{X'} \to 0.
    \end{equation}
    The canonical maps
    \[
      \Omega^1_{Y/T} \otimes_{\sO_Y} \sO_X \to \Omega^1_{X/S}, \qquad
      \Omega^1_{Y'/T} \otimes_{\sO_{Y'}} \sO_{X'} \to \Omega^1_{X'/S}
    \]
    are isomorphisms, and the sheaves of differentials $\Omega^1_{Y/T}$ and $\Omega^1_{Y'/T}$ are locally free over $\sO_Y$ and $\sO_{Y'}$ (since $Y$ and $Y'$ are smooth over $T$), so we furthermore obtain exact sequences
    \begin{equation}
      \label{degenp-cartier-lift-local-differentials-multp}
      0 \to \Omega^1_{X/S} \lblto{[p]} \Omega^1_{Y/T} \to \Omega^1_{X/S} \to 0, \qquad
      0 \to \Omega^1_{X'/S} \lblto{[p]} \Omega^1_{Y'/T} \to \Omega^1_{X'/S} \to 0.
    \end{equation}

    Next, the assumed compatibility \cref{degenp-cartier-lift-local-frob} of $F$ with $\rF_{X/S}$ gives us commutative diagrams
    \begin{equation}
      \label{degenp-cartier-lift-local-frob-differentials}
      \begin{tikzcd}
        \Omega^1_{Y'/T} \ar[r] \ar[d, "F^\sharp", swap] &
        \Omega^1_{X'/S} \ar[d, "\rF_{X/S}^\sharp"] \\
        \Omega^1_{Y/T} \ar[r]  &
        \Omega^1_{X/S}.
      \end{tikzcd}
    \end{equation}
    We make two observations about the map $F^\sharp \c \Omega^1_{Y'/T} \to \Omega^1_{Y/T}$:
    \begin{enumerate}
    \item By \cref{cartier-d-frob}\cref{cartier-d-frob-zero}, the map $\rF_{X/S}^\sharp$ in \cref{degenp-cartier-lift-local-frob-differentials} is zero. Thus the image of $F^\sharp$ is contained in the kernel of the bottom horizontal map. By \cref{degenp-cartier-lift-local-differentials-multp} this kernel is precisely the image of $[p] \c \Omega^1_{X/S} \to \Omega^1_{Y/T}$.
    \item Since $p^2 \Omega^1_{Y/T} = 0$, it follows from the previous observation that the image of $[p] \c \Omega^1_{X'/S} \to \Omega^1_{Y'/T}$ is contained in the kernel of $F^\sharp$. Therefore $F^\sharp$ factors through the cokernel of $[p]$, and by \cref{degenp-cartier-lift-local-differentials-multp} we see that this is precisely $\Omega^1_{X'/S}$.
    \end{enumerate}
    We conclude that there exists a (unique) map $\phi \c \Omega^1_{X'/S} \to \Omega^1_{X/S}$ making the diagram
    \[
      \begin{tikzcd}
        \Omega^1_{Y'/T} \ar[r, "F^\sharp"] \ar[d] &
        \Omega^1_{Y/T}  \\
        \Omega^1_{X'/S} \ar[r, "\phi"]  &
        \Omega^1_{X/S} \ar[u, "{[p]}", swap]
      \end{tikzcd}
    \]
    commute. Note that we may think of $\phi$ as ``$F^\sharp/p$'', as indicated in \cref{degenp-cartier-lift-local-intuition}.

    To finish, we just need to show that the image of $\phi$ lies in the $1$-cycles $\ker(\rd \c \Omega^1_{X/S} \to \Omega^2_{X/S})$, and that the resulting map $\o\phi \c \Omega^1_{X'/S} \to \rH^1(\Omega^\bullet_{X/S})$, obtained by passing to homology, agrees with the Cartier map $\gamma^1$ of \cref{cartier-iso}.

    Analogous to \cref{degenp-cartier-lift-local-frob-differentials}, we have at the level of functions a commutative diagram
    \begin{equation}
      \label{degenp-cartier-lift-local-frob-functions}
      \begin{tikzcd}
        \sO_{Y'} \ar[r] \ar[d, "F^\sharp", swap] &
        \sO_{X'} \ar[d, "\rF_{X/S}^\sharp"] \\
        \sO_{Y} \ar[r]  &
        \sO_{X},
      \end{tikzcd}
    \end{equation}
    and by \cref{degenp-cartier-lift-local-functions-multp} the horizontal maps are reduction mod $p$. Let $t$ be a local section of $\sO_X$. Let $t' \ce 1 \otimes t$ be the associated section of $\sO_{X'}$. Let $u$ and $u'$ be any sections of $\sO_Y$ and $\sO_{Y'}$ lifting $t$ and $t'$ in the reduction maps $\sO_{Y} \to \sO_{X}$ and $\sO_{Y'} \to \sO_{X'}$ (such lifts exist locally, and of course it suffices to work locally). Then $\rF^\sharp_{X/S}(t') = t^p$ is lifted by $u^p$, so the commutativity of \cref{degenp-cartier-lift-local-frob-functions} implies that
    \[
      F^\sharp(u') = u^p + pt_0
    \]
    for some section $t_0$ of $\sO_X$. It follows that
    \[
      F^\sharp(\rd u') = \rd(u^p + pt_0) = pu^{p-1}\rd u + p\rd t_0.
    \]
    As $u$ reduces to $t$ mod $p$, this is the image of $t^{p-1} \rd t + \rd t_0$ under the map $[p] \c \Omega^1_{X/S} \to \Omega^1_{Y/T}$. By definition of $\phi$ we deduce that
    \[
      \phi(\rd t') = t^{p-1} \rd t + \rd t_0.
    \]
    This formula proves that $\phi$ satisfies the desired properties.
  \end{proof}
\end{proposition}

\begin{proposition}
  \label{degenp-cartier-lift-honest-multiplicative}
  Suppose given a map of chain complexes $\phi^1 \c \Omega_{X'/S}^1[1] \to \Omega_{X/S}^\bullet$ inducing the Cartier isomorphism $\gamma^1$ in degree $1$ homology. Then we may extend this to a map of chain complexes
  \[
    \phi = \bigoplus \phi^i \c \bigoplus \Omega^i_{X'/S}[i] \to \Omega_{X/S}^\bullet
  \]
  inducing the full Cartier isomorphism $\gamma$ in homology.

  \begin{proof}
    As $\phi^1$ is given by an honest map $\Omega^1_{X'/S} \to \Omega^1_{X/S}$, we may simply define the map $\phi^i \c \Omega^i_{X'/S} \to \Omega^i_{X/S}$ to be the $i$-th exterior power of $\phi^1$. This works because $\gamma$ was defined to be multiplicative.
  \end{proof}
\end{proposition}

It follows from \cref{degenp-cartier-lift-local} and \cref{degenp-cartier-lift-honest-multiplicative} that the Cartier isomorphism may be fully lifted to a map at the chain level when given complete global lifting data over $\bZ/p^2$, i.e. a flat lift of $S$, smooth lifts of $X$ and $X'$, \emph{and} a lift of relative Frobenius $\rF_{X/S} \c X \to X'$. The next question to address is when such lifts can be obtained. Basic deformation theory (the necessary statements will be recalled below) says that in general there are obstructions to the existence of such lifts, valued in certain sheaf cohomology groups. These cohomology groups will vanish locally (namely on affines), and hence the above can be used to obtain local lifts of the Cartier map. We will then try to understand to what extend these lifts can be glued to obtain one globally.

\begin{proposition}
  \label{degenp-cartier-lift-global}
  Suppose given a flat lifting $T$ of $S$ from $\Spec(\bF_p)$ to $\Spec(\bZ/p^2)$, and a smooth lifting  $Y'$ of $X'$ from $S$ to $T$. Then there exists a map of $\phi^1 \c \Omega_{X'/S}^1[1] \to \Omega_{X/S}^\bullet$ in the derived category $D(X')$ inducing the Cartier isomorphism $\gamma^1$ in degree $1$ homology.
\end{proposition}

\begin{lemma}
  \label{degen--lift--extend}
  Suppose there is a map $\phi^1 \c \Omega_{X'/S}^1[-1] \to \Omega_{X/S}^\bullet$ in $D(X')$ which induces the Cartier isomorphism $\gamma^1$ in degree $1$ cohomology. Then there is a map $\phi^i \c \Omega_{X'/S}^i[-i] \to \Omega_{X/S}^\bullet$ inducing the Cartier isomorphism $\gamma^i$ in degree $i$ cohomology for all $0 \le i < p$.
\end{lemma}

\begin{proof}
  Firstly, we always have $\phi^0 \ce F_{X/S}^\sharp \c \sO_{X'} \to \sO_X$ inducing the Cartier isomorphism in degree $0$ by definition, so we just need to worry about $2 \le i < p$.

  Since $X'$ and $X$ are smooth over $S$, $\Omega_{X'/S}^i$ is a locally free $\sO_{X'}$-module and $\Omega_{X/S}^i$ is a locally free $\sO_X$-module; but by \cref{degen--cartier--frobenius-smooth}, $\sO_X$ is a locally free $\sO_{X'}$-module, so $\Omega_{X/S}^i$ is locally free as an $\sO_{X'}$-module as well. It follows that the $i$-fold \emph{derived} tensor powers in $D(X')$ of $\Omega_{X'/S}^1[-1]$ and $\Omega_{X/S}^\bullet$ are their ordinary tensor powers. Thus taking the $i$-fold derived tensor power of $\phi^1$ gives us a map
  \[
    (\phi^1)^{\otimes i} \c \Omega_{X'/S}^{\otimes i}[-i] \to (\Omega_{X/S}^\bullet)^{\otimes i}
  \]
  in $D(X')$. We can compose with the product map $(\Omega_{X/S}^\bullet)^{\otimes i} \to \Omega_{X/S}^\bullet$ on the right-hand side, and on the left-hand side we can precompose with the standard map
  \[
    \Omega_{X'/S}^i \to \Omega_{X'/S}^{\otimes i}, \quad
    \omega_1 \wedge \cdots \wedge \omega_i \mapsto \frac{1}{i!} \sum_{\sigma \in \rA_i} \operatorname{sgn}(\sigma)\, \omega_{\sigma(1)} \otimes \cdots \otimes \omega_{\sigma(i)},
  \]
  which makes sense for $i < p$ in characteristic $p$, begetting our desired map
  \[
    \phi^i \c \Omega_{X'/S}^i[-i] \to \Omega_{X/S}^\bullet.
  \]
  That $\phi^i$ in fact induces the Cartier isomorphism $\gamma^i$ follows from the fact that $\gamma^i$ was defined by extending $\gamma^1$ multiplicatively.
\end{proof}

\begin{definition}
  Suppose $S = \Spec \kappa$ for $\kappa$ a perfect field of characteristic $p$. Note that $\kappa$ being perfect means that $F_S$ is an isomorphism, and hence its base-change $G_{X/S}$ is an isomorphism of $S$-schemes $X' \iso X$.

  Then there is a unique flat lift $T = \Spec W_2(\kappa)$ of $S$ over $\lZ/p^2$. The ring $W_2(k)$ is called the \emph{ring of Witt vectors of length $2$} over $\kappa$.
\end{definition}

\begin{theorem}
  \label{degen--lift--degen}
  Let $\kappa$ a perfect field of characteristic $p$. Let $X$ be smooth, proper $\kappa$-scheme of dimension $< p$. If $X$ can be lifted over $W_2(\kappa)$, then the Hodge--de Rham spectral sequence
  \[
    E_1^{i,j} = \rH^j(X;\Omega_{X/\kappa}^i) \quad \Rightarrow \quad
    \rH_\dR^{i+j}(X/\kappa)
  \]
  degenerates at $E_1$.
\end{theorem}

\begin{proof}
  We implement our strategy \cref{degen--strategy}. It suffices to show that
  \[
    \dim_\kappa \rH_\dR^k(X/\kappa) = \sum_{i+j=k} \dim_\kappa \rH^j(X;\Omega_{X/\kappa}^i),
  \]
  where all dimensions are finite since $X$ is proper. Let $S \ce \Spec \kappa$; since $\kappa$ is perfect, the Frobenius $F_S$ is an isomorphism, and hence we have a base change isomorphism
  \[
    F_S^*\rH^j(X';\Omega_{X'/\kappa}^i) \iso \rH^j(X;\Omega_{X/\kappa}^i) \implies \rH^j(X';\Omega_{X'/\kappa}^i) = \dim_\kappa \rH^j(X;\Omega_{X/\kappa}^i).
  \]
  By \cref{degen--lift--main} and \cref{degen--lift--extend}, the hypotheses that $X$ can be lifted over $W_2(\kappa)$ and has dimension $< p$ implies that we have an equivalence
  \[
    \bigoplus_i \Omega_{X'/k}^k[-i] \iso \Omega_{X/k}^\bullet
  \]
  in $D(X')$; after applying sheaf cohomology we obtain an isomorphism
  \[
    \bigoplus_{i+j=k} \rH^j(X;\Omega_{X/\kappa}^i) = \rH_\dR^k(X/\kappa)
  \]
  Putting everything we've said together proves the claim.
\end{proof}

% ---------------------------------------------------------------------

\section{Hodge degeneration in characteristic zero}
\label{degenz}

% ---------------------------------------------------------------------

% \bibliographystyle{amsalpha}
% \bibliography{refs}

\end{document}
